% !TeX program = XeLaTeX
% !TeX root = ../../shloka.tex

\sect{दक्षिणामूर्त्यष्टकम्}

\fourlineindentedshloka
{विश्वं दर्पणदृश्यमाननगरीतुल्यं निजान्तर्गतं}
{पश्यन्नात्मनि मायया बहिरिवोद्भूतं यदा निद्रया}
{यः साक्षात्कुरुते प्रबोधसमये स्वात्मानमेवाद्वयं}
{तस्मै श्रीगुरुमूर्तये नम इदं श्रीदक्षिणामूर्तये}

\fourlineindentedshloka
{बीजस्यान्तरिवाङ्कुरो जगदिदं प्राङ्\mbox{}निर्विकल्पं पुनः}
{मायाकल्पितदेशकालकलनावैचित्र्यचित्रीकृतम्}
{मायावीव विजृम्भयत्यपि महायोगीव यः स्वेच्छया}
{तस्मै श्रीगुरुमूर्तये नम इदं श्रीदक्षिणामूर्तये}

\fourlineindentedshloka
{यस्यैव स्फुरणं सदाऽऽत्मकमसत्कल्पार्थकं भासते}
{साक्षात् तत्त्वमसीति वेदवचसा यो बोधयत्याश्रितान्}
{यत्साक्षात्करणाद्भवेन्न पुनरावृत्तिर्भवाम्भोनिधौ}
{तस्मै श्रीगुरुमूर्तये नम इदं श्रीदक्षिणामूर्तये}

\fourlineindentedshloka
{नानाच्छिद्रघटोदरस्थितमहादीपप्रभाभास्वरं}
{ज्ञानं यस्य तु चक्षुरादिकरणद्वारा बहिः स्पन्दते}
{जानामीति तमेव भान्तमनुभात्येतत्समस्तं जगत्}
{तस्मै श्रीगुरुमूर्तये नम इदं श्रीदक्षिणामूर्तये}

\fourlineindentedshloka
{देहं प्राणमपीन्द्रियाण्यपि चलां बुद्धिं च शून्यं विदुः}
{स्त्रीबालान्धजडोपमास्त्वहमिति भ्रान्ता भृशं वादिनः}
{मायाशक्तिविलासकल्पितमहाव्यामोहसंहारिणे}
{तस्मै श्रीगुरुमूर्तये नम इदं श्रीदक्षिणामूर्तये}

\fourlineindentedshloka
{राहुग्रस्तदिवाकरेन्दुसदृशो मायासमाच्छादनात्}
{सन्मात्रः करणोपसंहरणतो योऽभूत्सुषुप्तः पुमान्}
{प्रागस्वाप्समिति प्रबोधसमये यः प्रत्यभिज्ञायते}
{तस्मै श्रीगुरुमूर्तये नम इदं श्रीदक्षिणामूर्तये}

\fourlineindentedshloka
{बाल्यादिष्वपि जागृदादिषु तथा सर्वास्ववस्थास्वपि}
{व्यावृत्तास्वनुवर्तमानमहमित्यन्तः स्फुरन्तं सदा}
{स्वात्मानं प्रकटीकरोति भजतां यो मुद्रया भद्रया}
{तस्मै श्रीगुरुमूर्तये नम इदं श्रीदक्षिणामूर्तये}

\fourlineindentedshloka
{विश्वं पश्यति कार्यकारणतया स्वस्वामिसम्बन्धतः}
{शिष्याचार्यतया तथैव पितृपुत्राद्यात्मना भेदतः}
{स्वप्ने जागृति वा य एष पुरुषो मायापरिभ्रामितः}
{तस्मै श्रीगुरुमूर्तये नम इदं श्रीदक्षिणामूर्तये}

\fourlineindentedshloka*
{भूरम्भांस्यनलोऽनिलोऽम्बरमहर्नाथो हिमांशुः पुमान्}
{इत्याभाति चराचरात्मकमिदं यस्यैव मूर्त्यष्टकम्}
{नान्यत् किञ्चन विद्यते विमृशतां यस्मात्परस्माद्विभोः}
{तस्मै श्रीगुरुमूर्तये नम इदं श्रीदक्षिणामूर्तये}

\fourlineindentedshloka*
{सर्वात्मत्वमिति स्फुटीकृतमिदं यस्मादमुष्मिन् स्तवे}
{तेनास्य श्रवणात्तदर्थमननाद्ध्यानाच्च सङ्कीर्तनात्}
{सर्वात्मत्वमहाविभूतिसहितं स्यादीश्वरत्वं स्वतः}
{सिध्येत् तत्पुनरष्टधा परिणतं चैश्वर्यमव्याहतम्}

॥इति श्रीमत्परमहंसपरिव्राजकाचार्यस्य श्री-गोविन्द-भगवत्पूज्य-पाद-शिष्यस्य
श्रीमच्छङ्करभगवतः कृतौ श्री-दक्षिणामूर्त्यष्टकं सम्पूर्णम्॥

\closesection
\fourlineindentedshloka*
{वटविटपिसमीपे भूमिभागे निषण्णं}
{सकलमुनिजनानां ज्ञानदातारमारात्}
{त्रिभुवनगुरुमीशं दक्षिणामूर्तिदेवं}
{जननमरणदुःखच्छेददक्षं नमामि}