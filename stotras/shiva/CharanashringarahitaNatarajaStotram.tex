% !TeX program = XeLaTeX
% !TeX root = ../../shloka.tex
\sect{चरणशृङ्गरहित-नटराज-स्तोत्रम्}
\setlength{\shlokaspaceskip}{4pt}
\fourlineindentedshloka
{सदञ्चित-मुदञ्चित-निकुञ्चित-पदं झलझलञ्चलित-मञ्जु-कटकम्}
{पतञ्जलि-दृगञ्जन-मनञ्जन-मचञ्चलपदं जनन-भञ्जन-करम्}
{कदम्बरुचिमम्बरवसं परममम्बुद-कदम्बक-विडम्बक-गलम्}
{चिदम्बुधि-मणिं बुध-हृदम्बुज-रविं-पर-चिदम्बर-नटं हृदि भज} %1

\fourlineindentedshloka
{हरं त्रिपुर-भञ्जनम् अनन्तकृतकङ्कणम् अखण्डदय-मन्तरहितम्}
{विरिञ्चिसुरसंहतिपुरन्धर-विचिन्तितपदं तरुणचन्द्रमकुटम्}
{परं पद-विखण्डितयमं भसित-मण्डिततनुं मदनवञ्चन-परम्}
{चिरन्तनममुं प्रणवसञ्चितनिधिं पर-चिदम्बर-नटं हृदि भज} %2

\fourlineindentedshloka
{अनन्तमखिलं जगदभङ्ग-गुणतुङ्गममतं धृतविधुं सुरसरित्}
{तरङ्ग-निकुरम्ब-धृति-लम्पट-जटं शमनदम्बसुहरं भवहरम्}
{शिवं दशदिगन्तर-विजृम्भितकरं करलसन्मृगशिशुं पशुपतिम्}
{हरं शशिधनञ्जयपतङ्गनयनं पर-चिदम्बर-नटं हृदि भज} %3

\fourlineindentedshloka
{अनन्तनवरत्नविलसत्कटककिङ्किणिझलं झलझलं झलरवम्}
{मुकुन्दविधि-हस्तगतमद्दल-लयध्वनिधिमिद्धिमित-नर्तन-पदम्}
{शकुन्तरथ-बर्हिरथ-नन्दिमुख-शृङ्गिरिटि-भृङ्गिगण-सङ्घनिकटम्}
{सनन्दसनक-प्रमुख-वन्दित-पदं पर-चिदम्बर-नटं हृदि भज} %4

\fourlineindentedshloka
{अनन्तमहसं त्रिदशवन्द्य-चरणं मुनि-हृदन्तर-वसन्तममलम्}
{कबन्ध-वियदिन्द्ववनि-गन्धवह-वह्निमख-बन्धुरविमञ्जु-वपुषम्}
{अनन्तविभवं त्रिजगदन्तर-मणिं त्रिनयनं त्रिपुर-खण्डन-परम्}
{सनन्द-मुनि-वन्दित-पदं सकरुणं पर-चिदम्बर-नटं हृदि भज} %5

\fourlineindentedshloka
{अचिन्त्यमलिवृन्द-रुचि-बन्धुरगलं कुरित-कुन्द-निकुरम्ब-धवलम्}
{मुकुन्द-सुरवृन्द-बलहन्तृ-कृतवन्दन-लसन्तम्-अहिकुण्डल-धरम्}
{अकम्पमनुकम्पित-रतिं सुजन-मङ्गलनिधिं गजहरं पशुपतिम्}
{धनञ्जयनुतं प्रणत-रञ्जनपरं पर-चिदम्बर-नटं हृदि भज} %6

\fourlineindentedshloka
{परं सुरवरं पुरहरं पशुपतिं जनित-दन्तिमुख-षण्मुखममुम्}
{मृडं कनक-पिङ्गल-जटं सनकपङ्कज-रविं सुमनसं हिमरुचिम्}
{असङ्घमनसं जलधि-जन्मकरलं कवलयन्त-मतुलं गुणनिधिम्}
{सनन्द-वरदं शमितमिन्दु-वदनं पर-चिदम्बर-नटं हृदि भज} %7

\fourlineindentedshloka
{अजं क्षितिरथं भुजगपुङ्गवगुणं कनक-शृङ्गि-धनुषं करलसत्}
{कुरङ्ग-पृथु-टङ्क-परशुं रुचिर-कुङ्कुम-रुचिं डमरुकं च दधतम्}
{मुकुन्द-विशिखं नमदवन्ध्य-फलदं निगम-वृन्द-तुरगं निरुपमम्}
{सचण्डिकममुं झटिति संहृतपुरं पर-चिदम्बर-नटं हृदि भज} %8

\fourlineindentedshloka
{अनङ्गपरिपन्थिनमजं क्षिति-धुरन्धरमलं करुणयन्तमखिलम्}
{ज्वलन्तमनलं दधतमन्तकरिपुं सततमिन्द्रमुख-वन्दितपदम्}
{उदञ्चदरविन्दकुल-बन्धुशत-बिम्बरुचि-संहति-सुगन्धि-वपुषम्}
{पतञ्जलिनुतं प्रणवपञ्जर-शुकं पर-चिदम्बर-नटं हृदि भज} %9

\fourlineindentedshloka
{इति स्तवममुं भुजगपुङ्गव-कृतं प्रतिदिनं पठति यः कृतमुखः}
{सदः प्रभुपद-द्वितयदर्शनपदं सुललितं चरण-शृङ्ग-रहितम्}
{सरःप्रभव-सम्भव-हरित्पति-हरिप्रमुख-दिव्यनुत-शङ्करपदम्}
{स गच्छति परं न तु जनुर्जलनिधिं परमदुःखजनकं दुरितदम्} %10

॥इति श्रीपतञ्जलिमुनिप्रणीतं चरणशृङ्गरहित-नटराजस्तोत्रं सम्पूर्णम्॥
\setlength{\shlokaspaceskip}{24pt}