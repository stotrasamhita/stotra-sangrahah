% !TeX program = XeLaTeX
% !TeX root = ../../shloka.tex
\sect{शम्भुस्तुतिः}

\fourlineindentedshloka
{नमामि शम्भुं पुरुषं पुराणं}
{नमामि सर्वज्ञमपारभावम्}
{नमामि रुद्रं प्रभुमक्षयं तं}
{नमामि शर्वं शिरसा नमामि}% ॥ १॥

\fourlineindentedshloka
{नमामि देवं परमव्ययं तम्}
{उमापतिं लोकगुरुं नमामि}
{नमामि दारिद्र्यविदारणं तं}
{नमामि रोगापहरं नमामि}% ॥ २॥

\fourlineindentedshloka
{नमामि कल्याणमचिन्त्यरूपं}
{नमामि विश्वोद्भवबीजरूपम्}
{नमामि विश्वस्थितिकारणं तं}
{नमामि संहारकरं नमामि}% ॥ ३॥

\fourlineindentedshloka
{नमामि गौरीप्रियमव्ययं तं}
{नमामि नित्यं क्षरमक्षरं तम्}
{नमामि चिद्रूपममेयभावं}
{त्रिलोचनं तं शिरसा नमामि}% ॥ ४॥

\fourlineindentedshloka
{नमामि कारुण्यकरं भवस्य}
{भयङ्करं वाऽपि सदा नमामि}
{नमामि दातारमभीप्सितानां}
{नमामि सोमेशमुमेशमादौ}% ॥ ५॥

\fourlineindentedshloka
{नमामि वेदत्रयलोचनं तं}
{नमामि मूर्तित्रयवर्जितं तम्}
{नमामि पुण्यं सदसद्व्यतीतं}
{नमामि तं पापहरं नमामि}% ॥ ६॥

\fourlineindentedshloka
{नमामि विश्वस्य हिते रतं तं}
{नमामि रूपाणि बहूनि धत्ते}
{यो विश्वगोप्ता सदसत्प्रणेता}
{नमामि तं विश्वपतिं नमामि}% ॥ ७॥

\fourlineindentedshloka
{यज्ञेश्वरं सम्प्रति हव्यकव्यं}
{तथागतिं लोकसदाशिवो यः}
{आराधितो यश्च ददाति सर्वं}
{नमामि दानप्रियमिष्टदेवम्}% ॥ ८॥

\fourlineindentedshloka
{नमामि सोमेश्वरमस्वतन्त्रम्}
{उमापतिं तं विजयं नमामि}
{नमामि विघ्नेश्वरनन्दिनाथं}
{पुत्रप्रियं तं शिरसा नमामि}% ॥ ९॥

\fourlineindentedshloka
{नमामि देवं भवदुःखशोक-}
{विनाशनं चन्द्रधरं नमामि}
{नमामि गङ्गाधरमीशमीड्यम्}
{उमाधवं देववरं नमामि}% ॥ १०॥

\fourlineindentedshloka
{नमाम्यजादीशपुरन्दरादि-}
{सुरासुरैरर्चितपादपद्मम्}
{नमामि देवीमुखवादनानाम्}
{ईक्षार्थमक्षित्रितयं य ऐच्छत्}% ॥ ११॥

\fourlineindentedshloka
{पञ्चामृतैर्गन्धसुधूपदीपैर्-}
{विचित्रपुष्पैर्विविधैश्च मन्त्रैः}
{अन्नप्रकारैः सकलोपचारैः}
{सम्पूजितं सोममहं नमामि}% ॥ १२॥

॥इति श्रीब्रह्ममहापुराणे त्रयोविंशाधिकशततमाध्यायान्तर्गतं श्रीरामकृतं शम्भुस्तुतिः सम्पूर्णा॥

% ब्रह्मपुराणम् --- अध्याय १२३ (गौतमीय ५४) --- १९५-२०६॥
