% !TeX program = XeLaTeX
% !TeX root = ../../shloka.tex

\sect{कृष्णाष्टकम् १}
\fourlineindentedshloka
{श्रियाश्लिष्टो विष्णुः स्थिरचरगुरुर्वेदविषयो}
{धियां साक्षी शुद्धो हरिरसुरहन्ताब्जनयनः}
{गदी शङ्खी चक्री विमलवनमाली स्थिररुचिः}
{शरण्यो लोकेशो मम भवतु कृष्णोऽक्षिविषयः}

\fourlineindentedshloka
{यतः सर्वं जातं वियदनिलमुख्यं जगदिदम्}
{स्थितौ निःशेषं योऽवति निजसुखांशेन मधुहा}
{लये सर्वं स्वस्मिन् हरति कलया यस्तु स विभुः}
{शरण्यो लोकेशो मम भवतु कृष्णोऽक्षिविषयः}

\fourlineindentedshloka
{असूनायम्यादौ यमनियममुख्यैः सुकरणैः}
{निरुद्‌ध्येदं चित्तं हृदि विलयमानीय सकलम्}
{यमीड्यं पश्यन्ति प्रवरमतयो मायिनमसौ}
{शरण्यो लोकेशो मम भवतु कृष्णोऽक्षिविषयः}

\fourlineindentedshloka
{पृथिव्यां तिष्ठन् यो यमयति महीं वेद न धरा}
{यमित्यादौ वेदो वदति जगतामीशममलम्}
{नियन्तारं ध्येयं मुनिसुरनृणां मोक्षदमसौ}
{शरण्यो लोकेशो मम भवतु कृष्णोऽक्षिविषयः}

\fourlineindentedshloka
{महेन्द्रादिर्देवो जयति दितिजान् यस्य बलतो}
{न कस्य स्वातन्त्र्यं क्वचिदपि कृतौ यत्कृतिमृते}
{बलारातेर्गर्वं परिहरति योऽसौ विजयिनः}
{शरण्यो लोकेशो मम भवतु कृष्णोऽक्षिविषयः}

\fourlineindentedshloka
{विना यस्य ध्यानं व्रजति पशुतां सूकरमुखाम्}
{विना यस्य ज्ञानं जनिमृतिभयं याति जनता}
{विना यस्य स्मृत्या कृमिशतजनिं याति स विभुः}
{शरण्यो लोकेशो मम भवतु कृष्णोऽक्षिविषयः}

\fourlineindentedshloka
{नरातङ्कोट्टङ्कः शरणशरणो भ्रान्तिहरणो}
{घनश्यामो वामो व्रजशिशुवयस्योऽर्जुनसखः}
{स्वयम्भूर्भूतानां जनक उचिताचारसुखदः}
{शरण्यो लोकेशो मम भवतु कृष्णोऽक्षिविषयः}

\fourlineindentedshloka
{यदा धर्मग्लानिर्भवति जगतां क्षोभकरणी}
{तदा लोकस्वामी प्रकटितवपुः सेतुधृगजः}
{सतां धाता स्वच्छो निगमगणगीतो व्रजपतिः}
{शरण्यो लोकेशो मम भवतु कृष्णोऽक्षिविषयः}

॥इति श्रीमत्परमहंसपरिव्राजकाचार्यस्य श्री-गोविन्द-भगवत्पूज्य-पाद-शिष्यस्य
श्रीमच्छङ्करभगवतः कृतौ श्री-कृष्णाष्टकं सम्पूर्णम्॥