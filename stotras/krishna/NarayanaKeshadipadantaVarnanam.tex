% !TeX program = XeLaTeX
% !TeX root = ../../shloka.tex

\sect{नारायण केशादिपादवर्णनम्}

\fourlineindentedshloka
{अग्रे पश्यामि तेजो निबिडतरकलायावलीलोभनीयं}
{पीयूषाप्लावितोऽहं तदनु तदुदरे दिव्यकैशोरवेषम्}
{तारुण्यारम्भरम्यं परमसुखरसास्वादरोमाञ्चिताङ्गै-}
{रावीतं नारदाद्यैर्विलसदुपनिषत्सुन्दरीमण्डलैश्च}

\fourlineindentedshloka
{नीलाभं कुञ्चिताग्रं घनममलतरं संयतं चारुभङ्ग्या}
{रत्नोत्तंसाभिरामं वलयितमुदयच्चन्द्रकैः पिञ्छजालैः}
{मन्दारस्रङ्निवीतं तव पृथुकबरीभारमालोकयेऽहं}
{स्निग्धश्चेतोर्ध्वपुण्ड्रामपि च सुललितां फालबालेन्दुवीथीम्}

\fourlineindentedshloka
{हृद्यं पूर्णानुकम्पार्णवमृदुलहरीचञ्चलभ्रूविलासै-}
{रानीलस्निग्धपक्ष्मावलिपरिलसितं नेत्रयुग्मं विभो ते}
{सान्द्रच्छायं विशालारुणकमलदलाकारमामुग्धतारं}
{कारुण्यालोकलीलाशिशिरितभुवनं क्षिप्यतां मय्यनाथे}

\fourlineindentedshloka
{उत्तुङ्गोल्लासिनासं हरिमणिमुकुरप्रोल्लसद्गण्डपाली-}
{व्यालोलत्कर्णपाशाञ्चितमकरमणीकुण्डलद्वन्द्वदीप्रम्}
{उन्मीलद्दन्तपङ्क्तिस्फुरदरुणतरच्छायबिम्बाधरान्तः}
{प्रीतिप्रस्यन्दिमन्दस्मितमधुरतरं वक्त्रमुद्भासतां मे}

\fourlineindentedshloka
{बाहुद्वन्द्वेन रत्नोज्ज्वलवलयभृता शोणपाणिप्रवाले-}
{नोपात्तां वेणुनालीं प्रसृतनखमयूखाङ्गुलीसङ्गशाराम्}
{कृत्वा वक्त्रारविन्दे सुमधुरविकसद्रागमुद्भाव्यमानैः}
{शब्दब्रह्मामृतैस्त्वं शिशिरितभुवनैस्सिञ्च मे कर्णवीथीम्}

\fourlineindentedshloka
{उत्सर्पत्कौस्तुभश्रीततिभिररुणितं कोमलं कण्ठदेशं}
{वक्षः श्रीवत्सरम्यं तरलतरसमुद्दीप्रहारप्रतानम्}
{नानावर्णप्रसूनावलिकिसलयिनीं वन्यमालां विलोल-}
{ल्लोलम्बां लम्बमानामुरसि तव तथा भावये रत्नमालाम्}

\fourlineindentedshloka
{अङ्गे पञ्चाङ्गरागैरतिशयविकसत्सौरभाकृष्टलोकं}
{लीनानेकत्रिलोकीविततिमपि कृशां बिभ्रतं मध्यवल्लीम्}
{शक्राश्मन्यस्ततप्तोज्वलकनकनिभं पीतचेलं दधानं}
{ध्यायामो दीप्तरश्मिस्फुटमणिरशनाकिङ्गिणीमण्डितं त्वाम्}

\fourlineindentedshloka
{ऊरू चारू तवोरू घनमसृणरुचौ चित्तचोरौ रमायाः}
{विश्वक्षोभं विशङ्क्य ध्रुवमनिशमुभौ पीतचेलावृताङ्गौ}
{आनम्राणां पुरस्तान्न्यसनधृतसमस्तार्थपालीसमुद्ग-}
{च्छायां जानुद्वयं च क्रमपृथुलमनोज्ञे च जङ्घे निषेवे}

\fourlineindentedshloka
{मञ्जीरं मञ्जुनादैरिव पदभजनं श्रेय इत्यालपन्तं}
{पादाग्रं भ्रान्तिमज्जत्प्रणतजनमनोमन्दरोद्धारकूर्मम्}
{उत्तुङ्गाताम्रराजन्नखरहिमकरज्योत्स्नया चाऽश्रितानां}
{सन्तापध्वान्तहन्त्रीं ततिमनुकलये मङ्गलामङ्गुलीनाम्}

\fourlineindentedshloka
{योगीन्द्राणां त्वदङ्गेष्वधिकसुमधुरं मुक्तिभाजां निवासो}
{भक्तानां कामवर्षद्युतरुकिसलयं नाथ ते पादमूलम्}
{नित्यं चित्तस्थितं मे पवनपुरपते कृष्ण कारुण्यसिन्धो}
{हृत्वा निःशेषतापान्प्रदिशतु परमानन्दसन्दोहलक्ष्मीम्}

\fourlineindentedshloka
{अज्ञात्वा ते महत्त्वं यदिह निगदितं विश्वनाथ क्षमेथाः}
{स्तोत्रं चैतत्सहस्रोत्तरमधिकतरं त्वत्प्रसादाय भूयात्}
{द्वेधा नारायणीयं श्रुतिषु च जनुषा स्तुत्यतावर्णनेन}
{स्फीतं लीलावतारैरिदमिह कुरुतामायुरारोग्यसौख्यम्}

{॥इति श्रीमन्नारायणीये शततम-दशकं सम्पूर्णम्॥}
