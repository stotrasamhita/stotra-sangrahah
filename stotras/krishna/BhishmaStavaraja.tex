% !TeX program = XeLaTeX
% !TeX root = ../../shloka.tex
\sect{भीष्म-स्तवराजः}

\chapter{अध्यायः ४६}
\uvacha{जनमेजय उवाच}


\twolineshloka
{शरतल्पे शयानस्तु भरतानां पितामहः}
{कथमुत्सृष्टवान् देहं कं च योगमधारयत्} % १

\uvacha{वैशम्पायन उवाच}


\twolineshloka
{शृणुष्वावहितो राजञ्शुचिर्भूत्वा समाहितः}
{भीष्मस्य कुरुशार्दूल देहोत्सर्गं महात्मनः} % २


\twolineshloka
{प्रवृत्तमात्रे त्वयनमुत्तरेण दिवाकरे}
{शुक्लपक्षस्य चाष्टम्यां माघमासस्य पार्थिव} % ३


\twolineshloka
{प्राजापत्ये च नक्षत्रे मध्यं प्राप्ते दिवाकरे}
{समावेशयदात्मानमात्मन्येव समाहितः} % ४


\twolineshloka
{विकीर्णांशुरिवादित्यो भीष्मः शरशतैश्चितः}
{शुशुभे परया लक्ष्म्या वृतो ब्राह्मणसत्तमैः} % ५


\twolineshloka
{व्यासेन देवश्रवसा नारदेन सुरर्षिणा}
{देवस्थानेन वात्स्येन तथाऽश्मकसुमन्तुना} % ६


\twolineshloka
{तथा जैमिनिना चैव पैलेन च महात्मना}
{शाण्डिल्यदेवलाभ्यां च मैत्रेयेण च धीमता} % ७


\twolineshloka
{असितेन वसिष्ठेन कौशिकेन महात्मना}
{हारीतलोमशाभ्यां च तथाऽऽत्रेयेण धीमता} % ८


\twolineshloka
{बृहस्पतिश्च शुक्रश्च च्यवनश्च महामुनिः}
{सनत्कुमारः कपिलो वामीकिस्तुम्बुरुः कुरुः} % ९


\twolineshloka
{मौद्गल्यो भार्गवो रामस्तृणबिन्दुर्महामुनिः}
{पिप्पलादोऽथ वायुश्च सवर्तः पुलहः कचः} % १०


\twolineshloka
{काश्यपश्च पुलस्त्यश्च क्रतुर्दक्षः पराशरः}
{मरीचिरङ्गिराः काश्यो गौतमो गालवो मुनिः} % ११


\threelineshloka
{धौम्यो विभाण्डो माण्डव्योधौम्रः कृष्णानुभौतिकः}
{उलूकः परमो विप्रो मार्कण्डेयो महामुनिः}
{भास्करिः पूरणः कृष्णः सूतः परमधार्मिकः} % १२


\twolineshloka
{एतैश्चान्यैर्मुनिगणैर्महाभागैर्महात्मभिः}
{श्रद्धादमशमोपेतैर्वृतश्चन्द्र इव ग्रहैः} % १३


\twolineshloka
{भीष्मस्तु पुरुषव्याघ्रः कर्मणा मनसा गिरा}
{शरतल्पगतः कृष्णं प्रदध्यौ प्राञ्जलिः शुचिः} % १४


\threelineshloka
{स्वरेण हृष्टपुष्टेन तुष्टाव मधुसूदनम्}
{योगेश्वरं पद्मनाभं विष्णुं जिष्णुं जगत्पतिम्}
{अनादिनिधनं विष्णुमात्मयोनिं सनातनम्} % १५


\twolineshloka
{कृताञ्जलिपुटो भूत्वा वाग्विदां प्रवरः प्रभुः}
{भीष्मः परमधर्मात्मा वासुदेवमथास्तुवत्} % १६


\uvacha{भीष्म उवाच}


\twolineshloka
{आरिराधयिषुः कृष्णं वाचं जिगदिषामि याम्}
{तया व्याससमासिन्या प्रीयतां पुरुषोत्तमः} % १७


\twolineshloka
{शुचिं शुचिपदं हंसं तत्परं परमेष्ठिनम्}
{युक्त्वा सर्वात्मनाऽऽत्मानं तं प्रपद्ये प्रजापतिम्} % १८


\twolineshloka
{अनाद्यन्तं परं ब्रह्म न देवा नर्षयो विदुः}
{एकोऽयं वेद भगवान् धाता नारायणो हरिः} % १९


\twolineshloka
{नारायणादृषिगणास्तथा सिद्धमहोरगाः}
{देवा देवर्षयश्चैव यं विदुर्दुःखभेषजम्} % २०


\twolineshloka
{देवदानवगन्धर्वा यक्षराक्षसपन्नगाः}
{यं न जानन्ति को ह्येष कुतो वा भगवानिति} % २१


\twolineshloka
{यमाहुर्जगतः कोशं यस्मिंश्च निहिताः प्रजाः}
{यस्मिँल्लोकाः स्फुरन्त्येते जाले शकुनयो यथा} % २२


\twolineshloka
{यस्मिन् विश्वानि भूतानि तिष्ठन्ति च विशन्ति च}
{गुणभूतानि भूतेशे सूत्रे मणिगणा इव} % २३


\twolineshloka
{यं च विश्वस्य कर्तारं जगतस्तस्थुषां पतिम्}
{वदन्ति जगतोऽध्यक्षमध्यात्मपरिचिन्तकाः} % २४


\twolineshloka
{यस्मिन्नित्ये तते तन्तौ दृढे स्रगिव तिष्ठति}
{सदसद् ग्रथितं विश्वं विश्वाङ्गे विश्वकर्मणि} % २५


\twolineshloka
{हरिं सहस्रशिरसं सहस्रचरणेक्षणम्}
{सहस्रबाहुमकुटं सहस्रवदनोज्ज्वलम्} % २६


\threelineshloka
{प्राहुर्नारायणं देवं यं विश्वस्य परायणम्}
{अणीयसामणीयांसं स्थविष्ठं च स्थवीयसाम्}
{गरीयसां गरिष्ठं च श्रेष्ठं च श्रेयसामपि} % २७


\twolineshloka
{यं वाकेष्वनुवाकेषु निषत्सूपनिषत्सु च}
{गृणन्ति सत्यकर्माणं सत्यं सत्येषु सामसु} % २८


\twolineshloka
{चतुर्भिश्चतुरात्मानं सत्वस्थं सात्वतां पतिम्}
{यं दिव्यैर्देवमर्चन्ति गुह्यैः परमनामभिः} % २९


\twolineshloka
{यस्मिन्नित्यं तपस्तप्तं यदङ्गेष्वनुतिष्ठति}
{सर्वात्मा सर्ववित् सर्वः सर्वज्ञः सर्वभावनः} % ३०


\twolineshloka
{यं देवं देवकी देवी वसुदेवादजीजनत्}
{भौमस्य ब्रह्मणो गुप्त्यै दीप्तमग्निमिवारणिः} % ३१


\twolineshloka
{यमनन्यो व्यपेताशीरात्मानं वीतकल्मषम्}
{इष्ट्वानन्त्याय गोविन्दं पश्यत्यात्मानमात्मनि} % ३२


\threelineshloka
{अप्रतर्क्यमविज्ञेयं हरिं नारायणं विभुम्}
{अतिवाय्विन्द्रकर्माणमतिसूर्याग्नितेजसम्}
{अतिबुद्धीन्द्रियात्मानं तं प्रपद्ये प्रजापतिम्} % ३३


\twolineshloka
{पुराणे पुरुषं प्रोक्तं ब्रह्मप्रोक्तं युगादिषु}
{क्षये सङ्कर्षणं प्रोक्तं तमुपास्यमुपास्महे} % ३४


\twolineshloka
{यमेकं बहुधात्मानं प्रादुर्भूतमधोक्षजम्}
{नान्यभक्ताः क्रियावन्तो यजन्ते सर्वकामदम्} % ३५


\twolineshloka
{ऋतमेकाक्षरं ब्रह्म यत्तत्सदसतः परम्}
{अनादिमध्यपर्यन्तं न देवा नर्षयो विदुः} % ३६


\twolineshloka
{यं सुरासुरगन्धर्वाः सिद्धा ऋषिमहोरगाः}
{प्रयता नित्यमर्चन्ति परमं सुखभेषजम्} % ३७


\twolineshloka
{अनादिनिधनं देवमात्मयोनिं सनातनम्}
{अप्रेक्ष्यमनभिज्ञेयं हरिं नारायणं प्रभुम्} % ३८

\dnsub{अथ भीष्मस्तवराजः}

\twolineshloka
{हिरण्यवर्णं यं गर्भमदितिर्दैत्यनाशनम्}
{एकं द्वादशधा जज्ञे तस्मै सूर्यात्मने नमः} % ३९


\twolineshloka
{शुक्ले देवान् पितॄन् कृष्णे तर्पयत्यमृतेन यः}
{यश्च राजा द्विजातीनां तस्मै सोमात्मने नमः} % ४०


\twolineshloka
{हुताशनमुखैर्देवैर्धार्यते सकलं जगत्}
{हविः प्रथमभोक्ता यस्तस्मै होत्रात्मने नमः} % ४१


\twolineshloka
{महतस्तमसः पारे पुरुषं ह्यतितेजसम्}
{यं ज्ञात्वा मृत्युमत्येति तस्मै ज्ञेयात्मने नमः} % ४२


\twolineshloka
{यं बृहन्तं बृहत्युक्थे यमग्नौ यं महाध्वरे}
{यं विप्रसङ्घा गायन्ति तस्मै वेदात्मने नमः} % ४३


\twolineshloka
{पादाङ्गं सन्धिपर्वाणं स्वरव्यञ्जनभूषितम्}
{यमाहुरक्षरं विप्रास्तस्मै वागात्मने नमः} % ४४


\twolineshloka
{यज्ञाङ्गो यो वराहो वै भूत्वा गामुज्जहार ह}
{लोकत्रयहितार्थाय तस्मै वीर्यात्मने नमः} % ४५


\twolineshloka
{ऋग्यजुःसामधामानं दशार्धहविराकृतिम्}
{यं सप्ततन्तुं तन्वन्ति तस्मै यज्ञात्मने नमः} % ४६


\twolineshloka
{चतुर्भिश्च चतुर्भिश्च द्वाभ्यां पञ्चभिरेव च}
{हूयते च पुनर्द्वाभ्यां तस्मै होमात्मने नमः} % ४७


\twolineshloka
{यः सुपर्णो यजुर्नाम च्छन्दोगात्रस्त्रिवृच्छिराः}
{रथन्तरबृहत्पक्षस्तस्मै स्तोत्रात्मने नमः} % ४८


\twolineshloka
{यः सहस्रसवे सत्रे जज्ञे विश्वसृजामृषिः}
{हिरण्यपक्षः शकुनिस्तस्मै तार्क्ष्यात्मने नमः} % ४९


\twolineshloka
{यश्चिनोति सतां सेतुमृतेनामृतयोनिना}
{धर्मार्थव्यवहाराङ्गैस्तस्मै सत्यात्मने नमः} % ५०


\twolineshloka
{यं पृथग्धर्मचरणाः पृथग्धर्मफलैषिणः}
{पृथग्धर्मैः समर्चन्ति तस्मै धर्मात्मने नमः} % ५१


\twolineshloka
{यतः सर्वे प्रसूयन्ते ह्यनङ्गात्माङ्गदेहिनः}
{उन्मादः सर्वभूतानां तस्मै कामात्मने नमः} % ५२


\twolineshloka
{यं तं व्यक्तस्थमव्यक्तं विचिन्वन्ति महर्षयः}
{क्षेत्रे क्षेत्रज्ञमासीनं तस्मै क्षेत्रात्मने नमः} % ५३


\twolineshloka
{यं दृगात्मानमात्मस्थं वृतं षोडशभिर्गुणैः}
{प्राहुः सप्तदशंसाङ्ख्यास्तस्मै साङ्ख्यात्मने नमः} % ५४


\twolineshloka
{यं विनिद्रा जितश्वासाः सन्तुष्टाः संयतेन्द्रियाः}
{ज्योतिः पश्यन्ति युञ्जानास्तस्मै योगात्मेन नमः} % ५५


\twolineshloka
{अपुण्यपुण्योपरमे यं पुनर्भवनिर्भयाः}
{शान्ताः सन्न्यासिनो यान्ति तस्मै मोक्षात्मने नमः} % ५६


\twolineshloka
{यस्याग्रिरास्यं द्यौर्मूर्धा खं नाभिश्चरणौ क्षितिः}
{सूर्यश्चक्षुर्दिशः श्रोत्रं तस्मै लोकात्मने नमः} % ५७


\twolineshloka
{युगेष्वावर्तते योंऽशैर्मासर्त्वयनहायनैः}
{सर्गप्रलययोः कर्ता तस्मै कालात्मने नमः} % ५८


\twolineshloka
{योऽसौ युगसहस्रान्ते प्रदीप्तार्चिर्विभावसुः}
{सम्भक्षयति भूतानि तस्मै घोरात्मने नमः} % ५९


\twolineshloka
{सम्भक्ष्य सर्वभूतानि कृत्वा चैकार्णवं जगत्}
{बालः स्वपिति यश्चैकस्तस्मै मायात्मने नमः} % ६०


\twolineshloka
{सहस्रशिरसे तस्मै पुरुषायामितात्मने}
{चतुःसमुद्रपयसि योगनिद्रात्मने नमः} % ६१


\twolineshloka
{अजस्य नाभावध्येकं यस्मिन् विश्वं प्रतिष्ठितम्}
{पुष्करं पुष्कराक्षस्य तस्मै पद्मात्मने नमः} % ६२


\twolineshloka
{यस्य केशेषु जीमूता नद्यः सर्वाङ्गसन्धिषु}
{कुक्षौ समुद्राश्चत्वारस्तस्मै तोयात्मने नमः} % ६३


\twolineshloka
{यस्मात् सर्गाः प्रवर्तन्ते सर्गप्रलयविक्रियाः}
{यस्मिंश्चैव प्रलीयन्ते तस्मै हेत्वात्मने नमः} % ६४


\twolineshloka
{यो निषण्णो भवेद्रात्रौ दिवा भवति विष्ठितः}
{इष्टानिष्टस्य च द्रष्टा तस्मै द्रष्ट्रात्मने नमः} % ६५


\twolineshloka
{अकार्यः सर्वकार्येषु धर्मकार्यार्थमुद्यतः}
{वैकुण्ठस्य हि तद्रूपं तस्मै कार्यात्मने नमः} % ६६


\twolineshloka
{ब्रह्म वक्त्रं भुजौ क्षत्रं कृत्स्नमूरूदरं विशः}
{पादौ यस्याऽऽश्रिताः शूद्रास्तस्मै वर्णात्मने नमः} % ६७


\twolineshloka
{अन्नपानेन्धनमयो रसप्राणविवर्धनः}
{यो धारयति भूतानि तस्मै प्राणात्मने नमः} % ६८


\twolineshloka
{प्राणानां धारणार्थाय योऽन्नं भुङ्क्ते चतुर्विधम्}
{अन्तर्भूतः पचत्यग्निस्तस्मै पाकात्मने नमः} % ६९


\twolineshloka
{विषये वर्तमानानां यं तं वैषयिकैर्गुणैः}
{प्राहुर्विषयगोप्तारं तस्मै गोप्त्रात्मने नमः} % ७०


\twolineshloka
{अप्रमेयशरीराय सर्वतो बुद्धिचक्षुषे}
{अपारपरिमाणाय तस्मै दिव्यात्मने नमः} % ७१


\twolineshloka
{परः कालात् परो यज्ञात् परः सदसतश्च यः}
{अनादिरादिर्विश्वस्य तस्मै विश्वात्मने नमः} % ७२


\twolineshloka
{वैद्युतो जाठरश्चैव पावकः शुचिरेव च}
{दहनः सर्वभक्षाणां तस्मै वह्न्यात्मने नमः} % ७३


\twolineshloka
{रसातलगतः श्रीमाननन्तो भगवान् प्रभुः}
{जगद् धारयते योऽसौ तस्मै शेषात्मने नमः} % ७४


\twolineshloka
{ज्वलनार्केन्दुताराणां ज्योतिषां दिव्यमूर्तिनाम्}
{यस्तेजयति तेजांसि तस्मै तेजात्मने नमः} % ७५


\twolineshloka
{आत्मज्ञानमिदं ज्ञानं ज्ञात्वा पञ्चस्ववस्थितम्}
{यं ज्ञानेनाधिगच्छन्ति तस्मै ज्ञानात्मने नमः} % ७६


\twolineshloka
{साङ्ख्यैर्योगैर्विनिश्चित्य साध्यैश्च परमर्षिभिः}
{यस्य तु ज्ञायते तत्त्वं तस्मै गुह्यात्मने नमः} % ७७


\twolineshloka
{जटिने दण्डिने नित्यं लम्बोदरशरीरिणे}
{कमण्डलुनिषङ्गाय तस्मै ब्रह्मात्मने नमः} % ७८


\twolineshloka
{शूलिने त्रिदशेशाय त्र्यम्बकाय महात्मने}
{भस्मदिग्धोर्ध्वलिङ्गाय तस्मा रुद्रात्मने नमः} % ७९


\twolineshloka
{चन्द्रार्धकृतशीर्षाय व्यालयज्ञोपवीतिने}
{पिनाकशूलहस्ताय तस्मै उग्रात्मने नमः} % ८०


\twolineshloka
{यो जातो वसुदेवेन देवक्यां यदुनन्दनः}
{शङ्खचक्रगदापाणिर्वासुदेवात्मने नमः} % ८१


\twolineshloka
{शिरःकपालमालाय व्याघ्रचर्मनिवासिने}
{भस्मदिग्धशरीराय तस्मै रुद्रात्मने नमः} % ८२


\twolineshloka
{यो मोहयति भूतानि सर्वपाशानुबन्धनैः}
{सर्वस्य रक्षणार्थाय तस्मै मोहात्मने नमः} % ८३


\twolineshloka
{चैतन्यं सर्वतो नित्यं सर्वप्राणिहृदि स्थितम्}
{सर्वातीततरं सूक्ष्मं तस्मै सूक्ष्मात्मने नमः} % ८४


\twolineshloka
{पञ्चभूतात्मभूताय भूतादिनिधनाय च}
{अक्रोधद्रोहमोहाय तस्मै शान्तात्मने नमः} % ८५


\twolineshloka
{यस्मिन् सर्वं यतः सर्वं यः सर्वं सर्वतश्च यः}
{यश्च सर्वमयो देवस्तस्मै सर्वात्मने नमः} % ८६


\twolineshloka
{यः शेते क्षीरपर्यङ्के दिव्यनागविभूषिते}
{फणासहस्ररचिते तस्मै निद्रात्मने नमः} % ८७


\twolineshloka
{विश्वे च मरुतश्चैव रुद्रादित्याश्विनावपि}
{वसवः सिद्धसाध्याश्च तस्मै देवात्मने नमः} % ८८


\twolineshloka
{अव्यक्तं बुद्ध्यहङ्कारो मनोबुद्धीन्द्रियाणि च}
{तन्मात्राणि विशेषाश्च तस्मै तत्त्वात्मने नमः} % ८९


\twolineshloka
{भूतं भव्यं भविष्यच्च भूतादिप्रभवाव्ययः}
{योऽग्रजः सर्वभूतानां तस्मै भूतात्मने नमः} % ९०


\twolineshloka
{यं हि सूक्ष्मं विचिन्वन्ति परं सूक्ष्मविदो जनाः}
{सूक्ष्मात् सूक्ष्मं च यद् ब्रह्म तस्मै सूक्ष्मात्मने नमः} % ९१


\twolineshloka
{मत्स्यो भूत्वा विरिञ्चाय येन वेदाः समाहृताः}
{रसातलगतः शीघ्रं तस्मै मत्स्यात्मने नमः} % ९२


\twolineshloka
{मन्दराद्रिर्धृतो येन प्राप्ते ह्यमृतमन्थने}
{अतिकर्कशदेहाय तस्मै कूर्मात्मने नमः} % ९३


\twolineshloka
{वाराहं रूपमास्थाय महीं सवनपर्वताम्}
{उद्धरत्येकदंष्ट्रेण तस्मै क्रोडात्मने नमः} % ९४


\twolineshloka
{नारसिंहवपुः कृत्वा सर्वलोकभयङ्करम्}
{हिरण्यकशिपुं जघ्ने तस्मै सिंहात्मने नमः} % ९५


\twolineshloka
{पिङ्गेक्षणसटं यस्य रूपं दंष्ट्रानखैर्युतम्}
{दानवेन्द्रान्तकरणं तस्मै दृप्तात्मने नमः} % ९६


\twolineshloka
{यं न देवा न गन्धर्वा न दैत्या न च दानवाः}
{तत्त्वतो हि विजानन्ति तस्मै सूक्ष्मात्मने नमः} % ९७


\twolineshloka
{वामनं रूपमास्थाय बलिं संयम्य मायया}
{त्रैलोक्यं क्रान्तवान् यस्तु तस्मै क्रान्तात्मने नमः} % ९८


\twolineshloka
{जमदग्निसुतो भूत्वा रामः शस्त्रभृतां वरः}
{महीं निःक्षत्रियां चक्रे तस्मै रामात्मने नमः} % ९९


\twolineshloka
{त्रिःसप्तकृत्वो यश्चैको धर्मे व्युत्क्रान्तिगौरवात्}
{जघान क्षत्रियान् सङ्ख्ये तस्मै क्रोधात्मने नमः} % १००


\twolineshloka
{विभज्य पञ्चधाऽऽत्मानं वायुर्भूत्वा शरीरगः}
{यश्चेष्टयति भूतानि तस्मै वाय्वात्मने नमः} % १०१


\twolineshloka
{रामो दशिरथिर्भूत्वा पुलस्त्यकुलनन्दनम्}
{जघान रावणं सङ्ख्ये तस्मै क्षत्रात्मने नमः} % १०२


\twolineshloka
{यो हली मुसली श्रीमान्नीलाम्बरधरः स्थितः}
{रामाय रौहिणेयाय तस्मै भोगात्मने नमः} % १०३


\twolineshloka
{शङ्खिने चक्रिणे नित्यं शार्ङ्गिणे पीतवाससे}
{वनमालाधरायैव तस्मै कृष्णात्मने नमः} % १०४


\twolineshloka
{वसुदेवसुतः श्रीमान् क्रीडितो नन्दगोकुले}
{कंसस्य निधनार्थाय तस्मै क्रीडात्मने नमः} % १०५


\twolineshloka
{वासुदेवत्वमागम्य यदोर्वंशसमुद्भवः}
{भूभारहरणं चक्रे तस्मै कृष्णात्मने नमः} % १०६


\twolineshloka
{सारथ्यमर्जुनस्याऽऽजौ कुर्वन् गीतामृतं ददौ}
{लोकत्रयोपकाराय तस्मै ब्रह्मात्मने नमः} % १०७


\twolineshloka
{दानवांस्तु वशे कृत्वा पुनर्बुद्धत्वमागतः}
{सर्गस्य रक्षणार्थाय तस्मै बुद्धात्मने नमः} % १०८


\twolineshloka
{हनिष्यति कलौ प्राप्ते म्लेच्छांस्तुरगवाहनः}
{धर्मसंस्थापको यस्तु तस्मै कल्क्यात्मने नमः} % १०९


\twolineshloka
{तारान्वये कालनेमिं हत्वा दानवपुङ्गवम्}
{ददौ राज्यं महेन्द्राय तस्मै साङ्ख्यात्मने नमः} % ११०


\twolineshloka
{यः सर्वप्राणिनां देहे साक्षिभूतो ह्यवस्थितः}
{अक्षरः क्षरमाणानां तस्मै साक्ष्यात्मने नमः} % १११


\twolineshloka
{नमोऽस्तु ते महादेव नमस्ते भक्तवत्सल}
{सुब्रह्मण्य नमस्तेऽस्तु प्रसीद परमेश्वर} % ११२


\twolineshloka
{अव्यक्तव्यक्तरूपेण व्याप्तं सर्वं त्वया विभो}
{नारायणं सहस्राक्षं सर्वलोकमहेश्वरम्} % ११३


\twolineshloka
{हिरण्यनाभं यज्ञाङ्गममृतं विश्वतोमुखम्}
{सर्वदा सर्वकार्येषु नास्ति तेषाममङ्गलम्} % ११४


\twolineshloka
{येषां हृदिस्थो देवेशो मङ्गलायतनं हरिः}
{मङ्गल भगवान् विष्णुर्मङ्गलं मधुसूदनः} % ११५


\twolineshloka
{मङ्गलं पुण्डरीकाक्षो मङ्गलं गरुडध्वजः}
{विश्वकर्मन्नमस्तेऽस्तु विश्वात्मन् विश्वसम्भव} % ११६


\twolineshloka
{अपवर्गस्थभूतानां पञ्चानां परमास्थित}
{नमस्ते त्रिषु लोकेषु वमस्ते परतस्त्रिषु} % ११७


\twolineshloka
{नमस्ते दिक्षु सर्वासु त्वं हि सर्वपरायणम्}
{नमस्ते भगवन् विष्णो लोकानां प्रभवाव्यय} % ११८


\twolineshloka
{त्वं हि कर्ता हृषीकेशः संहर्ता चापराजितः}
{तेन पश्यामि ते दिव्यान् भावान् हि त्रिषु वर्त्मसु} % ११९


\threelineshloka
{तच्च पश्यामि तत्त्वेन यत्ते रूपं सनातनम्}
{दिवं ते शिरसा व्याप्तं पद्भ्यां देवी वसुन्धरा}
{विक्रमेण त्रयो लोकाः पुरुषोऽसि सनातनः} % १२०


\twolineshloka
{दिशो भुजा रविश्चक्षुर्वीर्ये शुक्रः प्रतिष्ठितः}
{सप्तमार्गा निरुद्धास्ते वायोरमिततेजसः} % १२१


\twolineshloka
{व्यक्ताव्यक्तस्वरूपेण व्याप्तं सर्वं त्वया विभो}
{अव्यक्तं ब्राह्मणं रूपं व्यक्तमेतच्चराचरम्} % १२२


\twolineshloka
{अतसीपुष्पसङ्काशं पीतवाससमच्युतम्}
{ये नमस्यन्ति गोविन्दं न तेषां विद्यते भयम्} % १२३


\fourlineindentedshloka
{एकोऽपि कृष्णस्य कृतः प्रणामो}
{दशाश्वमेधावभृथेन तुल्यः}
{दशाश्वमेधी पुनरेति जन्म}
{कृष्णप्रणामी न पुनर्भवाय} % १२४


\twolineshloka
{कृष्णव्रताः कृष्णमनुस्मरन्तो}
{रात्रौ च कृष्णं पुनरुत्थिता ये}
{ते कृष्णदेहाः प्रविशन्ति कृष्णम्}
{आज्यं यथा मन्त्रहुतं हुताशे} % १२५


\twolineshloka
{नमो नरकसन्त्रासरक्षामण्डलकारिणे}
{संसारनिम्नगावर्ततरिकाष्ठाय विष्णवे} % १२६


\twolineshloka
{नमो ब्रह्मण्यदेवाय गोब्राह्मणहिताय च}
{जगद्धिताय कृष्णाय गोविन्दाय नमो नमः} % १२७


\twolineshloka
{प्राणकान्तारपाथेयं संसारोच्छेदभेषजम्}
{दुःखशोकपरित्राणं हरिरित्यक्षरद्वयम्} % १२८


\twolineshloka
{नारायणपरं ब्रह्म नारायणपरं तपः}
{नारायणपरं सत्यं नारायणपरं परम्} % १२९


\twolineshloka
{यथा विष्णुमयं सत्यं यथा विष्णुमयं हविः}
{तथा विष्णुमयं सर्वं पाप्मा मे नश्यतां तथा} % १३०


\twolineshloka
{तस्य यज्ञवराहस्य विष्णोरमिततेजसः}
{प्रणामं येऽपि कुर्वन्ति तेषामपि नमो नमः} % १३१


\twolineshloka
{त्वां प्रपन्नाय भक्ताय गतिमिष्टां जिगीषवे}
{यच्छ्रेयः पुण्डरीकाक्ष तद्ध्यायस्व सुरोत्तम} % १३२


\twolineshloka
{इति विद्यातपोयोनिरयोनिर्विष्णुरीडितः}
{वाग्यज्ञेनार्चितो देवः प्रीयतां मे जनार्दनः} % १३३

\uvacha{वैशम्पायन उवाच}

\twolineshloka
{एतावदुक्त्वा वचनं भीष्मस्तद्रतमानसः}
{नम इत्येव कृष्णाय प्रणाममकरोत् तदा} % १३४


\twolineshloka
{तस्मिन्नुपरते वाक्ये ततस्ते ब्रह्मवादिनः}
{भीष्मं वाग्भिर्बाष्पगलास्तमानर्चुर्महाद्युतिम्} % १३५


\twolineshloka
{तेऽस्तुवन्तश्च विप्रेन्द्राः केशवं पुरुषोत्तमम्}
{भीष्मं च शनकैः सर्वे प्रशशंसुः पुनः पुनः} % १३६


\twolineshloka
{अधिगम्य तु योगेन भक्तिं भीष्मस्य माधवः}
{त्रैलोक्यदर्शनं ज्ञानं दिव्यं दत्त्वा ययौ हरिः} % १३७


\twolineshloka
{विदित्वा भक्तियोगं तं भीष्मस्य पुरुषोत्तमः}
{सहसोत्थाय तं हृष्टो यानमेवान्वपद्यत} % १३८


\twolineshloka
{केशवः सात्यकिश्चैव रथेनकेन जग्मतुः}
{अपरेण महात्मानौ युधिष्ठिरधनञ्जयौ} % १३९


\twolineshloka
{भीमसेनो यमौ चोभौ रथमेकं समास्थिताः}
{कृपो युयुत्सुः सूतश्च सञ्जयश्चापरं रथम्} % १४०


\twolineshloka
{ते रथैर्नगराकारैः प्रयाताः पुरुषर्षभाः}
{नेमिघोषेण महता कम्पयन्ते वसुन्धराम्} % १४१


\fourlineindentedshloka
{ततो गिरः पुरुषवरस्तवेरितान्}
{द्विजेरिताः पथिषु मनाक् स शुश्रुवे}
{कृताञ्जलिं प्रणतमथापरं जनं}
{स केशिहा मुदितमनास्थनन्दत} % १४२


\fourlineindentedshloka
{इति स्मरन् पठति च शार्ङ्गधन्वनः}
{शृणोतु वा यदुकुलनन्दनस्तवम्}
{स चक्रभृत् प्रतिहतसर्वकिल्बिषो}
{जनार्दनं प्रविशति देहसङ्क्षये} % १४३


\fourlineindentedshloka
{यं योगिनः प्राणवियोगकाले}
{यत्नेन चित्ते विनिवेशयन्ति}
{स तं पुरस्ताद्धरिमीक्षमाणः}
{प्राणाञ्जहौ प्राप्तफलो हि भीष्मः} % १४४


\twolineshloka
{स्तवराजः समाप्तोऽयं विष्णोरद्भुतकर्मणः}
{गाङ्गेयेन पुरा गीतो महापातकनाशनः} % १४५


\fourlineindentedshloka
{इमं नरः स्तवराजं मुमुक्षुः}
{पठञ्शुचिः कलुषितकल्मषापहम्}
{अतीत्य लोकान् मलिनः समागतान्}
{पदं स गच्छत्यमृतं महात्मनः} % १४६

॥इति श्रीमन्महाभारते शान्तिपर्वणि राजधर्मपर्वणि षट्चत्वारिंशोऽध्यायः॥