% !TeX program = XeLaTeX
% !TeX root = ../../shloka.tex

\sect{अच्युताष्टकम्}

\fourlineindentedshloka
{अच्युतं केशवं राम-नारायणं}
{कृष्ण-दामोदरं वासुदेवं हरिम्}
{श्रीधरं माधवं गोपिकावल्लभं}
{जानकीनायकं रामचन्द्रं भजे}

\fourlineindentedshloka
{अच्युतं केशवं सत्यभामाधवं}
{माधवं श्रीधरं राधिकाराधितम्}
{इन्दिरा मन्दिरं चेतसा सुन्दरं}
{देवकीनन्दनं नन्दनं सन्दधे}

\fourlineindentedshloka
{विष्णवे जिष्णवे शङ्खिने चक्रिणे}
{रुक्मिणी-रागिने जानकी-जानये}
{वल्लवी-वल्लभायाऽर्चितायात्मने}
{कंस-विध्वंसिने वंशिने ते नमः}

\fourlineindentedshloka
{कृष्ण गोविन्द हे राम नारायण}
{श्रीपते वासुदेवार्जित-श्रीनिधे}
{अच्युतानन्त हे माधवाधोक्षज}
{द्वारका-नायक द्रौपदी-रक्षक}

\fourlineindentedshloka
{राक्षसक्षोभितः सीतया शोभितो}
{दण्डकारण्य-भू-पुण्यता-कारणः}
{लक्ष्मणेनान्वितो वानरैः सेवितो-}
{ऽगस्त्सम्पूजितो राघवः पातु माम्}

\fourlineindentedshloka
{धेनुकारिष्टकोऽनिष्टकृद्-द्वेषिणां}
{केशिहा कंसहृद्-वंशिकावादकः}
{पूतनाकोपकः सूरजा-खेलनो}
{बाल-गोपालकः पातु मां सर्वदा}

\fourlineindentedshloka
{विद्युदाद्योतवान् प्रस्फुरद्वाससं}
{प्रावृडम्भोदवत् प्रोल्लसद्विग्रहम्}
{वन्यया मालया शोभितोरस्थलं}
{लोहिताङ्घ्रिद्वयं वारिजाक्षं भजे}

\fourlineindentedshloka
{कुञ्चितैः कुन्तलैर्भ्राजिमानाननं}
{रत्नमौलिं लसत् कुण्डलं गण्डयोः}
{हारकेयूरकं कङ्कण-प्रोज्ज्वलं}
{किङ्किणी-मञ्जुलं श्यामलं तं भजे}

\fourlineindentedshloka*
{अच्युतस्याष्टकं यः पठेदिष्टदं}
{प्रेमतः प्रत्यहं पूरुषः सस्पृहम्}
{वृत्ततः सुन्दरं वेद्यविश्वम्बरं}
{तस्य वश्यो हरिर्जायते सत्वरम्}

॥इति श्रीमत्परमहंसपरिव्राजकाचार्यस्य श्री-गोविन्द-भगवत्पूज्य-पाद-शिष्यस्य
श्रीमच्छङ्करभगवतः कृतौ श्री-अच्युताष्टकं सम्पूर्णम्॥