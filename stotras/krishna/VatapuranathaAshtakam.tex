% !TeX program = XeLaTeX
% !TeX root = ../../shloka.tex
\sect{गुरुवातपुरीशपञ्चरत्नम्}

\fourlineindentedshloka
{कुन्दसुमवृन्दसममन्दहसितास्यं}
{नन्दकुलनन्दभरतुन्दलनकन्दम्}
{पूतनिजगीतलवधूतदुरितं तं}
{वातपुरनाथमिममातनु हृदब्जे} %॥१॥


\fourlineindentedshloka
{नीलतरजालधरभालहरिरम्यं}
{लोलतरशीलयुतबालजनलीलम्}
{जालनतिशीलमपि पालयितुकामं}
{वातपुरनाथमिममातनु हृदब्जे} %॥२॥

\fourlineindentedshloka
{कंसरणहिंसमिह संसरणजात-}
{क्लान्तिभरशान्तिकरकान्तिझरवीतम्}
{वातमुखधातुजनिपातभयघातं}
{वातपुरनाथमिममातनु हृदब्जे} %॥३॥


\fourlineindentedshloka
{जातुधुरिपातुकमिहातुरजनं द्राक्}
{शोकभरमूकमपि तोकमिव पान्तम्}
{भृङ्गरुचिसङ्गरकृदङ्गलतिकं तं}
{वातपुरनाथमिममातनु हृदब्जे} %॥४॥


\fourlineindentedshloka
{पापभवतापभरकोपशमनार्था-}
{श्वासकरभासमृदुहासरुचिरास्यम्}
{रोगचयभोगभयवेगहरमेकं}
{वातपुरनाथमिममातनु हृदब्जे} %॥५॥


\fourlineindentedshloka
{घोषकुलदोषहरवेषमुपयान्तं}
{पूषशतदूषकविभूषणगणाढ्यम्}
{भुक्तिमपिमुक्तिमतिभक्तिषु ददानं}
{वातपुरनाथमिममातनु हृदब्जे} %॥६॥


\fourlineindentedshloka
{पापकदुरापमतितापहरशोभ-}
{स्वापघनमामतदुमापतिसमेतम्}
{दूनतरदीनसुखदानकृतदीक्षं}
{वातपुरनाथमिममातनु हृदब्जे} %॥७॥


\fourlineindentedshloka
{पादपतदादरणमोदपरिपूर्णं}
{जीवमुखदेवजनसेवनफलाङ्घ्रिम्}
{रूक्षभवमोक्षकृतदीक्षनिजवीक्षं}
{वातपुरनाथमिममातनु हृदब्जे} %॥८॥


\fourlineindentedshloka*
{भृत्यगणपत्युदितनुत्युचितमोदं}
{स्पष्टमिदमष्टकमदुष्टकरणार्हम्}
{आदधतमादरदमादिलयशून्यं}
{वातपुरनाथमिममातनु हृदब्जे} %॥९॥


॥इति महामहोपाध्याय-ब्रह्मश्री-पैङ्गानाडु-गणपतीशास्त्रिभिः विरचितं श्रीवातपुरनाथाष्टकम् सम्पूर्णम्॥
