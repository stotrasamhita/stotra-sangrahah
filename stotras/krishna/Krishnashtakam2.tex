% !TeX program = XeLaTeX
% !TeX root = ../../shloka.tex

\sect{कृष्णाष्टकम् २}
\twolineshloka*
{नित्यानन्दैकरसं सच्चिन्मात्रं स्वयं ज्योतिः}
{पुरुषोत्तममजमीशं वन्दे श्रीयादवाधीशम्}

\fourlineindentedshloka
{भजे व्रजैकमण्डनं समस्तपापखण्डनम्}
{स्वभक्तचित्तरञ्जनं सदैव नन्दनन्दनम्}
{सुपिच्छगुच्छमस्तकं सुनादवेणुहस्तकम्}
{अनङ्गरङ्गसागरं नमामि कृष्णनागरम्}

\fourlineindentedshloka
{मनोजगर्वमोचनं विशाललोललोचनम्}
{विधूतगोपशोचनं नमामि पद्मलोचनम्}
{करारविन्दभूधरं स्मितावलोकसुन्दरम्}
{महेन्द्रमानदारणं नमामि कृष्णवारणम्}

\fourlineindentedshloka
{कदम्बसूनकुण्डलं सुचारुगण्डमण्डलम्}
{व्रजाङ्गनैकवल्लभं नमामि कृष्णदुर्लभम्}
{यशोदया समोदया सगोपया सनन्दया}
{युतं सुखैकदायकं नमामि गोपनायकम्}

\fourlineindentedshloka
{सदैव पादपङ्कजं मदीय मानसे निजम्}
{दधानमुक्तमालकं नमामि नन्दबालकम्}
{समस्तदोषशोषणं समस्तलोकपोषणम्}
{समस्तगोपमानसं नमामि नन्दलालसम्}

\fourlineindentedshloka
{भुवो भरावतारकं भवाब्धिकर्णधारकम्}
{यशोमतीकिशोरकं नमामि चित्तचोरकम्}
{दृगन्तकान्तभङ्गिनं सदा सदालिसङ्गिनम्}
{दिने दिने नवं नवं नमामि नन्दसम्भवम्}

\fourlineindentedshloka
{गुणाकरं सुखाकरं कृपाकरं कृपापरम्}
{सुरद्विषन्निकन्दनं नमामि गोपनन्दनम्}
{नवीनगोपनागरं नवीनकेलिलम्पटम्}
{नमामि मेघसुन्दरं तडित्प्रभालसत्पटम्}

\fourlineindentedshloka
{समस्तगोपनन्दनं हृदम्बुजैकमोदनम्}
{नमामि कुञ्जमध्यगं प्रसन्नभानुशोभनम्}
{निकामकामदायकं दृगन्तचारुसायकम्}
{रसालवेणुगायकं नमामि कुञ्जनायकम्}

\fourlineindentedshloka
{विदग्धगोपिकामनोमनोज्ञतल्पशायिनम्}
{नमामि कुञ्जकानने प्रवृद्धवह्निपायिनम्}
{किशोरकान्तिरञ्जितं दृगञ्जनं सुशोभितम्}
{गजेन्द्रमोक्षकारिणं नमामि श्रीविहारिणम्}

\fourlineindentedshloka*
{यदा तदा यथा तथा तथैव कृष्णसत्कथा}
{मया सदैव गीयतां तथा कृपा विधीयताम्}
{प्रमाणिकाष्टकद्वयं जपत्यधीत्य यः पुमान्}
{भवेत् स नन्दनन्दने भवे भवे सुभक्तिमान्}
॥इति श्रीमच्छङ्कराचार्यविरचितं श्री~कृष्णाष्टकं सम्पूर्णम्॥