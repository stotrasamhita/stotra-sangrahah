% !TeX program = XeLaTeX
% !TeX root = ../../shloka.tex

\sect{मधुराष्टकम्}
\twolineshloka
{अधरं मधुरं वदनं मधुरं नयनं मधुरं हसितं मधुरम्}
{हृदयं मधुरं गमनं मधुरं मधुराधिपतेरखिलं मधुरम्}

\twolineshloka
{वचनं मधुरं चरितं मधुरं वसनं मधुरं वलितं मधुरम्}
{चलितं मधुरं भ्रमितं मधुरं मधुराधिपतेरखिलं मधुरम्}

\twolineshloka
{वेणुर्मधुरो रेणुर्मधुरः पाणिर्मधुरः पादौ मधुरौ}
{नृत्यं मधुरं सख्यं मधुरं मधुराधिपतेरखिलं मधुरम्}

\twolineshloka
{गीतं मधुरं पीतं मधुरं भुक्तं मधुरं सुप्तं मधुरम्}
{रूपं मधुरं तिलकं मधुरं मधुराधिपतेरखिलं मधुरम्}

\twolineshloka
{करणं मधुरं तरणं मधुरं हरणं मधुरं रमणं मधुरम्}
{वमितं मधुरं शमितं मधुरं मधुराधिपतेरखिलं मधुरम्}

\twolineshloka
{गुञ्जा मधुरा माला मधुरा यमुना मधुरा वीची मधुरा}
{सलिलं मधुरं कमलं मधुरं मधुराधिपतेरखिलं मधुरम्}

\twolineshloka
{गोपी मधुरा लीला मधुरा युक्तं मधुरं मुक्तं मधुरम्}
{दृष्टं मधुरं शिष्टं मधुरं मधुराधिपतेरखिलं मधुरम्}

\twolineshloka
{गोपा मधुरा गावो मधुरा यष्टिर्मधुरा सृष्टिर्मधुरा}
{दलितं मधुरं फलितं मधुरं मधुराधिपतेरखिलं मधुरम्}
{॥इति श्रीमद्वल्लभाचार्यविरचितं मधुराष्टकं सम्पूर्णम्॥}
