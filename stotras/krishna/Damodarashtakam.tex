% !TeX program = XeLaTeX
% !TeX root = ../../shloka.tex

\sect{दामोदराष्टकम्}
\fourlineindentedshloka
{नमामीश्वरं सच्चिदानन्दरूपं}
{लसत्कुण्डलं गोकुले भ्राजमानम्}
{यशोदाभियोलूखलाद्-धावमानं}
{परामृष्टमत्यन्ततो द्रुत्य गोप्या}

\fourlineindentedshloka
{रुदन्तं मुहुर्नेत्रयुग्मं मृजन्तं}
{कराम्भोजयुग्मेन सातङ्कनेत्रम्}
{मुहुः श्वासकम्पत्रिरेखाङ्ककण्ठ-}
{स्थितग्रैव-दामोदरं भक्तिबद्धम्}

\fourlineindentedshloka
{इतीदृक् स्वलीलाभिरानन्दकुण्डे}
{स्वघोषं निमज्जन्तमाख्यापयन्तम्}
{तदीयेषिताज्ञेषु भक्तैर्जितत्वं}
{पुनः प्रेमतस्तं शतावृत्ति वन्दे}

\fourlineindentedshloka
{वरं देव मोक्षं न मोक्षावधिं वा}
{न चान्यं वृणेऽहं वरेषादपीह}
{इदं ते वपुर्नाथ गोपालबालं}
{सदा मे मनस्याविरास्तां किमन्यैः}

\fourlineindentedshloka
{इदं ते मुखाम्भोजमत्यन्तनीलैर्-}
{वृतं कुन्तलैः स्निग्ध-रक्तैश्च गोप्या}
{मुहुश्चुम्बितं बिम्बरक्ताधरं मे}
{मनस्याविरास्ताम् अलं लक्षलाभैः}

\fourlineindentedshloka
{नमो देव दामोदरानन्त विष्णो}
{प्रसीद प्रभो दुःखजालाब्धिमग्नम्}
{कृपादृष्टिवृष्ट्यातिदीनं बतानु}
{गृहाणेश माम् अज्ञमेध्यक्षिदृश्यः}

\fourlineindentedshloka
{कुवेरात्मजौ बद्धमूर्त्यैव यद्वत्}
{त्वया मोचितौ भक्तिभाजौ कृतौ च}
{तथा प्रेमभक्तिं स्वकं मे प्रयच्छ}
{न मोक्षे ग्रहो मेऽस्ति दामोदरेह}

\fourlineindentedshloka
{नमस्तेऽस्तु दाम्ने स्फुरद्दीप्तिधाम्ने}
{त्वदीयोदरायाथ विश्वस्य धाम्ने}
{नमो राधिकायै त्वदीयप्रियायै}
{नमोऽनन्तलीलाय देवाय तुभ्यम्}

॥इति श्रीमद्पद्मपुराणे श्री-दामोदराष्टकं सम्पूर्णम्॥