% !TeX program = XeLaTeX
% !TeX root = ../../shloka.tex

\sect{पाण्डुरङ्गाष्टकम्} 


\fourlineindentedshloka
{महायोगपीठे तटे भीमरथ्या}
{वरं पुण्डरीकाय दातुं मुनीन्द्रैः}
{समागत्य तिष्ठन्तमानन्दकन्दं}
{परब्रह्मलिङ्गं भजे पाण्डुरङ्गम्} %॥१॥

\fourlineindentedshloka
{तटिद्वाससं नीलमेघावभासं}
{रमामन्दिरं सुन्दरं चित्प्रकाशम्}
{वरं त्विष्टकायां समन्यस्तपादं}
{परब्रह्मलिङ्गं भजे पाण्डुरङ्गम्} %॥२॥

\fourlineindentedshloka
{प्रमाणं भवाब्धेरिदं मामकानां}
{नितम्बः कराभ्यां धृतो येन तस्मात्}
{विधातुर्वसत्यै धृतो नाभिकोशः}
{परब्रह्मलिङ्गं भजे पाण्डुरङ्गम्} %॥३॥

\fourlineindentedshloka
{स्फुरत्कौस्तुभालङ्कृतं कण्ठदेशे}
{श्रिया जुष्टकेयूरकं श्रीनिवासम्}
{शिवं शान्तमीड्यं वरं लोकपालं}
{परब्रह्मलिङ्गं भजे पाण्डुरङ्गम्} %॥४॥

\fourlineindentedshloka
{शरच्चन्द्रबिम्बाननं चारुहासं}
{लसत्कुण्डलाक्रान्तगण्डस्थलान्तम्}
{जपारागबिम्बाधरं कञ्जनेत्रं}
{परब्रह्मलिङ्गं भजे पाण्डुरङ्गम्॥} %॥५॥

\fourlineindentedshloka
{किरीटोज्वलत्सर्वदिक्प्रान्तभागं}
{सुरैरर्चितं दिव्यरत्नैरनर्घैः}
{त्रिभङ्गाकृतिं बर्हमाल्यावतंसं}
{परब्रह्मलिङ्गं भजे पाण्डुरङ्गम्॥} %॥६॥

\fourlineindentedshloka
{विभुं वेणुनादं चरन्तं दुरन्तं}
{स्वयं लीलया गोपवेषं दधानम्}
{गवां वृन्दकानन्ददं चारुहासं}
{परब्रह्मलिङ्गं भजे पाण्डुरङ्गम्} %॥७॥

\fourlineindentedshloka
{अजं रुक्मिणीप्राणसञ्जीवनं तं}
{परं धाम कैवल्यमेकं तुरीयम्}
{प्रसन्नं प्रपन्नार्तिहं देवदेवं}
{परब्रह्मलिङ्गं भजे पाण्डुरङ्गम्} %॥८॥

\fourlineindentedshloka
{स्तवं पाण्डुरङ्गस्य वै पुण्यदं ये}
{पठन्त्येकचित्तेन भक्त्या च नित्यम्}
{भवाम्भोनिधिं ते वितीर्त्वान्तकाले}
{हरेरालयं शाश्वतं प्राप्नुवन्ति} 

॥इति श्रीमत्परमहंसपरिव्राजकाचार्यस्य श्रीगोविन्दभगवत्पूज्यपादशिष्यस्य श्रीमच्छङ्करभगवतः कृतौ पाण्डुरङ्गाष्टकं सम्पूर्णम्॥