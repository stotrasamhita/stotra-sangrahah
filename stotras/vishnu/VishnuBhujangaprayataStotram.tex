% !TeX program = XeLaTeX
% !TeX root = ../../shloka.tex

\sect{विष्णुभुजङ्गप्रयातस्तोत्रम्}
\begin{AutoCols}[\maxColumns]
\fourlineindentedshloka
{चिदंशं विभुं निर्मलं निर्विकल्पं}
{निरीहं निराकारमोङ्कारगम्यम्}
{गुणातीतमव्यक्तमेकं तुरीयं}
{परं ब्रह्म यं वेद तस्मै नमस्ते}% १}%

\fourlineindentedshloka
{विशुद्धं शिवं शान्तमाद्यन्तशून्यं}
{जगज्जीवनं ज्योतिरानन्दरूपम्}
{अदिग्देशकालव्यवच्छेदनीयं}
{त्रयी वक्ति यं वेद तस्मै नमस्ते}% २}%

\fourlineindentedshloka
{महायोगपीठे परिभ्राजमाने}
{धरण्यादितत्त्वात्मके शक्तियुक्ते}
{गुणाहस्करे वह्निबिम्बार्धमध्ये}
{समासीनमोङ्कर्णिकेऽष्टाक्षराब्जे}% ३}%

\fourlineindentedshloka
{समानोदितानेकसूर्येन्दुकोटि-}
{प्रभापूरतुल्यद्युतिं दुर्निरीक्षम्}
{न शीतं न चोष्णं सुवर्णावदात-}
{प्रसन्नं सदानन्दसंवित्स्वरूपम्}% ४}%

\fourlineindentedshloka
{सुनासापुटं सुन्दरभ्रूललाटं}
{किरीटोचिताकुञ्चितस्निग्धकेशम्}
{स्फुरत् पुण्डरीकाभिरामायताक्षं}
{समुत्फुल्लरत्नप्रसूनावतंसम्}% ५}%

\fourlineindentedshloka
{लसत् कुण्डलामृष्टगण्डस्थलान्तं}
{जपारागचोराधरं चारुहासम्}
{अलिव्याकुलामोलिमन्दारमालं}
{महोरस्फुरत्कौस्तुभोदारहारम्}% ६}%

\fourlineindentedshloka
{सुरत्नाङ्गदैरन्वितं बाहुदण्डैः}
{चतुर्भिश्चलत्कङ्कणालङ्कृताग्रैः}
{उदारोदरालङ्कृतं पीतवस्त्रं}
{पदद्वन्द्वनिर्धूतपद्माभिरामम्}% ७}%

\fourlineindentedshloka
{स्वभक्तेषु सन्दर्शिताकारमेवं}
{सदा भावयन् सन्निरुद्धेन्द्रियाश्वः}
{दुरापं नरो याति संसारपारं}
{परस्मै परेभ्योऽपि तस्मै नमस्ते}% ८}%

\fourlineindentedshloka
{श्रिया शातकुम्भद्युतिस्निग्धकान्त्या}
{धरण्या च दूर्वादलश्यामलाङ्ग्या}
{कलत्रद्वयेनामुना तोषिताय}
{त्रिलोकीगृहस्थाय विष्णो नमस्ते}% ९}%

\fourlineindentedshloka
{शरीरं कलत्रं सुतं बन्धुवर्गं}
{वयस्यं धनं सद्म भृत्यं भुवं च}
{समस्तं परित्यज्य हा कष्टमेको}
{गमिष्यामि दुःखेन दूरं किलाहम्}% १०}%

\fourlineindentedshloka
{जरेयं पिशाचीव हा जीवतो मे}
{वसामक्ति रक्तं च मांसं बलं च}
{अहो देव सीदामि दीनानुकम्पिन्}
{किमद्यापि हन्त त्वयोदासितव्यम्}% ११}%

\fourlineindentedshloka
{कफव्याहतोष्णोल्बणश्वासवेग-}
{व्यथाविस्फुरत्सर्वमर्मास्थिबन्धाम्}
{विचिन्त्याहमन्त्यामसङ्ख्यामवस्थां}
{बिभेमि प्रभो किं करोमि प्रसीद}% १२}%

\fourlineindentedshloka
{लपन्नच्युतानन्त गोविन्द विष्णो}
{मुरारे हरे नाथ नारायणेति}
{यथाऽनुस्मरिष्यामि भक्त्या भवन्तं}
{तथा मे दयाशील देव प्रसीद}% १३}%

\fourlineindentedshloka
{भुजङ्गप्रयातं पठेद्यस्तु भक्त्या}
{समाधाय चित्ते भवन्तं मुरारे}
{स मोहं विहायाऽऽशु युष्मत्प्रसादात्}
{समाश्रित्य योगं व्रजत्यच्युतं त्वाम्}% १४}%
\end{AutoCols}

॥इति श्रीमत्परमहंसपरिव्राजकाचार्यस्य श्री-गोविन्द-भगवत्पूज्य-पाद-शिष्यस्य
श्रीमच्छङ्करभगवतः कृतौ श्री-विष्णुभुजङ्गप्रयातस्तोत्रं सम्पूर्णम्‌॥

