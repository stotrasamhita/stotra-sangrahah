\sect{रङ्गनाथस्तोत्रम्}

\fourlineindentedshloka*
{सप्तप्राकारमध्ये सरसिजमुकुलोद्भासमाने विमाने}
{कावेरीमध्यदेशे फणिपतिशयने शेषपर्यङ्कभागे}
{निद्रामुद्राभिरामं कटिनिकटशिरः पार्श्वविन्यस्तहस्तम्}
{पद्माधात्रीकराभ्यां परिचितचरणं रङ्गराजं भजेऽहम्}

\fourlineindentedshloka
{आनन्दरूपे  निजबोधरूपे}
{ब्रह्मस्वरूपे श्रुतिमूर्तिरूपे}
{शशाङ्करूपे रमणीयरूपे}
{श्रीरङ्गरूपे रमतां मनो मे}

\fourlineindentedshloka
{कावेरितीरे करुणाविलोले}
{मन्दारमूले धृतचारुचेले}
{दैत्यान्तकालेऽखिललोकलीले}
{श्रीरङ्गलीले रमतां मनो मे}

\fourlineindentedshloka
{लक्ष्मीनिवासे जगतां निवासे}
{हृत्पद्मवासे रविबिम्बवासे}
{कृपानिवासे गुणबृन्दवासे}
{श्रीरङ्गवासे रमतां मनो मे}


\fourlineindentedshloka
{ब्रह्मादिवन्द्ये जगदेकवन्द्ये}
{मुकुन्दवन्द्ये सुरनाथवन्द्ये}
{व्यासादिवन्द्ये सनकादिवन्द्ये}
{श्रीरङ्गवन्द्ये रमतां मनो मे}

\fourlineindentedshloka
{ब्रह्माधिराजे गरुडाधिराजे}
{वैकुण्ठराजे सुरराजराजे}
{त्रैलोक्यराजेऽखिललोकराजे}
{श्रीरङ्गराजे रमतां मनो मे}

\fourlineindentedshloka
{अमोघमुद्रे परिपूर्णनिद्रे}
{श्रीयोगनिद्रे ससमुद्रनिद्रे}
{श्रितैकभद्रे जगदेकनिद्रे}
{श्रीरङ्गभद्रे रमतां मनो मे}


\fourlineindentedshloka
{स चित्रशायी भुजगेन्द्रशायी}
{नन्दाङ्कशायी कमलाङ्कशायी}
{क्षीराब्धिशायी वटपत्रशायी}
{श्रीरङ्गशायी रमतां मनो मे}


\fourlineindentedshloka
{इदं हि रङ्गं त्यजतामिहाङ्गम्}
{पुनर्नचाङ्कं यदि चाङ्गमेति}
{पाणौ रथाङ्गं चरणेम्बु गाङ्गम्}
{याने विहङ्गं शयने भुजङ्गम्}


\fourlineindentedshloka*
{रङ्गनाथाष्टकं पुण्यम्}
{प्रातरुत्थाय यः पठेत्}
{सर्वान् कामानवाप्नोति}
{रङ्गिसायुज्यमाप्नुयात्}

{॥इति श्रीमत्परमहंसपरिव्राजकाचार्यस्य श्री-गोविन्द-भगवत्पूज्य-पाद-शिष्यस्य 
श्रीमच्छङ्करभगवतः कृतौ श्री-रङ्गनाथाष्टकं सम्पूर्णम्॥}