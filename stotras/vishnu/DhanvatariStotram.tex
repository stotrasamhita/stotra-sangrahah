% !TeX program = XeLaTeX
% !TeX root = ../../shloka.tex

\sect{धन्वन्तरि स्तोत्रम् (मत्स्य पुराणान्तर्गतम्)}

\twolineshloka*
{क्षीरोदमोतं दिव्य-गन्धानुलेपनम्}
{सुधा-कलश-हस्तं तं वन्दे धन्वन्तरिं हरिम्}

\uvacha{देव-दानवा ऊचुः}

\twolineshloka
{नमो लोक-त्र्याध्यक्ष तेजसा जित-भास्कर}
{नमो विष्णो नमो जिष्णो नमस्ते कैटभार्दन}% ॥ १ ॥

\twolineshloka
{नमः सर्ग-क्रिया-कर्त्रे जगत् पालयते नमः}
{नमः स्मृतार्ति-नाशाय नमः पुष्कर मालिने}% ॥ २ ॥

\twolineshloka
{दिव्यौषधि-स्वरूपाय सुधा-कलश-पाणये}
{शङ्ख-चक्र-गदा-पद्म-धारिणे वनमालिने}% ॥ ३ ॥

\twolineshloka
{देवेन्द्रादि-सुरेड्याय नमः क्षीराब्धि-जन्मने}
{निर्गुणाय विशेषाय हरये ब्रह्म-रूपिणे}% ॥ ४ ॥

\twolineshloka
{जगत् प्रतिष्ठितं यत्र जगतां यो न दृश्यते}
{नमः सूक्ष्मातिसूक्ष्माय तस्मै देवाय शङ्खिने}% ॥ ५ ॥

\twolineshloka
{यं न पश्यन्ति पश्यन्तं जगदप्यखिलं नराः}
{अपश्यद्भिर्जगद् यश्च दृश्यते हृदि संस्थितः}% ॥ ६ ॥

\twolineshloka
{यस्मिन् वनानि पर्वता नद्यश्चैवाखिलं जगत्}
{तस्मै नमोऽस्तु जगताम् आधाराय नमो नमः}% ॥ ७ ॥

\twolineshloka
{आद्य-प्रजापतिर्यश्च यः पितॄणां परः पतिः}
{पतिः सुराणां यस्तस्मै नमः कृष्णाय वेधसे}% ॥ ८ ॥

\twolineshloka
{यः प्रवृत्तौ निवृत्तौ च इज्यते कर्मभिः स्वकैः}
{स्वर्गापवर्ग-फल-दो नमस्तस्मै गदा-भृते}% ॥ ९ ॥

\twolineshloka
{यश्चिन्त्यमानो मनसा सद्यः पापं व्यपोहति}
{नमस्तस्मै विशुद्धाय पराय हरि-मेधसे}% ॥ १० ॥

\twolineshloka
{यं बुद्ध्वा सर्वभूतानि देव-देवेशमव्ययम्}
{न पुनर्जन्म-मरणे प्राप्नुवन्ति नमामि तम्}% ॥ ११ ॥

\twolineshloka
{यो यज्ञे यज्ञ-परमैरिज्यते यज्ञ-संज्ञितः}
{तं यज्ञ-पुरुषं विष्णुं नमामि प्रभुमीश्वरम्}% ॥ १२ ॥

\twolineshloka
{गीयते सर्व-वेदेषु वेद-विद्भिर्विदां गतिः}
{यस्तस्मै वेद-वेद्याय विष्णवे जिष्णवे नमः}% ॥ १३ ॥

\twolineshloka
{यो विश्वं समुत्पन्नं यस्मिंश्च लयमेष्यति}
{विश्वोद्भव-प्रतिष्ठाय नमस्तस्मै महात्मने}% ॥ १४ ॥

\twolineshloka
{ब्रह्मादि-स्तम्ब-पर्यन्तं येन विश्वमिदं ततम्}
{माया जालं समुत्तर्त्तं तमुपेन्द्रं नमाम्यहम्}% ॥ १५ ॥

\twolineshloka
{विषाद-तोष-रोषाद्यैर्योऽजस्रं सुख-दुःखजम्}
{नृत्यत्यखिल-भूत-स्थस्तमुपेन्द्रं नमाम्यहम्}% ॥ १६ ॥

\twolineshloka
{यमाराध्य विशुद्धेन कर्मणा मनसा गिरा}
{तरन्त्यविद्याम् अखिलाम् आदि-वैद्यं नमाम्यहम्}% ॥ १७ ॥

\twolineshloka
{यः स्थितो विश्व-रूपेण बिभर्ति ह्यखिलौषधीः}
{तं रत्न-कलशोद्भासि-हस्तं धन्वन्तरिं नुमः}% ॥ १८ ॥

\twolineshloka
{विश्वं विश्व-पतिं विष्णुं तं नमामि प्रजापतिम्}
{मूर्त्या चासुरमय्या तु तद्विधान् विनिहन्ति यः}% ॥ १९ ॥

\twolineshloka
{रात्रि-रूपं सूर्य-रूपं भजेत्तं सान्ध्य-रूपिणम्}
{हन्ति विद्या-प्रदानेन यो वा अज्ञान-जं तमः}% ॥ २० ॥

\threelineshloka
{यस्तु भेषज-रूपेण जगदाप्याययेत् सदा}
{यस्याक्षिणी चन्द्र-सूर्यौ सर्व-लोक-शुभङ्करः}
{पश्यतः कर्म सततं तं च धन्वन्तरिं नुमः}% ॥ २१ ॥

\twolineshloka
{यस्मिन् सर्वेश्वरे वश्यं जगत् स्थावर-जङ्गमम्}
{आभाति तमजं विष्णुं नमामि प्रभुमव्ययम्}% ॥ २२ ॥

॥इति मत्स्य-पुराणान्तर्गतं धन्वन्तरि-स्तोत्रं सम्पूर्णम्॥
