% !TeX program = XeLaTeX
% !TeX root = ../../shloka.tex

\sect{श्री-लक्ष्मी-नृसिंह-पञ्चरत्न-स्तोत्रम्}
\begingroup
\setlength{\shlokaspaceskip}{10pt}
\fourlineindentedshloka
{त्वत्-प्रभु-जीव-प्रियमिच्छसि चेन्नरहरि-पूजां कुरु सततं}
{प्रतिबिम्बालङ्कृति-धृति-कुशलो बिम्बालङ्कृतिमातनुते}
{चेतो-भृङ्ग भ्रमसि वृथा भव-मरु-भूमौ विरसायां}
{भज भज लक्ष्मी-नरसिंहानघ-पद-सरसिज-मकरन्दम्}% ॥ १ ॥

\fourlineindentedshloka
{शुक्तौ रजत-प्रतिभा जाता कटकाद्यर्थ-समर्था चेद्}
{दुःखमयी ते संसृतिरेषा निर्वृति-दाने निपुणा स्यात्}
{चेतो-भृङ्ग भ्रमसि वृथा भव-मरु-भूमौ विरसायां}
{भज भज लक्ष्मी-नरसिंहानघ-पद-सरसिज-मकरन्दम्}% ॥ २ ॥

\fourlineindentedshloka
{आकृति-साम्याच्छाल्मलि-कुसुमे स्थल-नलिनत्व-भ्रममकरोः}
{गन्ध-रसाविह किमु विद्येते विफलं भ्राम्यसि भृश-विरसेऽस्मिन्}
{चेतो-भृङ्ग भ्रमसि वृथा भव-मरु-भूमौ विरसायां}
{भज भज लक्ष्मी-नरसिंहानघ-पद-सरसिज-मकरन्दम्}% ॥ ३ ॥

\fourlineindentedshloka
{स्रक्-चन्दन-वनितादीन् विषयान् सुखदान् मत्वा तत्र विहरसे}
{गन्ध-फली-सदृशा ननु तेऽमी भोगानन्तर-दुःख-कृतः स्युः}
{चेतो-भृङ्ग भ्रमसि वृथा भव-मरु-भूमौ विरसायां}
{भज भज लक्ष्मी-नरसिंहानघ-पद-सरसिज-मकरन्दम्}% ॥ ४ ॥

\fourlineindentedshloka
{तव हितमेकं वचनं वक्ष्ये श‍ृणु सुख-कामो यदि सततं}
{स्वप्ने दृष्टं सकलं हि मृषा जाग्रति च स्मर तद्वदिति}
{चेतो-भृङ्ग भ्रमसि वृथा भव-मरु-भूमौ विरसायां}
{भज भज लक्ष्मी-नरसिंहानघ-पद-सरसिज-मकरन्दम्}% ॥ ५ ॥

॥इति~श्रीमत्परमहंसपरिव्राजकाचार्यस्य श्री-गोविन्द-भगवत्पूज्य-पाद-शिष्यस्य
श्रीमच्छङ्करभगवतः कृतौ श्री-लक्ष्मी-नृसिंह-पञ्चरत्न-स्तोत्रं सम्पूर्णम्॥

\endgroup