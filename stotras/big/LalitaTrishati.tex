% !TeX program = XeLaTeX
% !TeX root = ../../shloka.tex

\sect{ललितात्रिशतीस्तोत्रम्}
\dnsub{न्यासः}
अस्य  श्रीललितात्रिशतीस्तोत्रमहामत्रस्य।भगवान्  हयग्रीव  ऋषिः। \\
अनुष्टुप्  छन्दः। श्रीललितात्रिपुरसुन्दरी  देवता।\\
ऐं  बीजम्।  क्लीं  शक्तिः।  सौः  कीलकम्।\\
सकल-चिन्तितफलावाप्त्यर्थे  जपे  विनियोगः॥\\

\dnsub{ध्यानम्}
\twolineshloka*
{अतिमधुरचापहस्ताम्  अपरिमितामोदबाण-सौभाग्याम्}
{अरुणाम्  अतिशयकरुणाम्  अभिनवकुलसुन्दरीं  वन्दे}

\dnsub{स्तोत्रम्}
\uvacha{हयग्रीव  उवाच}
\twolineshloka
{ककाररूपा कल्याणी कल्याणगुणशालिनी}
{कल्याणशैलनिलया कमनीया कलावती}

\twolineshloka
{कमलाक्षी कल्मषघ्नी करुणामृतसागरा}
{कदम्बकाननावासा कदम्बकुसुमप्रिया}

\twolineshloka
{कन्दर्पविद्या कन्दर्प-जनकापाङ्ग-वीक्षणा}
{कर्पूरवीटि-सौरभ्य-कल्लोलित-ककुप्तटा}

\twolineshloka
{कलिदोषहरा कञ्जलोचना कम्रविग्रहा}
{कर्मादिसाक्षिणी कारयित्री कर्मफलप्रदा}

\twolineshloka
{एकाररूपा चैकाक्षर्येकानेकाक्षराकृतिः}
{एतत्तदित्यनिर्देश्या चैकानन्द-चिदाकृतिः}

\twolineshloka
{एवमित्यागमाबोध्या चैकभक्ति-मदर्चिता}
{एकाग्रचित्त-निर्ध्याता चैषणा-रहितादृता}

\twolineshloka
{एलासुगन्धिचिकुरा चैनःकूटविनाशिनी}
{एकभोगा चैकरसा चैकैश्वर्य-प्रदायिनी}

\twolineshloka
{एकातपत्र-साम्राज्य-प्रदा चैकान्तपूजिता}
{एधमानप्रभा चैजदनेकजगदीश्वरी}

\twolineshloka
{एकवीरादि-संसेव्या चैकप्राभव-शालिनी}
{ईकाररूपा चेशित्री चेप्सितार्थ-प्रदायिनी}

\twolineshloka
{ईदृगित्य-विनिर्देश्या चेश्वरत्व-विधायिनी}
{ईशानादि-ब्रह्ममयी चेशित्वाद्यष्टसिद्धिदा}

\twolineshloka
{ईक्षित्रीक्षण-सृष्टाण्ड-कोटिरीश्वर-वल्लभा}
{ईडिता चेश्वरार्धाङ्ग-शरीरेशाधि-देवता}

\twolineshloka
{ईश्वर-प्रेरणकरी चेशताण्डव-साक्षिणी}
{ईश्वरोत्सङ्ग-निलया चेतिबाधा-विनाशिनी}

\twolineshloka
{ईहाविरहिता चेशशक्ति-रीषत्‌-स्मितानना}
{लकाररूपा ललिता लक्ष्मी-वाणी-निषेविता}

\twolineshloka
{लाकिनी ललनारूपा लसद्दाडिम-पाटला}
{ललन्तिकालसत्फाला ललाट-नयनार्चिता}

\twolineshloka
{लक्षणोज्ज्वल-दिव्याङ्गी लक्षकोट्यण्ड-नायिका}
{लक्ष्यार्था लक्षणागम्या लब्धकामा लतातनुः}

\twolineshloka
{ललामराजदलिका लम्बिमुक्तालताञ्चिता}
{लम्बोदर-प्रसूर्लभ्या लज्जाढ्या लयवर्जिता}

\twolineshloka
{ह्रीङ्काररूपा ह्रीङ्कारनिलया ह्रीम्पदप्रिया}
{ह्रीङ्कारबीजा ह्रीङ्कारमन्त्रा ह्रीङ्कारलक्षणा}

\twolineshloka
{ह्रीङ्कारजपसुप्रीता ह्रीम्मती ह्रींविभूषणा}
{ह्रींशीला ह्रीम्पदाराध्या ह्रीङ्गर्भा ह्रीम्पदाभिधा}

\twolineshloka
{ह्रीङ्कारवाच्या ह्रीङ्कारपूज्या ह्रीङ्कारपीठिका}
{ह्रीङ्कारवेद्या ह्रीङ्कारचिन्त्या ह्रीं ह्रीं-शरीरिणी}

\twolineshloka
{हकाररूपा हलधृक्पूजिता हरिणेक्षणा}
{हरप्रिया हराराध्या हरिब्रह्मेन्द्रवन्दिता}

\twolineshloka
{हयारूढा-सेविताङ्घ्रिर्हयमेध-समर्चिता}
{हर्यक्षवाहना हंसवाहना हतदानवा}

\twolineshloka
{हत्यादिपापशमनी हरिदश्वादि-सेविता}
{हस्तिकुम्भोत्तुङ्गकुचा हस्तिकृत्ति-प्रियाङ्गना}

\twolineshloka
{हरिद्राकुङ्कुमादिग्धा हर्यश्वाद्यमरार्चिता}
{हरिकेशसखी हादिविद्या हालामदालसा}

\twolineshloka
{सकाररूपा सर्वज्ञा सर्वेशी सर्वमङ्गला}
{सर्वकर्त्री सर्वभर्त्री सर्वहन्त्री सनातना}

\twolineshloka
{सर्वानवद्या सर्वाङ्गसुन्दरी सर्वसाक्षिणी}
{सर्वात्मिका सर्वसौख्यदात्री सर्वविमोहिनी}

\twolineshloka
{सर्वाधारा सर्वगता सर्वावगुणवर्जिता}
{सर्वारुणा सर्वमाता सर्वभूषण-भूषिता}

\twolineshloka
{ककारार्था कालहन्त्री कामेशी कामितार्थदा}
{कामसञ्जीवनी कल्या कठिनस्तन-मण्डला}

\twolineshloka
{करभोरूः कलानाथ-मुखी कचजिताम्बुदा}
{कटाक्षस्यन्दि-करुणा कपालि-प्राणनायिका}

\twolineshloka
{कारुण्य-विग्रहा कान्ता कान्तिधूत-जपावलिः}
{कलालापा कम्बुकण्ठी करनिर्जित-पल्लवा}

\twolineshloka
{कल्पवल्ली-समभुजा कस्तूरी-तिलकाञ्चिता}
{हकारार्था हंसगतिर्हाटकाभरणोज्ज्वला}

\twolineshloka
{हारहारि-कुचाभोगा हाकिनी हल्यवर्जिता}
{हरित्पति-समाराध्या हठात्कार-हतासुरा}

\twolineshloka
{हर्षप्रदा हविर्भोक्त्री हार्दसन्तमसापहा}
{हल्लीसलास्य-सन्तुष्टा हंसमन्त्रार्थ-रूपिणी}

\twolineshloka
{हानोपादान-निर्मुक्ता हर्षिणी हरिसोदरी}
{हाहाहूहू-मुख-स्तुत्या हानि-वृद्धि-विवर्जिता}

\twolineshloka
{हय्यङ्गवीन-हृदया हरिगोपारुणांशुका}
{लकाराख्या लतापूज्या लयस्थित्युद्भवेश्वरी}

\twolineshloka
{लास्य-दर्शन-सन्तुष्टा लाभालाभ-विवर्जिता}
{लङ्घ्येतराज्ञा लावण्य-शालिनी लघु-सिद्धिदा}

\twolineshloka
{लाक्षारस-सवर्णाभा लक्ष्मणाग्रज-पूजिता}
{लभ्येतरा लब्धभक्ति-सुलभा लाङ्गलायुधा}

\twolineshloka
{लग्न-चामर-हस्त-श्री-शारदा-परिवीजिता}
{लज्जापद-समाराध्या लम्पटा लकुलेश्वरी}

\twolineshloka
{लब्धमाना लब्धरसा लब्धसम्पत्समुन्नतिः}
{ह्रीङ्कारिणी  ह्रीङ्काराद्या ह्रीम्मध्या ह्रींशिखामणिः}

\twolineshloka
{ह्रीङ्कार-कुण्डाग्नि-शिखा ह्रीङ्कार-शशिचन्द्रिका}
{ह्रीङ्कार-भास्कररुचिर्ह्रीङ्काराम्भोद-चञ्चला}

\twolineshloka
{ह्रीङ्कार-कन्दाङ्कुरिका ह्रीङ्कारैक-परायणा}
{ह्रीङ्कार-दीर्घिकाहंसी ह्रीङ्कारोद्यान-केकिनी}

\twolineshloka
{ह्रीङ्कारारण्य-हरिणी ह्रीङ्कारावाल-वल्लरी}
{ह्रीङ्कार-पञ्जरशुकी ह्रीङ्काराङ्गण-दीपिका}

\twolineshloka
{ह्रीङ्कार-कन्दरा-सिंही ह्रीङ्काराम्भोज-भृङ्गिका}
{ह्रीङ्कार-सुमनो-माध्वी ह्रीङ्कार-तरुमञ्जरी}

\twolineshloka
{सकाराख्या समरसा सकलागम-संस्तुता}
{सर्ववेदान्त-तात्पर्यभूमिः सदसदाश्रया}

\twolineshloka
{सकला सच्चिदानन्दा साध्या सद्गतिदायिनी}
{सनकादिमुनिध्येया सदाशिव-कुटुम्बिनी}

\twolineshloka
{सकलाधिष्ठान-रूपा सत्यरूपा समाकृतिः}
{सर्वप्रपञ्च-निर्मात्री समानाधिक-वर्जिता}

\twolineshloka
{सर्वोत्तुङ्गा सङ्गहीना सगुणा सकलेष्टदा}
{ककारिणी काव्यलोला कामेश्वरमनोहरा}

\twolineshloka
{कामेश्वर-प्राणनाडी कामेशोत्सङ्गवासिनी}
{कामेश्वरालिङ्गिताङ्गी कामेश्वर-सुखप्रदा}

\twolineshloka
{कामेश्वर-प्रणयिनी कामेश्वर-विलासिनी}
{कामेश्वर-तपःसिद्धिः कामेश्वर-मनःप्रिया}

\twolineshloka
{कामेश्वर-प्राणनाथा कामेश्वर-विमोहिनी}
{कामेश्वर-ब्रह्मविद्या कामेश्वर-गृहेश्वरी}

\twolineshloka
{कामेश्वराह्लादकरी कामेश्वर-महेश्वरी}
{कामेश्वरी कामकोटिनिलया काङ्क्षितार्थदा}

\twolineshloka
{लकारिणी लब्धरूपा लब्धधीर्लब्ध-वाञ्छिता}
{लब्धपाप-मनोदूरा लब्धाहङ्कार-दुर्गमा}

\twolineshloka
{लब्धशक्तिर्लब्धदेहा लब्धैश्वर्यसमुन्नतिः}
{लब्धवृद्धिर्लब्धलीला लब्धयौवनशालिनी}

\twolineshloka
{लब्धातिशय-सर्वाङ्ग-सौन्दर्या लब्धविभ्रमा}
{लब्धरागा लब्धपतिर्लब्ध-नानागमस्थितिः}

\twolineshloka
{लब्धभोगा लब्धसुखा लब्धहर्षाभिपूरिता}
{ह्रीङ्कार-मूर्तिर्ह्रीङ्कार-सौधशृङ्गकपोतिका}

\twolineshloka
{ह्रीङ्कार-दुग्धाब्धि-सुधा ह्रीङ्कार-कमलेन्दिरा}
{ह्रीङ्कार-मणिदीपार्चिर्ह्रीङ्कार-तरुशारिका}

\twolineshloka
{ह्रीङ्कार-पेटक-मणिर्ह्रीङ्कारादर्श-बिम्बिता}
{ह्रीङ्कार-कोशासिलता ह्रीङ्कारास्थान-नर्तकी}

\twolineshloka
{ह्रीङ्कार-शुक्तिका-मुक्तामणिर्ह्रीङ्कार-बोधिता}
{ह्रीङ्कारमय-सौवर्णस्तम्भ-विद्रुम-पुत्रिका}

\twolineshloka
{ह्रीङ्कार-वेदोपनिषध्रीङ्काराध्वर-दक्षिणा}
{ह्रीङ्कार-नन्दनाराम-नवकल्पक-वल्लरी}

\twolineshloka
{ह्रीङ्कार-हिमवद्गङ्गा ह्रीङ्कारार्णव-कौस्तुभा}
{ह्रीङ्कार-मन्त्र-सर्वस्वा ह्रीङ्कार-परसौख्यदा}

॥इति~श्री-ब्रह्माण्डपुराणे उत्तराखण्डे श्री-हयग्रीवागस्त्यसंवादे
श्री-ललितात्रिशती स्तोत्रकथनं सम्पूर्णम्॥
