% !TeX program = XeLaTeX
% !TeX root = ../../shloka.tex

\sect{मातृभूतशतकम्}

\fourlineindentedshloka
{श्रेयांसि यं सकृदनुस्मरतां जनाना-}
{माविर्भवन्ति सुमहान्त्यनपायवन्ति}
{तं त्वामनन्यशरणश्शरणं भजामि}
{श्रीमातृभूत शिव पालय मां नमस्ते} % ॥ १॥

\fourlineindentedshloka
{आनन्द सच्चिदमलात्मनि शङ्करे त्वय्-}
{यन्तर्निषीदति भवन्तमहो विहाय}
{भ्राम्यामि मोहविवशो भवशोकखिन्नः}
{श्रीमातृभूत शिव पालय मां नमस्ते} % ॥ २॥

\fourlineindentedshloka
{मद्दाससूनुरयमित्यनुकम्पनीये}
{मय्यर्भके समधिरोप्य वृथाऽपराधान्}
{युक्तं किमेवमधिगन्तुमुदासिकां ते}
{श्रीमातृभूत शिव पालय मां नमस्ते} % ॥ ३॥

\fourlineindentedshloka
{सौशील्यघस्मरधनस्मयनष्टसंज्ञान्}
{अज्ञो विषीदति यदेष वृथानुधावन्}
{तत्पश्यतस्तव दया न कथं दयालो}
{श्रीमातृभूत शिव पालय मां नमस्ते} % ॥ ४॥

\fourlineindentedshloka
{भक्तिं त्वयि श्रितवतां वद कः प्रहीणः}
{सा वा तवाहमिति किं न सकृत्प्रपत्तिः}
{किं तद्वतामनवितासि कुतोऽस्म्युपेक्ष्यः}
{श्रीमातृभूत शिव पालय मां नमस्ते} % ॥ ५॥

\fourlineindentedshloka
{दासस्तवाहमिति यः सकृदप्युपैति}
{रक्ष्यः स एष इति ते विधृतव्रतस्य}
{रक्षोऽहमस्म्यथ कियानफलो विलम्बः}
{श्रीमातृभूत शिव पालय मां नमस्ते} % ॥ ६॥

\fourlineindentedshloka
{आज्ञापयेः कथमये तव गर्भवासं}
{दुःखापवर्गिणि गिरीश समाश्रयेयम्}
{आस्तां सहस्रमपि नानुगुणं त्विदं ते}
{श्रीमातृभूत शिव पालय मां नमस्ते} % ॥ ७॥

\fourlineindentedshloka
{सर्वज्ञ शङ्करमहेशदयालु शब्दाः}
{किं पारिभाषिकतयोपगता भवन्तम्}
{विज्ञापिते विलपितेऽपि किमित्युदास्से}
{श्रीमातृभूत शिव पालय मां नमस्ते} % ॥ ८॥

\fourlineindentedshloka
{हीनैः सदाधिकृतिकैरपि धृष्यमाणे}
{सावज्ञमेवमपदे मयि दृश्यमाने}
{नेदं क्षणं क्षणमुपेक्षणमाश्रयेथाः}
{श्रीमातृभूत शिव पालय मां नमस्ते} % ॥ ९॥

\fourlineindentedshloka
{आलस्यदुर्विषयसक्त्यपलापनिद्रा-}
{पद्मामदाधिगदनीचसमागमादीन्}
{विघ्नान् धुनीहि घटय त्वयि मेऽभिलाषं}
{श्रीमातृभूत शिव पालय मां नमस्ते} % ॥ १०॥

\fourlineindentedshloka
{क्वासावनन्यसुलभा भवदीयता मे}
{भूयिष्ठनिष्ठुर विपद्विषमा क्व वृत्तिः}
{एतन्न किं तव यशोभरभङ्गभीतं}
{श्रीमातृभूत शिव पालय मां नमस्ते} % ॥ ११॥

\fourlineindentedshloka
{सम्फुल्लचम्पकलसन्नवमल्लिकादि-}
{हृद्योपचार रुचिरं भवदर्चनं मे}
{पाण्योस्तरङ्गय निरन्तरमन्तरात्मन्}
{श्रीमातृभूत शिव पालय मां नमस्ते} % ॥ १२॥

\fourlineindentedshloka
{दारैः सुतैश्शुभतमैर्द्रविणैः प्रशान्तैः}
{स्वर्गापवर्गफलसिद्धिभिरभ्युपेताः}
{नन्दन्ति ते करुणया किमसाध्यमस्याः}
{श्रीमातृभूत शिव पालय मां नमस्ते} % ॥ १३॥

\fourlineindentedshloka
{कालं कियन्तमघदन्तुरमन्तरङ्गं}
{त्वच्चिन्तनं तनितुमक्षममीक्षसे त्वम्}
{त्वद्भक्तये स्पृहयतश्शिव मे दयस्व}
{श्रीमातृभूत शिव पालय मां नमस्ते} % ॥ १४॥

\fourlineindentedshloka
{वित्तस्मयान्ध गृहकानन सम्प्रचार}
{धौरेयभावमवधूय कदा मदङ्घ्री}
{स्तां त्वत्प्रदक्षिणविधौ करुणावशात्ते}
{श्रीमातृभूत शिव पालय मां नमस्ते} % ॥ १५॥

\fourlineindentedshloka
{स्यात्त्वत्पदं विदलदम्बुजसौरभश्री-}
{विश्रामसीमभिरुदञ्चितवाक्प्रपञ्चैः}
{स्तोत्रैर्विभो मम कदा नु सदानुबन्धि}
{श्रीमातृभूत शिव पालय मां नमस्ते} % ॥ १६॥

\fourlineindentedshloka
{पादौ प्रदक्षिणविधौ भवदर्चनेषु}
{पाणी स्तवेषु फणितीरपि मेऽन्तरात्मन्}
{ध्याने मनश्च विनियुङ्क्ष्व दयस्व मङ्क्षु}
{श्रीमातृभूत शिव पालय मां नमस्ते} % ॥ १७॥

\fourlineindentedshloka
{मूढोऽपटुर्भवदुपश्रयणक्रमेषु}
{रूढोपतापविततिर्विपदां कदम्बैः}
{सीमा तृषामहमतोऽस्मि तवानुकम्प्यः}
{श्रीमातृभूत शिव पालय मां नमस्ते} % ॥ १८॥

\fourlineindentedshloka
{अम्बेयमप्रतिहतप्रतिपत्तिशक्तिः}
{सर्वज्ञता च भवतो भुवने निरूढा}
{भूतिश्च वां निरवधिस्तनयस्तु सोऽहं}
{श्रीमातृभूत शिव पालय मां नमस्ते} % ॥ १९॥

\fourlineindentedshloka
{पुत्रादयो न सुखयन्ति हि भावभेदात्}
{सोपाधिकप्रणयसम्भ्रमिभिः किमेतैः}
{माता पिता च निरुपाधिकृपानिधिस्त्वं}
{श्रीमातृभूत शिव पालय मां नमस्ते} % ॥ २०॥

\fourlineindentedshloka
{उद्गद्गदस्वनगलं गलदम्बुनेत्रं}
{रोमाञ्चकञ्चुकितमञ्चितघर्मबिन्दु}
{त्वन्नामनि श्रवणगामिनि कल्पयाङ्गं}
{श्रीमातृभूत शिव पालय मां नमस्ते} % ॥ २१॥

\fourlineindentedshloka
{दुर्लभ्यदुर्विपरिणत्यसुखानुबन्धि-}
{विप्लावितेषु विषयेषु वृथा निपात्य}
{मां खेदयन्न कथमुद्वहसेऽनुकम्पां}
{श्रीमातृभूत शिव पालय मां नमस्ते} % ॥ २२॥

\fourlineindentedshloka
{कन्दर्पकोटिशतसुन्दरमिन्दुरेखा-}
{चूडं दरस्मितपरिस्फुरदाननेन्दुम्}
{कान्तं नगेन्द्रसुतया कलये भवन्तं}
{श्रीमातृभूत शिव पालय मां नमस्ते} % ॥ २३॥

\fourlineindentedshloka
{सृष्टौ स्थितावपि लयेऽस्ति तव प्रभुत्वं}
{दुःखे तु मे द्वितयगोचरमेव दृश्यम्}
{व्यङ्क्तुं तदन्तविषयं तव को विलम्बः}
{श्रीमातृभूत शिव पालय मां नमस्ते} % ॥ २४॥

\fourlineindentedshloka
{मन्दस्मितैर्दिशि दिशि स्फुरदिन्दुचूडैः}
{गौरीसखैः स्फटिकगौर मनोहराङ्गैः}
{त्वद्विग्रहैः कबलयाविरतं मनो मे}
{श्रीमातृभूत शिव पालय मां नमस्ते} % ॥ २५॥

\fourlineindentedshloka
{प्रेमातिरेकशिशिरां त्वदपाङ्गधाराम्}
{आनन्दसिन्धुवलमानतरङ्गरेखाम्}
{सञ्चारयन्मयि सहाम्बिकया दयेथाः}
{श्रीमातृभूत शिव पालय मां नमस्ते} % ॥ २६॥

\fourlineindentedshloka
{भ्रूकिङ्करास्तव विरिञ्चपुरन्दराद्याः}
{कस्त्वां विरोत्स्यति निरोत्स्यति किं ममैनः}
{दैन्यं ततो न बलवत् किमु दीनबन्धो}
{श्रीमातृभूत शिव पालय मां नमस्ते} % ॥ २७॥

\fourlineindentedshloka
{अन्वेति शङ्कर न यं करुणाङ्कुरस्ते}
{कस्तं विलोकयतु जल्पतु को धिनोतु}
{केयं विभो विमुखता मयि किङ्करेऽस्मिन्}
{श्रीमातृभूत शिव पालय मां नमस्ते}  % ॥ २८॥

\fourlineindentedshloka
{भान्तं भवन्तमनघं मणिभूषणौघैः}
{भान्त्योमया विमलविद्रुमपाटलाङ्ग्या}
{श‍ृङ्गारिताङ्कमुपयामि शशाङ्कगौरं}
{श्रीमातृभूत शिव पालय मां नमस्ते} % ॥ २९॥

\fourlineindentedshloka
{मन्दस्मितैः करुणया शिशिरैः कटाक्षैः}
{प्रेमातिरेकमधुरैरपि चाभिलापैः}
{आनन्दयिष्यसि कदा नु सहाम्बया त्वं}
{श्रीमातृभूत शिव पालय मां नमस्ते} % ॥ ३०॥

\fourlineindentedshloka
{गोपायितेति कुलदैवतमित्यजस्रं}
{विस्रम्भतोऽर्पितभरं त्वयि दीनबन्धो}
{मामीक्षसे न कृपया यदि किं करोमि}
{श्रीमातृभूत शिव पालय मां नमस्ते} % ॥ ३१॥

\fourlineindentedshloka
{एकं सकृद्यदलसो न पुराऽभजं त्वां}
{नानादुरीशसतताश्रयदुर्विपाकः}
{तस्यै न सोऽयमधुनाऽतिधुनाति चेतः}
{श्रीमातृभूत शिव पालय मां नमस्ते} % ॥ ३२॥

\fourlineindentedshloka
{एषा सुगन्धिचिकुरा जननी पिता त्वं}
{का ते सुते मयि विभो करुणा प्रहाणिः}
{किञ्चापि सा न खलु सार्द्रहृदो जनन्याः}
{श्रीमातृभूत शिव पालय मां नमस्ते} % ॥ ३३॥

\fourlineindentedshloka
{त्वत्संविदा तरति शोकमिति श्रुतीनां}
{भावो भवोन्नतिविधायि तवाभिधानम्}
{शोकापहं तदहमीश समाश्रयामि}
{श्रीमातृभूत शिव पालय मां नमस्ते} % ॥ ३४॥

\fourlineindentedshloka
{त्वन्नामकीर्तनमघापहमाशयेन}
{शंसन् पुराणनिकरो न किलार्थवादः}
{तत्संश्रयोऽहमकृपास्पदमस्मि किं ते}
{श्रीमातृभूत शिव पालय मां नमस्ते} % ॥ ३५॥

\fourlineindentedshloka
{कर्मानुरोधि फलमित्यवगम्यमाने}
{ह्येनोहृतौ शुभविधौ च पटीयसस्ते}
{कर्मेति कं प्रति तु वञ्चनचातुरीयं}
{श्रीमातृभूत शिव पालय मां नमस्ते} % ॥ ३६॥

\fourlineindentedshloka
{उत्सङ्गसीमनि सुगन्धिकचां निषण्णाम्}
{उल्लासयन् कुचतटीनखरावमर्शैः}
{आभासि शङ्कर निरन्तरमन्तरङ्गे}
{श्रीमातृभूत शिव पालय मां नमस्ते} % ॥ ३७॥

\fourlineindentedshloka
{मन्देतरैरखिलमान्तरमन्धकारं}
{माणिक्यभूषणमहोर्मिभिरंशुजालैः}
{धुन्वन्वपुस्तव विभासय मानसे मे}
{श्रीमातृभूत शिव पालय मां नमस्ते} % ॥ ३८॥

\fourlineindentedshloka
{अन्तर्वसद्भवदुदारपदारविन्द}
{सन्दर्शनाय मम तत्प्रतिरोधकं द्राक्}
{अंहः कवाटपटलं विघटय्य देव}
{श्रीमातृभूत शिव पालय मां नमस्ते} % ॥ ३९॥

\fourlineindentedshloka
{चेतो मम त्वदनुचिन्तनबद्धतृष्णं}
{मुष्णन्ति तच्च विषया मुहुरेधितार्थाः}
{विघ्नान् जहि त्वयि विजृम्भय भक्तियोगं}
{श्रीमातृभूत शिव पालय मां नमस्ते} % ॥ ४०॥

\fourlineindentedshloka
{दृक्कूणितेन जगतः खलु सृष्टिरेषा}
{तेनैव तेऽवनलयौ भुवने निरूढौ}
{तेनैव संहर ममाघततिं दयालो}
{श्रीमातृभूत शिव पालय मां नमस्ते} % ॥ ४१॥

\fourlineindentedshloka
{अङ्गं शिरीषमृदुलं परुषैर्ममाश्म-}
{प्रायैर्वचोभिरभिपीडितमीडनार्थे}
{यत्तावकं तमपराधमिमं सहस्व}
{श्रीमातृभूत शिव पालय मां नमस्ते} % ॥ ४२॥

\fourlineindentedshloka
{मन्दस्मितैर्भ्रुकुटिभिर्मधुरैः कटाक्षैः}
{वामभ्रुवां नवनवैश्च वपुर्विलासैः}
{त्वां नोऽभजं तरलितस्तदिदं सहस्व}
{श्रीमातृभूत शिव पालय मां नमस्ते} % ॥ ४३॥

\fourlineindentedshloka
{मन्दस्मितं मदनसुन्दरमिन्दुचूडम्}
{उत्सङ्गविस्फुरदुमोरसिजस्पृशं त्वाम्}
{ध्यायन्वसानि दिवसानि नयन् कदा वा}
{श्रीमातृभूत शिव पालय मां नमस्ते} % ॥ ४४॥

\fourlineindentedshloka
{पाटीरसारलहरीं तुहिनांशुधारां}
{सौधाकरीमपि झरीमवमन्यमानः}
{त्वच्चिन्तनप्रभवनिर्वृतिभिः कदा स्यां}
{श्रीमातृभूत शिव पालय मां नमस्ते} % ॥ ४५॥

\fourlineindentedshloka
{त्वामस्म्यसावजमुपासितुमक्षमोऽहं}
{मोहान्धया बत धिया न यदेतदेनः}
{हार्या त्वयैव कृपया मम दुर्दशेयं}
{श्रीमातृभूत शिव पालय मां नमस्ते} % ॥ ४६॥

\fourlineindentedshloka
{वित्तार्जनस्य सकलः किमसावनेहा}
{नेह स्थितेः फलमहो भवतो धनेहा}
{ इत्थं दुरीश्वरगिरां पदमस्मि दीनः}
{श्रीमातृभूत शिव पालय मां नमस्ते} % ॥ ४७॥

\fourlineindentedshloka
{सांसारिकानवधिखेदपरम्परासू-}
{न्मज्जन्निमज्जदिव सीदति मानसं मे}
{आश्वासयाशु कृपया न किमीक्षसे त्वं}
{श्रीमातृभूत शिव पालय मां नमस्ते} % ॥ ४८॥

\fourlineindentedshloka
{लक्ष्मीपतिस्तव पदाभरणे स शेते}
{भिक्षाटनेन वद कं प्रतिविप्रलम्भः}
{जानाम्यहं न किमशेषजगत्पतिं त्वां}
{श्रीमातृभूत शिव पालय मां नमस्ते} % ॥ ४९॥

\fourlineindentedshloka
{संसृत्युदन्वति वितन्वति दुःखमोह-}
{ग्राहैर्व्यथामहह मे दृढमज्जनेन}
{त्वं क्रीडसीति सदृशं तव किं दयालो}
{श्रीमातृभूत शिव पालय मां नमस्ते} % ॥ ५०॥

\fourlineindentedshloka
{मात्रा सुगन्धिकचया भवता च पित्रा}
{युक्तः प्रसूजनकहीन इवावसीदन्}
{अस्मीति वां जगदधीश्वरयोरयुक्तं}
{श्रीमातृभूत शिव पालय मां नमस्ते} % ॥ ५१॥

\fourlineindentedshloka
{मौढ्यादपाटवभरादपि पूजनं ते}
{नाहं तनोमि तदहं व्यथितो भवामि}
{विघ्नानपास्य वितर त्वदुपासनां मे}
{श्रीमातृभूत शिव पालय मां नमस्ते} % ॥ ५२॥

\fourlineindentedshloka
{त्वद्वन्दनं ननु जगत्त्रयवन्द्यतायै}
{त्वत्पूजनं तु भुवनत्रयपूज्यतायै}
{पुण्योच्चयेन तदिदं भवदीयता मे}
{श्रीमातृभूत शिव पालय मां नमस्ते} % ॥ ५३॥

\fourlineindentedshloka
{अल्पं महच्च सकलं फलमर्पणीयं}
{देव त्वयैव खलु तत्किमपीहमानः}
{त्वामाश्रयाम्यगतिको वद किं तवागः}
{श्रीमातृभूत शिव पालय मां नमस्ते} % ॥ ५४॥

\fourlineindentedshloka
{कन्दर्पदर्पहरसुन्दररूपधेये}
{मन्दस्मिते भुवनमङ्गलनामधेये}
{खेदापहे भवति खेलय मानसं मे}
{श्रीमातृभूत शिव पालय मां नमस्ते} % ॥ ५५॥

\fourlineindentedshloka
{सूर्योदये तिमिरजालमिवार्तिजालं}
{धावन्न किं मम विभो यदि दृक्पथे स्याः}
{किं मे तथा न दयसेऽनुचितो विलम्बः}
{श्रीमातृभूत शिव पालय मां नमस्ते} % ॥ ५६॥

\fourlineindentedshloka
{त्वं त्रायसे यदि निवारयितुं क्षमः को}
{लभ्यं किमस्ति तव मय्यवसीदतीत्थम्}
{किं हीयते मदवने तव किं विलम्बैः}
{श्रीमातृभूत शिव पालय मां नमस्ते} % ॥ ५७॥

\fourlineindentedshloka
{चन्द्रोपलामलभवत्कमनीयमूर्ति-}
{सम्पूजनं सुदृढभक्त्यनघोपचारैः}
{ईहे सदा रचयितुं घटयाशु तन्मे}
{श्रीमातृभूत शिव पालय मां नमस्ते} % ॥ ५८॥

\fourlineindentedshloka
{संसारसिन्धुसमुदित्वरखेदमोद-}
{वीचीपरिभ्रमपरिश्रमनोदनं मे}
{आकल्पय त्वदभिराधनहर्षभूतं}
{श्रीमातृभूत शिव पालय मां नमस्ते} % ॥ ५९॥

\fourlineindentedshloka
{भक्तिस्पृशो भवति तस्य न शर्म तेन}
{त्राणव्रताश्रुतिषु वा न सकृत् प्रपन्नः}
{कस्मात् करोषि करुणां न मयि प्रपन्ने}
{श्रीमातृभूत शिव पालय मां नमस्ते} % ॥ ६०॥

\fourlineindentedshloka
{यन्मां न पालयसि तत्त्वदपाटवाद्वा}
{यद्वा मदीयदुरिताद्वद नाथ नाद्यः}
{सर्वेश्वरो ननु भवानत एव नान्त्यः}
{श्रीमातृभूत शिव पालय मां नमस्ते} % ॥ ६१॥

\fourlineindentedshloka
{विश्वाधिनेतरि विपन्नजनावनस्थे}
{नाथे सति त्वयि न यन्मम खेदभङ्गः}
{तत्ते यशोहरमतो विनिवेदयामि}
{श्रीमातृभूत शिव पालय मां नमस्ते} % ॥ ६२॥

\fourlineindentedshloka
{अस्योद्यतः समुपरोद्धुमनात्मनीनं}
{सत्यं भवानघफलार्त्यनुभावनासु}
{नेशस्तु सोढुमियदस्म्यहमार्तबन्धौ}
{श्रीमातृभूत शिव पालय मां नमस्ते} % ॥ ६३॥

\fourlineindentedshloka
{आमज्जतः प्रतिपदं विपदम्बुराशौ}
{दृप्तं मम श्वसितुमप्यनधीश्वरस्य}
{किं पश्यतस्तव दया न कियान् विलम्बः}
{श्रीमातृभूत शिव पालय मां नमस्ते} % ॥ ६४॥

\fourlineindentedshloka
{स्वर्धेनवो न कति नो कति कल्पवृक्षाः}
{भ्रूवश्यतां न कति यान्त्यपरेऽप्युदाराः}
{भक्तिस्पृशां त्वयि भजे तमहं भवन्तं}
{श्रीमातृभूत शिव पालय मां नमस्ते} % ॥ ६५॥

\fourlineindentedshloka
{एनामुपप्लवभरैर्जटिलामवस्थां}
{प्राप्तं हि मां परिहसन्ति नितान्तमन्ये}
{त्वञ्चेन्न पश्यसि दयार्द्रदृशा गतिः का}
{श्रीमातृभूत शिव पालय मां नमस्ते} % ॥ ६६॥

\fourlineindentedshloka
{श्रीशश्शरः कनकभूमिशरश्शरासो}
{वासस्तु रूप्यशिखरी धनदः सखा ते}
{त्वामाश्रितस्य मम किन्न धुनोषि दैन्यं}
{श्रीमातृभूत शिव पालय मां नमस्ते} % ॥ ६७॥

\fourlineindentedshloka
{अत्यङ्कुरार्णमुखखेदमहार्णवेऽस्मिन्}
{मग्नस्य शङ्कर ममोत्तरणं प्रकल्प्य}
{त्वद्पूजनं घटय सङ्क्षपितान्तरायं}
{श्रीमातृभूत शिव पालय मां नमस्ते} % ॥ ६८॥

\fourlineindentedshloka
{लोकेऽतिवर्तितुमलं तव शासनं कः}
{पद्मासनो भवतु पद्मविलोचनो वा}
{तद्दुर्लिपिं विधिकृतामपि मे विलुम्प}
{श्रीमातृभूत शिव पालय मां नमस्ते} % ॥ ६९॥

\fourlineindentedshloka
{निःसारतामपि विदन्विषयेष्वमीषु}
{निःसीमसक्तिरहमेषु हि गाढमूढः}
{रक्ष्यस्त्वयैव करुणानिधिना प्रसह्य}
{श्रीमातृभूत शिव पालय मां नमस्ते} % ॥ ७०॥

\fourlineindentedshloka
{मञ्चे तरङ्गितमणित्विषि पुष्पकऌप्ते}
{मन्दस्मितं सह कदाऽम्बिकया भवन्तम्}
{पश्यामि पश्चिमतनूपहितोपबर्हं}
{श्रीमातृभूत शिव पालय मां नमस्ते} % ॥ ७१॥

\fourlineindentedshloka
{मय्यात्मनीनविमुखे मलिनस्वभावे}
{निर्वेदसागरनिमज्जनविप्लवेऽस्मिन्}
{कस्माद् भवस्यकरुणः करुणानिधे त्वं}
{श्रीमातृभूत शिव पालय मां नमस्ते} % ॥ ७२॥

\fourlineindentedshloka
{रुद्राक्षभृद्भसितभूषणधन्यमूर्तिः}
{भावं वहन् प्रमुदितं भवदेकतानम्}
{कालं क्षिपेयमिति साधय कामनां मे}
{श्रीमातृभूत शिव पालय मां नमस्ते} % ॥ ७३॥

\fourlineindentedshloka
{त्वं चिद्घनोऽस्यहमसौ जडिमैकसीमा}
{दुःखाकरः पुनरहं सुखशेवधिस्त्वम्}
{युक्ता न ते स्थितिरियं तव गर्भवासे}
{श्रीमातृभूत शिव पालय मां नमस्ते} % ॥ ७४॥

\fourlineindentedshloka
{खेदं हरिष्यसि कदा शिशिरैः कटाक्षैः}
{मोदं करिष्यसि कदा मृदुभिर्वचोभिः}
{ इत्युत्सुकोऽस्मि करुणां कुरु किङ्करेऽस्मिन्}
{श्रीमातृभूत शिव पालय मां नमस्ते} % ॥ ७५॥

\fourlineindentedshloka
{पाति प्रसह्य न किमर्भकमात्मनीनं}
{संयोजकैर्गुरुजनः स्मर तं क्रमं त्वम्}
{भक्तिं बलाद्भवति कल्पय मे दयालो}
{श्रीमातृभूत शिव पालय मां नमस्ते} % ॥ ७६॥

\fourlineindentedshloka
{सोऽहं न किञ्चिदपि साधु करोमि पापः}
{तुङ्गं पदं त्वभिलषामि तदप्यलज्जः}
{त्वद्दातृतासदयतागरिमावमर्शी}
{श्रीमातृभूत शिव पालय मां नमस्ते} % ॥ ७७॥

\fourlineindentedshloka
{शक्तो न कश्चिदपि शासनमन्तरा ते}
{कः सेव्यतां त्वदितरः किमितः फलं वा}
{स्वामिन्नुपैमि शरणं करणैस्त्रिभिस्त्वां}
{श्रीमातृभूत शिव पालय मां नमस्ते} % ॥ ७८॥

\fourlineindentedshloka
{मृत्युञ्जयोऽसि कमलापतिवल्लभोऽसि}
{तन्मृत्युदैन्यविपदोरपदं त्वदीयम्}
{त्वामाश्रितस्य मम किं सदृशी दशेयं}
{श्रीमातृभूत शिव पालय मां नमस्ते} % ॥ ७९॥

\fourlineindentedshloka
{किं जामिता न भवतो मम दुःखधारा-}
{निर्मातुरस्मृतिपदं किमुपांशुयाजः}
{कस्मात्प्रभो किमपि शर्म न किं विधत्से}
{श्रीमातृभूत शिव पालय मां नमस्ते} % ॥ ८०॥

\fourlineindentedshloka
{क्वाहं क्व चोन्नतपदस्पृहयालुतेय-}
{मात्मैव मामपहसत्यधुना तथाऽपि}
{त्वत्संश्रये सति किमस्ति दुरापमीश}
{श्रीमातृभूत शिव पालय मां नमस्ते} % ॥ ८१॥

\fourlineindentedshloka
{कस्याग्रतः प्रकटयानिरदानकाण्डे}
{कं हर्षयाणि नुतिभिः कमुपाश्रयाणि}
{मोघः श्रमेषु निपतेयमतो दयेथाः}
{श्रीमातृभूत शिव पालय मां नमस्ते} % ॥ ८२॥

\fourlineindentedshloka
{किं विद्यया किमु धनेन किमात्मजैर्वा}
{किं जायया किमितरैस्त्वयि चेन्न चेतः}
{आनन्दशेवधिमतस्त्वयि देहि भक्तिं}
{श्रीमातृभूत शिव पालय मां नमस्ते} % ॥ ८३॥

\fourlineindentedshloka
{दासोऽस्मि दैन्यविवशस्त्वयि सर्वलोक-}
{साम्राज्यसम्पदमकम्पितमश्नुवाने}
{यत्तन्न ते समुचितं विमृश प्रसीद}
{श्रीमातृभूत शिव पालय मां नमस्ते} % ॥ ८४॥

\fourlineindentedshloka
{त्वामाश्रितस्त्वदितराश्रयदैन्यभङ्गम्}
{अङ्गीकरोमि जगदीश कथं कथं वा}
{त्वां नायशः स्पृशतु तत्त्यज मय्युपेक्षां}
{श्रीमातृभूत शिव पालय मां नमस्ते} % ॥ ८५॥

\fourlineindentedshloka
{श्रान्तं वपुर्दुरधिपाश्रयधावनान्मे}
{श्रान्तं वचो दुरधिपस्तुतिधोरणीभिः}
{क्रान्तं मनो दुरधिपाश्रयचिन्तयैव}
{श्रीमातृभूत शिव पालय मां नमस्ते} % ॥ ८६॥

\fourlineindentedshloka
{हेमस्फुरत्तनुतरङ्गितकान्तिधारा-}
{सम्प्लाव्यमानमणिभूषणदिव्यभासम्}
{मन्ये भवन्तमुमया मणिहेमपीठे}
{श्रीमातृभूत शिव पालय मां नमस्ते} % ॥ ८७॥

\fourlineindentedshloka
{स्वानन्दबिन्दुलव सम्प्लवमानसर्व-}
{ब्रह्माण्डमण्डलमखण्डितवैभवं त्वाम्}
{कस्तोषयेत्तदपि तुष्य परिभ्रमैर्मे}
{श्रीमातृभूत शिव पालय मां नमस्ते} % ॥ ८८॥

\fourlineindentedshloka
{कारुण्यनिर्भरसुधारसशीतलाभिः}
{काले कदा तव कटाक्षपरम्पराभिः}
{कष्टां दशामुपजहानि दृढोपगूढः}
{श्रीमातृभूत शिव पालय मां नमस्ते} % ॥ ८९॥

\fourlineindentedshloka
{भक्ताभिभूत्यसहनो भवसि प्रतीतो}
{भक्तोऽस्म्यहं त्वदभिधाग्रहणप्रसक्तः}
{तस्यास्य ते मयि कृपा न कथं विषण्णे}
{श्रीमातृभूत शिव पालय मां नमस्ते} % ॥ ९०॥

\fourlineindentedshloka
{दृक्केलयस्तव कृपारसशीकराङ्काः}
{यास्तासु तत्तदवनप्रवणासु चेत् स्यात्}
{एका पुनर्मदवनप्रवणा क्षतिः का}
{श्रीमातृभूत शिव पालय मां नमस्ते} % ॥ ९१॥

\fourlineindentedshloka
{श्रेयो रुणद्धि यदि पापगणस्तदा त्वां}
{त्वन्नामकीर्तनमघक्षपणक्षमं मे}
{किन्नास्ति किं न दयसे मयि हन्त खिन्ने}
{श्रीमातृभूत शिव पालय मां नमस्ते} % ॥ ९२॥

\fourlineindentedshloka
{तातप्रसूतनयदार सुहृद्भिरेवम्}
{अर्थोपपत्तिविरहादहहावसीदन्}
{नेतुं क्षणं न निपुणोऽस्मि ततः प्रसीद}
{श्रीमातृभूत शिव पालय मां नमस्ते} % ॥ ९३॥

\fourlineindentedshloka
{श्रीशैलराजसुतया शिशिरैः कटाक्षैः}
{मां वीक्ष्यमाणमसकृन्मधुरस्मितेन}
{सम्भावयन्मयि कदा नु दयिष्यसे त्वं}
{श्रीमातृभूत शिव पालय मां नमस्ते} % ॥ ९४॥

\fourlineindentedshloka
{पारे भवाख्यजलधेर्भगवन् कृपाकू-}
{पारे भवत्यवतरिष्यति मे कदा वा}
{प्रेम्णा तृणीकृतपदं सकलं वितन्वन्}
{श्रीमातृभूत शिव पालय मां नमस्ते} % ॥ ९५॥

\fourlineindentedshloka
{निर्धूतदुस्तिमिरमस्तसमस्ततत्तद्-}
{दुर्वासनं हृदयमेतदये वितन्वन्}
{अस्मिन् भविष्यसि कदा नु सुखं निषण्णः}
{श्रीमातृभूत शिव पालय मां नमस्ते} % ॥ ९६॥

\fourlineindentedshloka
{मद्वर्णनैस्तव कदाऽस्तु मनः प्रसादो}
{मद्दृक्पथेऽस्तु च कदा तव दिव्यमूर्तिः}
{मत्कर्णगं तव वचोऽस्तु कदेति साशं}
{श्रीमातृभूत शिव पालय मां नमस्ते} % ॥ ९७॥

\fourlineindentedshloka
{कालेऽर्कनन्दन लुलायकटुस्वनोद्य-}
{त्कर्णज्वरापनयनैर्भवदुक्षजैर्मे}
{ध्वानामृतैः किमपि शर्म विनिर्मिमाणः}
{श्रीमातृभूत शिव पालय मां नमस्ते} % ॥ ९८॥

\fourlineindentedshloka
{यं ते दया स्पृशति तन्नयनान्तदासाः}
{ब्रह्मादयो जगति दीनजनास्पदा सा}
{दीनाग्रणीरहमतः पुनरुक्तिरेषा}
{श्रीमातृभूत शिव पालय मां नमस्ते} % ॥ ९९॥

\fourlineindentedshloka
{अन्यन्न मेऽभिमतमीश तदर्प्यतां वा}
{मा वेप्सितं तु तदिदं भवताऽर्पणीयम्}
{यत्सर्वदा निभृतवृत्त्यभिधानमित्थं}
{श्रीमातृभूत शिव पालय मां नमस्ते} % ॥ १००॥

\fourlineindentedshloka
{हे शङ्कर स्मरहर प्रमथाधिनाथ}
{मन्नाथ साम्ब शशिचूड शिव त्रिशूलिन्}
{श्रीचित्सभेश करुणाकर फालनेत्र}
{श्रीमातृभूत शिव पालय मां नमस्ते} % ॥ १०१॥

\fourlineindentedshloka
{एवं स्तवं प्रपठतां उपश‍ृण्वताञ्चा-}
{रोग्यायुरच्छशिवभक्ति धनर्द्धि विद्याः}
{दद्याः सुतानपि यशश्च विमुक्तिमन्ते}
{श्रीमातृभूत शिव पालय मां नमस्ते} % ॥ १०२॥

\fourlineindentedshloka
{श्रीमातृभूत शिव चन्दनसान्द्रलिप्तां}
{सङ्क्ऌप्तहेमतिलकां श्रितपुष्पमाल्याम्}
{अक्ष्णाऽमुना तव तनुं पिबतोऽक्ष्युदन्या}
{स्यान्मे यथाशु फलिनीश तथा दयेथाः} % ॥ १०३॥

\fourlineindentedshloka
{मनो मे विक्रीतं तव चरणसेवां कलयितुं}
{कृपामूल्यं लब्धं सुरभिचिकुरे सम्प्रति मया}
{न जाने तद्दुष्टं निवसति च वा धावति च वा}
{कृतं भूयो भूयः करतलयुगास्फालनमुमे} % ॥ १०४॥

॥इति श्रीश्रीधरवेङ्कटेशार्यविरचितं मातृभूतशतकं सम्पूर्णम्।