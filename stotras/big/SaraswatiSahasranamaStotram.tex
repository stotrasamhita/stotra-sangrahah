% !TeX program = XeLaTeX
% !TeX root = ../../shloka.tex
\sect{सरस्वती-सहस्रनाम-स्तोत्रम्}

\dnsub{ध्यानम्}
\fourlineindentedshloka*
{श्रीमच्चन्दनचर्चितोज्ज्वलवपुः शुक्लाम्बरा मल्लिका-}
{मालालालित-कुन्तला प्रविलसन्मुक्तावलीशोभना}
{सर्वज्ञाननिधानपुस्तकधरा रुद्राक्षमालाङ्किता}
{वाग्देवी वदनाम्बुजे वसतु मे त्रैलोक्यमाता शुभा}

\uvacha{श्री-नारद उवाच}

\twolineshloka
{भगवन् परमेशान सर्वलोकैकनायक}
{कथं सरस्वती साक्षात्प्रसन्ना परमेष्ठिनः}%२

\twolineshloka
{कथं देव्या महावाण्याः सतत्प्राप सुदुर्लभम्}
{एतन्मे वद तत्त्वेन महायोगीश्वरप्रभो}%३


\uvacha{श्री-सनत्कुमार उवाच}

\twolineshloka
{साधु पृष्टं त्वया ब्रह्मन् गुह्याद्गुह्यमनुत्तमम्}
{भयानुगोपितं यत्नादिदानीं सत्प्रकाश्यते}%४

\twolineshloka
{पुरा पितामहं दृष्ट्वा जगत्स्थावरजङ्गमम्}
{निर्विकारं निराभासं स्तम्भीभूतमचेतसम्}%५

\twolineshloka
{सृष्ट्वा त्रैलोक्यमखिलं वागभावात्तथाविधम्}
{आधिक्याभावतः स्वस्य परमेष्ठी जगद्गुरुः}%६

\twolineshloka
{दिव्यवर्षायुतं तेन तपो दुष्करमुत्तमम्}
{ततः कदाचित्सञ्जाता वाणी सर्वार्थशोभिता}%७

\twolineshloka
{अहमस्मि महाविद्या सर्ववाचामधीश्वरी}
{मम नाम्नां सहस्रं तु उपदेक्ष्याम्यनुत्तमम्}%८

\twolineshloka
{अनेन संस्तुता नित्यं पत्नी तव भवाम्यहम्}
{त्वया सृष्टं जगत्सर्वं वाणीयुक्तं भविष्यति}%९

\twolineshloka
{इदं रहस्यं परमं मम नामसहस्रकम्}
{सर्वपापौघशमनं महासारस्वतप्रदम्}%१०

\twolineshloka
{महाकवित्वदं लोके वागीशत्वप्रदायकम्}
{त्वं वा परः पुमान्यस्तु स्तवेनानेन तोषयेत्}%११

\twolineshloka
{तस्याहं किङ्करी साक्षाद्भविष्यामि न संशयः}
{इत्युक्त्वाऽन्तर्दधे वाणी तदारभ्य पितामहः}%१२

\twolineshloka
{स्तुत्वा स्तोत्रेण दिव्येन तत्पतित्वमवाप्तवान्}
{वाणीयुक्तं जगत्सर्वं तदारभ्याभवन्मुने}%१३

\twolineshloka
{तत्तेहं सम्प्रवक्ष्यामि शृणु यत्नेन नारद}
{सावधानमना भूत्वा क्षणं शुद्धो मुनीश्वरः}%१४




\dnsub{स्तोत्रम्} \resetShloka


\twolineshloka
{वाग्वाणी वरदा वन्द्या वरारोहा वरप्रदा}
{वृत्तिर्वागीश्वरी वार्ता वरा वागीशवल्लभा}%१

\twolineshloka
{विश्वेश्वरी विश्ववन्द्या विश्वेशप्रियकारिणी}
{वाग्वादिनी च वाग्देवी वृद्धिदा वृद्धिकारिणी}%२

\twolineshloka
{वृद्धिर्वृद्धा विषघ्नी च वृष्टिर्वृष्टिप्रदायिनी}
{विश्वाराध्या विश्वमाता विश्वधात्री विनायका}%३

\twolineshloka
{विश्वशक्तिर्विश्वसारा विश्वा विश्वविभावरी}
{वेदान्तवेदिनी वेद्या वित्ता वेदत्रयात्मिका}%४

\twolineshloka
{वेदज्ञा वेदजननी विश्वा विश्वविभावरी}
{वरेण्या वाङ्मयी वृद्धा विशिष्टप्रियकारिणी}%५

\twolineshloka
{विश्वतोवदना व्याप्ता व्यापिनी व्यापकात्मिका}
{व्यालघ्नी व्यालभूषाङ्गी विरजा वेदनायिका}%६

\twolineshloka
{वेदवेदान्तसंवेद्या वेदान्तज्ञानरूपिणी}
{विभावरी च विक्रान्ता विश्वामित्रा विधिप्रिया}%७

\twolineshloka
{वरिष्ठा विप्रकृष्टा च विप्रवर्यप्रपूजिता}
{वेदरूपा वेदमयी वेदमूर्तिश्च वल्लभा}%८

\twolineshloka
{गौरी गुणवती गोप्या गन्धर्वनगरप्रिया}
{गुणमाता गुहान्तस्था गुरुरूपा गुरुप्रिया}%९

\twolineshloka
{गिरिविद्या गानतुष्टा गायकप्रियकारिणी}
{गायत्री गिरिशाराध्या गीर्गिरीशप्रियङ्करी}%१०

\twolineshloka
{गिरिज्ञा ज्ञानविद्या च गिरिरूपा गिरीश्वरी}
{गीर्माता गणसंस्तुत्या गणनीयगुणान्विता}%११

\twolineshloka
{गूढरूपा गुहा गोप्या गोरूपा गौर्गुणात्मिका}
{गुर्वी गुर्वम्बिका गुह्या गेयजा ग्रहनाशिनी}%१२

\twolineshloka
{गृहिणी गृहदोषघ्नी गवघ्नी गुरुवत्सला}
{गृहात्मिका गृहाराध्या गृहबाधाविनाशिनी}%१३

\twolineshloka
{गङ्गा गिरिसुता गम्या गजयाना गुहस्तुता}
{गरुडासनसंसेव्या गोमती गुणशालिनी}%१४

\twolineshloka
{शारदा शाश्वती शैवी शाङ्करी शङ्करात्मिका}
{श्रीः शर्वाणी शतघ्नी च शरच्चन्द्रनिभानना}%१५

\twolineshloka
{शर्मिष्ठा शमनघ्नी च शतसाहस्ररूपिणी}
{शिवा शम्भुप्रिया श्रद्धा श्रुतिरूपा श्रुतिप्रिया}%१६

\twolineshloka
{शुचिष्मती शर्मकरी शुद्धिदा शुद्धिरूपिणी}
{शिवा शिवङ्करी शुद्धा शिवाराध्या शिवात्मिका}%१७

\twolineshloka
{श्रीमती श्रीमयी श्राव्या श्रुतिः श्रवणगोचरा}
{शान्तिः शान्तिकरी शान्ता शान्ताचारप्रियङ्करी}%१८

\twolineshloka
{शीललभ्या शीलवती श्रीमाता शुभकारिणी}
{शुभवाणी शुद्धविद्या शुद्धचित्तप्रपूजिता}%१९

\twolineshloka
{श्रीकरी श्रुतपापघ्नी शुभाक्षी शुचिवल्लभा}
{शिवेतरघ्नी शबरी श्रवणीयगुणान्विता}%२० %शर्वरी?

\twolineshloka
{शारी शिरीषपुष्पाभा शमनिष्ठा शमात्मिका}
{शमान्विता शमाराध्या शितिकण्ठप्रपूजिता}%२१

\twolineshloka
{शुद्धिः शुद्धिकरी श्रेष्ठा श्रुतानन्ता शुभावहा}
{सरस्वती च सर्वज्ञा सर्वसिद्धिप्रदायिनी}%२२

\twolineshloka
{सरस्वती च सावित्री सन्ध्या सर्वेप्सितप्रदा}
{सर्वार्तिघ्नी सर्वमयी सर्वविद्याप्रदायिनी}%२३

\twolineshloka
{सर्वेश्वरी सर्वपुण्या सर्गस्थित्यन्तकारिणी}
{सर्वाराध्या सर्वमाता सर्वदेवनिषेविता}%२४

\twolineshloka
{सर्वैश्वर्यप्रदा सत्या सती सत्वगुणाश्रया}
{स्वरक्रमपदाकारा सर्वदोषनिषूदिनी}%२५

\twolineshloka
{सहस्राक्षी सहस्रास्या सहस्रपदसंयुता}
{सहस्रहस्ता साहस्रगुणालङ्कृतविग्रहा}%२६

\twolineshloka
{सहस्रशीर्षा सद्रूपा स्वधा स्वाहा सुधामयी}
{षड्ग्रन्थिभेदिनी सेव्या सर्वलोकैकपूजिता}%२७

\twolineshloka
{स्तुत्या स्तुतिमयी साध्या सवितृप्रियकारिणी}
{संशयच्छेदिनी साङ्ख्यवेद्या सङ्ख्या सदीश्वरी}%२८

\twolineshloka
{सिद्धिदा सिद्धसम्पूज्या सर्वसिद्धिप्रदायिनी}
{सर्वज्ञा सर्वशक्तिश्च सर्वसम्पत्प्रदायिनी}%२९

\twolineshloka
{सर्वाशुभघ्नी सुखदा सुखा संवित्स्वरूपिणी}
{सर्वसम्भीषणी सर्वजगत्सम्मोहिनी तथा}%३०

\twolineshloka
{सर्वप्रियङ्करी सर्वशुभदा सर्वमङ्गला}
{सर्वमन्त्रमयी सर्वतीर्थपुण्यफलप्रदा}%३१

\twolineshloka
{सर्वपुण्यमयी सर्वव्याधिघ्नी सर्वकामदा}
{सर्वविघ्नहरी सर्ववन्दिता सर्वमङ्गला}%३२

\twolineshloka
{सर्वमन्त्रकरी सर्वलक्ष्मीः सर्वगुणान्विता}
{सर्वानन्दमयी सर्वज्ञानदा सत्यनायिका}%३३

\twolineshloka
{सर्वज्ञानमयी सर्वराज्यदा सर्वमुक्तिदा}
{सुप्रभा सर्वदा सर्वा सर्वलोकवशङ्करी}%३४

\twolineshloka
{सुभगा सुन्दरी सिद्धा सिद्धाम्बा सिद्धमातृका}
{सिद्धमाता सिद्धविद्या सिद्धेशी सिद्धरूपिणी}%३५

\twolineshloka
{सुरूपिणी सुखमयी सेवकप्रियकारिणी}
{स्वामिनी सर्वदा सेव्या स्थूलसूक्ष्मापराम्बिका}%३६

\twolineshloka
{साररूपा सरोरूपा सत्यभूता समाश्रया}
{सितासिता सरोजाक्षी सरोजासनवल्लभा}%३७

\twolineshloka
{सरोरुहाभा सर्वाङ्गी सुरेन्द्रादिप्रपूजिता}
{महादेवी महेशानी महासारस्वतप्रदा}%३८

\twolineshloka
{महासरस्वती मुक्ता मुक्तिदा मलनाशिनी}
{महेश्वरी महानन्दा महामन्त्रमयी मही}%३९

\twolineshloka
{महालक्ष्मीर्महाविद्या माता मन्दरवासिनी}
{मन्त्रगम्या मन्त्रमाता महामन्त्रफलप्रदा}%४०

\twolineshloka
{महामुक्तिर्महानित्या महासिद्धिप्रदायिनी}
{महासिद्धा महामाता महदाकारसंयुता}%४१

\twolineshloka
{महा महेश्वरी मूर्तिर्मोक्षदा मणिभूषणा}
{मेनका मानिनी मान्या मृत्युघ्नी मेरुरूपिणी}%४२

\twolineshloka
{मदिराक्षी मदावासा मखरूपा मखेश्वरी}
{महामोहा महामाया मातॄणां~मूर्ध्निसंस्थिता}%४३

\twolineshloka
{महापुण्या मुदावासा महासम्पत्प्रदायिनी}
{मणिपूरैकनिलया मधुरूपा महोत्कटा}%४४

\twolineshloka
{महासूक्ष्मा महाशान्ता महाशान्तिप्रदायिनी}
{मुनिस्तुता मोहहन्त्री माधवी माधवप्रिया}%४५

\twolineshloka
{मा महादेवसंस्तुत्या महिषीगणपूजिता}
{मृष्टान्नदा च माहेन्द्री महेन्द्रपददायिनी}%४६

\twolineshloka
{मतिर्मतिप्रदा मेधा मर्त्यलोकनिवासिनी}
{मुख्या महानिवासा च महाभाग्यजनाश्रिता}%४७

\twolineshloka
{महिला महिमा मृत्युहारी मेधाप्रदायिनी}
{मेध्या महावेगवती महामोक्षफलप्रदा}%४८

\twolineshloka
{महाप्रभाभा महती महादेवप्रियङ्करी}
{महापोषा महर्द्धिश्च मुक्ताहारविभूषणा}%४९

\twolineshloka
{माणिक्यभूषणा मन्त्रा मुख्यचन्द्रार्धशेखरा}
{मनोरूपा मनःशुद्धिर्मनःशुद्धिप्रदायिनी}%५०

\twolineshloka
{महाकारुण्यसम्पूर्णा मनोनमनवन्दिता}
{महापातकजालघ्नी मुक्तिदा मुक्तभूषणा}%५१

\twolineshloka
{मनोन्मनी महास्थूला महाक्रतुफलप्रदा}
{महापुण्यफलप्राप्या मायात्रिपुरनाशिनी}%५२

\twolineshloka
{महानसा महामेधा महामोदा महेश्वरी}
{मालाधरी महोपाया महातीर्थफलप्रदा}%५३

\twolineshloka
{महामङ्गलसम्पूर्णा महादारिद्र्यनाशिनी}
{महामखा महामेघा महाकाली महाप्रिया}%५४

\twolineshloka
{महाभूषा महादेहा महाराज्ञी मुदालया}
{भूरिदा भाग्यदा भोग्या भोग्यदा भोगदायिनी}%५५

\twolineshloka
{भवानी भूतिदा भूतिर्भूमिर्भूमिसुनायिका}
{भूतधात्री भयहरी भक्तसारस्वतप्रदा}%५६

\twolineshloka
{भुक्तिर्भुक्तिप्रदा भेकी भक्तिर्भक्तिप्रदायिनी}
{भक्तसायुज्यदा भक्तस्वर्गदा भक्तराज्यदा}%५७

\twolineshloka
{भागीरथी भवाराध्या भाग्यासज्जनपूजिता}
{भवस्तुत्या भानुमती भवसागरतारणी}%५८

\twolineshloka
{भूतिर्भूषा च भूतेशी भाललोचनपूजिता}
{भूता भव्या भविष्या च भवविद्या भवात्मिका}%५९

\twolineshloka
{बाधापहारिणी बन्धुरूपा भुवनपूजिता}
{भवघ्नी भक्तिलभ्या च भक्तरक्षणतत्परा}%६०

\twolineshloka
{भक्तार्तिशमनी भाग्या भोगदानकृतोद्यमा}
{भुजङ्गभूषणा भीमा भीमाक्षी भीमरूपिणी}%६१

\twolineshloka
{भाविनी भ्रातृरूपा च भारती भवनायिका}
{भाषा भाषावती भीष्मा भैरवी भैरवप्रिया}%६२

\twolineshloka
{भूतिर्भासितसर्वाङ्गी भूतिदा भूतिनायिका}
{भास्वती भगमाला च भिक्षादानकृतोद्यमा}%६३

\twolineshloka
{भिक्षुरूपा भक्तिकरी भक्तलक्ष्मीप्रदायिनी}
{भ्रान्तिघ्ना भ्रान्तिरूपा च भूतिदा भूतिकारिणी}%६४

\twolineshloka
{भिक्षणीया भिक्षुमाता भाग्यवद्दृष्टिगोचरा}
{भोगवती भोगरूपा भोगमोक्षफलप्रदा}%६५

\twolineshloka
{भोगश्रान्ता भाग्यवती भक्ताघौघविनाशिनी}
{ब्राह्मी ब्रह्मस्वरूपा च बृहती ब्रह्मवल्लभा}%६६

\twolineshloka
{ब्रह्मदा ब्रह्ममाता च ब्रह्माणी ब्रह्मदायिनी}
{ब्रह्मेशी ब्रह्मसंस्तुत्या ब्रह्मवेद्या बुधप्रिया}%६७

\twolineshloka
{बालेन्दुशेखरा बाला बलिपूजाकरप्रिया}
{बलदा बिन्दुरूपा च बालसूर्यसमप्रभा}%६८

\twolineshloka
{ब्रह्मरूपा ब्रह्ममयी ब्रध्नमण्डलमध्यगा}
{ब्रह्माणी बुद्धिदा बुद्धिर्बुद्धिरूपा बुधेश्वरी}%६९

\twolineshloka
{बन्धक्षयकरी बाधनाशनी बन्धुरूपिणी}
{बिन्द्वालया बिन्दुभूषा बिन्दुनादसमन्विता}%७०

\twolineshloka
{बीजरूपा बीजमाता ब्रह्मण्या ब्रह्मकारिणी}
{बहुरूपा बलवती ब्रह्मजा ब्रह्मचारिणी}%७१

\twolineshloka
{ब्रह्मस्तुत्या ब्रह्मविद्या ब्रह्माण्डाधिपवल्लभा}
{ब्रह्मेशविष्णुरूपा च ब्रह्मविष्ण्वीशसंस्थिता}%७२

\twolineshloka
{बुद्धिरूपा बुधेशानी बन्धी बन्धविमोचनी}
{अक्षमालाऽक्षराकाराऽक्षराऽक्षरफलप्रदा}%७३ %फलदा?

\twolineshloka
{अनन्ताऽऽनन्दसुखदाऽनन्तचन्द्रनिभानना}
{अनन्तमहिमाऽघोराऽनन्तगम्भीरसम्मिता}%७४

\twolineshloka
{अदृष्टाऽदृष्टदाऽनन्ताऽदृष्टभाग्यफलप्रदा}
{अरुन्धत्यव्ययीनाथाऽनेकसद्गुणसंयुता}%७५

\twolineshloka
{अनेकभूषणाऽदृश्याऽनेकलेखनिषेविता}
{अनन्ताऽनन्तसुखदाऽघोराऽघोरस्वरूपिणी}%७६

\twolineshloka
{अशेषदेवतारूपाऽमृतरूपाऽमृतेश्वरी}
{अनवद्याऽनेकहस्ताऽनेकमाणिक्यभूषणा}%७७

\twolineshloka
{अनेकविघ्नसंहर्त्री ह्यनेकाभरणान्विता}
{अविद्याऽज्ञानसंहर्त्री ह्यविद्याजालनाशिनी}%७८

\twolineshloka
{अभिरूपाऽनवद्याङ्गी ह्यप्रतर्क्यगतिप्रदा}
{अकलङ्कारूपिणी च ह्यनुग्रहपरायणा}%७९

\twolineshloka
{अम्बरस्थाऽम्बरमयाऽम्बरमालाऽम्बुजेक्षणा}
{अम्बिकाऽब्जकराऽब्जस्थांऽशुमत्यंशुशतान्विता}%८०

\twolineshloka
{अम्बुजाऽनवराऽखण्डाऽम्बुजासनमहाप्रिया}
{अजरामरसंसेव्याऽजरसेवितपद्युगा}%८१

\twolineshloka
{अतुलार्थप्रदाऽर्थैक्याऽत्युदारा त्वभयान्विता}
{अनाथवत्सलाऽनन्तप्रियाऽनन्तेप्सितप्रदा}%८२

\twolineshloka
{अम्बुजाक्ष्यम्बुरूपाऽम्बुजातोद्भवमहाप्रिया}
{अखण्डा त्वमरस्तुत्याऽमरनायकपूजिता}%८३

\twolineshloka
{अजेया त्वजसङ्काशाऽज्ञाननाशिन्यभीष्टदा}
{अक्ताऽघनेना चास्त्रेशी ह्यलक्ष्मीनाशिनी तथा}%८४

\twolineshloka
{अनन्तसाराऽनन्तश्रीरनन्तविधिपूजिता}
{अभीष्टाऽमर्त्यसम्पूज्या ह्यस्तोदयविवर्जिता}%८५

\twolineshloka
{आस्तिकस्वान्तनिलयाऽस्त्ररूपाऽस्त्रवती तथा}
{अस्खलत्यस्खलद्रूपाऽस्खलद्विद्याप्रदायिनी}%८६

\twolineshloka
{अस्खलत्सिद्धिदाऽऽनन्दाऽम्बुजाताऽमरनायिका}
{अमेयाऽशेषपापघ्न्यक्षयसारस्वतप्रदा}%८७

\twolineshloka
{जया जयन्ती जयदा जन्मकर्मविवर्जिता}
{जगत्प्रिया जगन्माता जगदीश्वरवल्लभा}%८८

\twolineshloka
{जातिर्जया जितामित्रा जप्या जपनकारिणी}
{जीवनी जीवनिलया जीवाख्या जीवधारिणी}%८९

\twolineshloka
{जाह्नवी ज्या जपवती जातिरूपा जयप्रदा}
{जनार्दनप्रियकरी जोषनीया जगत्स्थिता}%९०

\twolineshloka
{जगज्ज्येष्ठा जगन्माया जीवनत्राणकारिणी}
{जीवातुलतिका जीवजन्मी जन्मनिबर्हणी}%९१

\twolineshloka
{जाड्यविध्वंसनकरी जगद्योनिर्जयात्मिका}
{जगदानन्दजननी जम्बूश्च जलजेक्षणा}%९२

\twolineshloka
{जयन्ती जङ्गपूगघ्नी जनितज्ञानविग्रहा}
{जटा जटावती जप्या जपकर्तृप्रियङ्करी}%९३

\twolineshloka
{जपकृत्पापसंहर्त्री जपकृत्फलदायिनी}
{जपापुष्पसमप्रख्या जपाकुसुमधारिणी}%९४

\twolineshloka
{जननी जन्मरहिता ज्योतिर्वृत्यभिदायिनी}
{जटाजूटनचन्द्रार्धा जगत्सृष्टिकरी तथा}%९५

\twolineshloka
{जगत्त्राणकरी जाड्यध्वंसकर्त्री जयेश्वरी}
{जगद्बीजा जयावासा जन्मभूर्जन्मनाशिनी}%९६

\twolineshloka
{जन्मान्त्यरहिता जैत्री जगद्योनिर्जपात्मिका}
{जयलक्षणसम्पूर्णा जयदानकृतोद्यमा}%९७

\twolineshloka
{जम्भराद्यादिसंस्तुत्या जम्भारिफलदायिनी}
{जगत्त्रयहिता ज्येष्ठा जगत्त्रयवशङ्करी}%९८

\twolineshloka
{जगत्त्रयाम्बा जगती ज्वाला ज्वालितलोचना}
{ज्वालिनी ज्वलनाभासा ज्वलन्ती ज्वलनात्मिका}%९९

\twolineshloka
{जितारातिसुरस्तुत्या जितक्रोधा जितेन्द्रिया}
{जरामरणशून्या च जनित्री जन्मनाशिनी}%१००

\twolineshloka
{जलजाभा जलमयी जलजासनवल्लभा}
{जलजस्था जपाराध्या जनमङ्गलकारिणी}%१०१

\twolineshloka
{कामिनी कामरूपा च काम्या कामप्रदायिनी}
{कमाली कामदा कर्त्री क्रतुकर्मफलप्रदा}%१०२

\twolineshloka
{कृतघ्नघ्नी क्रियारूपा कार्यकारणरूपिणी}
{कञ्जाक्षी करुणारूपा केवलामरसेविता}%१०३

\twolineshloka
{कल्याणकारिणी कान्ता कान्तिदा कान्तिरूपिणी}
{कमला कमलावासा कमलोत्पलमालिनी}%१०४

\twolineshloka
{कुमुद्वती च कल्याणी कान्तिः कामेशवल्लभा}
{कामेश्वरी कमलिनी कामदा कामबन्धिनी}%१०५

\twolineshloka
{कामधेनुः काञ्चनाक्षी काञ्चनाभा कलानिधिः}
{क्रिया कीर्तिकरी कीर्तिः क्रतुश्रेष्ठा कृतेश्वरी}%१०६

\twolineshloka
{क्रतुसर्वक्रियास्तुत्या क्रतुकृत्प्रियकारिणी}
{क्लेशनाशकरी कर्त्री कर्मदा कर्मबन्धिनी}%१०७

\twolineshloka
{कर्मबन्धहरी कृष्टा क्लमघ्नी कञ्जलोचना}
{कन्दर्पजननी कान्ता करुणा करुणावती}%१०८

\twolineshloka
{क्लीङ्कारिणी कृपाकारा कृपासिन्धुः कृपावती}
{करुणार्द्रा कीर्तिकरी कल्मषघ्नी क्रियाकरी}%१०९

\twolineshloka
{क्रियाशक्तिः कामरूपा कमलोत्पलगन्धिनी}
{कला कलावती कूर्मी कूटस्था कञ्जसंस्थिता}%११०

\twolineshloka
{कालिका कल्मषघ्नी च कमनीयजटान्विता}
{करपद्मा कराभीष्टप्रदा क्रतुफलप्रदा}%१११

\twolineshloka
{कौशिकी कोशदा काव्या कर्त्री कोशेश्वरी कृशा}
{कूर्मयाना कल्पलता कालकूटविनाशिनी}%११२

\twolineshloka
{कल्पोद्यानवती कल्पवनस्था कल्पकारिणी}
{कदम्बकुसुमाभासा कदम्बकुसुमप्रिया}%११३

\twolineshloka
{कदम्बोद्यानमध्यस्था कीर्तिदा कीर्तिभूषणा}
{कुलमाता कुलावासा कुलाचारप्रियङ्करी}%११४

\twolineshloka
{कुलानाथा कामकला कलानाथा कलेश्वरी}
{कुन्दमन्दारपुष्पाभा कपर्दस्थितचन्द्रिका}%११५

\twolineshloka
{कवित्वदा काव्यमाता कविमाता कलाप्रदा}
{तरुणी तरुणीताता ताराधिपसमानना}%११६

\twolineshloka
{तृप्तिस्तृप्तिप्रदा तर्क्या तपनी तापिनी तथा}
{तर्पणी तीर्थरूपा च त्रिदशा त्रिदशेश्वरी}%११७

\twolineshloka
{त्रिदिवेशी त्रिजननी त्रिमाता त्र्यम्बकेश्वरी}
{त्रिपुरा त्रिपुरेशानी त्र्यम्बका त्रिपुराम्बिका}%११८

\twolineshloka
{त्रिपुरश्रीस्त्रयीरूपा त्रयीवेद्या त्रयीश्वरी}
{त्रय्यन्तवेदिनी ताम्रा तापत्रितयहारिणी}%११९

\twolineshloka
{तमालसदृशी त्राता तरुणादित्यसन्निभा}
{त्रैलोक्यव्यापिनी तृप्ता तृप्तिकृत्तत्त्वरूपिणी}%१२०

\twolineshloka
{तुर्या त्रैलोक्यसंस्तुत्या त्रिगुणा त्रिगुणेश्वरी}
{त्रिपुरघ्नी त्रिमाता च त्र्यम्बका त्रिगुणान्विता}%१२१

\twolineshloka
{तृष्णाच्छेदकरी तृप्ता तीक्ष्णा तीक्ष्णस्वरूपिणी}
{तुला तुलादिरहिता तत्तद्ब्रह्मस्वरूपिणी}%१२२

\twolineshloka
{त्राणकर्त्री त्रिपापघ्नी त्रिपदा त्रिदशान्विता}
{तथ्या त्रिशक्तिस्त्रिपदा तुर्या त्रैलोक्यसुन्दरी}%१२३

\twolineshloka
{तेजस्करी त्रिमूर्त्याद्या तेजोरूपा त्रिधामता}
{त्रिचक्रकर्त्री त्रिभगा तुर्यातीतफलप्रदा}%१२४

\twolineshloka
{तेजस्विनी तापहारी तापोपप्लवनाशिनी}
{तेजोगर्भा तपःसारा त्रिपुरारिप्रियङ्करी}%१२५

\twolineshloka
{तन्वी तापससन्तुष्टा तपताङ्गजभीतिनुत्}
{त्रिलोचना त्रिमार्गा च तृतीया त्रिदशस्तुता}%१२६

\twolineshloka
{त्रिसुन्दरी त्रिपथगा तुरीयपददायिनी}
{शुभा शुभावती शान्ता शान्तिदा शुभदायिनी}%१२७

\twolineshloka
{शीतला शूलिनी शीता श्रीमती च शुभान्विता}
{योगसिद्धिप्रदा योग्या यज्ञेनपरिपूरिता}%१२८

\twolineshloka
{यज्या यज्ञमयी यक्षी यक्षिणी यक्षिवल्लभा}
{यज्ञप्रिया यज्ञपूज्या यज्ञतुष्टा यमस्तुता}%१२९

\twolineshloka
{यामिनीयप्रभा याम्या यजनीया यशस्करी}
{यज्ञकर्त्री यज्ञरूपा यशोदा यज्ञसंस्तुता}%१३०

\twolineshloka
{यज्ञेशी यज्ञफलदा योगयोनिर्यजुस्तुता}
{यमिसेव्या यमाराध्या यमिपूज्या यमीश्वरी}%१३१

\twolineshloka
{योगिनी योगरूपा च योगकर्तृप्रियङ्करी}
{योगयुक्ता योगमयी योगयोगीश्वराम्बिका}%१३२

\twolineshloka
{योगज्ञानमयी योनिर्यमाद्यष्टाङ्गयोगता}
{यन्त्रिताघौघसंहारा यमलोकनिवारिणी}%१३३

\twolineshloka
{यष्टिव्यष्टीशसंस्तुत्या यमाद्यष्टाङ्गयोगयुक्}
{योगीश्वरी योगमाता योगसिद्धा च योगदा}%१३४

\twolineshloka
{योगारूढा योगमयी योगरूपा यवीयसी}
{यन्त्ररूपा च यन्त्रस्था यन्त्रपूज्या च यन्त्रिता}%१३५

\twolineshloka
{युगकर्त्री युगमयी युगधर्मविवर्जिता}
{यमुना यमिनी याम्या यमुनाजलमध्यगा}%१३६

\twolineshloka
{यातायातप्रशमनी यातनानान्निकृन्तनी}
{योगावासा योगिवन्द्या यत्तच्छब्दस्वरूपिणी}%१३७

\twolineshloka
{योगक्षेममयी यन्त्रा यावदक्षरमातृका}
{यावत्पदमयी यावच्छब्दरूपा यथेश्वरी}%१३८

\twolineshloka
{यत्तदीया यक्षवन्द्या यद्विद्या यतिसंस्तुता}
{यावद्विद्यामयी यावद्विद्याबृन्दसुवन्दिता}%१३९

\twolineshloka
{योगिहृत्पद्मनिलया योगिवर्यप्रियङ्करी}
{योगिवन्द्या योगिमाता योगीशफलदायिनी}%१४०

\twolineshloka
{यक्षवन्द्या यक्षपूज्या यक्षराजसुपूजिता}
{यज्ञरूपा यज्ञतुष्टा यायजूकस्वरूपिणी}%१४१

\twolineshloka
{यन्त्राराध्या यन्त्रमध्या यन्त्रकर्तृप्रियङ्करी}
{यन्त्रारूढा यन्त्रपूज्या योगिध्यानपरायणा}%१४२

\twolineshloka
{यजनीया यमस्तुत्या योगयुक्ता यशस्करी}
{योगबद्धा यतिस्तुत्या योगज्ञा योगनायकी}%१४३

\twolineshloka
{योगिज्ञानप्रदा यक्षी यमबाधाविनाशिनी}
{योगिकाम्यप्रदात्री च योगिमोक्षप्रदायिनी}%१४४


\dnsub{फलश्रुतिः} \resetShloka

\twolineshloka
{इति नाम्नां सरस्वत्याः सहस्रं समुदीरितम्}
{मन्त्रात्मकं महागोप्यं महासारस्वतप्रदम्}%१

\twolineshloka
{यः पठेच्छृणुयाद्भक्त्या त्रिकालं साधकः पुमान्}
{सर्वविद्यानिधिः साक्षात् स एव भवति ध्रुवम्}%२

\twolineshloka
{लभते सम्पदः सर्वाः पुत्रपौत्रादिसंयुताः}
{मूकोऽपि सर्वविद्यासु चतुर्मुख इवापरः}%३

\twolineshloka
{भूत्वा प्राप्नोति सान्निध्यम् अन्ते धातुर्मुनीश्वर}
{सर्वमन्त्रमयं सर्वविद्यामानफलप्रदम्}%४

\twolineshloka
{महाकवित्वदं पुंसां महासिद्धिप्रदायकम्}
{कस्मैचिन्न प्रदातव्यं प्राणैः कण्ठगतैरपि}%५

\twolineshloka
{महारहस्यं सततं वाणीनामसहस्रकम्}
{सुसिद्धमस्मदादीनां स्तोत्रं ते समुदीरितम्}%६


॥इति श्रीस्कान्दपुराणान्तर्गत-सनत्कुमार-संहितायां नारद-सनत्कुमार-संवादे सरस्वती-सहस्रनाम-स्तोत्रं सम्पूर्णम्॥
