% !TeX program = XeLaTeX
% !TeX root = ../../shloka.tex

\sect{गकारादि श्री-गणपति सहस्रनाम स्तोत्रम्}

अस्य श्रीगणपतिगकारादिसहस्रनाममालामन्त्रस्य।\\
दुर्वासा ऋषिः। अनुष्टुप् छन्दः। श्री-गणपतिर्देवता।\\
गं बीजम्। स्वाहा शक्तिः। ग्लौं कीलकम् ।\\
 सकलाभीष्टसिद्ध्यर्थे जपे विनियोगः ॥

\dnsub{करन्यासः}
ओं अङ्गुष्ठाभ्यां नमः। श्रीं तर्जनीभ्यां नमः।\\
ह्रीं मध्यमाभ्यां नमः। क्रीं अनामिकाभ्यां नमः।\\
ग्लौं कनिष्ठिकाभ्यां नमः। गं करतलकरपृष्ठाभ्यां नमः।\\
एवं हृदयादिन्यासः।\\
अथवा षड्दीर्घभाजागमितिबीजेन कराङ्गन्यासः॥

\dnsub{ध्यानम्}

\fourlineindentedshloka*
{ओङ्कार-सन्निभमिभाननमिन्दुभालम्}{मुक्ताग्रबिन्दुममलद्युतिमेकदन्तम्}
{लम्बोदरं कलचतुर्भुजमादिदेवम्}{ध्यायेन्महागणपतिं मतिसिद्धिकान्तम्}

\dnsub{स्तोत्रम्}

\twolineshloka
{ओं गणेश्वरो गणाध्यक्षो गणाराध्यो गणप्रियः}
{गणनाथो गणस्वामी गणेशो गणनायकः}% १॥

\twolineshloka
{गणमूर्तिर्गणपतिर्गणत्राता गणञ्जयः}%
{गणपोऽथ गणक्रीडो गणदेवो गणाधिपः}% २॥

\twolineshloka
{गणज्येष्ठो गणश्रेष्ठो गणप्रेष्ठो गणाधिराट्}%
{गणराड् गणगोप्ताथ गणाङ्गो गणदैवतम्}% ३॥

\twolineshloka
{गणबन्धुर्गणसुहृद्गणाधीशो गणप्रथः}%
{गणप्रियसखः शश्वद्गणप्रियसुहृत् तथा}% ४॥

\twolineshloka
{गणप्रियरतो नित्यं गणप्रीतिविवर्धनः}%
{गणमण्डलमध्यस्थो गणकेलिपरायणः}% ५॥

\twolineshloka
{गणाग्रणीर्गणेशानो गणगीतो गणोच्छ्रयः}%
{गण्यो गणहितो गर्जद्गणसेनो गणोद्धतः}% ६॥

\twolineshloka
{गणभीतिप्रमथनो गणभीत्यपहारकः}%
{गणनार्हो गणप्रौढो गणभर्ता गणप्रभुः}% ७॥

\twolineshloka
{गणसेनो गणचरो गणप्रज्ञो गणैकराट्}%
{गणाग्र्यो गणनामा च गणपालनतत्परः}% ८॥

\twolineshloka
{गणजिद्गणगर्भस्थो गणप्रवणमानसः}%
{गणगर्वपरीहर्ता गणो गणनमस्कृतः}% ९॥

\twolineshloka
{गणार्चिताङ्घ्रियुगलो गणरक्षणकृत् सदा}%
{गणध्यातो गणगुरुर्गणप्रणयतत्परः}% १०॥

\twolineshloka
{गणागणपरित्राता गणाधिहरणोद्धुरः}%
{गणसेतुर्गणनुतो गणकेतुर्गणाग्रगः}% ११॥

\twolineshloka
{गणहेतुर्गणग्राही गणानुग्रहकारकः}%
{गणागणानुग्रहभूर्गणागणवरप्रदः}% १२॥

\twolineshloka
{गणस्तुतो गणप्राणो गणसर्वस्वदायकः}%
{गणवल्लभमूर्तिश्च गणभूतिर्गणेष्टदः}% १३॥

\twolineshloka
{गणसौख्यप्रदाता च गणदुःखप्रणाशनः}%
{गणप्रथितनामा च गणाभीष्टकरः सदा}% १४॥

\twolineshloka
{गणमान्यो गणख्यातो गणवीतो गणोत्कटः}%
{गणपालो गणवरो गणगौरवदायकः}% १५॥

\twolineshloka
{गणगर्जितसन्तुष्टो गणस्वच्छन्दगः सदा}%
{गणराजो गणश्रीदो गणाभयकरः क्षणात्}% १६॥

\twolineshloka
{गणमूर्धाभिषिक्तश्च गणसैन्यपुरस्सरः}%
{गुणातीतो गुणमयो गुणत्रयविभागकृत्}%  १७॥

\twolineshloka
{गुणी गुणाकृतिधरो गुणशाली गुणप्रियः}%
{गुणपूर्णो गुणाम्भोधिर्गुणभाग् गुणदूरगः}% १८॥

\twolineshloka
{गुणागुणवपुर्गौणशरीरो गुणमण्डितः}%
{गुणस्त्रष्टा गुणेशानो गुणेशोऽथ गुणेश्वरः}% १९॥

\twolineshloka
{गुणसृष्टजगत्सङ्घो गुणसङ्घो गुणैकराट्}%
{गुणप्रवृष्टो गुणभूर्गुणीकृतचराचरः}% २०॥

\twolineshloka
{गुणप्रवणसन्तुष्टो गुणहीनपराङ्मुखः}%
{गुणैकभूर्गुणश्रेष्ठो गुणज्येष्ठो गुणप्रभुः}% २१॥

\twolineshloka
{गुणज्ञो गुणसम्पूज्यो गुणैकसदनं सदा}%
{गुणप्रणयवान् गौणप्रकृतिर्गुणभाजनम्}% २२॥

\twolineshloka
{गुणिप्रणतपादाब्जो गुणिगीतो गुणोज्ज्वलः}%
{गुणवान् गुणसम्पन्नो गुणानन्दितमानसः}% २३॥

\twolineshloka
{गुणसञ्चारचतुरो गुणसञ्चयसुन्दरः}%
{गुणगौरो गुणाधारो गुणसंवृतचेतनः}% २४॥

\twolineshloka
{गुणकृद्गुणभृन्नित्यं गुणाग्र्यो गुणपारदृक्}%
{गुणप्रचारी गुणयुग् गुणागुणविवेककृत्}% २५॥

\twolineshloka
{गुणाकरो गुणकरो गुणप्रवणवर्धनः}%
{गुणगूढचरो गौणसर्वसंसारचेष्टितः}% २६॥

\twolineshloka
{गुणदक्षिणसौहार्दो गुणलक्षणतत्त्ववित्}%
{गुणहारी गुणकलो गुणसङ्घसखः सदा}% २७॥

\twolineshloka
{गुणसंस्कृतसंसारो गुणतत्त्वविवेचकः}%
{गुणगर्वधरो गौणसुखदुःखोदयो गुणः}% २८॥

\twolineshloka
{गुणाधीशो गुणलयो गुणवीक्षणलालसः}%
{गुणगौरवदाता च गुणदाता गुणप्रदः}% २९॥

\twolineshloka
{गुणकृद्गुणसम्बन्धो गुणभृद्गुणबन्धनः}%
{गुणहृद्यो गुणस्थायी गुणदायी गुणोत्कटः}% ३०॥

\twolineshloka
{गुणचक्रधरो गौणावतारो गुणबान्धवः}%
{गुणबन्धुर्गुणप्रज्ञो गुणप्राज्ञो गुणालयः}% ३१॥

\twolineshloka
{गुणधाता गुणप्राणो गुणगोपो गुणाश्रयः}%
{गुणयायी गुणाधायी गुणपो गुणपालकः}% ३२॥

\twolineshloka
{गुणाहृततनुर्गौणो गीर्वाणो गुणगौरवः}%
{गुणवत्पूजितपदो गुणवत्प्रीतिदायकः}% ३३॥

\twolineshloka
{गुणवद्गीतकीर्तिश्च गुणवद्बद्धसौहृदः}%
{गुणवद्वरदो नित्यं गुणवत्प्रतिपालकः}% ३४॥

\twolineshloka
{गुणवद्गुणसन्तुष्टो गुणवद्रचितस्तवः}%
{गुणवद्रक्षणपरो गुणवत्प्रणयप्रियः}% ३५॥

\twolineshloka
{गुणवच्चक्रसञ्चारो गुणवत्कीर्तिवर्धनः}%
{गुणवद्गुणचित्तस्थो गुणवद्गुणरक्षकः}% ३६॥

\twolineshloka
{गुणवत्पोषणकरो गुणवच्छत्रुसूदनः}%
{गुणवत्सिद्धिदाता च गुणवद्गौरवप्रदः}% ३७॥

\twolineshloka
{गुणवत्प्रवणस्वान्तो गुणवद्गुणभूषणः}%
{गुणवत्कुलविद्वेषिविनाशकरणक्षमः}% ३८॥

\twolineshloka
{गुणिस्तुतगुणो गर्जप्रलयाम्बुदनिःस्वनः}%
{गजो गजपतिर्गर्जद्गजयुद्धविशारदः}% ३९॥

\twolineshloka
{गजास्यो गजकर्णोऽथ गजराजो गजाननः}%
{गजरूपधरो गर्जद्गजयूथोद्धुरध्वनिः}% ४०॥

\twolineshloka
{गजाधीशो गजाधारो गजासुरजयोद्धुरः}%
{गजदन्तो गजवरो गजकुम्भो गजध्वनिः}% ४१॥

\twolineshloka
{गजमायो गजमयो गजश्रीर्गजगर्जितः}%
{गजामयहरो नित्यं गजपुष्टिप्रदायकः}% ४२॥

\twolineshloka
{गजोत्पत्तिर्गजत्राता गजहेतुर्गजाधिपः}%
{गजमुख्यो गजकुलप्रवरो गजदैत्यहा}% ४३॥

\twolineshloka
{गजकेतुर्गजाध्यक्षो गजसेतुर्गजाकृतिः}%
{गजवन्द्यो गजप्राणो गजसेव्यो गजप्रभुः}% ४४॥

\twolineshloka
{गजमत्तो गजेशानो गजेशो गजपुङ्गवः}%
{गजदन्तधरो गुञ्जन्मधुपो गजवेषभृत्}% ४५॥

\twolineshloka
{गजच्छन्नो गजाग्रस्थो गजयायी गजाजयः}%
{गजराड्गजयूथस्थो गजगञ्जकभञ्जकः}% ४६॥

\twolineshloka
{गर्जितोज्ञितदैत्यासुर्गर्जितत्रातविष्टपः}%
{गानज्ञो गानकुशलो गानतत्त्वविवेचकः}% ४७॥

\twolineshloka
{गानश्लाघी गानरसो गानज्ञानपरायणः}%
{गानागमज्ञो गानाङ्गो गानप्रवणचेतनः}% ४८॥

\twolineshloka
{गानकृद्गानचतुरो गानविद्याविशारदः}%
{गानध्येयो गानगम्यो गानध्यानपरायणः}% ४९॥

\twolineshloka
{गानभूर्गानशीलश्च गानशाली गतश्रमः}%
{गानविज्ञानसम्पन्नो गानश्रवणलालसः}% ५०॥

\twolineshloka
{गानयत्तो गानमयो गानप्रणयवान् सदा}%
{गानध्याता गानबुद्धिर्गानोत्सुकमनाः पुनः}% ५१॥

\twolineshloka
{गानोत्सुको गानभूमिर्गानसीमा गुणोज्ज्वलः}%
{गानङ्गज्ञानवान् गानमानवान् गानपेशलः}% ५२॥

\twolineshloka
{गानवत्प्रणयो गानसमुद्रो गानभूषणः}%
{गानसिन्धुर्गानपरो गानप्राणो गणाश्रयः}% ५३॥

\twolineshloka
{गानैकभूर्गानहृष्टो गानचक्षुर्गाणैकदृक्}%
{गानमत्तो गानरुचिर्गानविद्गानवित्प्रियः}%  ५४॥

\twolineshloka
{गानान्तरात्मा गानाढ्यो गानभ्राजत्सभः सदा}%
{गानमयो गानधरो गानविद्याविशोधकः}% ५५॥

\twolineshloka
{गानाहितघ्रो गानेन्द्रो गानलीनो गतिप्रियः}%
{गानाधीशो गानलयो गानाधारो गतीश्वरः}% ५६॥

\twolineshloka
{गानवन्मानदो गानभूतिर्गानैकभूतिमान्}%
{गानतानततो गानतानदानविमोहितः}% ५७॥

\twolineshloka
{गुरुर्गुरुदरश्रोणिर्गुरुतत्त्वार्थदर्शनः}%
{गुरुस्तुतो गुरुगुणो गुरुमायो गुरुप्रियः}% ५८॥

\twolineshloka
{गुरुकीर्तिर्गुरुभुजो गुरुवक्षा गुरुप्रभः}%
{गुरुलक्षणसम्पन्नो गुरुद्रोहपराङ्मुखः}% ५९॥

\twolineshloka
{गुरुविद्यो गुरुप्राणो गुरुबाहुबलोच्छ्रयः}%
{गुरुदैत्यप्राणहरो गुरुदैत्यापहारकः}% ६०॥

\twolineshloka
{गुरुगर्वहरो गुह्यप्रवरो गुरुदर्पहा}%
{गुरुगौरवदायी च गुरुभीत्यपहारकः}% ६१॥

\twolineshloka
{गुरुशुण्डो गुरुस्कन्धो गुरुजङ्घो गुरुप्रथः}%
{गुरुभालो गुरुगलो गुरुश्रीर्गुरुगर्वनुत्}

\twolineshloka
{गुरूरुर्गुरुपीनांसो गुरुप्रणयलालसः}%
{गुरुमुख्यो गुरुकुलस्थायी गुरुगुणः सदा}% ६३॥

\twolineshloka
{गुरुसंशयभेत्ता च गुरुमानप्रदायकः}%
{गुरुधर्मसदाराध्यो गुरुधर्मनिकेतनः}% ६४॥

\onelineshloka
{गुरुदैत्यकुलच्छेत्ता गुरुसैन्यो गुरुद्युतिः}% ६५॥

\twolineshloka
{गुरुधर्माग्रगण्योऽथ गुरुधर्मधुरन्धरः}%
{गरिष्ठो गुरुसन्तापशमनो गुरुपूजितः}% ६६॥

\twolineshloka
{गुरुधर्मधरो गौरधर्माधारो गदापहः}%
{गुरुशास्त्रविचारज्ञो गुरुशास्त्रकृतोद्यमः}% ६७॥

\twolineshloka
{गुरुशास्त्रार्थनिलयो गुरुशास्त्रालयः सदा}%
{गुरुमन्त्रो गुरुशेष्ठो गुरुमन्त्रफलप्रदः}% ६८॥

\twolineshloka
{गुरुस्त्रीगमनोद्दामप्रायश्चित्तनिवारकः}%
{गुरुसंसारसुखदो गुरुसंसारदुःखभित्}% ६९॥

\twolineshloka
{गुरुश्लाघापरो गौरभानुखण्डावतंसभृत्}%
{गुरुप्रसन्नमूर्तिश्च गुरुशापविमोचकः}% ७०॥

\twolineshloka
{गुरुकान्तिर्गुरुमयो गुरुशासनपालकः}%
{गुरुतन्त्रो गुरुप्रज्ञो गुरुभो गुरुदैवतम्}% ७१॥

\twolineshloka
{गुरुविक्रमसञ्चारो गुरुदृग्गुरुविक्रमः}%
{गुरुक्रमो गुरुप्रेष्ठो गुरुपाखण्डखण्डकः}% ७२॥

\twolineshloka
{गुरुगर्जितसम्पूर्णब्रह्माण्डो गुरुगर्जितः}%
{गुरुपुत्रप्रियसखो गुरुपुत्रभयापहः}% ७३॥

\twolineshloka
{गुरुपुत्रपरित्राता गुरुपुत्रवरप्रदः}%
{गुरुपुत्रार्तिशमनो गुरुपुत्राधिनाशनः}% ७४॥

\twolineshloka
{गुरुपुत्रप्राणदाता गुरुभक्तिपरायणः}%
{गुरुविज्ञानविभवो गौरभानुवरप्रदः}% ७५॥

\twolineshloka
{गौरभानुस्तुतो गौरभानुत्रासापहारकः}%
{गौरभानुप्रियो गौरभानुर्गौरववर्धनः}% ७६॥

\twolineshloka
{गौरभानुपरित्राता गौरभानुसखः सदा}%
{गौरभानुर्प्रभुर्गौरभानुभीतिप्रणाशनः}% ७७॥

\twolineshloka
{गौरीतेजःसमुत्पन्नो गौरीहृदयनन्दनः}%
{गौरीस्तनन्धयो गौरीमनोवाञ्छितसिद्धिकृत्}% ७८॥

\twolineshloka
{गौरो गौरगुणो गौरप्रकाशो गौरभैरवः}%
{गौरीशनन्दनो गौरीप्रियपुत्रो गदाधरः}% ७९॥

\twolineshloka
{गौरीवरप्रदो गौरीप्रणयो गौरसच्छविः}%
{गौरीगणेश्वरो गौरीप्रवणो गौरभावनः}% ८०॥

\twolineshloka
{गौरात्मा गौरकीर्तिश्च गौरभावो गरिष्ठदृक्}%
{गौतमो गौतमीनाथो गौतमीप्राणवल्लभः}% ८१॥

\twolineshloka
{गौतमाभीष्टवरदो गौतमाभयदायकः}%
{गौतमप्रणयप्रह्वो गौतमाश्रमदुःखहा}% ८२॥

\twolineshloka
{गौतमीतीरसञ्चारी गौतमीतीर्थनायकः}%
{गौतमापत्परिहारो गौतमाधिविनाशनः}% ८३॥

\twolineshloka
{गोपतिर्गोधनो गोपो गोपालप्रियदर्शनः}%
{गोपालो गोगणाधीशो गोकश्मलनिवर्तकः}% ८४॥

\twolineshloka
{गोसहस्रो गोपवरो गोपगोपीसुखावहः}%
{गोवर्धनो गोपगोपो गोपो गोकुलवर्धनः}% ८५॥

\twolineshloka
{गोचरो गोचराध्यक्षो गोचरप्रीतिवृद्धिकृत्}%
{गोमी गोकष्टसन्त्राता गोसन्तापनिवर्तकः}% ८६॥

\twolineshloka
{गोष्ठो गोष्ठाश्रयो गोष्ठपतिर्गोधनवर्धनः}%
{गोष्ठप्रियो गोष्ठमयो गोष्ठामयनिवर्तकः}% ८७॥

\twolineshloka
{गोलोको गोलको गोभृद्गोभर्ता गोसुखावहः}%
{गोधुग्गोधुग्गणप्रेष्ठो गोदोग्धा गोमयप्रियः}% ८८॥

\twolineshloka
{गोत्रं गोत्रपतिर्गोत्रप्रभुर्गोत्रभयापहः}%
{गोत्रवृद्धिकरो गोत्रप्रियो गोत्रार्तिनाशनः}% ८९॥

\twolineshloka
{गोत्रोद्धारपरो गोत्रप्रवरो गोत्रदैवतम्}%
{गोत्रविख्यातनामा च गोत्री गोत्रप्रपालकः}% ९०॥

\twolineshloka
{गोत्रसेतुर्गोत्रकेतुर्गोत्रहेतुर्गतक्लमः}%
{गोत्रत्राणकरो गोत्रपतिर्गोत्रेशपूजितः}% ९१॥

\twolineshloka
{गोत्रभिद्गोत्रभित्त्राता गोत्रभिद्वरदायकः}%
{गोत्रभित्पूजितपदो गोत्रभिच्छत्रुसूदनः}% ९२॥

\twolineshloka
{गोत्रभित्प्रीतिदो नित्यं गोत्रभिद्गोत्रपालकः}%
{गोत्रभिद्गीतचरितो गोत्रभिद्राज्यरक्षकः}% ९३॥

\twolineshloka
{गोत्रभिज्जयदायी च गोत्रभित्प्रणयः सदा}%
{गोत्रभिद्भयसम्भेत्ता गोत्रभिन्मानदायकः}% ९४॥

\twolineshloka
{गोत्रभिद्गोपनपरो गोत्रभित्सैन्यनायकः}%
{गोत्राधिपप्रियो गोत्रपुत्रीपुत्रो गिरिप्रियः}% ९५॥

\twolineshloka
{ग्रन्थज्ञो ग्रन्थकृद्ग्रन्थग्रन्थिभिद्ग्रन्थविघ्नहा}%
{ग्रन्थादिर्ग्रन्थसञ्चारो ग्रन्थश्रवणलोलुपः}% ९६॥

\twolineshloka
{ग्रन्थादीनक्रियो ग्रन्थप्रियो ग्रन्थार्थतत्त्ववित्}%
{ग्रन्थसंशयसञ्छेदी ग्रन्थवक्ता ग्रहाग्रणीः}% ९७॥

\twolineshloka
{ग्रन्थगीतगुणो ग्रन्थगीतो ग्रन्थादिपूजितः}%
{ग्रन्थारम्भस्तुतो ग्रन्थग्राही ग्रन्थार्थपारदृक्}% ९८॥

\twolineshloka
{ग्रन्थदृग्ग्रन्थविज्ञानो ग्रन्थसन्दर्भषोधकः}%
{ग्रन्थकृत्पूजितो ग्रन्थकरो ग्रन्थपरायणः}% ९९॥

\twolineshloka
{ग्रन्थपारायणपरो ग्रन्थसन्देहभञ्जकः}%
{ग्रन्थकृद्वरदाता च ग्रन्थकृद्वन्दितः सदा}% १००॥

\twolineshloka
{ग्रन्थानुरक्तो ग्रन्थज्ञो ग्रन्थानुग्रहदायकः}%
{ग्रन्थान्तरात्मा ग्रन्थार्थपण्डितो ग्रन्थसौहृदः}% १०१॥

\twolineshloka
{ग्रन्थपारङ्गमो ग्रन्थगुणविद्ग्रन्थविग्रहः}%
{ग्रन्थसेतुर्ग्रन्थहेतुर्ग्रन्थकेतुर्ग्रहाग्रगः}% १०२॥

\twolineshloka
{ग्रन्थपूज्यो ग्रन्थगेयो ग्रन्थग्रथनलालसः}%
{ग्रन्थभूमिर्ग्रहश्रेष्ठो ग्रहकेतुर्ग्रहाश्रयः}% १०३॥

\twolineshloka
{ग्रन्थकारो ग्रन्थकारमान्यो ग्रन्थप्रसारकः}%
{ग्रन्थश्रमज्ञो ग्रन्थाङ्गो ग्रन्थभ्रमनिवारकः}% १०४॥

\twolineshloka
{ग्रन्थप्रवणसर्वाङ्गो ग्रन्थप्रणयतत्परः}%
{गीतं गीतगुणो गीतकीर्तिर्गीतविशारदः}% १०५॥

\twolineshloka
{गीतस्फीतयशा गीतप्रणयो गीतचञ्चुरः}%
{गीतप्रसन्नो गीतात्मा गीतलोलो गतस्पृहः}% १०६॥

\twolineshloka
{गीताश्रयो गीतमयो गीततत्त्वार्थकोविदः}%
{गीतसंशयसञ्छेत्ता गीतसङ्गीतशासनः}% १०७॥

\twolineshloka
{गीतार्थज्ञो गीततत्त्वो गीतातत्त्वं गताश्रयः}%
{गीतासारोऽथ गीताकृद्गीताकृद्विघ्ननाशनः}% १०८॥

\twolineshloka
{गीताशक्तो गीतलीनो गीताविगतसञ्ज्वरः}%
{गीतैकदृग्गीतभूतिर्गीतप्रीतो गतालसः}% १०९॥

\twolineshloka
{गीतवाद्यपटुर्गीतप्रभुर्गीतार्थतत्त्ववित्}%
{गीतागीतविवेकज्ञो गीताप्रवणचेतनः}% ११०॥

\twolineshloka
{गतभीर्गतविद्वेषो गतसंसारबन्धनः}%
{गतमायो गतत्रासो गतदुःखो गतज्वरः}% १११॥

\twolineshloka
{गतासुहृद्गतज्ञानो गतदुष्टाशयो गतः}%
{गतार्तिर्गतसङ्कल्पो गतदुष्टविचेष्टितः}% ११२॥

\twolineshloka
{गताहङ्कारसञ्चारो गतदर्पो गताहितः}%
{गतविघ्नो गतभयो गतागतनिवारकः}% ११३॥

\twolineshloka
{गतव्यथो गतापायो गतदोषो गतेः परः}%
{गतसर्वविकारोऽथ गतगञ्जितकुञ्जरः}% ११४॥

\twolineshloka
{गतकम्पितभूपृष्ठो गतरुग्गतकल्मषः}%
{गतदैन्यो गतस्तैन्यो गतमानो गतश्रमः}% ११५॥

\twolineshloka
{गतक्रोधो गतग्लानिर्गतम्लानो गतभ्रमः}%
{गताभावो गतभवो गततत्त्वार्थसंशयः}% ११६॥

\twolineshloka
{गयासुरशिरश्छेत्ता गयासुरवरप्रदः}%
{गयावासो गयानाथो गयावासिनमस्कृतः}% ११७॥

\twolineshloka
{गयातीर्थफलाध्यक्षो गयायात्राफलप्रदः}%
{गयामयो गयाक्षेत्रं गयाक्षेत्रनिवासकृत्}% ११८॥

\twolineshloka
{गयावासिस्तुतो गयान्मधुव्रतलसत्कटः}%
{गायको गायकवरो गायकेष्टफलप्रदः}% ११९॥

\twolineshloka
{गायकप्रणयी गाता गायकाभयदायकः}%
{गायकप्रवणस्वान्तो गायकः प्रथमः सदा}% १२०॥

\twolineshloka
{गायकोद्गीतसम्प्रीतो गायकोत्कटविघ्नहा}%
{गानगेयो गानकेशो गायकान्तरसञ्चरः}% १२१॥

\twolineshloka
{गायकप्रियदः शश्वद्गायकाधीनविग्रहः}%
{गेयो गेयगुणो गेयचरितो गेयतत्त्ववित्}% १२२॥

\twolineshloka
{गायकत्रासहा ग्रन्थो ग्रन्थतत्त्वविवेचकः}%
{गाढानुरागो गाढाङ्गो गाढागङ्गाजलोऽन्वहम्}% १२३॥

\twolineshloka
{गाढावगाढजलधिर्गाढप्रज्ञो गतामयः}%
{गाढप्रत्यर्थिसैन्योऽथ गाढानुग्रहतत्परः}% १२४॥

\twolineshloka
{गाढश्लेषरसाभिज्ञो गाढनिर्वृतिसाधकः}%
{गङ्गाधरेष्टवरदो गङ्गाधरभयापहः}% १२५॥

\twolineshloka
{गङ्गाधरगुरुर्गङ्गाधरध्यातपदः सदा}
{गङ्गाधरस्तुतो गङ्गाधराराध्यो गतस्मयः}% १२६॥

\twolineshloka
{गङ्गाधरप्रियो गङ्गाधरो गङ्गाम्बुसुन्दरः}%
{गङ्गाजलरसास्वादचतुरो गाङ्गतीरयः}% १२७॥

\twolineshloka
{गङ्गाजलप्रणयवान् गङ्गातीरविहारकृत्}%
{गङ्गाप्रियो गङ्गाजलावगाहनपरः सदा}% १२८॥

\twolineshloka
{गन्धमादनसंवासो गन्धमादनकेलिकृत्}%
{गन्धानुलिप्तसर्वाङ्गो गन्धलुब्धमधुव्रतः}% १२९॥

\twolineshloka
{गन्धो गन्धर्वराजोऽथ गन्धर्वप्रियकृत् सदा}%
{गन्धर्वविद्यातत्त्वज्ञो गन्धर्वप्रीतिवर्धनः}% १३०॥

\twolineshloka
{गकारबीजनिलयो गकारो गर्विगर्वनुत्}%
{गन्धर्वगणसंसेव्यो गन्धर्ववरदायकः}% १३१॥

\twolineshloka
{गन्धर्वो गन्धमातङ्गो गन्धर्वकुलदैवतम्}%
{गन्धर्वगर्वसञ्छेत्ता गन्धर्ववरदर्पहा}% १३२॥

\twolineshloka
{गन्धर्वप्रवणस्वान्तो गन्धर्वगणसंस्तुतः}%
{गन्धर्वार्चितपादाब्जो गन्धर्वभयहारकः}% १३३॥

\twolineshloka
{गन्धर्वाभयदः शश्वद्गन्धर्वप्रतिपालकः}%
{गन्धर्वगीतचरितो गन्धर्वप्रणयोत्सुकः}% १३४॥

\twolineshloka
{गन्धर्वगानश्रवणप्रणयी गर्वभञ्जनः}%
{गन्धर्वत्राणसन्नद्धो गन्धर्वसमरक्षमः}% १३५॥

\twolineshloka
{गन्धर्वस्त्रीभिराराध्यो गानं गानपटुः सदा}%
{गच्छो गच्छपतिर्गच्छनायको गच्छगर्वहा}% १३६॥

\twolineshloka
{गच्छराजोऽथ गच्छेशो गच्छराजनमस्कृतः}%
{गच्छप्रियो गच्छगुरुर्गच्छत्राणकृतोद्यमः}% १३७॥

\twolineshloka
{गच्छप्रभुर्गच्छचरो गच्छप्रियकृतोद्यमः}%
{गच्छगीतगुणो गच्छमर्यादाप्रतिपालकः}% १३८॥

\twolineshloka
{गच्छधाता गच्छभर्ता गच्छवन्द्यो गुरोर्गुरुः}%
{गृत्सो गृत्समदो गृत्समदाभीष्टवरप्रदः}% १३९॥

\twolineshloka
{गीर्वाणगीतचरितो गीर्वाणगणसेवितः}%
{गीर्वाणवरदाता च गीर्वाणभयनाशकृत्}% १४०॥

\twolineshloka
{गीर्वाणगुणसंवीतो गीर्वाणारातिसूदनः}%
{गीर्वाणधाम गीर्वाणगोप्ता गीर्वाणगर्वहृत्}% १४१॥

\twolineshloka
{गीर्वाणार्तिहरो नित्यं गीर्वाणवरदायकः}%
{गीर्वाणशरणं गीतनामा गीर्वाणसुन्दरः}% १४२॥

\twolineshloka
{गीर्वाणप्राणदो गन्ता गीर्वाणानीकरक्षकः}%
{गुहेहापूरको गन्धमत्तो गीर्वाणपुष्टिदः}% १४३॥

\twolineshloka
{गीर्वाणप्रयुतत्राता गीतगोत्रो गताहितः}%
{गीर्वाणसेवितपदो गीर्वाणप्रथितो गलत्}% १४४॥

\twolineshloka
{गीर्वाणगोत्रप्रवरो गीर्वाणफलदायकः}%
{गीर्वाणप्रियकर्ता च गीर्वाणागमसारवित्}% १४५॥

\twolineshloka
{गीर्वाणागमसम्पत्तिर्गीर्वाणव्यसनापहः}%
{गीर्वाणप्रणयो गीतग्रहणोत्सुकमानसः}% १४६॥

\twolineshloka
{गीर्वाणभ्रमसम्भेत्ता गीर्वाणगुरुपूजितः}%
{ग्रहो ग्रहपतिर्ग्राहो ग्रहपीडाप्रणाशनः}% १४७॥

\twolineshloka
{ग्रहस्तुतो ग्रहाध्यक्षो ग्रहेशो ग्रहदैवतम्}%
{ग्रहकृद्ग्रहभर्ता च ग्रहेशानो ग्रहेश्वरः}% १४८॥

\twolineshloka
{ग्रहाराध्यो ग्रहत्राता ग्रहगोप्ता ग्रहोत्कटः}%
{ग्रहगीतगुणो ग्रन्थप्रणेता ग्रहवन्दितः}% १४९॥

\twolineshloka
{गवी गवीश्वरो गर्वी गर्विष्ठो गर्विगर्वहा}%
{गवाम्प्रियो गवान्नाथो गवीशानो गवाम्पती}% १५०॥

\twolineshloka
{गव्यप्रियो गवाङ्गोप्ता गविसम्पत्तिसाधकः}%
{गविरक्षणसन्नद्धो गवाम्भयहरः क्षणात्}% १५१॥

\twolineshloka
{गविगर्वहरो गोदो गोप्रदो गोजयप्रदः}%
{गजायुतबलो गण्डगुञ्जन्मत्तमधुव्रतः}% १५२॥

\twolineshloka
{गण्डस्थललसद्दानमिलन्मत्तालिमण्डितः}%
{गुडो गुडप्रियो गुण्डगलद्दानो गुडाशनः}% १५३॥

\twolineshloka
{गुडाकेशो गुडाकेशसहायो गुडलड्डुभुक्}%
{गुडभुग्गुडभुग्गणयो गुडाकेशवरप्रदः}% १५४॥

\twolineshloka
{गुडाकेशार्चितपदो गुडाकेशसखः सदा}%
{गदाधरार्चितपदो गदाधरवरप्रदः}% १५५॥

\twolineshloka
{गदायुधो गदापाणिर्गदायुद्धविशारदः}%
{गदहा गददर्पघ्नो गदगर्वप्रणाशनः}% १५६॥

\twolineshloka
{गदग्रस्तपरित्राता गदाडम्बरखण्डकः}%
{गुहो गुहाग्रजो गुप्तो गुहाशायी गुहाशयः}% १५७॥

\twolineshloka
{गुहप्रीतिकरो गूढो गूढगुल्फो गुणैकदृक्}%
{गीर्गीष्पतिर्गिरीशानो गीर्देवीगीतसद्गुणः}% १५८॥

\twolineshloka
{गीर्देवो गीष्प्रियो गीर्भूर्गीरात्मा गीष्प्रियङ्करः}%
{गीर्भूमिर्गीरसन्नोऽथ गीःप्रसन्नो गिरीश्वरः}% १५९॥

\twolineshloka
{गिरीशजो गिरौशायी गिरिराजसुखावहः}%
{गिरिराजार्चितपदो गिरिराजनमस्कृतः}% १६०॥

\twolineshloka
{गिरिराजगुहाविष्टो गिरिराजाभयप्रदः}%
{गिरिराजेष्टवरदो गिरिराजप्रपालकः}% १६१॥

\twolineshloka
{गिरिराजसुतासूनुर्गिरिराजजयप्रदः}%
{गिरिव्रजवनस्थायी गिरिव्रजचरः सदा}% १६२॥

\twolineshloka
{गर्गो गर्गप्रियो गर्गदेहो गर्गनमस्कृतः}%
{गर्गभीतिहरो गर्गवरदो गर्गसंस्तुतः}% १६३॥

\twolineshloka
{गर्गगीतप्रसन्नात्मा गर्गानन्दकरः सदा}%
{गर्गप्रियो गर्गमानप्रदो गर्गारिभञ्जकः}% १६४॥

\twolineshloka
{गर्गवर्गपरित्राता गर्गसिद्धिप्रदायकः}%
{गर्गग्लानिहरो गर्गभ्रमहृद्गर्गसङ्गतः}% १६५॥

\twolineshloka
{गर्गाचार्यो गर्गमुनिर्गर्गसम्मानभाजनः}%
{गम्भीरो गणितप्रज्ञो गणितागमसारवित्}% १६६॥

\twolineshloka
{गणको गणकश्लाघ्यो गणकप्रणयोत्सुकः}%
{गणकप्रवणस्वान्तो गणितो गणितागमः}% १६७॥

\twolineshloka
{गद्यं गद्यमयो गद्यपद्यविद्याविशारदः}%
{गललग्नमहानागो गलदर्चिर्गलन्मदः}% १६८॥

\twolineshloka
{गलत्कुष्ठिव्यथाहन्ता गलत्कुष्ठिसुखप्रदः}%
{गम्भीरनाभिर्गम्भीरस्वरो गम्भीरलोचनः}% १६९॥

\twolineshloka
{गम्भीरगुणसम्पन्नो गम्भीरगतिशोभनः}%
{गर्भप्रदो गर्भरूपो गर्भापद्विनिवारकः}% १७०॥

\twolineshloka
{गर्भागमनसन्नाशो गर्भदो गर्भशोकनुत्}%
{गर्भत्राता गर्भगोप्ता गर्भपुष्टिकरः सदा}% १७१॥

\twolineshloka
{गर्भाश्रयो गर्भमयो गर्भामयनिवारकः}%
{गर्भाधारो गर्भधरो गर्भसन्तोषसाधकः}% १७२॥

\twolineshloka
{गर्भगौरवसन्धानसन्धानं गर्भवर्गहृत्}%
{गरीयान् गर्वनुद्गर्वमर्दी गरदमर्दकः}% १७३॥

{गरसन्तापशमनो गुरुराज्यसुखप्रदः।}%

\dnsub{फलश्रुतिः}
\onelineshloka
{नाम्नां सहस्रमुदितं महद्गणपतेरिदम्}% १७४॥


\twolineshloka
{गकारादि जगद्वन्द्यं गोपनीयं प्रयत्नतः}%
{य इदं प्रयतः प्रातस्त्रिसन्ध्यं वा पठेन्नरः}% १७५॥

\twolineshloka
{वाञ्छितं समवाप्नोति नात्र कार्या विचारणा}%
{पुत्रार्थी लभते पुत्रान् धनार्थी लभते धनम्}% १७६॥

\twolineshloka
{विद्यार्थी लभते विद्यां सत्यं सत्यं न संशयः}%
{भूर्जत्वचि समालिख्य कुङ्कुमेन समाहितः}% १७७॥

\twolineshloka
{चतुर्थां भौमवारो च चन्द्रसूर्योपरागके}%
{पूजयित्वा गणधीशं यथोक्तविधिना पुरा}% १७८॥

\twolineshloka
{पूजयेद्यो यथाशक्त्या जुहुयाच्च शमीदलैः}%
{गुरुं सम्पूज्य वस्त्राद्यैः कृत्वा चापि प्रदक्षिणम्}% १७९॥

\twolineshloka
{धारयेद्यः प्रयत्नेन स साक्षाद्गणनायकः}%
{सुराश्चासुरवर्याश्च पिशाचाः किन्नरोरगः}% १८०॥

\twolineshloka
{प्रणमन्ति सदा तं वै दुष्ट्वा विस्मितमानसाः}%
{राजा सपदि वश्यः स्यात् कामिन्यस्तद्वशो स्थिराः}% १८१॥

\twolineshloka
{तस्य वंशो स्थिरा लक्ष्मीः कदाऽपि न विमुञ्चति}%
{निष्कामो यः पठेदेतद्गणेश्वरपरायणः}% १८२॥

\twolineshloka
{स प्रतिष्ठां परां प्राप्य निजलोकमवाप्नुयात्}%
{इदं ते कीर्तितं नाम्नां सहस्रं देवि पावनम्}% १८३॥

\twolineshloka
{न देयं कृपणायाथ शठाय गुरुविद्विषे}%
{दत्त्वा च भ्रंशमाप्नोति देवतायाः प्रकोपतः}% १८४॥

\twolineshloka
{इति श्रुत्वा महादेवी तदा विस्मितमानसा}%
{पूजयामास विधिवद्गणेश्वरपदद्वयम्}% १८५॥

॥इति श्री-रुद्रयामले महागुप्तसारे शिवपार्वतीसंवादे गकारादि श्री-गणपतिसहस्रनामस्तोत्रं सम्पूर्णम्॥
