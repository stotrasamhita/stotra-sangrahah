% !TeX program = XeLaTeX
% !TeX root = ../../shloka.tex
\sect{शिवसहस्रनामस्तोत्रम् (विष्णुकृतम्)}
\label{sec:start_ShivaSahasranamaStotram-VishnuKrtam}

\twolineshloka*
{शुक्लाम्बरधरं विष्णुं शशिवर्णं चतुर्भुजम्}
{प्रसन्नवदनं ध्यायेत् सर्वविघ्नोपशान्तये}

\twolineshloka*
{नमोऽस्तु ते व्यास विशालबुद्धे फुल्लारविन्दायतपत्रनेत्र}
{येन त्वया भारततैलपूर्णः प्रज्वालितो ज्ञानमयः प्रदीपः}% .. 2..


\twolineshloka*
{नारायणं नमस्कृत्य नरं चैव नरोत्तमम्}
{देवीं सरस्वतीं व्यासं ततो जयमुदीरयेत्}


\fourlineindentedshloka*
{वन्दे शम्भुमुमापतिं सुरगुरुं वन्दे जगत्कारणं}
{वन्दे पन्नगभूषणं मृगधरं वन्दे पशूनां पतिम्}
{वन्दे सूर्यशशाङ्कवह्निनयनं वन्दे मुकुन्दप्रियं}
{वन्दे भक्तजनाश्रयं च वरदं वन्दे शिवं शङ्करम्}


\uvacha{ऋषय ऊचुः}

\twolineshloka
{कथं देवेन वै सूत देवदेवान्महेश्वरात्}
{सुदर्शनाख्यं वै लब्धं वक्तुमर्हसि विष्णुना} %॥१॥

\uvacha{सूत उवाच}

\twolineshloka
{देवानामसुरेन्द्राणामभवच्च सुदारुणः}
{सर्वेषामेव भूतानां विनाशकरणो महान्} %॥२॥

\twolineshloka
{ते देवाः शक्तिमुसलैः सायकैर्नतपर्वभिः}
{प्रभिद्यमानाः कुन्तैश्च दुद्रुवुर्भयविह्वलाः} %॥३॥

\twolineshloka
{पराजितास्तदा देवा देवदेवेश्वरं हरिम्}
{प्रणेमुस्तं सुरेशानं शोकसंविग्नमानसाः} %॥४॥

\twolineshloka
{तान् समीक्ष्याथ भगवान् देवदेवेश्वरो हरिः}
{प्रणिपत्य स्थितान् देवानिदं वचनमब्रवीत्} %॥५॥

\twolineshloka
{वत्साः किमिति वै देवाश्च्युतालङ्कारविक्रमाः}
{समागताः ससन्तापा वक्तुमर्हथ सुव्रताः} %॥६॥

\twolineshloka
{तस्य तद् वचनं श्रुत्वा तथाभूताः सुरोत्तमाः}
{प्रणम्याहुर्यथावृत्तं देवदेवाय विष्णवे} %॥७॥

\twolineshloka
{भगवन् देवदेवेश विष्णो जिष्णो जनार्दन}
{दानवैः पीडिताः सर्वे वयं शरणमागताः} %॥८॥

\twolineshloka
{त्वमेव देवदेवेश गतिर्नः पुरुषोत्तम}
{त्वमेव परमात्मा हि त्वं पिता जगतामपि} %॥९॥

\twolineshloka
{त्वमेव भर्ता हर्ता च भोक्ता दाता जनार्दन}
{हन्तुमर्हसि तस्मात् त्वं दानवान् दानवार्दन} %॥१०॥

\twolineshloka
{दैत्याश्च वैष्णवैर्ब्राह्मै रौद्रैर्याम्यैः सुदारुणैः}
{कौबेरैश्चैव सौम्यैश्च नैर्ऋत्यैर्वारुणैर्दृढैः} %॥११॥

\twolineshloka
{वायव्यैश्च तथाऽऽग्नेयैरेशानैर्वार्षिकैः शुभैः}
{सौरै रौद्रैस्तथा भीमैः कम्पनैर्जृम्भणैर्दृढैः} %॥१२॥

\twolineshloka
{अवध्या वरलाभात् ते सर्वे वारिजलोचन}
{सूर्यमण्डलसम्भूतं त्वदीयं चक्रमुद्यतम्} %॥१३॥

\twolineshloka
{कुण्ठितं हि दधीचेन च्यावनेन जगद्गुरो}
{दण्डं शार्ङ्गं तवास्त्रं च लब्धं दैत्यैः प्रसादतः} %॥१४॥

\twolineshloka
{पुरा जलन्धरं हेतुं निर्मितं त्रिपुरारिणा}
{रथाङ्गं सुशितं घोरं तेन तान् हन्तुमर्हसि} %॥१५॥

\twolineshloka
{तस्मात् तेन निहन्तव्या नान्यैः शस्त्रशतैरपि}
{ततो निशम्य तेषां वै वचनं वारिजेक्षणः} %॥१६॥


{वाचस्पतिमुखानाह स हरिश्चक्रभृत् स्वयम्।}

\uvacha{श्रीविष्णुरुवाच}
{भो भो देवा महादेवं सर्वैर्देवैः सनातनैः॥१७॥} %॥१७॥
\addtocounter{shlokacount}{1}

\twolineshloka
{सम्प्राप्य साम्प्रतं सर्वं करिष्यामि दिवौकसाम्}
{देवा जलन्धरं हन्तुं निर्मितं हि पुरारिणा} %॥१८॥

\twolineshloka
{लब्ध्वा रथाङ्गं तेनैव निहत्य च महासुरान्}
{सर्वान् धुन्धुमुखान् दैत्यान् अष्टषष्टिशतान् सुरान्} %॥१९॥

{सबान्धवान् क्षणादेव युष्मान् सन्तारयाम्यहम्।}

\uvacha{सूत उवाच}

{एवमुक्त्वा सुरश्रेष्ठान् सुरश्रेष्ठमनुस्मरन्॥२०॥} %॥२०॥

\addtocounter{shlokacount}{1}

\twolineshloka
{सुरश्रेष्ठस्तदा श्रेष्ठं पूजयामास शङ्करम्}
{लिङ्गं स्थाप्य यथान्यायं हिमवच्छिखरे शुभे} %॥२१॥

\twolineshloka
{मेरुपर्वतसङ्काशं निर्मितं विश्वकर्मणा}
{त्वरिताख्येन रुद्रेण रौद्रेण च जनार्दनः} %॥२२॥

\twolineshloka
{स्नाप्य सम्पूज्य गन्धाद्यैर्ज्वालाकारं मनोरमम्}
{तुष्टाव च तदा रुद्रं सम्पूज्याग्नौ प्रणम्य च} %॥२३॥

\twolineshloka
{देवं नाम्नां सहस्रेण भवाद्येन यथाक्रमम्}
{पूजयामास च शिवं प्रणवाद्यं नमोन्तकम्} %॥२४॥

\twolineshloka
{देवं नाम्नां सहस्रेण भवाद्येन महेश्वरम्}
{प्रतिनाम स पद्मेन पूजयामास शङ्करम्} %॥२५॥

\twolineshloka
{अग्नौ च नामभिर्देवं भवाद्यैः समिदादिभिः}
{स्वाहान्तैर्विधिवद्धुत्वा प्रत्येकमयुतं प्रभुम्} %॥२६॥

तुष्टाव च पुनः शम्भुं भवाद्यैर्भवमीश्वरम्।

\dnsub{न्यासः}

अस्य श्रीशिवसहस्रनामस्तोत्रमहामन्त्रस्य \textsf{}\\
श्रीविष्णुः ऋषिः, अनुष्टुप् छन्दः, परमात्मा श्रीशङ्करो देवता।

\dnsub{ध्यानम्}

\fourlineindentedshloka*
{मूले कल्पद्रुमस्य द्रुतकनकनिभं चारुपद्मासनस्थं}
{वामाङ्कारूढगौरीनिबिडकुचभराभोगगाढोपगूढम्}
{नानालङ्कारकान्तं वरपरशुमृगाभीतिहस्तं त्रिनेत्रं}
{वन्दे बालेन्दुमौलिं गजवदनगुहाश्लिष्टपार्श्वं महेशम्}

\dnsub{स्तोत्रम्}
\resetShloka
\uvacha{श्रीविष्णुरुवाच}

\twolineshloka
{भवः शिवो हरो रुद्रः पुरुषः पद्मलोचनः}
{अर्थितव्यः सदाचारः सर्वशम्भुमहेश्वरः}

\twolineshloka
{ईश्वरः स्थाणुरीशानः सहस्राक्षः सहस्रपात्}
{वरीयान् वरदो वन्द्यः शङ्करः परमेश्वरः}

\twolineshloka
{गङ्गाधरः शूलधरः परार्थैकप्रयोजनः}
{सर्वज्ञः सर्वदेवादिगिरिधन्वा जटाधरः}

\twolineshloka
{चन्द्रापीडश्चन्द्रमौलिर्विद्वान् विश्वामरेश्वरः}
{वेदान्तसारसन्दोहः कपाली नीललोहितः}

\twolineshloka
{ध्यानाधारोऽपरिच्छेद्यो गौरीभर्ता गणेश्वरः}
{अष्टमूर्तिर्विश्वमूर्तिस्त्रिवर्गः स्वर्गसाधनः}

\twolineshloka
{ज्ञानगम्यो दृढप्रज्ञो देवदेवस्त्रिलोचनः}
{वामदेवो महादेवः पाण्डुः परिदृढोऽदृढः}

\twolineshloka
{विश्वरूपो विरूपाक्षो वागीशः शुचिरन्तरः}
{सर्वप्रणयसंवादी वृषाङ्को वृषवाहनः}

\twolineshloka
{ईशः पिनाकी खट्वाङ्गी चित्रवेषश्चिरन्तनः}
{तमोहरो महायोगी गोप्ता ब्रह्माङ्गहृज्जटी}

\twolineshloka
{कालकालः कृत्तिवासाः सुभगः प्रणवात्मकः}
{उन्मत्तवेषश्चक्षुष्यो दुर्वासाः स्मरशासनः}

\twolineshloka
{दृढायुधः स्कन्दगुरुः परमेष्ठी परायणः}
{अनादिमध्यनिधनो गिरिशो गिरिबान्धवः}

\twolineshloka
{कुबेरबन्धुः श्रीकण्ठो लोकवर्णोत्तमोत्तमः}
{सामान्यदेवः कोदण्डी नीलकण्ठः परश्वधी}

\twolineshloka
{विशालाक्षो मृगव्याधः सुरेशः सूर्यतापनः}
{धर्मकर्माक्षमः क्षेत्रं भगवान् भगनेत्रभित्}

\twolineshloka
{उग्रः पशुपतिस्तार्क्ष्यप्रियभक्तः प्रियंवदः}
{दान्तोदयाकरो दक्षः कपर्दी कामशासनः}

\twolineshloka
{श्मशाननिलयः सूक्ष्मः श्मशानस्थो महेश्वरः}
{लोककर्ता भूतपतिर्महाकर्ता महौषधी}

\twolineshloka
{उत्तरो गोपतिर्गोप्ता ज्ञानगम्यः पुरातनः}
{नीतिः सुनीतिः शुद्धात्मा सोमसोमरतः सुखी}

\twolineshloka
{सोमपोऽमृतपः सोमो महानीतिर्महामतिः}
{अजातशत्रुरालोकः सम्भाव्यो हव्यवाहनः}

\twolineshloka
{लोककारो वेदकारः सूत्रकारः सनातनः}
{महर्षिः कपिलाचार्यो विश्वदीप्तिस्त्रिलोचनः}

\twolineshloka
{पिनाकपाणिर्भूर्देवः स्वस्तिदः स्वस्तिकृत् सदा}
{त्रिधामा सौभगः शर्वसर्वज्ञः सर्वगोचरः}

\twolineshloka
{ब्रह्मधृग्विश्वसृक्स्वर्गः कर्णिकारः प्रियः कविः}
{शाखो विशाखो गोशाखः शिवो नैकः क्रतुः समः}

\twolineshloka
{गङ्गाप्लवोदको भावः सकलः स्थपतिस्थिरः}
{विजितात्मा विधेयात्मा भूतवाहनसारथिः}

\twolineshloka
{सगणो गणकार्यश्च सुकीर्तिश्छिन्नसंशयः}
{कामदेवः कामपालो भस्मोद्धूलितविग्रहः}

\twolineshloka
{भस्मप्रियो भस्मशायी कामी कान्तः कृतागमः}
{समायुक्तो निवृत्तात्मा धर्मयुक्तः सदाशिवः}

\twolineshloka
{चतुर्मुखश्चतुर्बाहुर्दुरावासो दुरासदः}
{दुर्गमो दुर्लभो दुर्गः सर्वायुधविशारदः}

\twolineshloka
{अध्यात्मयोगनिलयः सुतन्तुस्तन्तुवर्धनः}
{शुभाङ्गो लोकसारङ्गो जगदीशोऽमृताशनः}

\twolineshloka
{भस्मशुद्धिकरो मेरुरोजस्वी शुद्धविग्रहः}
{हिरण्यरेतास्तरणिर्मरीचिर्महिमालयः}

\twolineshloka
{महाह्रदो महागर्भः सिद्धवृन्दारवन्दितः}
{व्याघ्रचर्मधरो व्याली महाभूतो महानिधिः}

\twolineshloka
{अमृताङ्गोऽमृतवपुः पञ्चयज्ञः प्रभञ्जनः}
{पञ्चविंशतितत्त्वज्ञः पारिजातः परावरः}

\twolineshloka
{सुलभः सुव्रतः शूरो वाङ्मयैकनिधिर्निधिः}
{वर्णाश्रमगुरुर्वर्णी शत्रुजिच्छत्रुतापनः}

\twolineshloka
{आश्रमः क्षपणः क्षामो ज्ञानवानचलाचलः}
{प्रमाणभूतो दुर्ज्ञेयः सुपर्णो वायुवाहनः}

\twolineshloka
{धनुर्धरो धनुर्वेदो गुणराशिर्गुणाकरः}
{अनन्तदृष्टिरानन्दो दण्डो दमयिता दमः}

\twolineshloka
{अभिवाद्यो महाचार्यो विश्वकर्मा विशारदः}
{वीतरागो विनीतात्मा तपस्वी भूतभावनः}

\twolineshloka
{उन्मत्तवेषः प्रच्छन्नो जितकामोऽजितप्रियः}
{कल्याणप्रकृतिः कल्पः सर्वलोकप्रजापतिः}

\twolineshloka
{तपःस्वी तारको धीमान् प्रधानप्रभुरव्ययः}
{लोकपालोऽन्तर्हितात्मा कल्पादिः कमलेक्षणः}

\twolineshloka
{वेदशास्त्रार्थतत्त्वज्ञो नियमो नियमाश्रयः}
{चन्द्रः सूर्यः शनिः केतुर्विरामो विद्रुमच्छविः}

\twolineshloka
{भक्तिगम्यः परब्रह्ममृगबाणार्पणोऽनघः}
{अद्रिराजालयः कान्तः परमात्मा जगद्गुरुः}

\twolineshloka
{सर्वकर्माचलस्त्वष्टा मङ्गल्यो मङ्गलावृतः}
{महातपा दीर्घतपाः स्थविष्ठः स्थविरो ध्रुवः}

\twolineshloka
{अहः संवत्सरो व्याप्तिः प्रमाणं परमं तपः}
{संवत्सरकरो मन्त्रः प्रत्ययः सर्वदर्शनः}

\twolineshloka
{अजः सर्वेश्वरः स्निग्धो महारेता महाबलः}
{योगी योग्यो महारेताः सिद्धः सर्वादिरग्निदः}

\twolineshloka
{वसुर्वसुमनाः सत्यसर्वपापहरो हरः}
{अमृतः शाश्वतः शान्तो बाणहस्तः प्रतापवान्}

\twolineshloka
{कमण्डलुधरो धन्वी वेदाङ्गो वेदविन्मुनिः}
{भ्राजिष्णुर्भोजनं भोक्ता लोकनेता दुराधरः}

\twolineshloka
{अतीन्द्रियो महामायः सर्वावासश्चतुष्पथः}
{कालयोगी महानादो महोत्साहो महाबलः}

\twolineshloka
{महाबुद्धिर्महावीर्यो भूतचारी पुरन्दरः}
{निशाचरः प्रेतचारिमहाशक्तिर्महाद्युतिः}

\twolineshloka
{अनिर्देश्यवपुः श्रीमान् सर्वहार्यमितो गतिः}
{बहुश्रुतो बहुमयो नियतात्मा भवोद्भवः}

\twolineshloka
{ओजस्तेजोद्युतिकरो नर्तकः सर्वकामकः}
{नृत्यप्रियो नृत्यनृत्यः प्रकाशात्मा प्रतापनः}

\twolineshloka
{बुद्धस्पष्टाक्षरो मन्त्रः सन्मानः सारसम्प्लवः}
{युगादिकृद् युगावर्तो गम्भीरो वृषवाहनः}

\twolineshloka
{इष्टो विशिष्टः शिष्टेष्टः शरभः शरभो धनुः}
{अपां निधिरधिष्ठानविजयो जयकालवित्}

\twolineshloka
{प्रतिष्ठितः प्रमाणज्ञो हिरण्यकवचो हरिः}
{विरोचनः सुरगणो विद्येशो विबुधाश्रयः}

\twolineshloka
{बालरूपो बलोन्माथी विवर्तो गहनो गुरुः}
{करणं कारणं कर्ता सर्वबन्धविमोचनः}

\twolineshloka
{विद्वत्तमो वीतभयो विश्वभर्ता निशाकरः}
{व्यवसायो व्यवस्थानः स्थानदो जगदादिजः}

\twolineshloka
{दुन्दुभो ललितो विश्वो भवात्मात्मनि संस्थितः}
{वीरेश्वरो वीरभद्रो वीरहा वीरभृद्विराट्}

\twolineshloka
{वीरचूडामणिर्वेत्ता तीव्रनादो नदीधरः}
{आज्ञाधारस्त्रिशूली च शिपिविष्टः शिवालयः}

\twolineshloka
{वालखिल्यो महाचापस्तिग्मांशुर्निधिरव्ययः}
{अभिरामः सुशरणः सुब्रह्मण्यः सुधापतिः}

\twolineshloka
{मघवान् कौशिको गोमान् विश्रामः सर्वशासनः}
{ललाटाक्षो विश्वदेहः सारः संसारचक्रभृत्}

\twolineshloka
{अमोघदण्डी मध्यस्थो हिरण्यो ब्रह्मवर्चसी}
{परमार्थः परमयः शम्बरो व्याघ्रकोऽनलः}

\twolineshloka
{रुचिर्वररुचिर्वन्द्यो वाचस्पतिरहर्पतिः}
{रविर्विरोचनः स्कन्धः शास्ता वैवस्वतोऽजनः}

\twolineshloka
{युक्तिरुन्नतकीर्तिश्च शान्तरागः पराजयः}
{कैलासपतिकामारिः सविता रविलोचनः}

\twolineshloka
{विद्वत्तमो वीतभयो विश्वहर्ताऽनिवारितः}
{नित्यो नियतकल्याणः पुण्यश्रवणकीर्तनः}

\twolineshloka
{दूरश्रवा विश्वसहो ध्येयो दुःस्वप्ननाशनः}
{उत्तारको दुष्कृतिहा दुर्धर्षो दुःसहोऽभयः}

\twolineshloka
{अनादिर्भूर्भुवो लक्ष्मीः किरीटित्रिदशाधिपः}
{विश्वगोप्ता विश्वभर्ता सुधीरो रुचिराङ्गदः}

\twolineshloka
{जननो जनजन्मादिः प्रीतिमान् नीतिमान् नयः}
{विशिष्टः काश्यपो भानुर्भीमो भीमपराक्रमः}

\twolineshloka
{प्रणवः सप्तधाचारो महाकायो महाधनुः}
{जन्माधिपो महादेवः सकलागमपारगः}

\twolineshloka
{तत्त्वातत्त्वविवेकात्मा विभूष्णुर्भूतिभूषणः}
{ऋषिर्ब्राह्मणविज्जिष्णुर्जन्ममृत्युजरातिगः}

\twolineshloka
{यज्ञो यज्ञपतिर्यज्वा यज्ञान्तोऽमोघविक्रमः}
{महेन्द्रो दुर्भरः सेनी यज्ञाङ्गो यज्ञवाहनः}

\twolineshloka
{पञ्चब्रह्मसमुत्पत्तिर्विश्वेशो विमलोदयः}
{आत्मयोनिरनाद्यन्तः षड्विंशत्सप्तलोकधृक्}

\twolineshloka
{गायत्रीवल्लभः प्रांशुर्विश्वावासः प्रभाकरः}
{शिशुर्गिरिरतः सम्राट्सुषेणः सुरशत्रुहा}

\twolineshloka
{अमोघोऽरिष्टमथनो मुकुन्दो विगतज्वरः}
{स्वयञ्ज्योतिरनुज्योतिरात्मज्योतिरचञ्चलः}

\twolineshloka
{पिङ्गलः कपिलश्मश्रुः शास्त्रनेत्रत्रयीतनुः}
{ज्ञानस्कन्धो महाज्ञानी निरुत्पत्तिरुपप्लवः}

\twolineshloka
{भगो विवस्वानादित्यो योगाचार्यो बृहस्पतिः}
{उदारकीर्तिरुद्योगी सद्योगी सदसन्मयः}

\twolineshloka
{नक्षत्रमाली राकेशः साधिष्ठानः षडाश्रयः}
{पवित्रपाणिः पापारिर्मणिपूरो मनोगतिः}

\twolineshloka
{हृत्पुण्डरीकमासीनः शुक्लः शान्तो वृषाकपिः}
{विष्णुर्ग्रहपतिः कृष्णः समर्थोऽर्थननाशनः}

\twolineshloka
{अधर्मशत्रुरक्षय्यः पुरुहूतः पुरुष्टुतः}
{ब्रह्मगर्भो बृहद्गर्भो धर्मधेनुर्धनागमः}

\twolineshloka
{जगद्धितैषिसुगतः कुमारः कुशलागमः}
{हिरण्यवर्णो ज्योतिष्मान् नानाभूतधरो ध्वनिः}

\twolineshloka
{अरोगो नियमाध्यक्षो विश्वामित्रो द्विजोत्तमः}
{बृहज्ज्योतिः सुधामा च महाज्योतिरनुत्तमः}

\twolineshloka
{मातामहो मातरिश्वा नभस्वान् नागहारधृक्}
{पुलस्त्यः पुलहोऽगस्त्यो जातूकर्ण्यः पराशरः}

\twolineshloka
{निरावरणधर्मज्ञो विरिञ्चो विष्टरश्रवाः}
{आत्मभूरनिरुद्धोऽत्रिज्ञानमूर्तिर्महायशाः}

\twolineshloka
{लोकचूडामणिर्वीरश्चण्डसत्यपराक्रमः}
{व्यालकल्पो महाकल्पो महावृक्षः कलाधरः}

\twolineshloka
{अलङ्करिष्णुस्त्वचलो रोचिष्णुर्विक्रमोत्तमः}
{आशुशब्दपतिर्वेगी प्लवनः शिखिसारथिः}

\twolineshloka
{असंसृष्टोऽतिथिः शक्रः प्रमाथी पापनाशनः}
{वसुश्रवाः कव्यवाहः प्रतप्तो विश्वभोजनः}

\twolineshloka
{जर्यो जराधिशमनो लोहितश्च तनूनपात्}
{पृषदश्वो नभो योनिः सुप्रतीकस्तमिस्रहा}

\twolineshloka
{निदाघस्तपनो मेघः पक्षः परपुरञ्जयः}
{मुखानिलः सुनिष्पन्नः सुरभिः शिशिरात्मकः}

\twolineshloka
{वसन्तो माधवो ग्रीष्मो नभस्यो बीजवाहनः}
{अङ्गिरा मुनिरात्रेयो विमलो विश्ववाहनः}

\twolineshloka
{पावनः पुरुजिच्छक्रस्त्रिविद्यो नरवाहनः}
{मनोबुद्धिरहङ्कारः क्षेत्रज्ञः क्षेत्रपालकः}

\twolineshloka
{तेजोनिधिर्ज्ञाननिधिर्विपाको विघ्नकारकः}
{अधरोऽनुत्तरो ज्ञेयो ज्येष्ठो निःश्रेयसालयः}

\twolineshloka
{शैलो नगस्तनुर्दोहो दानवारिररिन्दमः}
{चारुधीर्जनकश्चारुविशल्यो लोकशल्यकृत्}

\twolineshloka
{चतुर्वेदश्चतुर्भावश्चतुरश्चतुरप्रियः}
{आम्नायोऽथ समाम्नायस्तीर्थदेवशिवालयः}

\twolineshloka
{बहुरूपो महारूपः सर्वरूपश्चराचरः}
{न्यायनिर्वाहको न्यायो न्यायगम्यो निरञ्जनः}

\twolineshloka
{सहस्रमूर्धा देवेन्द्रः सर्वशस्त्रप्रभञ्जनः}
{मुण्डो विरूपो विकृतो दण्डी दान्तो गुणोत्तमः}

\twolineshloka
{पिङ्गलाक्षोऽथ हर्यक्षो नीलग्रीवो निरामयः}
{सहस्रबाहुः सर्वेशः शरण्यः सर्वलोकभृत्}

\twolineshloka
{पद्मासनः परञ्ज्योतिः परावरपरम्फलः}
{पद्मगर्भो महागर्भो विश्वगर्भो विचक्षणः}

\twolineshloka
{परावरज्ञो बीजेशः सुमुखः सुमहास्वनः}
{देवासुरगुरुर्देवो देवासुरनमस्कृतः}

\twolineshloka
{देवासुरमहामात्रो देवासुरमहाश्रयः}
{देवादिदेवो देवर्षिदेवासुरवरप्रदः}

\twolineshloka
{देवासुरेश्वरो दिव्यो देवासुरमहेश्वरः}
{सर्वदेवमयोऽचिन्त्यो देवतात्माऽऽत्मसम्भवः}

\twolineshloka
{ईड्योऽनीशः सुरव्याघ्रो देवसिंहो दिवाकरः}
{विबुधाग्रवरश्रेष्ठः सर्वदेवोत्तमोत्तमः}

\twolineshloka
{शिवज्ञानरतः श्रीमान् शिखिश्रीपर्वतप्रियः}
{जयस्तम्भो विशिष्टम्भो नरसिंहनिपातनः}

\twolineshloka
{ब्रह्मचारी लोकचारी धर्मचारी धनाधिपः}
{नन्दी नन्दीश्वरो नग्नो नग्नव्रतधरः शुचिः}

\twolineshloka
{लिङ्गाध्यक्षः सुराध्यक्षो युगाध्यक्षो युगावहः}
{स्ववशः सवशः स्वर्गस्वरः स्वरमयस्वनः}

\twolineshloka
{बीजाध्यक्षो बीजकर्ता धनकृद् धर्मवर्धनः}
{दम्भोऽदम्भो महादम्भः सर्वभूतमहेश्वरः}

\twolineshloka
{श्मशाननिलयस्तिष्यः सेतुरप्रतिमाकृतिः}
{लोकोत्तरस्फुटालोकस्त्र्यम्बको नागभूषणः}

\twolineshloka
{अन्धकारिर्मखद्वेषी विष्णुकन्धरपातनः}
{वीतदोषोऽक्षयगुणो दक्षारिः पूषदन्तहृत्}

\twolineshloka
{धूर्जटिः खण्डपरशुः सकलो निष्कलोऽनघः}
{आधारः सकलाधारः पाण्डुराभो मृडो नटः}

\twolineshloka
{पूर्णः पूरयिता पुण्यः सुकुमारः सुलोचनः}
{सामगेयः प्रियकरः पुण्यकीर्तिरनामयः}

\twolineshloka
{मनोजवस्तीर्थकरो जटिलो जीवितेश्वरः}
{जीवितान्तकरो नित्यो वसुरेता वसुप्रियः}

\twolineshloka
{सद्गतिः सत्कृतिः सक्तः कालकण्ठः कलाधरः}
{मानी मान्यो महाकालः सद्भूतिः सत्परायणः}

\twolineshloka
{चन्द्रसञ्जीवनः शास्ता लोकगूढोऽमराधिपः}
{लोकबन्धुर्लोकनाथः कृतज्ञः कृतिभूषणः}

\twolineshloka
{अनपाय्यक्षरः कान्तः सर्वशास्त्रभृतां वरः}
{तेजोमयो द्युतिधरो लोकमायोऽग्रणीरणुः}

\twolineshloka
{शुचिस्मितः प्रसन्नात्मा दुर्जयो दुरतिक्रमः}
{ज्योतिर्मयो निराकारो जगन्नाथो जलेश्वरः}

\twolineshloka
{तुम्बवीणी महाकायो विशोकः शोकनाशनः}
{त्रिलोकात्मा त्रिलोकेशः शुद्धः शुद्धी रथाक्षजः}

\twolineshloka
{अव्यक्तलक्षणोऽव्यक्तो व्यक्ताव्यक्तो विशां पतिः}
{वरशीलो वरतुलो मानो मानधनो मयः}

\twolineshloka
{ब्रह्मा विष्णुः प्रजापालो हंसो हंसगतिर्यमः}
{वेधा धाता विधाता च अत्ता हर्ता चतुर्मुखः}

\twolineshloka
{कैलासशिखरावासी सर्वावासी सतां गतिः}
{हिरण्यगर्भो हरिणः पुरुषः पूर्वजः पिता}

\twolineshloka
{भूतालयो भूतपतिर्भूतिदो भुवनेश्वरः}
{संयोगी योगविद्ब्रह्मा ब्रह्मण्यो ब्राह्मणप्रियः}

\twolineshloka
{देवप्रियो देवनाथो देवज्ञो देवचिन्तकः}
{विषमाक्षः कलाध्यक्षो वृषाङ्को वृषवर्धनः}

\twolineshloka
{निर्मदो निरहङ्कारो निर्मोहो निरुपद्रवः}
{दर्पहा दर्पितो दृप्तः सर्वर्तुपरिवर्तकः}

\twolineshloka
{सप्तजिह्वः सहस्रार्चिः स्निग्धः प्रकृतिदक्षिणः}
{भूतभव्यभवन्नाथः प्रभवो भ्रान्तिनाशनः}

\twolineshloka
{अर्थोऽनर्थो महाकोशः परकार्यैकपण्डितः}
{निष्कण्टकः कृतानन्दो निर्व्याजो व्याजमर्दनः}

\twolineshloka
{सत्त्ववान् सात्त्विकः सत्यकीर्तिस्तम्भकृतागमः}
{अकम्पितो गुणग्राही नैकात्मा नैककर्मकृत्}

\twolineshloka
{सुप्रीतः सुमुखः सूक्ष्मः सुकरो दक्षिणोऽनलः}
{स्कन्धः स्कन्धधरो धुर्यः प्रकटः प्रीतिवर्धनः}

\twolineshloka
{अपराजितः सर्वसहो विदग्धः सर्ववाहनः}
{अधृतः स्वधृतः साध्यः पूर्तमूर्तिर्यशोधरः}

\twolineshloka
{वराहशृङ्गधृग् वायुर्बलवानेकनायकः}
{श्रुतिप्रकाशः श्रुतिमानेकबन्धुरनेकधृक्}

\twolineshloka
{श्रीवल्लभशिवारम्भः शान्तभद्रः समञ्जसः}
{भूशयो भूतिकृद्भूतिर्भूषणो भूतवाहनः}

\twolineshloka
{अकायो भक्तकायस्थः कालज्ञानी कलावपुः}
{सत्यव्रतमहात्यागी निष्ठाशान्तिपरायणः}

\twolineshloka
{परार्थवृत्तिर्वरदो विविक्तः श्रुतिसागरः}
{अनिर्विण्णो गुणग्राही कलङ्काङ्कः कलङ्कहा}

\twolineshloka
{स्वभावरुद्द्रो मध्यस्थः शत्रुघ्नो मध्यनाशकः}
{शिखण्डी कवची शूली चण्डी मुण्डी च कुण्डली}

\twolineshloka
{मेखली कवची खड्गी मायी संसारसारथिः}
{अमृत्युः सर्वदृक् सिंहस्तेजोराशिर्महामणिः}

\twolineshloka
{असङ्ख्येयोऽप्रमेयात्मा वीर्यवान् कार्यकोविदः}
{वेद्यो वेदार्थविद्गोप्ता सर्वाचारो मुनीश्वरः}

\twolineshloka
{अनुत्तमो दुराधर्षो मधुरः प्रियदर्शनः}
{सुरेशः शरणं सर्वः शब्दब्रह्मसतां गतिः}

\twolineshloka
{कालभक्षः कलङ्कारिः कङ्कणीकृतवासुकिः}
{महेष्वासो महीभर्ता निष्कलङ्को विशृङ्खलः}

\twolineshloka
{द्युमणिस्तरणिर्धन्यः सिद्धिदः सिद्धिसाधनः}
{निवृत्तः संवृतः शिल्पो व्यूढोरस्को महाभुजः}

\twolineshloka
{एकज्योतिर्निरातङ्को नरो नारायणप्रियः}
{निर्लेपो निष्प्रपञ्चात्मा निर्व्यग्रो व्यग्रनाशनः}

\twolineshloka
{स्तव्यस्तवप्रियः स्तोता व्यासमूर्तिरनाकुलः}
{निरवद्यपदोपायो विद्याराशिरविक्रमः}

\twolineshloka
{प्रशान्तबुद्धिरक्षुद्रः क्षुद्रहा नित्यसुन्दरः}
{धैर्याग्र्यधुर्यो धात्रीशः शाकल्यः शर्वरीपतिः}

\twolineshloka
{परमार्थगुरुर्दृष्टिर्गुरुराश्रितवत्सलः}
{रसो रसज्ञः सर्वज्ञः सर्वसत्त्वावलम्बनः}


॥इति श्रीलिङ्गमहापुराणे पूर्वभागे अष्टनवतितमे अध्याये भगवता विष्णुना प्रोक्तं श्रीशिवसहस्रनामस्तोत्रम्॥


\uvacha{सूत उवाच}
{एवं नाम्नां सहस्रेण तुष्टाव वृषभध्वजम्॥१५९॥}%॥१५९ ॥

\resetShloka
\addtocounter{shlokacount}{159}


\twolineshloka
{स्नापयामास च विभुः पूजयामास पङ्कजैः}
{परीक्षार्थं हरेः पूजाकमलेषु महेश्वरः}%॥१६० ॥

\twolineshloka
{गोपयामास कमलं तदैकं भुवनेश्वरः}
{हृतपुष्पो हरिस्तत्र किमिदं त्वभ्यचिन्तयत्}%॥६१ ॥

\twolineshloka
{ज्ञात्वा स्वनेत्रमुद्धृत्य सर्वसत्त्वावलम्बनम्}
{पूजयामास भावेन नाम्ना तेन जगद्गुरुम्}%॥१६२ ॥

\twolineshloka
{ततस्तत्र विभुर्दृष्ट्वा तथाभूतं हरो हरिम्}
{तस्मादवतताराशु मण्डलात् पावकस्य च}%॥१६३ ॥

\twolineshloka
{कोटिभास्करसङ्काशं जटामुकुटमण्डितम्}
{ज्वालामालावृतं दिव्यं तीक्ष्णदंष्ट्रं भयङ्करम्}%॥१६४ ॥

\twolineshloka
{शूलटङ्कगदाचक्रकुन्तपाशधरं हरम्}
{वरदाभयहस्तं च द्वीपिचर्मोत्तरीयकम्}%॥१६५ ॥

\twolineshloka
{इत्थम्भूतं तदा दृष्ट्वा भवं भस्मविभूषितम्}
{हृष्टो नमश्चकाराशु देवदेवं जनार्दनः}%॥१६६ ॥

\twolineshloka
{दुद्रुवुस्तं परिक्रम्य सेन्द्रा देवास्त्रिलोचनम्}
{चचाल ब्रह्मभुवनं चकम्पे च वसुन्धरा}%॥१६७ ॥

\twolineshloka
{ददाह तेजस्तच्छम्भोः प्रान्तं वै शतयोजनम्}
{अधस्ताच्चोर्ध्वतश्चैव हाहेत्यकृत भूतले}%॥१६८ ॥

\twolineshloka
{तदा प्राह महादेवः प्रहसन्निव शङ्करः}
{सम्प्रेक्ष्य प्रणयाद् विष्णुं कृताञ्जलिपुटं स्थितम्}%॥१६९ ॥

\twolineshloka
{ज्ञातं मयेदमधुना देवकार्यं जनार्दन}
{सुदर्शनाख्यं चक्रं च ददामि तव शोभनम्}%॥१७० ॥

\twolineshloka
{यद् रूपं भवता दृष्टं सर्वलोकभयङ्करम्}
{हिताय तव यत्नेन तव भावाय सुव्रत}%॥१७१ ॥

\twolineshloka
{शान्तं रणाजिरे विष्णो देवानां दुःखसाधनम्}
{शान्तस्य चास्त्रं शान्तं स्याच्छान्तेनास्त्रेण किं फलम्}%॥१७२ ॥

\twolineshloka
{शान्तस्य समरे चास्त्रं शान्तिरेव तपस्विनम्}
{योद्धुः शान्त्या बलच्छेदः परस्य बलवृद्धिदः}%॥१७३ ॥

\twolineshloka
{देवैरशान्तैर्यद् रूपं मदीयं भावयाव्ययम्}
{किमायुधेन कार्यं वै योद्धुं देवारिसूदन}%॥१७४ ॥

\twolineshloka
{क्षमा युधि न कार्या वै योद्धुं देवारिसूदन}
{अनागते व्यतीते च दौर्बल्ये स्वजनोत्करे}%॥१७५ ॥

\twolineshloka
{अकालिके त्वधर्मे च अनर्थे वाऽरिसूदन}
{एवमुक्त्वा ददौ चक्रं सूर्यायुतसमप्रभम्}%॥१७६ ॥

\twolineshloka
{नेत्रं च नेता जगतां प्रभुर्वै पद्मसन्निभम्}
{तदाप्रभृति तं प्राहुः पद्माक्षमिति सुव्रतम्}%॥१७७ ॥

\twolineshloka
{दत्त्वैनं नयनं चक्रं विष्णवे नीललोहितः}
{पस्पर्श च कराभ्यां वै सुशुभाभ्यामुवाच ह}%॥१७८ ॥

\twolineshloka
{वरदोऽहं वरश्रेष्ठ वरान् वरय चेप्सितान्}
{भक्त्या वशीकृतो नूनं त्वयाऽहं पुरुषोत्तम}%॥१७९ ॥

\twolineshloka
{इत्युक्तो देवदेवेन देवदेवं प्रणम्य तम्}
{त्वयि भक्तिर्महादेव प्रसीद वरमुत्तमम्}%॥१८० ॥

\twolineshloka
{नान्यमिच्छामि भक्तानामार्तयो नास्ति यत् प्रभो}
{तच्छ्रुत्वा वचनं तस्य दयावान् सुतरां भवः}%॥१८१ ॥

\twolineshloka
{पस्पर्श च ददौ तस्मै श्रद्धां शीतांशुभूषणः}
{प्राह चैवं महादेवः परमात्मानमच्युतम्}%॥१८२ ॥

\twolineshloka
{मयि भक्तश्च वन्द्यश्च पूज्यश्चैव सुरासुरैः}
{भविष्यसि न सन्देहो मत्प्रसादात् सुरोत्तम}%॥१८३ ॥

\twolineshloka
{यदा सती दक्षपुत्री विनिन्द्यैव सुलोचना}
{मातरं पितरं दक्षं भविष्यति सुरेश्वरी}%॥१८४ ॥

\twolineshloka
{दिव्या हैमवती विष्णो तदा त्वमपि सुव्रत}
{भगिनीं तव कल्याणीं देवीं हैमवतीमुमाम्}%॥१८५ ॥

\twolineshloka
{नियोगाद् ब्रह्मणः साध्वीं प्रदास्यसि ममैव ताम्}
{मत्सम्बन्धी च लोकानां मध्ये पूज्यो भविष्यसि}%॥१८६ ॥

\twolineshloka
{मां दिव्येन च भावेन तदा प्रभृति शङ्करम्}
{द्रक्ष्यसे च प्रसन्नेन मित्रभूतमिवात्मना}%॥१८७ ॥

\twolineshloka
{इत्युक्त्वाऽन्तर्दधे रुद्रो भगवान् नीललोहितः}
{जनार्दनोऽपि भगवान् देवानामपि सन्निधौ}%॥१८८ ॥

\twolineshloka
{अयाचत महादेवं ब्रह्माणं मुनिभिः समम्}
{मया प्रोक्तं स्तवं दिव्यं पद्मयोने सुशोभनम्}%॥१८९ ॥

\twolineshloka
{यः पठेच्छृणुयाद् वाऽपि श्रावयेद् वा द्विजोत्तमान्}
{प्रतिनाम्नि हिरण्यस्य दत्तस्य फलमाप्नुयात्}%॥१९० ॥

\twolineshloka
{अश्वमेधसहस्रेण फलं भवति तस्य वै}
{घृताद्यैः स्नापयेद् रुद्रं स्थाल्या वै कलशैः शुभैः}%॥१९१ ॥

\twolineshloka
{नाम्नां सहस्रेणानेन श्रद्धया शिवमीश्वरम्}
{सोऽपि यज्ञसहस्रस्य फलं लब्ध्वा सुरेश्वरैः}%॥१९२ ॥

\twolineshloka
{पूज्यो भवति रुद्रस्य प्रीतिर्भवति तस्य वै}
{तथाऽस्त्विति तथा प्राह पद्मयोनिर्जनार्दनम्}%॥१९३ ॥

\threelineshloka
{जग्मतुः प्रणिपत्यैनं देवदेवं जगद्गुरुम्}
{तस्मान्नाम्नां सहस्रेण पूजयेदनघो द्विजाः}
{जपेन्नाम्नां सहस्रं च स याति परमां गतिम्}% ॥ १९४ ॥

{॥इति श्रीलिङ्गमहापुराणे पूर्वभागे सहस्रनामभिः पूजनाद् विष्णुचक्रलाभो नामाष्टनवतितमोऽध्यायः॥}%॥९८ ॥

\fourlineindentedshloka*
{दुःस्वप्नदुःशकुनदुर्गतिदौर्मनस्य}
{दुर्भिक्षदुर्व्यसनदुःसहदुर्यशांसि}
{उत्पाततापविषभीतिम् असद्‌ग्रहार्तिं}
{व्याधींश्च नाशयतु मे जगतामधीशः}

॥इति श्रीशिवसहस्रनामस्तोत्रं सम्पूर्णम्॥

\hyperref[sec:start_ShivaSahasranamaStotram-VishnuKrtam]{\closesection}