% !TeX program = XeLaTeX
% !TeX root = ../../shloka.tex
\sect{शिवसहस्रनामस्तोत्रम् (विष्णुकृतम्)}

\twolineshloka*
{शुक्लाम्बरधरं विष्णुं शशिवर्णं चतुर्भुजम्}
{प्रसन्नवदनं ध्यायेत् सर्वविघ्नोपशान्तये}

\twolineshloka*
{नमोऽस्तु ते व्यास विशालबुद्धे फुल्लारविन्दायतपत्रनेत्र}
{येन त्वया भारततैलपूर्णः प्रज्वालितो ज्ञानमयः प्रदीपः}% .. 2..


\twolineshloka*
{नारायणं नमस्कृत्य नरं चैव नरोत्तमम्}
{देवीं सरस्वतीं व्यासं ततो जयमुदीरयेत्}


\fourlineindentedshloka*
{वन्दे शम्भुमुमापतिं सुरगुरुं वन्दे जगत्कारणम्}
{वन्दे पन्नगभूषणं मृगधरं वन्दे पशूनां पतिम्}
{वन्दे सूर्यशशाङ्कवह्निनयनं वन्दे मुकुन्दप्रियम्}
{वन्दे भक्तजनाश्रयं च वरदं वन्दे शिवं शङ्करम्}


\uvacha{ऋषय ऊचुः}

\twolineshloka
{कथं देवेन वै सूत देव-देवान्महेश्वरात्}
{सुदर्शनाख्यं वै लब्धं वक्तुमर्हसि विष्णुना} %॥ १॥

\uvacha{सूत उवाच}

\twolineshloka
{देवानामसुरेन्द्राणामभवच्च सुदारुणः}
{सर्वेषामेव भूतानां विनाश-करणो महान्} %॥ २॥

\twolineshloka
{ते देवाः शक्ति-मुसलैः सायकैर्नत-पर्वभिः}
{प्रभिद्यमानाः कुन्तैश्च दुद्रुवुर्भय-विह्वलाः} %॥ ३॥

\twolineshloka
{पराजितास्तदा देवादेव-देवेश्वरं हरिम्}
{प्रणेमुस्तं सुरेशानं शोक-संविग्न-मानसाः} %॥ ४॥

\twolineshloka
{तान् समीक्ष्याथ भगवान् देव-देवेश्वरो हरिः}
{प्रणिपत्य स्थितान् देवानिदं वचनमब्रवीत्} %॥ ५॥

\twolineshloka
{वत्साः किमिति वै देवाश्च्युतालङ्कार-विक्रमाः}
{समागताः स-सन्तापा वक्तुमर्हथ सुव्रताः} %॥ ६॥

\twolineshloka
{तस्य तद् वचनं श्रुत्वा तथा-भूताः सुरोत्तमाः}
{प्रणम्याहुर्यथा-वृत्तं देव-देवाय विष्णवे} %॥ ७॥

\twolineshloka
{भगवन् देव-देवेश विष्णो जिष्णो जनार्दन}
{दानवैः पीडिताः सर्वे वयं शरणमागताः} %॥ ८॥

\twolineshloka
{त्वमेव देव-देवेश गतिर्नः पुरुषोत्तम}
{त्वमेव परमात्मा हि त्वं पिता जगतामपि} %॥ ९॥

\twolineshloka
{त्वमेव भर्ता हर्ता च भोक्ता दाता जनार्दन}
{हन्तुमर्हसि तस्मात् त्वं दानवान् दानवार्दन} %॥ १०॥

\twolineshloka
{दैत्याश्च वैष्णवैर्ब्राह्मै रौद्रैर्याम्यैः सुदारुणैः}
{कौबेरैश्चैव सौम्यैश्च नैर्ऋत्यैर्वारुणैर्दृढैः} %॥ ११॥

\twolineshloka
{वायव्यैश्च तथाग्नेयैरेशानैर्वार्षिकैः शुभैः}
{सौरै रौद्रैस्तथा भीमैः कम्पनैर्जृम्भणैर्दृढैः} %॥ १२॥

\twolineshloka
{अवध्या वर-लाभात् ते सर्वे वारिज-लोचन}
{सूर्य-मण्डल-सम्भूतं त्वदीयं चक्रमुद्यतम्} %॥ १३॥

\twolineshloka
{कुण्ठितं हि दधीचेन च्यावनेन जगद्गुरो}
{दण्डं शार्ङ्गं तवास्त्रं च लब्धं दैत्यैः प्रसादतः} %॥ १४॥

\twolineshloka
{पुरा जलन्धरं हेतुं निर्मितं त्रि-पुरारिणा}
{रथाङ्गं सुशितं घोरं तेन तान् हन्तुमर्हसि} %॥ १५॥

\twolineshloka
{तस्मात् तेन निहन्तव्या नान्यैः शस्त्र-शतैरपि}
{ततो निशम्य तेषां वै वचनं वारिजेक्षणः} %॥ १६॥


{वाचस्पति-मुखानाह स हरिश्चक्र-भृत् स्वयम्।}

\uvacha{श्री-विष्णुरुवाच}
{भो भो देवा महादेवं सर्वैर्देवैः सनातनैः॥१७॥} %॥ १७॥
\addtocounter{shlokacount}{1}

\twolineshloka
{सम्प्राप्य साम्प्रतं सर्वं करिष्यामि दिवौकसाम्}
{देवा जलन्धरं हन्तुं निर्मितं हि पुरारिणा} %॥ १८॥

\twolineshloka
{लब्ध्वा रथाङ्गं तेनैव निहत्य च महासुरान्}
{सर्वान् धुन्धु-मुखान् दैत्यान् अष्ट-षष्टि-शतान् सुरान्} %॥ १९॥

{स-बान्धवान् क्षणादेव युष्मान् सन्तारयाम्यहम्।}

\uvacha{सूत उवाच}

{एवमुक्त्वा सुर-श्रेष्ठान् सुर-श्रेष्ठमनुस्मरन्॥२०॥} %॥ २०॥

\addtocounter{shlokacount}{1}

\twolineshloka
{सुर-श्रेष्ठस्तदा श्रेष्ठं पूजयामास शङ्करम्}
{लिङ्गं स्थाप्य यथा-न्यायं हिमवच्छिखरे शुभे} %॥ २१॥

\twolineshloka
{मेरु-पर्वत-सङ्काशं निर्मितं विश्व-कर्मणा}
{त्वरिताख्येन रुद्रेण रौद्रेण च जनार्दनः} %॥ २२॥

\twolineshloka
{स्नाप्य सम्पूज्य गन्धाद्यैर्ज्वालाकारं मनोरमम्}
{तुष्टाव च तदा रुद्रं सम्पूज्याग्नौ प्रणम्य च} %॥ २३॥

\twolineshloka
{देवं नाम्नां सहस्रेण भवाद्येन यथा-क्रमम्}
{पूजयामास च शिवं प्रणवाद्यं नमोन्तकम्} %॥ २४॥

\twolineshloka
{देवं नाम्नां सहस्रेण भवाद्येन महेश्वरम्}
{प्रति-नाम स पद्मेन पूजयामास शङ्करम्} %॥ २५॥

\twolineshloka
{अग्नौ च नामभिर्देवं भवाद्यैः समिदादिभिः}
{स्वाहान्तैर्विधिवद्धुत्वा प्रत्येकमयुतं प्रभुम्} %॥ २६॥

तुष्टाव च पुनः शम्भुं भवाद्यैर्भवमीश्वरम्।

\uvacha{श्री-विष्णुरुवाच}



\dnsub{स्तोत्रम्}
\resetShloka

अस्य श्री-शिव-सहस्र-नाम-स्तोत्र-महा-मन्त्रस्य \textsf{---}\\
श्री-विष्णुः ऋषिः, अनुष्टुप् छन्दः, परमात्मा श्री-शङ्करो देवता।

\dnsub{ध्यानम्}

\fourlineindentedshloka*
{मूले कल्प-द्रुमस्य द्रुत-कनक-निभं चारु-पद्मासन-स्थं}
{वामाङ्कारूढ-गौरी-निबिड-कुच-भराभोग-गाढोपगूढम्}
{नानालङ्कार-कान्तं वर-परशु-मृगाभीति-हस्तं त्रि-नेत्रं}
{वन्दे बालेन्दु-मौलिं गज-वदन-गुहाश्लिष्ट-पार्श्वं महेशम्}

\twolineshloka
{भवः शिवो हरो रुद्रः पुरुषः पद्म-लोचनः}
{अर्थितव्यः सदाचारः सर्व-शम्भु-महेश्वरः}

\twolineshloka
{ईश्वरः स्थाणुरीशानः सहस्राक्षः सहस्रपात्}
{वरीयान् वर-दो वन्द्यः शं-करः पर-मेश्वरः}

\twolineshloka
{गङ्गा-धरः शूल-धरः परार्थैक-प्रयोजनः}
{सर्व-ज्ञः सर्व-देवादि-गिरि-धन्वा जटा-धरः}

\twolineshloka
{चन्द्रापीडश्चन्द्रमौलिर्विद्वान् विश्वामरेश्वरः}
{वेदान्त-सार-सन्दोहः कपाली नील-लोहितः}

\twolineshloka
{ध्यानाधारोऽपरिच्छेद्यो गौरी-भर्ता गणेश्वरः}
{अष्ट-मूर्तिर्विश्व-मूर्तिस्त्रि-वर्गः स्वर्ग-साधनः}

\twolineshloka
{ज्ञान-गम्यो दृढ-प्रज्ञो देव-देवस्त्रि-लोचनः}
{वामदेवो महा-देवः पाण्डुः परिदृढोऽ-दृढः}

\twolineshloka
{विश्व-रूपो विरूपाक्षो वागीशः शुचिरन्तरः}
{सर्व-प्रणय-संवादी वृषाङ्को वृष-वाहनः}

\twolineshloka
{ईशः पिनाकी खट्वाङ्गी चित्र-वेषश्चिरन्तनः}
{तमो-हरो महा-योगी गोप्ता ब्रह्माङ्ग-हृज्जटी}

\twolineshloka
{काल-कालः कृत्ति-वासाः सु-भगः प्रणवात्मकः}
{उन्मत्त-वेषश्चक्षुष्यो दुर्वासाः स्मर-शासनः}

\twolineshloka
{दृढायुधः स्कन्द-गुरुः परमेष्ठी परायणः}
{अनादि-मध्य-निधनो गिरि-शो गिरि-बान्धवः}

\twolineshloka
{कुबेर-बन्धुः श्री-कण्ठो लोक-वर्णोत्तमोत्तमः}
{सामान्य-देवः कोदण्डी नील-कण्ठः परश्वधी}

\twolineshloka
{विशालाक्षो मृग-व्याधः सुरेशः सूर्य-तापनः}
{धर्म-कर्माक्षमः क्षेत्रं भगवान् भग-नेत्र-भित्}

\twolineshloka
{उग्रः पशु-पतिस्तार्क्ष्य-प्रिय-भक्तः प्रियं-वदः}
{दान्तोदया-करो दक्षः कपर्दी काम-शासनः}

\twolineshloka
{श्मशान-निलयः सूक्ष्मः श्मशान-स्थो महेश्वरः}
{लोक-कर्ता भूत-पतिः महा-कर्ता महौषधी}

\twolineshloka
{उत्तरो गो-पतिर्गोप्ता ज्ञान-गम्यः पुरातनः}
{नीतिः सुनीतिः शुद्धात्मा सोम-सोम-रतः सुखी}

\twolineshloka
{सोम-पोऽमृत-पः सोमो महा-नीतिर्महा-मतिः}
{अजात-शत्रुरालोकः सम्भाव्यो हव्यवाहनः}

\twolineshloka
{लोक-कारो वेद-कारः सूत्र-कारः सनातनः}
{महर्षिः कपिलाचार्यो विश्व-दीप्तिस्त्रि-लोचनः}

\twolineshloka
{पिनाक-पाणिर्भूर्देवः स्वस्ति-दः स्वस्ति-कृत् सदा}
{त्रि-धामा सौभ-गः शर्व-सर्व-ज्ञः सर्व-गोचरः}

\twolineshloka
{ब्रह्म-धृग्-विश्व-सृक्-स्वर्गः कर्णिकारः प्रियः कविः}
{शाखो विशाखो गो-शाखः शिवो नैकः क्रतुः समः}

\twolineshloka
{गङ्गा-प्लवोदको भावः सकलः स्थपति-स्थिरः}
{विजितात्मा विधेयात्मा भूत-वाहन-सारथिः}

\twolineshloka
{स-गणो गण-कार्यश्च सुकीर्तिश्छिन्न-संशयः}
{काम-देवः काम-पालो भस्मोद्धूलित-विग्रहः}

\twolineshloka
{भस्म-प्रियो भस्म-शायी कामी कान्तः कृतागमः}
{स-मायुक्तो निवृत्तात्मा धर्म-युक्तः सदा-शिवः}

\twolineshloka
{चतुर्मुखश्चतुर्बाहुर्दुरावासो दुरासदः}
{दुर्गमो दुर्लभो दुर्गः सर्वायुध-विशारदः}

\twolineshloka
{अध्यात्म-योग-निलयः सुतन्तुस्तन्तु-वर्धनः}
{शुभाङ्गो लोक-सारङ्गो जगदीशोऽमृताशनः}

\twolineshloka
{भस्म-शुद्धि-करो मेरुरोजस्वी शुद्ध-विग्रहः}
{हिरण्य-रेतास्तरणिर्मरीचिर्महिमालयः}

\twolineshloka
{महा-ह्रदो महा-गर्भः सिद्ध-वृन्दार-वन्दितः}
{व्याघ्र-चर्म-धरो व्याली महा-भूतो महा-निधिः}

\twolineshloka
{अमृताङ्गोऽमृत-वपुः पञ्च-यज्ञः प्रभञ्जनः}
{पञ्च-विंशति-तत्त्व-ज्ञः पारिजातः परावरः}

\twolineshloka
{सुलभः सुव्रतः शूरो वाङ्मयैक-निधिर्निधिः}
{वर्णाश्रम-गुरुर्वर्णी शत्रु-जिच्छत्रु-तापनः}

\twolineshloka
{आश्रमः क्षपणः क्षामो ज्ञान-वानचलाचलः}
{प्रमाण-भूतो दुर्ज्ञेयः सु-पर्णो वायु-वाहनः}

\twolineshloka
{धनुर्धरो धनुर्वेदो गुण-राशिर्गुणाकरः}
{अनन्त-दृष्टिरानन्दो दण्डो दमयिता दमः}

\twolineshloka
{अभिवाद्यो महाचार्यो विश्व-कर्मा विशारदः}
{वीत-रागो विनीतात्मा तपस्वी भूत-भावनः}

\twolineshloka
{उन्मत्त-वेषः प्रच्छन्नो जित-कामोऽजित-प्रियः}
{कल्याण-प्रकृतिः कल्पः सर्व-लोक-प्रजा-पतिः}

\twolineshloka
{तपः-स्वी तारको धी-मान् प्रधान-प्रभुरव्ययः}
{लोक-पालोऽन्तर्हितात्मा कल्पादिः कमलेक्षणः}

\twolineshloka
{वेद-शास्त्रार्थ-तत्त्व-ज्ञो नियमो नियमाश्रयः}
{चन्द्रः सूर्यः शनिः केतुर्वि-रामो विद्रुम-च्छविः}

\twolineshloka
{भक्ति-गम्यः पर-ब्रह्म-मृग-बाणार्पणोऽनघः}
{अद्रि-राजालयः कान्तः परमात्मा जगद्-गुरुः}

\twolineshloka
{सर्व-कर्माचलस्त्वष्टा मङ्गल्यो मङ्गलावृतः}
{महा-तपा दीर्घ-तपाः स्थविष्ठः स्थविरो ध्रुवः}

\twolineshloka
{अहः संवत्सरो व्याप्तिः प्रमाणं पर-मं तपः}
{संवत्सर-करो मन्त्रः प्रत्ययः सर्व-दर्शनः}

\twolineshloka
{अजः सर्वेश्वरः स्निग्धो महा-रेता महा-बलः}
{योगी योग्यो महा-रेताः सिद्धः सर्वादिरग्नि-दः}

\twolineshloka
{वसुर्वसु-मनाः सत्यः सर्व-पाप-हरो ह-रः}
{अमृतः शाश्वतः शान्तो बाण-हस्तः प्रताप-वान्}

\twolineshloka
{कमण्डलु-धरो धन्वी वेदाङ्गो वेद-विन्मुनिः}
{भ्राजिष्णुर्भोजनं भोक्ता लोक-नेता दुराधरः}

\twolineshloka
{अतीन्द्रियो महा-मायः सर्वावासश्चतुष्पथः}
{काल-योगी महा-नादो महोत्साहो महा-बलः}

\twolineshloka
{महा-बुद्धिर्महा-वीर्यो भूत-चारी पुरंदरः}
{निशा-चरः प्रेत-चारि-महा-शक्तिर्महा-द्युतिः}

\twolineshloka
{अनिर्देश्य-वपुः श्री-मान् सर्व-हार्यमितो गतिः}
{बहु-श्रुतो बहु-मयो नियतात्मा भवोद्भवः}

\twolineshloka
{ओजस्तेजोद्युति-करो नर्तकः सर्व-कामकः}
{नृत्य-प्रियो नृत्य-नृत्यः प्रकाशात्मा प्रतापनः}

\twolineshloka
{बुद्ध-स्पष्टाक्षरो मन्त्रः सन्मानः सार-सम्प्लवः}
{युगादि-कृद् युगावर्तो गम्भीरो वृष-वाहनः}

\twolineshloka
{इष्टो विशिष्टः शिष्टेष्टः शरभः शर-भो धनुः}
{अपां निधिरधिष्ठान-विजयो जय-काल-वित्}

\twolineshloka
{प्रतिष्ठितः प्रमाण-ज्ञो हिरण्य-कवचो हरिः}
{विरोचनः सुर-गणो विद्येशो विबुधाश्रयः}

\twolineshloka
{बाल-रूपो बलोन्माथी विवर्तो गहनो गुरुः}
{करणं कारणं कर्ता सर्व-बन्ध-विमोचनः}

\twolineshloka
{विद्वत्तमो वीत-भयो विश्व-भर्ता निशा-करः}
{व्यवसायो व्यवस्थानः स्थान-दो जगदादि-जः}

\twolineshloka
{दुन्दुभो ललितो विश्वो भवात्मात्मनि संस्थितः}
{वीरेश्वरो वीरभद्रो वीर-हा वीर-भृद्-विराट्}

\twolineshloka
{वीर-चूडा-मणिर्वेत्ता तीव्र-नादो नदी-धरः}
{आज्ञा-धारस्त्रिशूली च शिपि-विष्टः शिवालयः}

\twolineshloka
{वालखिल्यो महा-चापस्तिग्मांशुर्निधिरव्ययः}
{अभिरामः सु-शरणः सुब्रह्मण्यः सुधा-पतिः}

\twolineshloka
{मघवान् कौशिको गो-मान् वि-श्रामः सर्व-शासनः}
{ललाटाक्षो विश्व-देहः सारः संसार-चक्र-भृत्}

\twolineshloka
{अमोघ-दण्डी मध्य-स्थो हिरण्यो ब्रह्म-वर्चसी}
{परमार्थः पर-मयः शम्बरो व्याघ्र-कोऽनलः}

\twolineshloka
{रुचिर्वर-रुचिर्वन्द्यो वाचस्पतिरहर्पतिः}
{रविर्विरोचनः स्कन्धः शास्ता वैवस्वतोऽजनः}

\twolineshloka
{युक्तिरुन्नत-कीर्तिश्च शान्त-रागः परा-जयः}
{कैलास-पति-कामारिः सविता रवि-लोचनः}

\twolineshloka
{विद्वत्-तमो वीत-भयो विश्व-हर्ताऽनिवारितः}
{नित्यो नियत-कल्याणः पुण्य-श्रवण-कीर्तनः}

\twolineshloka
{दूर-श्रवा विश्व-सहो ध्येयो दुःस्वप्न-नाशनः}
{उत्तारको दुष्कृति-हा दुर्धर्षो दुःसहोऽभयः}

\twolineshloka
{अनादिर्भूर्भुवो लक्ष्मीः किरीटि-त्रिदशाधिपः}
{विश्व-गोप्ता विश्व-भर्ता सुधीरो रुचिराङ्गदः}

\twolineshloka
{जननो जन-जन्मादिः प्रीति-मान् नीतिमान् नयः}
{विशिष्टः काश्यपो भानुर्भीमो भीम-पराक्रमः}

\twolineshloka
{प्रणवः सप्तधा-चारो महा-कायो महा-धनुः}
{जन्माधिपो महा-देवः सकलागम-पार-गः}

\twolineshloka
{तत्त्वातत्त्व-विवेकात्मा विभूष्णुर्भूति-भूषणः}
{ऋषिर्ब्राह्मण-विज्जिष्णुर्जन्म-मृत्यु-जरातिगः}

\twolineshloka
{यज्ञो यज्ञ-पतिर्यज्वा यज्ञान्तोऽमोघ-विक्रमः}
{महेन्द्रो दुर्भरः सेनी यज्ञाङ्गो यज्ञ-वाहनः}

\twolineshloka
{पञ्च-ब्रह्म-समुत्पत्तिर्विश्वेशो विमलोदयः}
{आत्म-योनिरनाद्यन्तो षड्विंशत्-सप्त-लोक-धृक्}

\twolineshloka
{गायत्री-वल्लभः प्रांशुर्विश्वावासः प्रभा-करः}
{शिशुर्गिरि-रतः सम्राट्-सुषेणः सुर-शत्रु-हा}

\twolineshloka
{अ-मोघोऽरिष्ट-मथनो मुकुन्दो विगत-ज्वरः}
{स्वयं-ज्योतिरनु-ज्योतिरात्म-ज्योतिरचञ्चलः}

\twolineshloka
{पिङ्गलः कपिल-श्मश्रुः शास्त्र-नेत्रस्त्रयी-तनुः}
{ज्ञान-स्कन्धो महा-ज्ञानी निरुत्पत्तिरुपप्लवः}

\twolineshloka
{भगो विवस्वानादित्यो योगाचार्यो बृहस्पतिः}
{उदार-कीर्तिरुद्योगी सद्-योगी सदसन्मयः}

\twolineshloka
{नक्षत्र-माली राकेशः साधिष्ठानः षडाश्रयः}
{पवित्र-पाणिः पापारिर्मणि-पूरो मनोगतिः}

\twolineshloka
{हृत्-पुण्डरीकमासीनः शुक्लः शान्तो वृषा-कपिः}
{विष्णुर्ग्रह-पतिः कृष्णः समर्थोऽनर्थ-नाशनः}

\twolineshloka
{अधर्म-शत्रुरक्षय्यः पुरु-हूतः पुरु-ष्टुतः}
{ब्रह्म-गर्भो बृहद्-गर्भो धर्म-धेनुर्धनागमः}

\twolineshloka
{जगद्धितैषि-सुगतः कुमारः कुशलागमः}
{हिरण्य-वर्णो ज्योतिष्मान् नाना-भूत-धरो ध्वनिः}

\twolineshloka
{अरोगो नियमाध्यक्षो विश्वामित्रो द्विजोत्तमः}
{बृहज्ज्योतिः सुधामा च महा-ज्योतिरनुत्तमः}

\twolineshloka
{मातामहो मातरिश्वा नभस्वान् नाग-हार-धृक्}
{पुलस्त्यः पुलहोऽगस्त्यो जातूकर्ण्यः पराशरः}

\twolineshloka
{निरावरण-धर्म-ज्ञो विरिञ्चो विष्टर-श्रवाः}
{आत्म-भूरनिरुद्धोऽत्रि-ज्ञान-मूर्तिर्महा-यशाः}

\twolineshloka
{लोक-चूडामणिर्वीरः चण्ड-सत्य-पराक्रमः}
{व्याल-कल्पो महा-कल्पो महा-वृक्षः कला-धरः}

\twolineshloka
{अलं-करिष्णुस्त्वचलो रोचिष्णुर्विक्रमोत्तमः}
{आशु-शब्द-पतिर्वेगी प्लवनः शिखि-सारथिः}

\twolineshloka
{असंसृष्टोऽतिथिः शक्रः प्रमाथी पाप-नाशनः}
{वसु-श्रवाः कव्य-वाहः प्रतप्तो विश्व-भोजनः}

\twolineshloka
{जर्यो जराधि-शमनो लोहितश्च तनूनपात्}
{पृषदश्वो नभो योनिः सुप्रतीकस्तमिस्र-हा}

\twolineshloka
{निदाघस्तपनो मेघः पक्षः पर-पुरं-जयः}
{मुखानिलः सुनिष्पन्नः सुरभिः शिशिरात्मकः}

\twolineshloka
{वसन्तो माधवो ग्रीष्मो नभस्यो बीज-वाहनः}
{अङ्गिरा मुनिरात्रेयो वि-मलो विश्व-वाहनः}

\twolineshloka
{पावनः पुरुजिच्छक्रस्त्रि-विद्यो नर-वाहनः}
{मनो-बुद्धिरहङ्कारः क्षेत्र-ज्ञः क्षेत्र-पालकः}

\twolineshloka
{तेजो-निधिर्ज्ञान-निधिर्विपाको विघ्न-कारकः}
{अ-धरोऽनुत्तरो ज्ञेयो ज्येष्ठो निःश्रेयसालयः}

\twolineshloka
{शैलो नगस्तनुर्दोहो दानवारिररिं-दमः}
{चारु-धीर्जनकश्चारु-विशल्यो लोक-शल्य-कृत्}

\twolineshloka
{चतुर्वेदश्चतुर्भावश्चतुरश्चतुर-प्रियः}
{आम्नायोऽथ समाम्नायस्तीर्थ-देव-शिवालयः}

\twolineshloka
{बहु-रूपो महा-रूपः सर्व-रूपश्चराचरः}
{न्याय-निर्वाहको न्यायो न्याय-गम्यो निरञ्जनः}

\twolineshloka
{सहस्र-मूर्धा देवेन्द्रः सर्व-शस्त्र-प्रभञ्जनः}
{मुण्डो वि-रूपो वि-कृतो दण्डी दान्तो गुणोत्तमः}

\twolineshloka
{पिङ्गलाक्षोऽथ हर्यक्षो नील-ग्रीवो निरामयः}
{सहस्र-बाहुः सर्वेशः शरण्यः सर्व-लोक-भृत्}

\twolineshloka
{पद्मासनः परं-ज्योतिः परावर-परं-फलः}
{पद्म-गर्भो महा-गर्भो विश्व-गर्भो विचक्षणः}

\twolineshloka
{परावर-ज्ञो बीजेशः सु-मुखः सु-महा-स्वनः}
{देवासुर-गुरुर्देवो देवासुर-नमस्कृतः}

\twolineshloka
{देवासुर-महा-मात्रो देवासुर-महाश्रयः}
{देवादि-देवो देवर्षिर्देवासुर-वर-प्रदः}

\twolineshloka
{देवासुरेश्वरो दिव्यो देवासुर-महेश्वरः}
{सर्व-देव-मयोऽचिन्त्यो देवतात्माऽऽत्म-सम्भवः}

\twolineshloka
{ईड्योऽनीशः सुर-व्याघ्रो देव-सिंहो दिवा-करः}
{विबुधाग्र-वर-श्रेष्ठः सर्व-देवोत्तमोत्तमः}

\twolineshloka
{शिव-ज्ञान-रतः श्रीमान् शिखि-श्री-पर्वत-प्रियः}
{जय-स्तम्भो विशिष्टम्भो नरसिंह-निपातनः}

\twolineshloka
{ब्रह्म-चारी लोक-चारी धर्म-चारी धनाधिपः}
{नन्दी नन्दीश्वरो नग्नो नग्न-व्रत-धरः शुचिः}

\twolineshloka
{लिङ्गाध्यक्षः सुराध्यक्षो युगाध्यक्षो युगावहः}
{स्व-वशः स-वशः स्वर्ग-स्वरः स्वर-मय-स्वनः}

\twolineshloka
{बीजाध्यक्षो बीज-कर्ता धन-कृद् धर्म-वर्धनः}
{दम्भोऽदम्भो महा-दम्भः सर्व-भूत-महेश्वरः}

\twolineshloka
{श्मशान-निलयस्तिष्यः सेतुरप्रतिमाकृतिः}
{लोकोत्तर-स्फुटालोकस्त्र्यम्बको नाग-भूषणः}

\twolineshloka
{अन्धकारिर्मख-द्वेषी विष्णु-कन्धर-पातनः}
{वीत-दोषोऽक्षय-गुणो दक्षारिः पूष-दन्त-हृत्}

\twolineshloka
{धूर्जटिः खण्ड-परशुः स-कलो निष्कलोऽनघः}
{आ-धारः सकलाधारः पाण्डुराभो मृडो नटः}

\twolineshloka
{पूर्णः पूरयिता पुण्यः सुकुमारः सुलोचनः}
{साम-गेयः प्रिय-करः पुण्य-कीर्तिर-नाम-यः}

\twolineshloka
{मनोजवस्तीर्थ-करो जटिलो जीवितेश्वरः}
{जीवितान्त-करो नि-त्यो वसु-रेता वसु-प्रियः}

\twolineshloka
{सद्-गतिः सत्-कृतिः सक्तः काल-कण्ठः कला-धरः}
{मानी मान्यो महा-कालः सद्-भूतिः सत्-परायणः}

\twolineshloka
{चन्द्र-सञ्जीवनः शास्ता लोक-गूढोऽमराधिपः}
{लोक-बन्धुर्लोक-नाथः कृत-ज्ञः कृति-भूषणः}

\twolineshloka
{अनपाय्यक्षरः कान्तः सर्व-शास्त्र-भृतां वरः}
{तेजो-मयो द्युति-धरो लोक-मायोऽग्रणीरणुः}

\twolineshloka
{शुचि-स्मितः प्रसन्नात्मा दुर्जयो दुरतिक्रमः}
{ज्योतिर्मयो निराकारो जगन्नाथो जलेश्वरः}

\twolineshloka
{तुम्ब-वीणी महा-कायो वि-शोकः शोक-नाशनः}
{त्रि-लोकात्मा त्रि-लोकेशः शुद्धः शुद्धीरथाक्ष-जः}

\twolineshloka
{अव्यक्त-लक्षणोऽ-व्यक्तो व्यक्ताव्यक्तो विशां पतिः}
{वर-शीलो वर-तुलो मानो मान-धनो मयः}

\twolineshloka
{ब्रह्मा विष्णुः प्रजा-पालो हंसो हंस-गतिर्यमः}
{वेधा धाता विधाता च अत्ता हर्ता चतुर्मुखः}

\twolineshloka
{कैलास-शिखरावासी सर्वावासी सतां गतिः}
{हिरण्य-गर्भो हरिणः पुरुषः पूर्व-जः पिता}

\twolineshloka
{भूतालयो भूत-पतिर्भूति-दो भुवनेश्वरः}
{संयोगी योग-विद्-ब्रह्मा ब्रह्मण्यो ब्राह्मण-प्रियः}

\twolineshloka
{देव-प्रियो देव-नाथो देव-ज्ञो देव-चिन्तकः}
{विषमाक्षः कलाध्यक्षो वृषाङ्को वृष-वर्धनः}

\twolineshloka
{निर्मदो निरहङ्कारो निर्मोहो निरुपद्रवः}
{दर्प-हा दर्पितो दृप्तः सर्वर्तु-परिवर्तकः}

\twolineshloka
{सप्तजिह्वः सहस्रार्चिः स्निग्धः प्रकृति-दक्षिणः}
{भूत-भव्य-भवन्नाथः प्रभवो भ्रान्ति-नाशनः}

\twolineshloka
{अर्थोऽनर्थो महा-कोशः पर-कार्यैक-पण्डितः}
{निष्कण्टकः कृतानन्दो निर्व्याजो व्याज-मर्दनः}

\twolineshloka
{सत्त्ववान् सात्त्विकः सत्य-कीर्ति-स्तम्भ-कृतागमः}
{अकम्पितो गुण-ग्राही नैकात्मा नैक-कर्म-कृत्}

\twolineshloka
{सु-प्रीतः सु-मुखः सूक्ष्मः सु-करो दक्षिणोऽनलः}
{स्कन्धः स्कन्ध-धरो धुर्यः प्रकटः प्रीति-वर्धनः}

\twolineshloka
{अपराजितः सर्व-सहो विदग्धः सर्व-वाहनः}
{अधृतः स्व-धृतः साध्यः पूर्त-मूर्तिर्यशोधरः}

\twolineshloka
{वराह-शृङ्ग-धृग् वायुर्बलवानेक-नायकः}
{श्रुति-प्रकाशः श्रुति-मानेक-बन्धुरनेक-धृक्}

\twolineshloka
{श्री-वल्लभ-शिवारम्भः शान्त-भद्रः समञ्जसः}
{भू-शयो भूति-कृद्-भूतिर्भूषणो भूत-वाहनः}

\twolineshloka
{अकायो भक्त-काय-स्थः काल-ज्ञानी कला-वपुः}
{सत्य-व्रत-महा-त्यागी निष्ठा-शान्ति-परायणः}

\twolineshloka
{परार्थ-वृत्तिर्वर-दो विविक्तः श्रुति-सागरः}
{अनिर्विण्णो गुण-ग्राही कलङ्काङ्कः कलङ्क-हा}

\twolineshloka
{स्वभाव-रुद्-द्रो मध्य-स्थः शत्रु-घ्नो मध्य-नाशकः}
{शिखण्डी कवची शूली चण्डी मुण्डी च कुण्डली}

\twolineshloka
{मेखली कवची खड्गी मायी संसार-सारथिः}
{अमृत्युः सर्व-दृक् सिंहस्तेजो-राशिर्महा-मणिः}

\twolineshloka
{अ-सङ्ख्येयोऽप्रमेयात्मा वीर्यवान् कार्य-कोविदः}
{वेद्यो वेदार्थ-विद्-गोप्ता सर्वाचारो मुनीश्वरः}

\twolineshloka
{अ-नुत्तमो दुराधर्षो मधुरः प्रिय-दर्शनः}
{सुरेशः शरणं सर्वः शब्द-ब्रह्म-सतां गतिः}

\twolineshloka
{काल-भक्षः कलङ्कारिः कङ्कणी-कृत-वासुकिः}
{महेष्वासो मही-भर्ता निष्कलङ्को वि-शृङ्खलः}

\twolineshloka
{द्यु-मणिस्तरणिर्धन्यः सिद्धि-दः सिद्धि-साधनः}
{निवृत्तः संवृतः शिल्पो व्यूढोरस्को महा-भुजः}

\twolineshloka
{एक-ज्योतिर्निरातङ्को नरो नारायण-प्रियः}
{निर्लेपो निष्प्रपञ्चात्मा निर्व्यग्रो व्यग्र-नाशनः}

\twolineshloka
{स्तव्य-स्तव-प्रियः स्तोता व्यास-मूर्तिरनाकुलः}
{निरवद्य-पदोपायो विद्या-राशिरविक्रमः}

\twolineshloka
{प्रशान्त-बुद्धिरक्षुद्रः क्षुद्र-हा नित्य-सुन्दरः}
{धैर्याग्र्य-धुर्यो धात्रीशः शाकल्यः शर्वरी-पतिः}

\twolineshloka
{परमार्थ-गुरुर्दृष्टिर्गुरुराश्रित-वत्सलः}
{रसो रस-ज्ञः सर्व-ज्ञः सर्व-सत्त्वावलम्बनः}


॥इति श्री-लिङ्ग-महा-पुराणे पूर्व-भागे अष्टनवतितमे अध्याये भगवता विष्णुना प्रोक्तं श्री-शिव-सहस्र-नाम-स्तोत्रम्॥


\uvacha{सूत उवाच}
{एवं नाम्नां सहस्रेण तुष्टाव वृषभ-ध्वजम्॥१५९॥}%॥ १५९ ॥

\resetShloka
\addtocounter{shlokacount}{159}


\twolineshloka
{स्नापयामास च विभुः पूजयामास पङ्कजैः}
{परीक्षार्थं हरेः पूजा-कमलेषु महेश्वरः}%॥ १६० ॥

\twolineshloka
{गोपयामास कमलं तदैकं भुवनेश्वरः}
{हृत-पुष्पो हरिस्तत्र किमिदं त्वभ्यचिन्तयत्}%॥ ६१ ॥

\twolineshloka
{ज्ञात्वा स्व-नेत्रमुद्धृत्य सर्व-सत्त्वावलम्बनम्}
{पूजयामास भावेन नाम्ना तेन जगद्-गुरुम्}%॥ १६२ ॥

\twolineshloka
{ततस्तत्र विभुर्दृष्ट्वा तथा-भूतं हरो हरिम्}
{तस्मादवतताराशु मण्डलात् पावकस्य च}%॥ १६३ ॥

\twolineshloka
{कोटि-भास्कर-सङ्काशं जटा-मुकुट-मण्डितम्}
{ज्वाला-मालावृतं दिव्यं तीक्ष्ण-दंष्ट्रं भयं-करम्}%॥ १६४ ॥

\twolineshloka
{शूल-टङ्क-गदा-चक्र-कुन्त-पाश-धरं हरम्}
{वरदाभय-हस्तं च द्वीपि-चर्मोत्तरीयकम्}%॥ १६५ ॥

\twolineshloka
{इत्थं-भूतं तदा दृष्ट्वा भवं भस्म-विभूषितम्}
{हृष्टो नमश्चकाराशु देव-देवं जनार्दनः}%॥ १६६ ॥

\twolineshloka
{दुद्रुवुस्तं परिक्रम्य सेन्द्रा देवास्त्रि-लोचनम्}
{चचाल ब्रह्म-भुवनं चकम्पे च वसुन्धरा}%॥ १६७ ॥

\twolineshloka
{ददाह तेजस्तच्छम्भोः प्रान्तं वै शत-योजनम्}
{अधस्ताच्चोर्ध्वतश्चैव हाहेत्यकृत भूतले}%॥ १६८ ॥

\twolineshloka
{तदा प्राह महादेवः प्रहसन्निव शङ्करः}
{सम्प्रेक्ष्य प्रणयाद् विष्णुं कृताञ्जलि-पुटं स्थितम्}%॥ १६९ ॥

\twolineshloka
{ज्ञातं मयेदमधुना देव-कार्यं जनार्दन}
{सुदर्शनाख्यं चक्रं च ददामि तव शोभनम्}%॥ १७० ॥

\twolineshloka
{यद् रूपं भवता दृष्टं सर्व-लोक-भयं-करम्}
{हिताय तव यत्नेन तव भावाय सुव्रत}%॥ १७१ ॥

\twolineshloka
{शान्तं रणाजिरे विष्णो देवानां दुःख-साधनम्}
{शान्तस्य चास्त्रं शान्तं स्याच्छान्तेनास्त्रेण किं फलम्}%॥ १७२ ॥

\twolineshloka
{शान्तस्य समरे चास्त्रं शान्तिरेव तपस्विनम्}
{योद्धुः शान्त्या बल-च्छेदः परस्य बल-वृद्धि-दः}%॥ १७३ ॥

\twolineshloka
{देवैरशान्तैर्यद् रूपं मदीयं भावयाव्ययम्}
{किमायुधेन कार्यं वै योद्धुं देवारि-सूदन}%॥ १७४ ॥

\twolineshloka
{क्षमा युधि न कार्या वै योद्धुं देवारि-सूदन}
{अनागते व्यतीते च दौर्बल्ये स्व-जनोत्करे}%॥ १७५ ॥

\twolineshloka
{अकालिके त्वधर्मे च अनर्थे वाऽरि-सूदन}
{एवमुक्त्वा ददौ चक्रं सूर्यायुत-सम-प्रभम्}%॥ १७६ ॥

\twolineshloka
{नेत्रं च नेता जगतां प्रभुर्वै पद्म-सन्निभम्}
{तदा-प्रभृति तं प्राहुः पद्माक्षमिति सु-व्रतम्}%॥ १७७ ॥

\twolineshloka
{दत्त्वैनं नयनं चक्रं विष्णवे नील-लोहितः}
{पस्पर्श च कराभ्यां वै सुशुभाभ्यामुवाच ह}%॥ १७८ ॥

\twolineshloka
{वरदोऽहं वर-श्रेष्ठ वरान् वरय चेप्सितान्}
{भक्त्या वशी-कृतो नूनं त्वयाऽहं पुरुषोत्तम}%॥ १७९ ॥

\twolineshloka
{इत्युक्तो देव-देवेन देव-देवं प्रणम्य तम्}
{त्वयि भक्तिर्महादेव प्रसीद वरमुत्तमम्}%॥ १८० ॥

\twolineshloka
{नान्यमिच्छामि भक्तानामार्तयो नास्ति यत् प्रभो}
{तच्छ्रुत्वा वचनं तस्य दयावान् सुतरां भवः}%॥ १८१ ॥

\twolineshloka
{पस्पर्श च ददौ तस्मै श्रद्धां शीतांशु-भूषणः}
{प्राह चैवं महा-देवः परमात्मानमच्युतम्}%॥ १८२ ॥

\twolineshloka
{मयि भक्तश्च वन्द्यश्च पूज्यश्चैव सुरासुरैः}
{भविष्यसि न सन्देहो मत्-प्रसादात् सुरोत्तम}%॥ १८३ ॥

\twolineshloka
{यदा सती दक्ष-पुत्री विनिन्द्यैव सुलोचना}
{मातरं पितरं दक्षं भविष्यति सुरेश्वरी}%॥ १८४ ॥

\twolineshloka
{दिव्या हैमवती विष्णो तदा त्वमपि सुव्रत}
{भगिनीं तव कल्याणीं देवीं हैमवतीमुमाम्}%॥ १८५ ॥

\twolineshloka
{नियोगाद् ब्रह्मणः साध्वीं प्रदास्यसि ममैव ताम्}
{मत्-सम्बन्धी च लोकानां मध्ये पूज्यो भविष्यसि}%॥ १८६ ॥

\twolineshloka
{मां दिव्येन च भावेन तदा प्रभृति शङ्करम्}
{द्रक्ष्यसे च प्रसन्नेन मित्र-भूतमिवात्मना}%॥ १८७ ॥

\twolineshloka
{इत्युक्त्वाऽन्तर्दधे रुद्रो भगवान् नील-लोहितः}
{जनार्दनोऽपि भगवान् देवानामपि सन्निधौ}%॥ १८८ ॥

\twolineshloka
{अयाचत महादेवं ब्रह्माणं मुनिभिः समम्}
{मया प्रोक्तं स्तवं दिव्यं पद्म-योने सुशोभनम्}%॥ १८९ ॥

\twolineshloka
{यः पठेच्छृणुयाद् वाऽपि श्रावयेद् वा द्विजोत्तमान्}
{प्रतिनाम्नि हिरण्यस्य दत्तस्य फलमाप्नुयात्}%॥ १९० ॥

\twolineshloka
{अश्वमेध-सहस्रेण फलं भवति तस्य वै}
{घृताद्यैः स्नापयेद् रुद्रं स्थाल्या वै कलशैः शुभैः}%॥ १९१ ॥

\twolineshloka
{नाम्नां सहस्रेणानेन श्रद्धया शिवमीश्वरम्}
{सोऽपि यज्ञ-सहस्रस्य फलं लब्ध्वा सुरेश्वरैः}%॥ १९२ ॥

\twolineshloka
{पूज्यो भवति रुद्रस्य प्रीतिर्भवति तस्य वै}
{तथाऽस्त्विति तथा प्राह पद्मयोनिर्जनार्दनम्}%॥ १९३ ॥

\threelineshloka
{जग्मतुः प्रणिपत्यैनं देव-देवं जगद्-गुरुम्} 
{तस्मान्नाम्नां सहस्रेण पूजयेदनघो द्विजाः}
{जपेन्नाम्नां सहस्रं च स याति परमां गतिम्}% ॥ १९४ ॥


{॥इति श्री-लिङ्ग-महा-पुराणे पूर्व-भागे सहस्र-नामभिः पूजनाद् विष्णु-चक्र-लाभो नामाष्टनवतितमोऽध्यायः॥}%॥ ९८ ॥



\fourlineindentedshloka*
{दुःस्वप्न-दुःशकुन-दुर्गति-दौर्मनस्य}
{दुर्भिक्ष-दुर्व्यसन-दुःसह-दुर्यशांसि}
{उत्पात-ताप-विषभीतिम् असद्‌-ग्रहार्तिम्}
{व्याधींश्च नाशयतु मे जगतामधीशः}

॥इति श्री-शिवसहस्रनामस्तोत्रं सम्पूर्णम्॥