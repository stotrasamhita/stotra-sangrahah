\sect{श्रीकाञ्चीकामकोटिपीठजगद्गुरुपरम्परास्तवः}

(पञ्चषष्टितमैः पीठाधिपतिभिः श्रीमत्सुदर्शनमहादेवेन्द्रसरस्वतीश्रीचरणैः प्रणीतः)

\fourlineindentedshloka
{नारायणं पद्मभुवं वसिष्ठं}
{शक्तिं च तत्पुत्रपराशरं च}
{व्यासं शुकं गौडपदं महान्तं}
{गोविन्दयोगीन्द्रमथास्य शिष्यम्}% ॥ १ ॥

\twolineshloka
{श्रीशङ्कराचार्यमथास्य पद्मपादं च हस्तामलकं च शिष्यम्}
{तं तोटकं वार्तिककारमन्यान् अस्मद्गुरून् सन्ततमानतोऽस्मि}% ॥ २ ॥

\twolineshloka
{सदाशिवसमारम्भां शङ्कराचार्यमध्यमाम्}
{अस्मदाचार्यपर्यन्तां वन्दे गुरुपरम्पराम्}% ॥ ३ ॥

\twolineshloka
{सर्वतन्त्रस्वतन्त्राय सदाऽऽत्माद्वैतवेदिने}
{श्रीमते शङ्करार्याय वेदान्तगुरवे नमः}% ॥ ४ ॥

\twolineshloka
{अविप्लुतब्रह्मचर्यान् अन्वितेन्द्रसरस्वतीन्}
{आत्तमिथ्यावारपथान् अद्वैताचार्यसङ्कथान्}% ॥ ५ ॥

\twolineshloka
{आसेतुहिमवच्छैलं सदाचारप्रवर्तकान्}
{जगद्गुरून् स्तुमः काञ्चीशारदामठसंश्रयान्}% ॥ ६ ॥

\twolineshloka
{पवित्रितेतराद्वैतमठपीठीशिरोभुवे}
{श्रीकाञ्चीशारदापीठगुरवे भवभीरवे}% ॥ ७ ॥

\twolineshloka
{वार्तिकादिब्रह्मविद्याकर्त्रे ब्रह्मावतारिणे}
{सुरेश्वराचार्यनाम्ने योगीन्द्राय नमो नमः}% ॥ ८ ॥

\twolineshloka
{अपोऽश्नन्नेव जैनान् य आप्राग्ज्योतिषमाच्छिनत्}
{शिशुमाचार्यवाग्वेणीरयरोधिमहोबलम्}% ॥ ९

\twolineshloka
{सङ्क्षेपशारीरमुखप्रबन्धविवृताद्वयम्}
{ब्रह्मस्वरूपार्यभाष्यशान्त्याचार्यकपण्डितम्}% ॥ १० ॥

\twolineshloka
{सर्वज्ञचन्द्रनाम्ना च सर्वतो भुवि विश्रुतम्}
{सर्वज्ञसद्गुरुं वन्दे सर्वज्ञमिव भूगतम्}% ॥ ११ ॥

\twolineshloka
{मेधाविनं सत्यबोधं व्याधूतविमतोच्चयम्}
{प्राच्यभाष्यत्रयव्याख्याप्रवीणं प्रभुमाश्रये}% ॥ १२ ॥

\twolineshloka
{ज्ञानानन्दमुनीन्द्रार्यं ज्ञानोत्तमपराभिधम्}
{चन्द्रचूडपदासक्तं चन्द्रिकाकृतमाश्रये}% ॥ १३ ॥

\twolineshloka
{शुद्धानन्दमुनीन्द्राणां विद्धार्हतमतत्विषाम्}
{आनन्दज्ञानसेव्यानाम् आलम्बे चरणाम्बुजम्}% ॥ १४ ॥

\twolineshloka
{सर्वशाङ्करभाष्यौघभाष्यकर्तारमद्वयम्}
{सर्ववार्तिकसद्वृत्तिकृतं श्रीशैलगं भजे}% ॥ १५ ॥

\twolineshloka
{कैवल्यानन्दयोगीन्द्रान् केवलं राजयोगिनः}
{कैवल्यमात्रनिरतान् कलयेम जगद्गुरून्}% ॥ १६ ॥

\twolineshloka
{श्रीकृपाशङ्करार्याणां मर्यादातीततेजसाम्}
{षण्मताचार्यकजुषाम् अङ्घ्रिद्वन्द्वमहं श्रये}% ॥ १७ ॥

\twolineshloka
{महिष्ठाय नमस्तस्मै महादेवाय योगिने}
{सुरेश्वरापराख्याय गुरवे दोषभीरवे}% ॥ १८ ॥

\twolineshloka
{स्तुमः सदा शिवानन्दचिद्घनेन्द्रसरस्वतीन्}
{कामाक्षीचन्द्रमौल्यर्चाकलनैकलसन्मतीन्}% ॥ १९ ॥

\twolineshloka
{सार्वभौमाभिधमहाव्रतचर्यापरायणान्}
{वन्दे जगद्गुरूंश्चन्द्रशेखरेन्द्रसरस्वतीन्}% ॥ २० ॥

\twolineshloka
{समाद्वात्रिंशदत्युग्रकाष्ठमौनसमाश्रयान्}
{जितमृत्यून् महालिङ्गभूतान् सच्चिद्घनान् नुमः}% ॥ २१ ॥

\twolineshloka
{महाभैरवदुस्तन्त्रदुर्दान्तध्वान्तभास्करान्}
{विद्याघनान् नमस्यामि सर्वविद्याविचक्षणान्}% ॥ २२ ॥

\twolineshloka
{आचार्यपदपाथोजपरिचर्यापरायणम्}
{गङ्गाधरं नमस्यामः सदा गङ्गाधरार्चकम्}% ॥ २३ ॥

\twolineshloka
{जगज्जयिसुसौराष्ट्रजरदृष्टिमदापहान्}
{शकसिल्हकदर्पघ्नान् ईडीमहि महायतीन्}% ॥ २४ ॥

\twolineshloka
{चतुस्समुद्रीक्रोडस्थवर्णाश्रमविचारकान्}
{श्रितविप्रव्रजस्कन्धसुवर्णान्दोलिकाचरान्}% ॥ २५ ॥

\twolineshloka
{प्रत्यहं ब्रह्मसाहस्रसन्तर्पणधृतव्रतान्}
{सदाशिवसमाह्वानान् स्मरामः सद्गुरून् सदा}% ॥ २६ ॥

\twolineshloka
{मायालोकायतीभूतबृहस्पतिमदापहान्}
{वन्दे सुरेन्द्रवन्द्याङ्घ्रीन् श्रीसुरेन्द्रसरस्वतीन्}% ॥ २७ ॥

\twolineshloka
{श्रीविद्याकरुणालब्धब्रह्मविद्याहृतामयान्}
{वन्दे वशंवदप्राणान् मुनीन् विद्याघनान् मुहुः}% ॥ २८ ॥

\twolineshloka
{विद्याघनकृपालब्धसर्ववेदान्तविस्तरम्}
{कौतस्कुतोत्पातकेतुं निश्शङ्कं नौमि शङ्करम्}% ॥ २९ ॥

\twolineshloka
{चन्द्रचूडपदध्यानप्राप्तानन्दमहोदधीन्}
{यतीन्द्रांश्चन्द्रचूडेन्द्रान् स्मरामि मनसा सदा}% ॥ ३० ॥

\twolineshloka
{नमामि परिपूर्णश्रीबोधान् ग्रावाभिलापकान्}
{यदीक्षणात् पलायन्त प्राणिनामामयाधयः}% ॥ ३१ ॥

\twolineshloka
{सच्चित्सुखान् प्रपद्येऽहं सुखम् आप्तगुहास्थितीन्} % ।
{चित्सुखाचार्यमीडेऽहं सत्सुखं कोङ्कणाश्रयम्} %॥

\twolineshloka
{भजे श्रीसच्चिदानन्दघनेन्द्रान् रससाधनात्}
{लिङ्गात्मना परिणतान् प्रभासे योगसंश्रिते}% ॥ ३३ ॥

\twolineshloka
{भगवत्पादपादाब्जासक्तिनिर्णिक्तमानसान्}
{प्रज्ञाघनं चिद्विलासं महादेवं च मैथिलम्}% ॥ ३४ ॥

\twolineshloka
{पूर्णबोधं च बोधं च भक्तियोगप्रवर्तकम्}
{ब्रह्मानन्दघनेन्द्रं च नमामि नियतात्मनः}% ॥ ३५ ॥

\twolineshloka
{चिदानन्दघनेन्द्राणां लम्बिकायोगसेविनाम्}
{जीर्णपर्णाशिनां पादौ प्रपद्ये मनसा सदा}% ॥ ३६ ॥

\twolineshloka
{सच्चिदानन्दनामानं शिवार्चनपरायणम्}
{भाषापञ्चदशीप्राज्ञं भावयामि सदा मुदा}% ॥ ३७ ॥

\twolineshloka
{भूप्रदक्षिणकर्मैकसक्तं श्रीचन्द्रशेखरम्}
{त्रातदावाग्निसन्दग्धकिशोरकमुपास्महे}% ॥ ३८ ॥

\twolineshloka
{चित्सुखेन्द्रं सुखेनैव क्रान्तसह्यगुहागृहम्}
{कामरूपचरं नानारूपवन्तमुपास्महे}% ॥ ३९ ॥

\twolineshloka
{निर्दोषसंयमधरान् चित्सुखानन्दतापसान्}
{विद्याघनेन्द्रान् श्रीविद्यावशीकृतजनान् स्तुमः}

\twolineshloka
{शङ्करेन्द्रयतीन्द्राणां पादुके ब्रह्मसम्भृते}
{नमामि शिरसा याभ्यां त्रीन् लोकान् व्यचरन्मुनिः}% ॥ ४१ ॥

\twolineshloka
{सच्चिद्विलासयोगीन्द्रं महादेवेन्द्रमुज्ज्वलम्}
{गङ्गाधरेन्द्रमप्येतान् नौमि वादिशिरोमणीन्}

\twolineshloka
{ब्रह्मानन्दघनेन्द्राख्यांस्तथाऽऽनन्दघनान् अपि}
{पूर्णबोधमहर्षींश्च ज्ञाननिष्ठानुपास्महे} %॥ ४३

\twolineshloka
{वृत्त्याऽऽजगर्या श्रीशैलगुहागृहकृतस्थितीन्}
{श्रीमत्परशिवाभिख्यान् सर्वातीतान् श्रये सदा}% ॥ ४४ ॥

\twolineshloka
{अन्योन्यसदृशान्योन्यौ बोधश्रीचन्द्रशेखरौ}
{प्रणवोपासनासक्तमानसौ मनसा श्रये}% ॥ ४५ ॥

\twolineshloka
{मुक्तिलिङ्गार्चनानन्दविस्मृताशेषवृत्तये}
{चिदम्बररहस्यन्तर्लीनदेहाय योगिने}% ॥ ४६ ॥

\twolineshloka
{अद्वैतानन्दसाम्राज्यविद्रुताशेषपाप्मने}
{अद्वैतानन्दबोधाय नमो ब्रह्म समीयुषे}% ॥ ४७ ॥

\twolineshloka
{श्रये महादेवचन्द्रशेखरेन्द्रमहामुनी}
{महाव्रतसमारब्धकोटिहोमान्तगामिनौ}% ॥ ४८ ॥

\twolineshloka
{विद्यातीर्थसमाह्वानान् श्रीविद्यानाथयोगिनः}
{विद्यया शङ्करप्रख्यान् विद्यारण्यगुरून् भजे}% ॥ ४९ ॥

\twolineshloka
{सच्चिद्घनेन्द्रान् अद्वैतब्रह्मानन्दमुनीनपि}
{सान्द्रानन्दयतीन्द्रांश्च तथाप्यद्वैतशेवधीन्}% ॥ ५० ॥

\twolineshloka
{महादेवशिवाद्वैतसुखानन्दयतीश्वरौ}
{मनसा भावये नित्यं महासंयमधारिणौ}% ॥ ५१ ॥

\twolineshloka
{वीक्षणात् सर्वभूतानां विषव्याधिनिबर्हणम्}
{शिवयोगीश्वरं साक्षाच्चिन्तयामि सदा मुदा}% ॥ ५२ ॥

\twolineshloka
{प्रत्यग्ज्योतिःप्रकाशेन्द्रान् प्रत्यग्ज्योतिरुपासिनः}
{न्यक्कृताशेषदुस्तर्ककार्कश्यान् सततं स्तुमः}% ॥ ५३ ॥

\twolineshloka
{शङ्करानन्दयोगीन्द्रपदपङ्कजयोर्युगम्}
{बुक्कभूपशिरोरत्नं स्मरामि सततं हृदा}% ॥ ५४ ॥

\twolineshloka
{श्रीपूर्णानन्दमौनीन्द्रं नेपालनृपदेशिकम्}
{अव्याहतस्वसञ्चारं संश्रयामि जगद्गुरुम्}% ॥ ५५ ॥

\twolineshloka
{महादेवश्च तच्छिष्यश्चन्द्रशेखरयोग्यपि}
{स्तां मे हृदि सदा धीरावद्वैतमतदेशिकौ}% ॥ ५६ ॥

\twolineshloka
{प्रवीरसेतुभूपालसेविताङ्घ्रिसरोरुहान्}
{भजे सदाशिवेन्द्रश्रीबोधेश्वरगुरून् सदा}% ॥ ५७ ॥

\twolineshloka
{सदाशिवश्रीब्रह्मेन्द्रधृतस्वपदपादुकान्}
{धीरान् परशिवेन्द्रार्यान् ध्यायामि सततं हृदि}% ॥ ५८ ॥

\twolineshloka
{आत्मबोधयतीन्द्राणाम् आशीताचलचारिणाम्}
{अन्यश्रीशङ्कराचार्यधीकृतामङ्घ्रिमाश्रये}% ॥ ५९

\twolineshloka
{भगवन्नामसाम्राज्यलक्ष्मीसर्वस्वविग्रहान्}
{श्रीमद्बोधेन्द्रयोगीन्द्रदेशिकेन्द्रानुपास्महे}% ॥ ६० ॥

\twolineshloka
{अद्वैतात्मप्रकाशाय सर्वशास्त्रार्थवेदिने}
{विधूतसर्वभेदाय नमो विश्वातिशायिने}% ॥ ६१ ॥

\twolineshloka
{आ सप्तमाज्जीर्णपर्णजलवातारुणांशुभिः}
{कृतस्वप्राणयात्राय महादेवाय सन्नतिः}% ॥ ६२ ॥

\twolineshloka
{चोलकेरलचेरौड्रपाण्ड्यकर्णाटकोङ्कणान्}
{महाराष्ट्रान्ध्रसौराष्ट्रमगधादींश्च भूभुजः}% ॥ ६३ ॥

\twolineshloka
{शिष्यान् आसेतुशीताद्रि शासते पुण्यकर्मणे}
{श्रीचन्द्रशेखरेन्द्राय जगतो गुरवे नमः}% ॥ ६४ ॥

\twolineshloka
{निष्पापवृत्तये नित्यनिर्धूतभवकॢप्तये}
{महादेवाय सततं नमोऽस्तु नतरक्षिणे}% ॥ ६५ ॥

\twolineshloka
{श्रीविद्योपासनादार्ढ्यवशीकृतचराचरान्}
{श्रीचन्द्रशेखरेन्द्रार्यान् शङ्करप्रतिमान् नुमः}% ॥ ६६ ॥

\twolineshloka
{श्रीकाञ्चीशारदापीठसंस्थितानाम् इमां क्रमात्}
{स्तुतिं जगद्गुरूणां यः पठेत् स सुखभाग् भवेत्}% ॥ ६७ ॥

\medskip

\textbf{॥परिशिष्टम् - १॥}

(अत्र १-४ पद्यानि पोलहग्रामाभिजनैः श्रीरामशास्त्रिभिः प्रकाशितानि। ५ पद्यं नवषष्टितमैः पीठाधिपतिभिः श्रीमज्जयेन्द्रसरस्वतीश्रीपादैः रचितम्। ६-७ पद्ये शिमिलिग्रामाभिजनैः राधाकृष्णशास्त्रिभिः विरचिते।)


\twolineshloka
{कलानन्दपरब्रह्मानन्दानुभवतुन्दिलान्}
{महादेवेन्द्रयमिनः सततं संश्रये हृदा}% ॥ १ ॥

\fourlineindentedshloka
{प्रतिदिनविहितश्रीचन्द्रमौलीश्वरार्चा-}
{प्रसितसकलदेहः प्रौढपुण्यानुभावः}
{मनसि सततमास्तां श्रीमहादेवनामा}
{मम गुरुरितकाञ्चीशारदापीठसीमा}% ॥ २ ॥

\twolineshloka
{अतिबाल्यविधृतसंयमविमलमनःसेवितेशचरणयुगान्}
{श्रीचन्द्रशेखरेन्द्रान् साम्प्रतमाचार्यशेखरान् प्रणुमः}% ॥ ३ ॥

\twolineshloka
{लक्ष्मीनारायण इति पूर्वाश्रमनामभूषितं शान्तम्}
{ऋग्वेदे सम्यगधीतिनं महादेवमाश्रयामि गुरुम्}% ॥ ४ ॥

\twolineshloka
{अपारकरुणासिन्धुं ज्ञानदं शान्तरूपिणम्}
{श्रीचन्द्रशेखरगुरुं प्रणमामि मुदाऽन्वहम्}% ॥ ५ ॥

\fourlineindentedshloka
{परित्यज्य मौनं वटाधःस्थितिं च}
{व्रजन् भारतस्य प्रदेशात् प्रदेशम्}% ।
{मधुस्यन्दिवाचा जनान् धर्ममार्गे}
{नयन् श्रीजयेन्द्रो गुरुर्भाति चित्ते}% ॥ ६ ॥

\twolineshloka
{नमामः शङ्करान्वाख्यविजयेन्द्रसरस्वतीम्}
{श्रीगुरुं शिष्टमार्गानुनेतारं सन्मतिप्रदम्}% ॥ ७ ॥

\twolineshloka*
{देवे देहे च देशे च भक्त्यारोग्य-सुख-प्रदम्}
{बुध-पामर-सेव्यं तं श्री-जयेन्द्रं नमाम्यहम्}

\twolineshloka*
{नमामः शङ्करान्वाख्य-विजयेन्द्रसरस्वतीम्}
{श्रीगुरुं शिष्टमार्गानुनेतारं सन्मतिप्रदम्}

\twolineshloka*
{सत्यनारायणक्षेत्रात् सत्यव्रतमुपागतम्}
{आश्रये कामकोटीशं सत्यश्रीचन्द्रशेखरम्} 
