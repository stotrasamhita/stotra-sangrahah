% !TeX program = XeLaTeX
% !TeX root = ../../shloka.tex

\sect{हरिहरपुत्राष्टोत्तरशतनामस्तोत्रम्}

\twolineshloka
{महाशास्ता महादेवो महादेवसुतोऽव्ययः}
{लोककर्ता लोकभर्ता लोकहन्ता परात्परः}

\twolineshloka
{त्रिलोकरक्षको धन्वी तपस्वी भूतसैन्यकः}
{मन्त्रवेत्ता महावेत्ता मारुतो जगदीश्वरः}

\twolineshloka
{लोकाध्यक्षोऽग्रणीः श्रीमान् अप्रमेयपराक्रमः}
{सिंहारूढो गजारूढो हयारूढो महेश्वरः}

\twolineshloka
{नानाशास्त्रधरोऽनर्घो नानाविद्याविशारदः}
{नानारूपधरो वीरो नानाप्राणिनिषेवितः}

\twolineshloka
{भूतेशः पूजितो भृत्यो भुजङ्गाभरणोत्तमः}
{इक्षुधन्वी पुष्पबाणो महारूपो महाप्रभुः}

\twolineshloka
{मायादेवीसुतो मान्यो महानीतो महागुणः}
{महाशैवो महारुद्रो वैष्णवो विष्णुपूजकः}

\twolineshloka
{विघ्नेशो वीरभद्रेशो भैरवो षण्मुखध्रुवः}
{मेरुशृङ्गसमासीनो मुनिसङ्घनिषेवितः}

\twolineshloka
{देवो भद्रो जगन्नाथो गणनाथो गणेश्वरः}
{महायोगी महामायी महाज्ञानी महाधिपः}

\twolineshloka
{देवशास्ता भूतशास्ता भीमहासपराक्रमः}
{नागहारश्च नागेशो व्योमकेशः सनातनः}

\twolineshloka
{कालज्ञो निर्गुणो नित्यो नित्यतृप्तो निराश्रयः}
{लोकाश्रयो गुणाधीशश्चतुःषष्टिकलामयः}

\twolineshloka
{ऋग्यजुःसामरूपी च मल्लकासुरभञ्जनः}
{त्रिमूर्तिर्दैत्यमथनो प्रकृतिः पुरुषोत्तमः}

\twolineshloka
{सुगुणश्च महाज्ञानी कामदः कमलेक्षणः}
{कल्पवृक्षो महावृक्षो विद्यावृक्षो विभूतिदः}

\twolineshloka
{संसारतापविच्छेत्ता पशुलोकभयङ्करः}
{रोगहन्ता प्राणदाता परगर्वविभञ्जनः}

\twolineshloka
{सर्वशास्त्रार्थतत्त्वज्ञो नीतिमान् पापभञ्जनः}
{पुष्कलापूर्णसंयुक्तो परमात्मा सतां गतिः}

\twolineshloka
{अनन्तादित्यसङ्काशः सुब्रह्मण्यानुजो बली}
{भक्तानुकम्पी देवेशो भगवान् भक्तवत्सलः}

{॥इति श्री हरिहरपुत्राष्टोत्तरशतनामस्तोत्रं सम्पूर्णम्॥}
