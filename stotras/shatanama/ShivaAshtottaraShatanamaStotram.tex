% !TeX program = XeLaTeX
% !TeX root = ../../shloka.tex

\sect{शिवाष्टोत्तरशतनामस्तोत्रम्}

\dnsub{ध्यानम्}
\fourlineindentedshloka*
{ध्यायेन्नित्यं महेशं रजतगिरिनिभं चारुचन्द्रावतंसं}
{रत्नाकल्पोज्ज्वलाङ्गं परशुमृगवराभीतिहस्तं प्रसन्नम्}
{पद्मासीनं समन्तात् स्तुतममरगणैर्व्याघ्रकृत्तिं वसानं}
{विश्वाद्यं विश्वबीजं निखिलभयहरं पञ्चवक्त्रं त्रिनेत्रम्}

\dnsub{स्तोत्रम्}
\twolineshloka
{शिवो महेश्वरः शम्भुः पिनाकी शशिशेखरः}
{वामदेवो विरूपाक्षः कपर्दी नीललोहितः}

\twolineshloka
{शङ्करः शूलपाणिश्च खट्वाङ्गी विष्णुवल्लभः}
{शिपिविष्टोऽम्बिकानाथः श्रीकण्ठो भक्तवत्सलः}

\twolineshloka
{भवः शर्वस्त्रिलोकेशः शितिकण्ठः शिवाप्रियः}
{उग्रः कपालिः कामारिरन्धकासुरसूदनः}

\twolineshloka
{गङ्गाधरो ललाटाक्षः कालकालः कृपानिधिः}
{भीमः परशुहस्तश्च मृगपाणिर्जटाधरः}

\twolineshloka
{कैलासवासी कवची कठोरस्त्रिपुरान्तकः}
{वृषाङ्को वृषभारूढो भस्मोद्धूलितविग्रहः}

\twolineshloka
{सामप्रियः स्वरमयस्त्रयीमूर्तिरनीश्वरः}
{सर्वज्ञः परमात्मा च सोमसूर्याग्निलोचनः}

\twolineshloka
{हविर्यज्ञमयः सोमः पञ्चवक्त्रः सदाशिवः}
{विश्वेश्वरो वीरभद्रो गणनाथः प्रजापतिः}

\twolineshloka
{हिरण्यरेता दुर्धर्षो गिरीशो गिरिशोऽनघः}
{भुजङ्गभूषणो भर्गो गिरिधन्वा गिरिप्रियः}

\twolineshloka
{कृत्तिवासाः पुरारातिर्भगवान् प्रमथाधिपः}
{मृत्युञ्जयः सूक्ष्मतनुर्जगद्व्यापी जगद्गुरुः}

\twolineshloka
{व्योमकेशो महासेनजनकश्चारुविक्रमः}
{रुद्रो भूतपतिः स्थाणुरहिर्बुध्न्यो दिगम्बरः}

\twolineshloka
{अष्टमूर्तिरनेकात्मा सात्त्विकः शुद्धविग्रहः}
{शाश्वतः खण्डपरशुरजः पाशविमोचकः}

\twolineshloka
{मृडः पशुपतिर्देवो महादेवोऽव्ययः प्रभुः}
{पूषदन्तभिदव्यग्रो दक्षाध्वरहरो हरः}

\twolineshloka
{भगनेत्रभिदव्यक्तः सहस्राक्षः सहस्रपात्}
{अपवर्गप्रदोऽनन्तस्तारकः परमेश्वरः}

\dnsub{फलश्रुतिः}
\twolineshloka
{इमानि दिव्यनामानि जप्यन्ते सर्वदा मया}
{नामकल्पलतेयं मे सर्वाभीष्टप्रदायिनि}

\twolineshloka
{नामान्येतानि सुभगे शिवदानि न संशयः}
{वेदसर्वस्वभूतानि नामान्येतानि वस्तुतः}

\twolineshloka
{एतानि यानि नामानि तानि सर्वार्थदान्यतः}
{जप्यन्ते सादरं नित्यं मया नियमपूर्वकम्}

\twolineshloka
{वेदेषु शिवनामानि श्रेष्ठान्यघहराणि च}
{सन्त्यनन्तानि सुभगे वेदेषु विविधेष्वपि}

\twolineshloka
{तेभ्यो नामानि सङ्गृह्य कुमाराय महेश्वरः}
{अष्टोत्तरसहस्रं तु नाम्नामुपदिशत् पुरा}

{॥इति शाक्तप्रमोदे श्रीशिवाष्टोत्तरशतनामस्तोत्रं सम्पूर्णम्॥}