% !TeX program = XeLaTeX
% !TeX root = ../../shloka.tex
\sect{शिवाष्टोत्तरशतदिव्यस्थानीयनामस्तोत्रम्}

\twolineshloka
{अष्टोत्तरशतं भूमौ स्थितं क्षेत्रं वदाम्यहम्}
{कैवल्यशैले श्रीकण्ठः केदारो हिमवत्यपि}

\twolineshloka
{काशीपुर्यां विश्वनाथः श्रीशैले मल्लिकार्जुनः}
{प्रयागे नीलकण्ठेशो गयायां रुद्रनामकः}

\twolineshloka
{नीलकण्ठेश्वरः साक्षात् कालञ्जरपुरे शिवः}
{द्राक्षारामे तु भीमेशो मायूरे चाम्बिकेश्वरः}

\twolineshloka
{ब्रह्मावर्ते देवलिङ्गः प्रभासे शशिभूषणः}
{वृषध्वजाभिधः श्रीमतिः श्वेतहस्तिपुरेश्वरः}

\twolineshloka
{गोकर्णेशस्तु गोकर्णे सोमेशः सोमनाथके}
{श्रीरूपाख्ये त्यागराजो वेदे वेदपुरीश्वरः}

\twolineshloka
{भीमारामे तु भीमेशो मन्थने कालिकेश्वरः}
{मधुरायां चोक्कनाथो मानसे माधवेश्वरः}

\twolineshloka
{श्रीवाञ्छके चम्पकेशः पञ्चवट्यां वटेश्वरः}
{गजारण्ये तु वैद्येशस्तीर्थाद्रौ तीर्थकेश्वरः}

\twolineshloka
{कुम्भकोणे तु कुम्भेशो लेपाक्ष्यां पापनाशनः}
{कण्वपुर्यां तु कण्वेशो मध्ये मध्यार्जुनेश्वरः}

\twolineshloka
{हरिहरपुरे श्रीशङ्करनारायणेश्वरः}
{विरञ्चिपुर्यां मार्गेशः पञ्चनद्यां गिरीश्वरः}

\twolineshloka
{पम्पापुर्यां विरूपाक्षः सोमाद्रौ मल्लिकार्जुनः}
{त्रिमकूटे त्वगस्त्येशः सुब्रह्मण्येऽहिपेश्वरः}

\twolineshloka
{महाबलेश्वरः साक्षान्महाबलशिलोच्चये}
{रविणा पूजितो दक्षिणावर्तेऽर्केश्वरः स्वयम्}

\twolineshloka
{वेदारण्ये महापुण्ये वेदारण्येश्वराभिधः}
{मूर्तित्रयात्मकः सोमपुर्यां सोमेश्वराभिधः}

\twolineshloka
{अवन्त्यां रामलिङ्गेशः काश्मीरे विजयेश्वरः}
{महानन्दिपुरे साक्षान्महानन्दिपुरेश्वरः}

\twolineshloka
{कोटितीर्थे तु कोटीशो वृद्धे वृद्धाचलेश्वरः}
{महापुण्ये तत्र ककुद्गिरौ गङ्गाधरेश्वरः}

\twolineshloka
{चामराज्याख्यनगरे चामराजेश्वरः स्वयम्}
{नन्दीश्वरो नन्दिगिरौ चण्डेशो बधिराचले}

\twolineshloka
{नञ्जुण्डेशो गरपुरे शतशृङ्गे}
{घनानन्दाचले सोमो नल्लूरे विमलेश्वरः}

\twolineshloka
{नीडानाथपुरे साक्षान्नीडानाथेश्वरः स्वयम्}
{एकान्ते रामलिङ्गेशः श्रीनागे कुण्डलीश्वरः}

\twolineshloka
{श्रीकन्यायां त्रिभङ्गीश उत्सङ्गे राघवेश्वरः}
{मत्स्यतीर्थे तु तीर्थेशस्त्रिकूटे ताण्डवेश्वरः}

\twolineshloka
{प्रसन्नाख्यपुरे मार्गसहायेशो वरप्रदः}
{गण्डक्यां शिवनाभस्तु श्रीपतौ श्रीपतीश्वरः}

%% Unknown shloka format

\twolineshloka
{धर्मपुर्यां धर्मलिङ्गः कन्याकुब्जे कलाधरः}
{वाणिग्रामे विरिञ्चेशो नेपाले नकुलेश्वरः}

\twolineshloka
{मार्कण्डेयो जगन्नाथे स्वयम्भूर्नर्मदातटे}
{धर्मस्थले मञ्जुनाथो व्यासेशस्तु त्रिरूपके}

\twolineshloka
{स्वर्णावत्यां कलिङ्गेशो निर्मले पन्नगेश्वरः}
{पुण्डरीके जैमिनीशोऽयोध्यायां मधुरेश्वरः}

\twolineshloka
{सिद्धवट्यां तु सिद्धेशः श्रीकूर्मे त्रिपुरान्तकः}
{मणिकुण्डलतीर्थे तु मणिमुक्तानदीश्वरः}

\twolineshloka
{वटाटव्यां कृत्तिवासास्त्रिवेण्यां सङ्गमेश्वरः}
{स्तनिताख्ये तु मल्लेश इन्द्रकीलेऽर्जुनेश्वरः}

\twolineshloka
{शेषाद्रौ कपिलेशस्तु पुष्पे पुष्पगिरीश्वरः}
{भुवनेशश्चित्रकूटे तूज्जिन्यां कालिकेश्वरः}

\twolineshloka
{ज्वालामुख्यां शूलटङ्को मङ्गल्यां सङ्गमेश्वरः}
{बृहतीशस्तञ्जापुर्यां रामेशो वह्निपुष्करे}

\twolineshloka
{लङ्काद्वीपे तु मत्स्येशः कूर्मेशो गन्धमादने}
{विन्ध्याचले वराहेशो नृसिंहः स्यादहोबिले}

\twolineshloka
{कुरुक्षेत्रे वामनेशस्ततः कपिलतीर्थके}
{तथा परशुरामेशः सेतौ रामेश्वराभिधः}

\twolineshloka
{साकेते बलरामेशो बौद्धेशो वारणावते}
{तत्त्वक्षेत्रे च कल्कीशः कृष्णेशः स्यान्महेन्द्रके}

{॥इति ललितागमे ज्ञानपादे शिवलिङ्गप्रादुर्भावपटलान्तर्गते श्रीशिवाष्टोत्तरशतदिव्यस्थानीयनामस्तोत्रं सम्पूर्णम्॥}