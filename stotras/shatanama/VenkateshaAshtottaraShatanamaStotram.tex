%!TeX program = Xelatex
%!TeX root = ../../shloka.tex

\sect{वेङ्कटेशाष्टोत्तरशतनामस्तोत्रम्}

\uvacha{सिद्धा  ऊचुः}
\twolineshloka*
{भगवन्  वेङ्कटेशस्य  नाम्नामष्टोत्तरं  शतम्}
{अनुब्रूहि  दयासिन्धो  क्षिप्रसिद्धिप्रदं  नृणाम्}

\uvacha{नारद  उवाच}
\twolineshloka*
{सावधानेन  मनसा  शृण्वन्तु  तदिदं  शुभम्}
{जप्तं  वैखानसैः  पूर्वं  सर्वसौभाग्यवर्धनम्}

\dnsub{स्तोत्रम्}
\twolineshloka
{ओङ्कारपरमार्थश्च  नरनारायणात्मकः}
{मोक्षलक्ष्मीप्राणकान्तो वेङ्कटाचलनायकः}

\twolineshloka
{करुणापूर्णहृदयः टेङ्कारजपसौख्यदः}
{शास्त्रप्रमाणगम्यश्च यमाद्यष्टाङ्गगोचरः}

\twolineshloka
{भक्तलोकैकवरदो वरेण्यो भयनाशनः}
{यजमानस्वरूपश्च हस्तन्यस्तसुदर्शनः}

\twolineshloka
{रमावतारङ्गेशो णाकारजपसुप्रियः}
{यज्ञेशो गतिदाता च जगतीवल्लभो वरः}

\twolineshloka
{रक्षःसन्दोहसंहर्ता वर्चस्वी रघुपुङ्गवः}
{दानधर्मपरो याजी घनश्यामलविग्रहः}

\twolineshloka
{हरादिसर्वदेवेड्यो रामो यदुकुलाग्रणीः}
{श्रीनिवासो महात्मा च तेजस्वी तत्त्वसन्निधिः}

\twolineshloka
{त्वमर्थलक्ष्यरूपश्च रूपवान् पावनो यशः}
{सर्वेशो कमलाकान्तो लक्ष्मीसल्लापसम्मुखः}

\twolineshloka
{चतुर्मुखप्रतिष्ठाता राजराजवरप्रदः}
{चतुर्वेदशिरोरत्नं रमणो नित्यवैभवः}

\twolineshloka
{दासवर्गपरित्राता नारदादिमुनिस्तुतः}
{यादवाचलवासी च खिद्यद्भक्तार्तिभञ्जनः}

\twolineshloka
{लक्ष्मीप्रसादको विष्णुर्देवेशो रम्यविग्रहः}
{माधवो लोकनाथश्च लालिताखिलसेवकः}

\twolineshloka
{यक्षगन्धर्ववरदः कुमारो मातृकार्चितः}
{रटद्बालकपोषी च शेषशैलकृतस्थलः}

\twolineshloka
{षाड्गुण्यपरिपूर्णश्च द्वैतदोषनिवारणः}
{तिर्यग्जन्त्वर्चिताङ्घ्रिश्च नेत्रानन्दकरोत्सवः}

\twolineshloka
{द्वादशोत्तमलीलश्च दरिद्रजनरक्षकः}
{शत्रुकृत्यादिभीतिघ्नो भुजङ्गशयनप्रियः}

\twolineshloka
{जाग्रद्रहस्यावासो यः शिष्टानां  परिपालकः}
{वरेण्यः पूर्णबोधश्च जन्मसंसारभेषजम्}

\twolineshloka
{कार्त्तिकेयवपुर्धारी यतिशेखरभावितः}
{नरकादिभयध्वंसी रथोत्सवकलाधरः}

\twolineshloka
{लोकार्चामुख्यमूर्तिश्च केशवाद्यवतारवान्}
{शास्त्रश्रुतानन्तलीलो यमशिक्षानिबर्हणः}

\twolineshloka
{मानसंरक्षणपर इरिणाङ्कुरधान्यदः}
{नेत्रहीनाक्षिदायी च मतिहीनमतिप्रदः}

\twolineshloka
{हिरण्यदानसङ्ग्राही मोहजालनिकृन्तनः}
{दधिलाजाक्षतार्च्यश्च यातुधानविनाशनः}

\twolineshloka
{यजुर्वेदशिखागम्यो वेङ्कटो दक्षिणास्थितः}
{सारपुष्करणीतीरो रात्रौदेवगणार्चितः}

\twolineshloka
{यत्नवत्फलसन्धाता श्रीञ्जपाद्धनवृद्धिकृत्}
{क्लीङ्कारजापिकाम्यार्थप्रदानसदयान्तरः}

\twolineshloka
{सौं सर्वसिद्धिसन्धाता नमस्कर्तुरभीष्टदः}
{मोहिताखिललोकश्च नानारूपव्यवस्थितः}

\twolineshloka
{राजीवलोचनो यज्ञवराहो गणवेङ्कटः}
{तेजोराशीक्षणः स्वामी हार्दाविद्यानिवारणः}

\threelineshloka*
{इति श्रीवेङ्कटेशस्य नाम्नामष्टोत्तरं  शतम्}
{प्रातः प्रातः  समुत्थाय  यः  पठेद्भक्तिमान्नरः}
{सर्वेष्टार्थानवाप्नोति  वेङ्कटेशप्रसादतः}

॥इति  श्री-वेङ्कटेशाष्टोत्तरशतनामस्तोत्रं  सम्पूर्णम्॥