% !TeX program = XeLaTeX
% !TeX root = ../../shloka.tex
\sect{शक्त्यष्टोत्तरशतदिव्यस्थानीयनामस्तोत्रम्}

\uvacha{दक्ष उवाच}
\twolineshloka*
{एवमुक्तोऽब्रवीद्दक्षः केषु केषु मयाऽनघे}
{तीर्थेषु च त्वं द्रष्टव्या स्तोतव्या कैश्च नामभिः}

\uvacha{देव्युवाच}
\twolineshloka*
{सर्वगा सर्वभूतेषु द्रष्टव्या सर्वतो भुवि}
{सप्तलोकेषु यत्किञ्चिद्रहितं न मया विना}

\twolineshloka*
{तथाऽपि येषु स्थानेषु द्रष्टव्याः सिद्धिमीप्सुभिः}
{स्मर्तव्या भूतिकामैर्वा तानि वक्ष्यामि तत्त्वतः}

\dnsub{स्तोत्रम्}
\twolineshloka
{वाराणस्यां विशालाक्षी नैमिषे लिङ्गधारिणी}
{प्रयागे ललिता देवी कामुका गन्धमादने}

\twolineshloka
{मानसे कुमुदा नाम विश्वकाया तथेश्वरे}
{गोमन्ते गोमती नाम मन्दरे कामचारिणी}

\twolineshloka
{मदोत्कटा चैत्ररथे जयन्ती हस्तिनापुरे}
{कान्यकुब्जे तथा गौरी रम्भा मलयपर्वते}

\twolineshloka
{एकाम्रके कीर्तिमती विश्वां विश्वेश्वरे विदुः}
{पुष्करे पुरुहूतेति केदारे मार्गदायिनी}

\twolineshloka
{नन्दा हिमवतः पृष्ठे गोकर्णे भद्रकर्णिका}
{स्थानेश्वरे भवानी तु बिल्वके बिल्वपत्रिका}

\twolineshloka
{श्रीशैले माधवी नाम भद्रा भद्रेश्वरे तथा}
{जया वराहशैले तु कमला कमलालये}

\twolineshloka
{रुद्रकोट्यां च रुद्राणी काली कालञ्जरे गिरौ}
{महालिङ्गे तु कपिला मर्कोटे मुकुटेश्वरी}

\twolineshloka
{शालग्रामे महादेवी शिवलिङ्गे जलप्रिया}
{मायापुर्यां कुमारी तु सन्ताने ललिता तथा}

\twolineshloka
{उत्पलाक्षी सहस्राक्षे हिरण्याक्षे महोत्पला}
{गयायां मङ्गला नाम विमला पुरुषोत्तमे}

\twolineshloka
{विपाशायामघोराक्षी पाटला पुण्ड्रवर्धने}
{नारायणी सुपार्श्वे तु त्रिकूटे रुद्रसुन्दरी}

\twolineshloka
{विपुले विपुला नाम कल्याणी मलयाचले}
{कौटवी कोटितीर्थेषु सुगन्धा माधवे वने}

\twolineshloka
{गोदाश्रमे त्रिसन्ध्या तु गङ्गाद्वारे रतिप्रिया}
{शिवकुण्डे शिवानन्दा नन्दिनी देविकातटे}

\twolineshloka
{रुक्मिणी द्वारवत्यां तु राधा वृन्दावने वने}
{देवकी मथुरायां तु पाताले परमेश्वरी}

\twolineshloka
{चित्रकूटे तथा सीता विन्ध्ये विन्ध्याधिवासिनी}
{सह्यद्वारे करीरा तु हरिश्चन्द्रे तु चन्द्रिका}

\twolineshloka
{रमणा रामतीर्थे तु यमुनायां मृगावती}
{करवीरे महालक्ष्मीरुमादेवी विनायके}

\twolineshloka
{अरोगा वैद्यनाथे तु महाकाले महेश्वरी}
{अभयेत्युष्णतीर्थेषु चामृता विन्ध्यकन्दरे}

\twolineshloka
{माण्डव्ये माण्डवी नाम स्वाहा माहेश्वरे पुरे}
{छागलाण्डे प्रचण्डा तु चण्डिकामरकण्टके}

\twolineshloka
{सोमेश्वरे वरारोहा प्रभासे पुष्करावती}
{देवमाता सरस्वत्यां पारा पारातटे मता}

\twolineshloka
{महालये महाभागा पयोष्ण्यां पिङ्गलेश्वरी}
{सिंहिका कृतशौचे तु कार्त्तिकेये तु शङ्करी}

\twolineshloka
{उत्पलावर्तके लोला सुभद्रा शोणसङ्गमे}
{महासिद्धवने लक्ष्मीरनङ्गा भरताश्रमे}

\twolineshloka
{जालन्धरे विश्वमुखी तारा किष्किन्धपर्वते}
{देवदारुवने पुष्टिर्मेधा काश्मीरमण्डले}

\twolineshloka
{भीमादेवी हिमाद्रौ तु पुष्टिर्विश्वेश्वरे तथा}
{कपालमोचने शुद्धिर्माता कायावरोहणे}

\twolineshloka
{शङ्खोद्धारे ध्वनिर्नाम धृतिः पिण्डारके तथा}
{काला तु चन्द्रभागायामच्चोदे शिवकारिणी}

\twolineshloka
{वेणायाममृता नाम बदर्याम् उर्वशी तथा}
{औषधी चोत्तरकुरौ कुशद्वीपे कुशोदका}

\twolineshloka
{मन्मथा हेमकूटे तु मुकुटे सत्यवादिनी}
{अश्वत्थे वन्दनीया तु निधिर्वैश्रवणालये}

\twolineshloka
{गायत्री वेदवदने पार्वती शिवसन्निधौ}
{देवलोके तथेन्द्राणी ब्रह्मास्येषु सरस्वती}

\twolineshloka
{सूर्यबिम्बे प्रभा नाम मातॄणां वैष्णवी तथा}
{अरुन्धती सतीनां तु रामासु च तिलोत्तमा}

\twolineshloka
{चित्ते ब्रह्मकला नाम शक्तिः सर्वशरीरिणाम्}
{एतदुद्देशतः प्रोक्तं नामाष्टशतमुत्तमम्}

\twolineshloka
{अष्टोत्तरं च तीर्थानां शतमेतदुदाहृतम्}
{यः पठेच्छृणुयाद्वाऽपि सर्वपापैः प्रमुच्यते}

\twolineshloka
{एषु तीर्थेषु यः कृत्वा स्नानं पश्यन्ति मां नरः}
{सर्वपापविनिर्मुक्तः कल्पं शिवपुरे वसेत्}

\twolineshloka
{यस्तु मत्परमः कालं करोत्येतेषु मानवः}
{स भित्त्वा ब्रह्मसदनं पदमभ्येति शाङ्करम्}

\twolineshloka
{नामाष्टशतकं यस्तु श्रावयेच्छिवसन्निधौ}
{तृतीयायामथाष्टम्यां बहुपुत्रो भवेन्नरः}

\twolineshloka
{गोदाने श्राद्धकाले वा अहन्यहनि वा पुनः}
{देवार्चनविधौ विद्वान् पठन् ब्रह्माधिगच्छति}

\twolineshloka
{एवं वदन्ती सा तत्र ददाहात्मानमात्मना}
{स्वायम्भुवेऽपि काले च दक्षप्राचेतसोऽभवत्}

\twolineshloka
{पार्वती चाभवद्देवी शिवदेहार्धधारिणी}
{मेनागर्भसमुत्पन्ना भुक्तिमुक्तिफलप्रदा}

\twolineshloka
{अरुन्धती जपन्त्येतत्तदाप्ता योगमुत्तमम्}
{पुरूरवाश्च राजर्षिर्लोकेषु जयतामगात्}

\threelineshloka
{ययातिः पुत्रलाभं तु धनलाभं तु भार्गवः}
{तथाऽन्ये देवदैत्याश्च ब्राह्मणाः क्षत्रियास्तथा}
{वैश्याः शूद्राश्च बहवः सिद्धिमीयुर्यथेप्सिताम्}

\twolineshloka
{यत्रेतल्लिखितं तिष्ठेत् पूज्यते देवसन्निधौ}
{न तत्र शोकदौर्गत्यं कदाचिदपि जायते}

॥इति श्रीमत्स्यमहापुराणे त्रयोदशेऽध्याये श्री-शक्त्यष्टोत्तरशतदिव्यस्थानीयनामस्तोत्रं सम्पूर्णम्॥