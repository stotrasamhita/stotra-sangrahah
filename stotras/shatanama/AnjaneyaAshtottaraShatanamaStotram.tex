% !TeX program = XeLaTeX
% !TeX root = ../../shloka.tex

\sect{आञ्जनेयाष्टोत्तरशतनामस्तोत्रम्}

\twolineshloka
{आञ्जनेयो महावीरो हनुमान् मारुतात्मजः}
{तत्त्वज्ञानप्रदः सीतादेवीमुद्राप्रदायकः}

\twolineshloka
{अशोकवनिकाच्छेत्ता सर्वमायाविभञ्जनः}
{सर्वबन्धविमोक्ता च रक्षोविध्वंसकारकः}

\twolineshloka
{परविद्यापरीहर्ता परशौर्यविनाशकः}
{परमन्त्रनिराकर्ता परयन्त्रप्रभेदकः}

\twolineshloka
{सर्वग्रहविनाशी च भीमसेनसहायकृत्}
{सर्वदुःखहरः सर्वलोकचारी मनोजवः}

\twolineshloka
{पारिजातद्रुमूलस्थः सर्वमन्त्रस्वरूपवान्}
{सर्वतन्त्रस्वरूपी च सर्वयन्त्रात्मिकस्तथा}

\twolineshloka
{कपीश्वरो महाकायः सर्वरोगहरः प्रभुः}
{बलसिद्धिकरः सर्वविद्यासम्पत्प्रदायकः}

\twolineshloka
{कपिसेनानायकश्च भविष्यच्चतुराननः}
{कुमारब्रह्मचारी च रत्नकुण्डलदीप्तिमान्}

\twolineshloka
{चञ्चलद्वालसन्नद्धो लम्बमानशिखोज्ज्वलः}
{गन्धर्वविद्यातत्त्वज्ञो महाबलपराक्रमः}

\twolineshloka
{कारागृहविमोक्ता च शृङ्खलाबन्धमोचकः}
{सागरोत्तारकः प्राज्ञो रामदूतः प्रतापवान्}

\twolineshloka
{वानरः केसरीसूनुः सीताशोकनिवारणः}
{अञ्जनागर्भसम्भूतो बालार्कसदृशाननः}

\twolineshloka
{विभीषणप्रियकरो दशग्रीवकुलान्तकः}
{लक्ष्मणप्राणदाता च वज्रकायो महाद्युतिः}

\twolineshloka
{चिरञ्जीवी रामभक्तो दैत्यकार्यविघातकः}
{अक्षहन्ता काञ्चनाभः पञ्चवक्त्रो महातपाः}

\twolineshloka
{लङ्किणीभञ्जनः श्रीमान् सिंहिकाप्राणभञ्जनः}
{गन्धमादनशैलस्थो लङ्कापुरविदाहकः}

\twolineshloka
{सुग्रीवसचिवो धीरः शूरो दैत्यकुलान्तकः }
{सुरार्चितो महातेजो रामचूडामणिप्रदः}

\twolineshloka
{कामरूपी पिङ्गलाक्षो वर्धिमैनाकपूजितः}
{कबलीकृतमार्ताण्डमण्डलो विजितेन्द्रियः}

\twolineshloka
{रामसुग्रीवसन्धाता महिरावणमर्दनः}
{स्फटिकाभो वागधीशो नवव्याकृतिपण्डितः}

\twolineshloka
{चतुर्बाहुर्दीनबन्धुर्महात्मा भक्तवत्सलः}
{सञ्जीवननगाहर्ता शुचिर्वाग्मी धृतव्रतः}

\twolineshloka
{कालनेमिप्रमथनो हरिमर्कटमर्कटः}
{दान्तः शान्तः प्रसन्नात्मा शतकण्ठमदापहः}

\twolineshloka
{योगी रामकथालोलः सीतान्वेषणपण्डितः}
{वज्रदंष्ट्रो वज्रनखो रुद्रवीर्यसमुद्भवः}

\twolineshloka
{इन्द्रजित्प्रहितामोघब्रह्मास्त्रविनिवारकः}
{पार्थध्वजाग्रसंवासी शरपञ्जरहेलकः}

\twolineshloka
{दशबाहुर्लोकपूज्यो जाम्बवत्प्रीतिवर्धनः}
{सीतासमेतश्रीरामपादसेवाधुरन्धरः}

{॥इति श्री आञ्जनेयाष्टोत्तरशतनामस्तोत्रं सम्पूर्णम्॥}

