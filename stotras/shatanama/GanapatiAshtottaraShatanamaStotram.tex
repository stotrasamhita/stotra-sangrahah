% !TeX program = XeLaTeX
% !TeX root = ../../shloka.tex
\sect{गणपत्यष्टोत्तरशतनामस्तोत्रम्}

\dnsub{ध्यानम्}
\fourlineindentedshloka*
{ॐकारसन्निभमिभाननमिन्दुभालं}
{मुक्ताग्रबिन्दुममलद्युतिमेकदन्तम्}
{लम्बोदरं कलचतुर्भुजमादिदेवं}
{ध्यायेन्महागणपतिं मतिसिद्धिकान्तम्}

\dnsub{स्तोत्रम्}
\twolineshloka
{गणेश्वरो गणक्रीडो महागणपतिस्तथा}
{विश्वकर्ता विश्वमुखो दुर्जयो धूर्जयो जयः}

\twolineshloka
{सुरूपः सर्वनेत्राधिवासो वीरासनाश्रयः}
{योगाधिपस्तारकस्थः पुरुषो गजकर्णकः}

\twolineshloka
{चित्राङ्गः श्यामदशनो भालचन्द्रश्चतुर्भुजः}
{शम्भुतेजा यज्ञकायः सर्वात्मा सामबृंहितः}

\twolineshloka
{कुलाचलांसो व्योमनाभिः कल्पद्रुमवनालयः}
{निम्ननाभिः स्थूलकुक्षिः पीनवक्षा बृहद्भुजः}

\twolineshloka
{पीनस्कन्धः कम्बुकण्ठो लम्बोष्ठो लम्बनासिकः}
{सर्वायवसम्पूर्णः सर्वलक्षणलक्षितः}

\twolineshloka
{इक्षुचापधरः शूली कान्तिकन्दलिताश्रयः}
{अक्षमालाधरो ज्ञानमुद्रावान् विजयावहः}

\twolineshloka
{कामिनीकामनाकाममालिनीकेलिलालितः}
{अमोघसिद्धिराधार आधाराधेयवर्जितः}

\twolineshloka
{इन्दीवरदलश्याम इन्दुमण्डलनिर्मलः}
{कर्मसाक्षी कर्मकर्ता कर्माकर्मफलप्रदः}

\twolineshloka
{कमण्डलुधरः कल्पः कपर्दी कटिसूत्रभृत्}
{कारुण्यदेहः कपिलो गुह्यागमनिरूपितः}

\twolineshloka
{गुहाशयो गुहाब्धिस्थो घटकुम्भो घटोदरः}
{पूर्णानन्दः परानन्दो धनदो धरणीधरः}

\twolineshloka
{बृहत्तमो ब्रह्मपरो ब्रह्मण्यो ब्रह्मवित्प्रियः}
{भव्यो भूतालयो भोगदाता चैव महामनाः}

\twolineshloka
{वरेण्यो वामदेवश्च वन्द्यो वज्रनिवारणः}
{विश्वकर्ता विश्वचक्षुर्हवनं हव्यकव्यभुक्}

\twolineshloka
{स्वतन्त्रः सत्यसङ्कल्पस्तथा सौभाग्यवर्धनः}
{कीर्तिदः शोकहारी च त्रिवर्गफलदायकः}

\twolineshloka
{चतुर्बाहुश्चतुर्दन्तश्चतुर्थातिथिसम्भवः}
{सहस्रशीर्षा पुरुषः सहस्राक्षः सहस्रपात्}

\twolineshloka
{कामरूपः कामगतिर्द्विरदो द्वीपरक्षकः}
{क्षेत्राधिपः क्षमाभर्ता लयस्थो लड्डुकप्रियः}

\twolineshloka
{प्रतिवादिमुखस्तम्भो दुष्टचित्तप्रसादनः}
{भगवान् भक्तिसुलभो याज्ञिको याजकप्रियः}

\twolineshloka
{इत्येवं देवदेवस्य गणराजस्य धीमतः}
{शतमष्टोत्तरं नाम्नां सारभूतं प्रकीर्तितम्}

\threelineshloka
{सहस्रनाम्नामाकृष्य मया प्रोक्तं मनोहरम्}
{ब्राह्मे मुहूर्ते चोत्थाय स्मृत्वा देवं गणेश्वरम्}
{पठेत्स्तोत्रमिदं भक्त्या गणराजः प्रसीदति}

{॥ इति श्रीगणेशपुराणे उपासनाखण्डे श्रीगणपत्यष्टोत्तरशतनामस्तोत्रं सम्पूर्णम् ॥}
