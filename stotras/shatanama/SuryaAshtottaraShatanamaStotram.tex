%!TeX program = Xelatex
%!TeX root = ../../shloka.tex

\sect{सूर्याष्टोत्तरशतनामस्तोत्रम्}

धौम्य उवाच

\twolineshloka
{सूर्योऽर्यमा भगस्त्वष्टा पूषाऽर्कः सविता रविः}
{गभस्तिमानजः कालो मृत्युर्धाता प्रभाकरः}

\twolineshloka
{पृथिव्यापश्च तेजश्च खं वायुश्च परायणम्}
{सोमो बृहस्पतिः शुक्रो बुधोऽङ्गारक एव च}

\twolineshloka
{इन्द्रो विवस्वान् दीप्तांशुः शुचिः शौरिः शनैश्चरः}
{ब्रह्मा विष्णुश्च रुद्रश्च स्कन्दो वै वरुणो यमः}

\twolineshloka
{वैद्युतो जाठरश्चाग्निरैन्धनस्तेजसां पतिः}
{धर्मध्वजो वेदकर्ता वेदाङ्गो वेदवाहनः}

\twolineshloka
{कृतं त्रेता द्वापरश्च कलिः सर्वमलाश्रयः}
{कलाकाष्ठामुहूर्तश्च क्षपा यामस्तथा क्षणः}

\twolineshloka
{संवत्सरकरोऽश्वत्थः कालचक्रो विभावसुः}
{पुरुषः शाश्वतो योगी व्यक्ताव्यक्तः सनातनः}

\twolineshloka
{कालाध्यक्षः प्रजाध्यक्षो विश्वकर्मा तमोनुदः}
{वरुणः सागरोंऽशुश्च जीमूतो जीवनोऽरिहा}

\twolineshloka
{भूताश्रयो भूतपतिः सर्वलोकनमस्कृतः}
{स्रष्टा संवर्तको वह्निः सर्वस्यादिरलोलुपः}

\twolineshloka
{अनन्तः कपिलो भानुः कामदः सर्वतोमुखः}
{जयो विशालो वरदः सर्वभूतनिषेवितः}

\twolineshloka
{मनः सुपर्णो भूतादिः शीघ्रगः प्राणधारकः}
{धन्वतरिर्धूमकेतुरादिदेवोऽदितेः सुतः}

\twolineshloka
{द्वादशात्माऽरविन्दाक्षः पिता माता पितामहः}
{स्वर्गद्वारं प्रजाद्वारं मोक्षद्वारं त्रिविष्टपम्}

\twolineshloka
{देहकर्ता प्रशान्तात्मा विश्वात्मा विश्वतोमुखः}
{चराचरात्मा सूक्ष्मात्मा मैत्रेयः करुणान्वितः}

\twolineshloka
{एतद्वै कीर्तनीयस्य सूर्यस्यामिततेजसः}
{नामाष्टशतकं चेदं प्रोक्तमेतत् स्वयम्भुवा}

\twolineshloka
{सुरगणपितृयक्षसेवितं ह्यसुरनिशाचरसिद्धवन्दितम्}
{वरकनकहुताशनप्रभं प्रणिपतितोऽस्मि हिताय भास्करम्}

\fourlineindentedshloka
{सूर्योदये यः सुसमाहितः पठेत्}
{स पुत्रदारान् धनरत्नसञ्चयान्}
{लभेत जातिस्मरतां नरः सदा}
{ धृतिं च मेधां च स विन्दते पुमान्}

\fourlineindentedshloka
{इमं स्तवं देववरस्य यो नरः}
{प्रकीर्तयेच्छुद्धमनाः समाहितः}
{विमुच्यते शोकदवाग्निसागरात्}
{लभेत कामान् मनसा यथेप्सितान्}

॥इति श्रीमन्महाभारते वनपर्वणि धौम्ययुधिष्ठिरसंवादे श्री सूर्याष्टोत्तरशतनामस्तोत्रं सम्पूर्णम्॥
