% !TeX program = XeLaTeX
% !TeX root = ../../shloka.tex

\section{हनुमत्कृतं श्री-सीता-राम-स्तोत्रम्}

\twolineshloka
{अयोध्या-पुर-नेतारं मिथिला-पुर-नायिकाम्}
{इक्ष्वाकूणाम् अलंकारं वैदेहानाम् अलंक्रियाम्} % ॥ १ ॥

\twolineshloka
{रघूणां कुल-दीपं च निमीनां कुल-दीपिकाम्}
{सूर्य-वंश-समुद्भूतं सोम-वंश-समुद्भवाम्} % ॥ २ ॥

\twolineshloka
{पुत्रं दशरथस्यापि पुत्रीं जनक-भूपतेः}
{वसिष्ठ-अनुमताचारं शतानन्द-मतानुगाम्} % ॥ ३ ॥

\twolineshloka
{कौसल्या-गर्भ-सम्भूतं वेदि-गर्भोदितां स्वयम्}
{पुण्डरीक-विशालाक्षं स्फुरद्-इन्दीवरेक्षणाम्} % ॥ ५ ॥

\twolineshloka
{चन्द्र-कान्त-आननाम्भोजं चन्द्रबिम्ब-उपमाननाम्}
{मत्त-मातङ्ग-गमनं मत्त-सारस-गामिनीम्}

\twolineshloka
{चन्दनार्द्र-भुजा-मध्यं कुङ्कुमाक्त-कुच-स्थलीम्}
{चापालंकृत-हस्ताब्जं पद्मालंकृत-पाणिकाम्} % ॥ ७ ॥

\twolineshloka
{शरणागतगोप्तारं प्रणिपातप्रसादिकाम्}
{ताली-दल-श्यामलाङ्गं तप्त-चामीकर-प्रभाम्} % ॥  ६ ॥

\twolineshloka
{दिव्य-सिंहासनारूढं दिव्य-स्रग्-वस्त्र-भूषणाम्} % ॥ ९ ॥
{अनुक्षणं कटाक्षाभ्याम् अन्योन्य-ईक्षण-काङ्क्षिणौ}

\twolineshloka
{अन्योन्य-सदृशाकारौ त्रैलोक्य-गृह-दम्पती}
{इमौ युवां प्रणम्याहं भजाम्यद्य कृतार्थताम्} % ॥ १० ॥

\twolineshloka
{अनया स्तोति यः स्तुत्या रामं सीतां च भक्तितः}
{तस्य तौ तनुतां प्रीतौ संपदः सकला अपि} % ॥ ११ ॥

\threelineshloka
{इतीदं रामचन्द्रस्य जानक्याश्च विशेषतः}
{कृतं हनुमता पुण्यं स्तोत्रं सद्यो विमुक्ति-दम्}
{यः पठेत्प्रातरुत्थाय सर्वान् कामानवाप्नुयात्} % ॥ १२ ॥

॥इति श्री-हनूमत्कृतं श्री-सीतारामस्तोत्रं सम्पूर्णम्॥