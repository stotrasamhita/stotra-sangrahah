% !TeX program = XeLaTeX
% !TeX root = ../../shloka.tex

\sect{शम्भुकृत-रामस्तवः}

\addtocounter{shlokacount}{23}

\uvacha{महादेव उवाच}

\twolineshloka
{नमो मूलप्रकृतये नित्याय परमात्मने}
{सच्चिदानन्दरूपाय विश्वरूपाय वेधसे}% २४

\twolineshloka
{नमो निरन्तरानन्द कन्दमूलाय विष्णवे}
{जगत्त्रयकृतानन्द मूर्तये दिव्यमूर्त्तये}% २५

\twolineshloka
{नमो ब्रह्मेन्द्रपूज्याय शङ्कराभयदाय च}
{नमो विष्णुस्वरूपाय सर्वरूप नमो नमः}% २६

\twolineshloka
{उत्पत्तिस्थितिसंहारकारिणे त्रिगुणात्मने}
{नमोस्तु निर्गतोपाधिस्वरूपाय महात्मने}% २७

\twolineshloka
{अनया विद्यया देव्या सीतयोपाधिकारिणे}
{नमः पुम्प्रकृतिभ्यां च युवाभ्यां जगतां कृते}% २८

\twolineshloka
{जगन्मातापितृभ्यां च जनन्यै राघवाय च}
{नमः प्रपञ्चरूपिण्यै निष्प्रपञ्चस्वरूपिणे}% २९

\twolineshloka
{नमो ध्यानस्वरूपिण्यै योगिध्येयात्ममूर्तये}
{परिणामापरीणामरिक्ताभ्यां च नमो नमः}% ३०

\twolineshloka
{कूटस्थबीजरूपिण्यै सीतायै राघवाय च}
{सीता लक्ष्मीर्भवान् विष्णुः सीता गौरी भवान्शिवः}% ३१

\twolineshloka
{सीता स्वयं हि सावित्रि भवान् ब्रह्मा चतुर्मुखः}
{सीता शची भवान् शक्रः सीता स्वाहाऽनलो भवान्}% ३२

\twolineshloka
{सीता संहारिणी देवी यमरूपधरो भवान्}
{सीता हि सर्वसम्पत्तिः कुबेरस्त्वं रघूत्तम}% ३३

\twolineshloka
{सीता देवी च रुद्राणी भवान्रुद्रो महाबलः}
{सीता तु रोहिणी देवी चन्द्रस्त्वं लोकसौख्यदः}% ३४

\twolineshloka
{सीता संज्ञा भवान्सूर्यः सीता रात्रिर्दिवा भवान्}
{सीतादेवी महाकाली महाकालो भवान्सदा}% ३५

\twolineshloka
{स्त्रीलिङ्गेषु त्रिलोकेषु यत्तत्सर्वं हि जानकी}
{पुन्नाम लाञ्छितं यत्तु तत्सर्वं हि भवान्प्रभो}% ३६

\twolineshloka
{सर्वत्र सर्वदेवेश सीता सर्वत्र धारिणी}
{तदा त्वमपि च त्रातुं तच्छक्तिर्विश्वधारिणी}% ३७

\twolineshloka
{तस्मात्कोटिगुणं पुण्यं युवाभ्यां परिचिह्नितम्}
{चिह्नितं शिवशक्तिभ्यां चरितं तव शान्तिदम्}% ३८

\twolineshloka
{आवां राम जगत्पूज्यौ मम पूज्यौ सदा युवाम्}
{त्वन्नामजापिनी गौरी त्वन्मन्त्रजपवानहम्}% ३९

\twolineshloka
{मुमूर्षोर्मणिकर्ण्यां तु अर्धोदकनिवासिनः}
{अहं दिशामि ते मन्त्रं तारकं ब्रह्मदायकम्}% ४०

\twolineshloka
{अतस्त्वं जानकीनाथ परब्रह्मासि निश्चितम्}
{त्वन्मायामोहितास्सर्वे न त्वां जानंति तत्वतः}% ४१

\uvacha{ईश्वर उवाच}

\twolineshloka
{इत्युक्तः शम्भुना रामः प्रसादप्रवणोऽभवत्}
{दिव्यरूपधरः श्रीमानद्भुताद्भुतदर्शनः}% ४२

\twolineshloka
{तथा तं रूपमालोक्य नरवानरदेवताः}
{न द्रष्टुमपिशक्तास्ते तेजसं महदद्भुतम्}% ४३

\threelineshloka
{भयाद्वै त्रिदशश्रेष्ठाः प्रणेमुश्चातिभक्तितः}%
{भीता विज्ञाय रामोऽपि नरवानरदेवताः}
{मायामानुषतां प्राप्य स देवानब्रवीत्पुनः}% ४४

\uvacha{रामचन्द्र उवाच}

\twolineshloka
{शृणुध्वं देवता यो मां प्रत्यहं संस्तुविष्यति}
{स्तवेन शम्भुनोक्तेन देवतुल्यो भवेन्नरः}% ४५

\twolineshloka
{विमुक्तः सर्वपापेभ्यो मत्स्वरूपं समश्नुते}
{रणे जयमवाप्नोति न क्वचित्प्रतिहन्यते}% ४६

\twolineshloka
{भूतवेतालकृत्याभिर्ग्रहैश्चापि न बाध्यते}
{अपुत्रो लभते पुत्रं पतिं विन्दति कन्यका}% ४७

\twolineshloka
{दरिद्रः श्रियमाप्नोति सत्ववान् शीलवान्भवेत्}
{आत्मतुल्यबलः श्रीमाञ्जायते नात्र संशयः}% ४८

\twolineshloka
{निर्विघ्नं सर्वकार्येषु सर्वारम्भेषु वै नृणाम्}
{यं यं कामयते मर्त्यः सुदुर्लभमनोरथम्}% ४९

\threelineshloka
{षण्मासात्सिद्धिमाप्नोति स्तवस्यास्य प्रसादतः}
{यत्पुण्यं सर्वतीर्थेषु सर्वयज्ञेषु यत्फलम्}
{तत्फलं कोटिगुणितं स्तवेनानेन लभ्यते}% ५०

॥इति श्रीपाद्मे महापुराणे पञ्चपञ्चाशत्साहस्र्यां संहितायामुत्तरखण्डे उमामहेश्वरसंवादे विश्वदर्शनं नाम त्रिचत्वारिंशदधिकद्विशततमोऽध्याये श्री-शम्भुकृत-रामस्तवः सम्पूर्णः॥


