\sect{विवाह-मङ्गलाष्टकम्}

\fourlineindentedshloka
{लक्ष्मीर्यस्य परिग्रहः कमल-भूः सूनुर्गरुत्मान् रथः}
{पौत्रश्चन्द्र-विभूषणः सुर-गुरु शेषश्च शय्या पुनः}
{ब्रह्माण्डं वर-मन्दिरं सुर-गणाः यस्य प्रभोः सेवकाः}
{स त्रैलोक्य-कुटुम्ब-पालन-परः कुर्याद्धरिर्मङ्गलम्}

\fourlineindentedshloka
{ब्रह्मा-वायु-गिरीश-शेष-गरुडा देवेन्द्र-कामौ गुरुः}
{चन्द्रार्कौ वरुणानलौ मनु-यमौ वित्तेश-विघ्नेश्वरौ}
{नासत्यौ निरृतिर्मरुद् गण-युताः पर्जन्य-मित्रादयः}
{सस्त्रीकाः सुर-पुङ्गवाः प्रतिदिनं कुर्वन्तु नो मङ्गलम्}

\fourlineindentedshloka
{विश्वामित्र-पराशरौर्व-भृगवो-ऽगस्त्यः पुलस्त्यः क्रतुः}
{श्रीमानत्रि-मरीचि-कौत्स-पुलहाः शक्तिर्वसिष्ठोऽङ्गिराः}
{माण्डव्यो जमदग्नि-गौतम-भरद्वाजादयस्तापसाः}
{श्रीमद् विष्णु-पदाब्ज-भक्ति-निरताः कुर्वन्तु नो मङ्गलम्}

\fourlineindentedshloka
{मान्धाता नहुषोऽम्बरीष-सगरौ राजा पृथुर्हैहयः}
{श्रीमान् धर्म-सुतो नलो दशरथो रामो ययातिर्यदुः}
{इक्ष्वाकुश्च विभीषणश्च भरतश्चोत्तानपाद-ध्रुवा-}
{वित्येते भुवि भूभुजाः प्रतिदिनं कुर्वन्तु नो मङ्गलम्}

\fourlineindentedshloka
{श्री-मेरुर्हिमवांश्च मन्दर-गिरिः कैलास-शैलस्तथा}
{माहेन्द्रो मलयश्च विन्ध्य-निषधौ सिंहस्तथा रैवतः}
{सह्याद्रिर्वर-गन्धमादन-गिरिर्मैनाक-गोमान्तका-}
{वित्येते भुवि भूधराश्च प्रतिदिनं सततं कुर्वन्तु नो मङ्गलम्}

\fourlineindentedshloka
{गङ्गा-सिन्धु-सरस्वती च यमुना गोदावरी नर्मदा}
{कृष्णा भीमरथी च फल्गु-सरयूः श्री-गण्डकी गोमती}
{कावेरी-कपिला-प्रयाग-किटिजा-नेत्रावतीत्यादयो}
{नद्यः श्रीहरि-पाद-पङ्कज-भवाः कुर्वन्तु नो मङ्गलम्}

\fourlineindentedshloka
{वेदाश्चोपनिषद्-गणाश्च विविधाः साङ्गाः पुराणान्विता}
{वेदान्ता अपि मन्त्र-तन्त्र-सहितास्तर्काः स्मृतीनां गणाः}
{काव्यालङ्कृति-नीति-नाटक-युताः शब्दाश्च नाना-विधाः}
{श्रीविष्णोर्गुण-नाम-कीर्तन-पराः कुर्वन्तु नो मङ्गलम्}

\fourlineindentedshloka
{आदित्यादि-नव-ग्रहाः शुभ-करा मेषादयो राशयो}
{नक्षत्राणि स-योगकाश्च तिथयस्तद्-देवतास्तद्-गणाः}
{मासाब्दा ऋतवस्तथैव दिवसाः सन्ध्यास्तथा रात्रयः}
{सर्वे स्थावर-जङ्गमाः प्रतिदिनं कुर्वन्तु नो मङ्गलम्}

\fourlineindentedshloka*
{इत्येतद् वर-मङ्गलाष्टकमिदं श्रीवादिराजेश्वरे-}
{णाऽख्यातं जगतामभीष्ट-फल-दं सर्वाशुभ-ध्वंसनम्}
{माङ्गल्यादि-शुभ-क्रियासु सततं सन्ध्यासु वा यः पठेद्}
{धर्मार्थादि-समस्त-वाञ्छित-फलं प्राप्नोत्यसौ मानवः}

॥इति श्रीवादिराजेश्वरयतिविरचितं मङ्गलाष्टकं सम्पूर्णम्॥