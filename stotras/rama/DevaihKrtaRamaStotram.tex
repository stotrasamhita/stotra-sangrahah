% !TeX program = XeLaTeX
% !TeX root = ../../shloka.tex

\sect{देवैः कृतं राम-स्तोत्रम्}

\uvacha{शेष उवाच}

\twolineshloka
{अथाभिषिक्तं रामं तु तुष्टुवुः प्रणताः सुराः}
{रावणाभिधदैत्येन्द्र वधहर्षितमानसाः} %१

\uvacha{देवा ऊचुः}

\twolineshloka
{जय दाशरथे सुरार्तिहञ्जय जय दानववंशदाहक}
{जय देववराङ्गनागणग्रहणव्यग्रकरारिदारक} %२

\twolineshloka
{तवयद्दनुजेन्द्रनाशनं कवयो वर्णयितुं समुत्सुकाः}
{प्रलये जगतान्ततीः पुनर्ग्रससे त्वं भुवनेशलीलया} %३

\twolineshloka
{जय जन्मजरादिदुःखकैः परिमुक्तप्रबलोद्धरोद्धर}
{जय धर्मकरान्वयाम्बुधौ कृतजन्मन्नजरामराच्युत} %४

\twolineshloka
{तव देववरस्य नामभिर्बहुपापा अपि ते पवित्रिताः}
{किमु साधुद्विजवर्यपूर्वकाः सुतनुं मानुषतामुपागताः} %५

\twolineshloka
{हरविरिञ्चिनुतं तव पादयोर्युगलमीप्सितकामसमृद्धिदम्}
{हृदि पवित्रयवादिकचिह्नितैः सुरचितं मनसा स्पृहयामहे} %६

\twolineshloka
{यदि भवान्न दधात्यभयं भुवो मदनमूर्ति तिरस्करकान्तिभृत्}
{सुरगणा हि कथं सुखिनः पुनर्ननु भवन्ति घृणामय पावन} %७

\twolineshloka
{यदा यदास्मान्दनुजाहि दुःखदास्तदा तदा त्वं भुवि जन्मभाग्भवेः}
{अजोऽव्ययोऽपीशवरोऽपि सन्विभो स्वभावमास्थाय निजं निजार्चितः} %८

\twolineshloka
{मृतसुधासदृशैरघनाशनैः सुचरितैरवकीर्य महीतलम्}
{अमनुजैर्गुणशंसिभिरीडितः प्रविश चाशु पुनर्हि स्वकं पदम्} %९

\twolineshloka
{अनादिराद्योजररूपधारी हारी किरीटी मकरध्वजाभः}
{जयं करोतु प्रसभं हतारिः स्मरारि संसेवितपादपद्मः} %१०

\twolineshloka
{इत्युक्त्वा ते सुराः सर्वे ब्रह्मेन्द्रप्रमुखा मुहुः}
{प्रणेमुररिनाशेन प्रीणिता रघुनायकम्} %११

\twolineshloka
{इति स्तुत्यातिसंहृष्टो रघुनाथो महायशाः}
{प्रोवाच तान्सुरान्वीक्ष्य प्रणतान्नतकन्धरान्} %१२

\uvacha{श्रीराम उवाच}

\twolineshloka
{सुरा वृणुत मे यूयं वरं किञ्चित्सुदुर्लभम्}
{यं कोऽपि देवो दनुजो न यक्षः प्राप सादरः} %१३

\uvacha{सुरा ऊचुः}

\twolineshloka
{स्वामिन्भगवतः सर्वं प्राप्तमस्माभिरुत्तमम्}
{यदयं निहतः शत्रुरस्माकं तु दशाननः} %१४

\twolineshloka
{यदा यदाऽसुरोऽस्माकं बाधां परिदधाति भोः}
{तदा तदेति कर्तव्यमेतावद्वैरिनाशनम्} %१५

\onelineshloka*
{तथेत्युक्त्वा पुनर्वीरः प्रोवाच रघुनन्दनः}

\uvacha{श्रीराम उवाच}
\onelineshloka
{सुराः शृणुत मद्वाक्यमादरेण समन्विताः} %१६

\twolineshloka
{भवत्कृतं मदीयैर्वैगुणैर्ग्रथितमद्भुतम्}
{स्तोत्रं पठिष्यति मुहुः प्रातर्निशि सकृन्नरः} %१७

\twolineshloka
{तस्य वैरि पराभूतिर्न भविष्यति दारुणा}
{न च दारिद्र्यसंयोगो न च व्याधिपराभवौ} %१८

\twolineshloka
{मदीयचरणद्वन्द्वे भक्तिस्तेषां तु भूयसी}
{भविष्यति मुदायुक्ते स्वान्ते पुंसां तु पाठतः} %१९

\twolineshloka
{इत्युक्त्वा सोऽभवत्तूष्णीं नरदेवशिरोमणिः}
{सुराः सर्वे प्रहृष्टास्ते ययुर्लोकं स्वकं स्वकम्} %२०

॥इति श्रीपद्मपुराणे पातालखण्डे शेषवात्स्यायनसंवादे रामाश्वमेधे अगस्त्यसमागमोनाम पञ्चमोऽध्यायान्तर्गतं देवैः कृतं श्री-राम-स्तोत्रं सम्पूर्णम्॥
