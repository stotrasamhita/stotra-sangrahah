\sect{सीता-विवाह-चूर्णिका}

(१) अथ जनक-नृपतिर्-मनोरथ-महार्णव-पारं गत्वा,\\
(२) परमानन्द-जलधौ निमग्नः\\
(३) जगच्चक्षुः-कुल-भूषणस्य\\
(४) जगत्-त्रय-संरक्षण-दीक्षितस्य\\
(५) नत-जन-कल्प-तरोः\\
(६) श्रीकोदण्ड-दीक्षा-गुरोः\\
(७) विवाह-महोत्सवाय,\\

(८) क्षिति-रथ-वर–पूरित-मनो-रथ–भगीरथ-वंश-जातं\\
(९) अतिरथ-महारथ-पृथिवी-पति-वन्दितं\\
(१०) दशरथ-महाराजं\\
(११) रथ-गज-तुरग-पदाति-सहितं\\
(१२) स-मन्त्रि-पुरोहितं\\
(१३) स-पुत्र-कलत्रं\\
(१४) सामात्यं स-परिवारम् आनाय्य सत्कृत्य,\\

(१५) नाना-विध-क्रमुक-नारिकेल-रम्भादि-फल-युक्त-सहस्र-शातकुम्भ-स्तम्भ-भूषिते\\
(१६) जम्भारि-चिर-काङ्क्षिते\\
(१७) परिपूर्ण-मनोरथ-परमानन्द-पौर-भक्तजन-परस्परकृत-सल्लापे\\
(१८) सकल-जन-मनोहरालापे\\

(१९) किरीट-कवच-कुण्डल-केयूर-नूपुर-मुक्ताहार–विभूषित–नानाविध-मृदुल-चित्रासनस्थ–नानादेशागत–राजवर्य-वृन्दे\\
(२०) लोचनानन्दे\\
(२१) नानाविध-वर्ण-स्वर्णपुष्प-स्तबक-कोटि-देदीप्यमान–मलय-चन्दनारु-गुग्गुलु-गोघृतादि-दिव्य-परिमल-वासना-वस्तु-घुमघुमायमान-सभान्तरे\\
(२२) निरन्तर-मणि-खचित-मुकुर-परम्परे\\
(२३) भेरी-मृदङ्ग-मद्दल-काहल-दुन्दुभि-तुरीय-तुम्बुरु-वीणा-वेणु-नूपुर-मड्डुक-डिण्डिम-डमरुक-जञ्झरि-झल्लरि-धवलशङ्ख-पणव-पटहाद्यष्टादश-वाद्य-घोष-सन्तत-शब्दायमान-सकल-दिगन्तरे\\
(२४) नारद-तुम्बुरु-वीणागान-मनोहरे\\
(२५) सवत्स-गोसहस्राद्यनेक-कोटि-हिरण्य-वस्त्राभरण-भूदान-बहुमान-सन्तोषित–पात्रभूतोभय-सम्बन्धि-राजश्रेष्ठ-वंशावलि-नाम-गोत्र-प्रवरोच्चारण-निपुण–वसिष्ठ-वामदेव-विश्वामित्र-शतानन्दादि-सकल-मुनिगण-सङ्कीर्ण–विवाह-वेदि-मध्ये\\

(२६) कृत-स्तव-सिद्ध-साध्ये\\
(२७) अब्जभव-जिष्णुमुख-हृष्ट-सकल-त्रिदश-सप्तमरुद्-अष्टवसु-यक्षगण-सिद्धजन-रुद्रगण-सत्कृत–पवित्रतर–मित्र-कुल-रत्न-सुगुण-स्तव-समस्त-प्रशस्त-बिरुद-स्तोम-निनदे\\
(२८) निरस्त-धनदेन्द्र-बिरुदे\\
(२९) सङ्क्रन्दनाद्यमर-वृन्दावकीर्यमाण-मन्दार-कुसुम-मुख–नन्दक-वर्षे\\
(३०) नवरत्न-मणिखचित-कनकमय-मण्टपे\\

(३१) किञ्चिद्-आरब्ध-तारुण्यं\\
(३२) नत-जनैक-कारुण्यं\\
(३३) विदलित-महेश-चापं\\
(३४) विधिमुख-सुरादिविदित-प्रतापं\\
(३५) परिहृत-मुनिजन-तापं\\
(३६) भानुकुल-रत्न-दीपं\\
(३७) सम्पूरित-भक्त-जन-कामं\\
(३८) सजल-जलधर-श्यामं\\
(३९) सकल-लोकाभिरामं\\
(४०) श्रीरामं सन्निवेश्य,\\

(४१) केतकी-चम्पक-मल्लिका-कुसुमाद्यलङ्कृत-नीलालक-वेणीं\\
(४२) कलहंस-गामिनीं\\
(४३) करुणावलोकिनीं\\
(४४) अगणित-कल्याण-गुण-गण-शीलाम्\\
(४५) अनघ-मणि-खचित-हेमकण्ठीं\\
(४६) स्वर-जित-कल-कण्ठीं\\
(४७) हाटक-मणिमय-ताटङ्कालङ्कृत-कपोलाम्\\
(४८) अरविन्द-सुन्दर-मन्दहास-मुखीं\\
(४९) श्रीरघुनन्दनाभिमुखीं कृत्वा\\

(५०) ‘इयं सीता मम सुता सहधर्मचरी तव।\\
प्रतीच्छ चैनां भद्रं ते पाणिं गृह्णीष्व पाणिना॥’\\
(५१) इति क्षीराब्धिरिव महालक्ष्मीं श्रीमते नारायणाय\\
(५२) शोभन-मन्त्राक्षत-हिरण्योदक-धारा-सहितं\\
(५३) मन्द-स्मितां सीतां\\
(५४) श्रीराम-चन्द्र-करारविन्दे समर्पयामास॥\\

\closesub

\twolineshloka*
{कन्यां कनकसम्पन्नां सर्वाभरणभूषिताम्}
{दास्यामि विष्णवे तुभ्यं ब्रह्मलोकजिगीषया}

\twolineshloka*
{विश्वम्भरः सर्वभूताः साक्षिण्यः सर्वदेवताः}
{इमां कन्यां प्रदास्यामि पितॄणां तारणाय च}

ततः—

\twolineshloka*
{गौरीं कन्यामिमां विप्र यथाशक्ति विभूषिताम्}
{गोत्राय शर्मणे तुभ्यं दत्तां विप्र समाश्रय}

\twolineshloka*
{कन्ये ममाग्रतो भूयाः कन्ये मे देवि पार्श्वयोः}
{कन्ये मे पृष्ठतो भूयास्त्वद्दानान्मोक्षमाप्नुयाम्}

\twolineshloka*
{मम वंशकुले जाता पालिता वत्सराष्टकम्}
{तुभ्यं विप्र मया दत्ता पुत्रपौत्रप्रवर्धिनी} 

इति कन्यां निवेदयेत्।