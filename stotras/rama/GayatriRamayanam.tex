% !TeX program = XeLaTeX
% !TeX root = ../../shloka.tex

\sect{गायत्री-रामायणम्}

\twolineshloka*
{शुक्लाम्बरधरं विष्णुं शशिवर्णं चतुर्भुजम्}
{प्रसन्नवदनं ध्यायेत् सर्वविघ्नोपशान्तये}

\twolineshloka*
{वागीशाद्याः सुमनसः सर्वार्थानामुपक्रमे}
{यं नत्वा कृतकृत्याः स्युस्तं नमामि गजाननम्}

\dnsub{श्री-गुरु-प्रार्थना}

\twolineshloka*
{गुरुर्ब्रह्मा गुरुर्विष्णुर्गुरुर्देवो महेश्वरः}
{गुरुः साक्षात् परं ब्रह्म तस्मै श्री-गुरवे नमः}

\twolineshloka*
{सदाशिवसमारम्भां शङ्कराचार्यमध्यमाम्}
{अस्मदाचार्यपर्यन्तां वन्दे गुरुपरम्पराम्}

\twolineshloka*
{अखण्डमण्डलाकारं व्याप्तं येन चराचरम्}
{तत्पदं दर्शितं येन तस्मै श्री-गुरवे नमः}

\dnsub{श्री-सरस्वती-प्रार्थना}
\fourlineindentedshloka*
{दोर्भिर्युक्ता चतुर्भिः स्फटिकमणिनिभैरक्षमालां दधाना}
{हस्तेनैकेन पद्मं सितमपि च शुकं पुस्तकं चापरेण}
{भासा कुन्देन्दुशङ्खस्फटिकमणिनिभा भासमानाऽसमाना}
{सा मे वाग्देवतेयं निवसतु वदने सर्वदा सुप्रसन्ना}


\dnsub{श्री-वाल्मीकि-नमस्क्रिया}
\twolineshloka
{कूजन्तं राम रामेति मधुरं मधुराक्षरम्}
{आरुह्य कविताशाखां वन्दे वाल्मीकिकोकिलम्}

\twolineshloka
{वाल्मीकेर्मुनिसिंहस्य कवितावनचारिणः}
{शृण्वन् रामकथानादं को न याति परां गतिम्}

\twolineshloka
{यः पिबन् सततं रामचरितामृतसागरम्}
{अतृप्तस्तं मुनिं वन्दे प्राचेतसमकल्मषम्}

\resetShloka
\dnsub{श्री-हनुमन्नमस्क्रिया}

\twolineshloka
{गोष्पदीकृत-वाराशिं मशकीकृत-राक्षसम्}
{रामायण-महामाला-रत्नं वन्देऽनिलात्मजम्}

\twolineshloka
{अञ्जनानन्दनं वीरं जानकीशोकनाशनम्}
{कपीशमक्षहन्तारं वन्दे लङ्काभयङ्करम्}

\twolineshloka
{उल्लङ्घ्य सिन्धोः सलिलं सलीलं यः शोकवह्निं जनकात्मजायाः}
{आदाय तेनैव ददाह लङ्कां नमामि तं प्राञ्जलिराञ्जनेयम्}

\twolineshloka
{आञ्जनेयमतिपाटलाननं काञ्चनाद्रि-कमनीय-विग्रहम्}
{पारिजात-तरुमूल-वासिनं भावयामि पवमान-नन्दनम्}

\twolineshloka
{यत्र यत्र रघुनाथकीर्तनं तत्र तत्र कृतमस्तकाञ्जलिम्}
{बाष्पवारिपरिपूर्णलोचनं मारुतिं नमत राक्षसान्तकम्}

\twolineshloka
{मनोजवं मारुततुल्यवेगं जितेन्द्रियं बुद्धिमतां वरिष्ठम्}
{वातात्मजं वानरयूथमुख्यं श्रीरामदूतं शिरसा नमामि}

\resetShloka

\dnsub{श्री-रामायण-प्रार्थना}
\fourlineindentedshloka
{यः कर्णाञ्जलिसम्पुटैरहरहः सम्यक् पिबत्यादरात्}
{वाल्मीकेर्वदनारविन्दगलितं रामायणाख्यं मधु}
{जन्म-व्याधि-जरा-विपत्ति-मरणैरत्यन्त-सोपद्रवं}
{संसारं स विहाय गच्छति पुमान् विष्णोः पदं शाश्वतम्}

\twolineshloka
{तदुपगत-समास-सन्धियोगं सममधुरोपनतार्थ-वाक्यबद्धम्}
{रघुवरचरितं मुनिप्रणीतं दशशिरसश्च वधं निशामयध्वम्}

\twolineshloka
{वाल्मीकि-गिरिसम्भूता रामसागरगामिनी}
{पुनातु भुवनं पुण्या रामायणमहानदी}

\twolineshloka
{श्लोकसारजलाकीर्णं सर्गकल्लोलसङ्कुलम्}
{काण्डग्राहमहामीनं वन्दे रामायणार्णवम्}

\twolineshloka
{वेदवेद्ये परे पुंसि जाते दशरथात्मजे}
{वेदः प्राचेतसादासीत् साक्षाद्रामायणात्मना}

\resetShloka
\dnsub{श्री-राम-ध्यानम्}

\fourlineindentedshloka
{वैदेहीसहितं सुरद्रुमतले हैमे महामण्डपे}
{मध्ये पुष्पकमासने मणिमये वीरासने सुस्थितम्}
{अग्रे वाचयति प्रभञ्जनसुते तत्त्वं मुनिभ्यः परं}
{व्याख्यान्तं भरतादिभिः परिवृतं रामं भजे श्यामलम्}

\fourlineindentedshloka
{वामे भूमिसुता पुरश्च हनुमान् पश्चात् सुमित्रासुतः}
{शत्रुघ्नो भरतश्च पार्श्वदलयोर्वाय्वादिकोणेषु च}
{सुग्रीवश्च विभीषणश्च युवराट् तारासुतो जाम्बवान्}
{मध्ये नीलसरोजकोमलरुचिं रामं भजे श्यामलम्}

\twolineshloka
{रामं रामानुजं सीतां भरतं भरतानुजम्}
{सुग्रीवं वायुसूनुं च प्रणमामि पुनः पुनः}

\twolineshloka
{नमोऽस्तु रामाय सलक्ष्मणाय देव्यै च तस्यै जनकात्मजायै}
{नमोऽस्तु रुद्रेन्द्रयमानिलेभ्यो नमोऽस्तु चन्द्रार्कमरुद्गणेभ्यः}

\centerline{\textbf{ॐ श्री-गुरुभ्यो नमः।}}

\resetShloka
\dnsub{गायत्री रामयाणम्}
\annotwolineshloka
{तपः स्वाध्यायनिरतं तपस्वी वाग्विदां वरम्}
{नारदं परिपप्रच्छ वाल्मीकिर्मुनिपुङ्गवम्}{१-१-१}

\annotwolineshloka
{स हत्वा राक्षसान् सर्वान् यज्ञघ्नान् रघुनन्दनः}
{ऋषिभिः पूजितः सम्यक् यथेन्द्रो विजये पुरा}{१-३०-२३}

\annotwolineshloka
{विश्वामित्रस्तु धर्मात्मा श्रुत्वा जनकभाषितम्}
{वत्स राम धनुः पश्य इति राघवमब्रवीत्}{१-६७-१२}

\annotwolineshloka
{तुष्टावास्य तदा वंशं  प्रविश्य च विशाम्पतेः}
{शयनीयं नरेन्द्रस्य तदासाद्य व्यतिष्ठत}{२-१५-२०}

\annotwolineshloka
{वनवासं हि सङ्ख्याय वासांस्याभरणानि च}
{भर्तारमनुगच्छन्त्यै सीतायै श्वशुरो ददौ}{२-४०-१५}

\annotwolineshloka
{राजा सत्यं च धर्मं च  राजा कुलवतां कुलम्}
{राजा माता पिता चैव राजा हितकरो नृणाम्}{२-६७-३४}

\annotwolineshloka
{निरीक्ष्य स मुहूर्तं तु ददर्श भरतो गुरुम्}
{उटजे राममासीनं जटामण्डलधारिणम्}{२-९९-२५}

\annotwolineshloka
{यदि बुद्धिः कृता द्रष्टुम् अगस्त्यं तं महामुनिम्}
{अद्यैव गमने बुद्धिं रोचयस्व महायशाः}{३-११-४४}

\annotwolineshloka
{भरतस्यार्यपुत्रस्य श्वश्रूणां मम च प्रभो}
{मृगरूपमिदं व्यक्तं विस्मयं जनयिष्यति}{३-४३-१७}

\annotwolineshloka
{गच्छ शीघ्रमितो राम सुग्रीवं तं महाबलम्}
{वयस्यं तं कुरु क्षिप्रमितो गत्वाऽद्य राघव}{३-७२-१७}

\annotwolineshloka
{देशकालौ प्रतीक्षस्व क्षममाणः प्रियाप्रिये}
{सुखदुःखसहः काले  सुग्रीववशगो भव}{४-२२-२०}

\annotwolineshloka
{वन्द्यास्ते तु तपः सिद्धास्तपसा वीतकल्मषाः}
{प्रष्टव्याश्चापि सीतायाः प्रवृत्तिं विनयान्वितैः}{४-४३-३४}

\annotwolineshloka
{स निर्जित्य पुरीं श्रेष्ठां लङ्कां तां कामरूपिणीम्}
{विक्रमेण महातेजा हनूमान्मारुतात्मजः}{५-४-१}

\annotwolineshloka
{धन्या देवाः सगन्धर्वाः सिद्धाश्च परमर्षयः}
{मम पश्यन्ति ये नाथं रामं राजीवलोचनम्}{५-२६-४१}

\annotwolineshloka
{मङ्गलाभिमुखी तस्य सा तदासीन्महाकपेः}
{उपतस्थे विशालाक्षी प्रयता हव्यवाहनम्}{५-५३-२६}

\annofourlineindentedshloka
{हितं महार्थं मृदु हेतुसंहितम्}
{व्यतीतकालायतिसम्प्रतिक्षमम्}
{निशम्य तद्वाक्यमुपस्थितज्वरः}
{प्रसङ्गवानुत्तरमेतदब्रवीत्}{६-१०-२७}

\annotwolineshloka
{धर्मात्मा रक्षसां श्रेष्ठः सम्प्राप्तोऽयं विभीषणः}
{लङ्कैश्वर्यं ध्रुवं श्रीमानयं प्राप्नोत्यकण्टकम्}{६-४१-६८}

\annofourlineindentedshloka
{यो वज्रपाताशनिसन्निपातान्}{न चुक्षुभे नापि चचाल राजा}
{स रामबाणाभिहतो भृशार्तः}{चचाल चापं च मुमोच वीरः}{६-५९-१४०}

\annotwolineshloka
{यस्य विक्रममासाद्य राक्षसा निधनं गताः}
{तं मन्ये राघवं वीरं नारायणमनामयम्}{६-७२-११}

\annotwolineshloka
{न ते ददर्शिरे रामं दहन्तमरिवाहिनीम्}
{मोहिताः परमास्त्रेण गान्धर्वेण महात्मना}{६-९४-२६}

\annotwolineshloka
{प्रणम्य देवताभ्यश्च ब्राह्मणेभ्यश्च मैथिली}
{बद्धाञ्जलिपुटा चेदमुवाचाग्निसमीपतः}{६-११९-२३}

\annotwolineshloka
{चलनात्पर्वतेन्द्रस्य गणा देवाश्च कम्पिताः}
{चचाल पार्वती चापि तदाऽऽश्लिष्टा महेश्वरम्}{७-१६-२६}

\annotwolineshloka
{दाराः पुत्राः पुरं राष्ट्रं भोगाच्छादनभोजनम्}
{सर्वमेवाविभक्तं नौ भविष्यति हरीश्वर}{७-३४-४१}

\annotwolineshloka
{यामेव रात्रिं शत्रुघ्नः पर्णशालां समाविशत्}
{तामेव रात्रिं सीताऽपि प्रसूता दारकद्वयम्}{७-६६-१}

\twolineshloka*
{इदं रामायणं कृत्स्नं गायत्रीबीजसंयुतम्}
{त्रिसन्ध्यं यः पठेन्नित्यं सर्वपापैः प्रमुच्यते}

॥इति श्री-गायत्री रामायणं सम्पूर्णम्॥

\dnsub{मङ्गलश्लोकाः}
\resetShloka
\fourlineindentedshloka
{स्वस्ति प्रजाभ्यः परिपालयन्तां}
{न्यायेन मार्गेण महीं महीशाः}
{गोब्राह्मणेभ्यः शुभमस्तु नित्यं}
{लोकाः समस्ताः सुखिनो भवन्तु}

\twolineshloka
{काले वर्षतु पर्जन्यः पृथिवी सस्यशालिनी}
{देशोऽयं क्षोभरहितो ब्राह्मणाः सन्तु निर्भयाः}

\twolineshloka
{अपुत्राः पुत्रिणः सन्तु पुत्रिणः सन्तु पौत्रिणः}
{अधनाः सधनाः सन्तु जीवन्तु शरदां शतम्}

\twolineshloka
{चरितं रघुनाथस्य शतकोटि-प्रविस्तरम्}
{एकैकमक्षरं पुंसां महापातकनाशनम्}

\twolineshloka
{शृण्वन् रामायणं भक्त्या यः पादं पदमेव वा}
{स याति ब्रह्मणः स्थानं ब्रह्मणा पूज्यते सदा}

\twolineshloka
{रामाय रामभद्राय रामचन्द्राय वेधसे}
{रघुनाथाय नाथाय सीतायाः पतये नमः}

\twolineshloka
{यन्मङ्गलं सहस्राक्षे सर्वदेवनमस्कृते}
{वृत्रनाशे समभवत् तत्ते भवतु मङ्गलम्}

\twolineshloka
{यन्मङ्गलं सुपर्णस्य विनताऽकल्पयत् पुरा}
{अमृतं प्रार्थयानस्य तत्ते भवतु मङ्गलम्}

\twolineshloka
{अमृतोत्पादने दैत्यान् घ्नतो वज्रधरस्य यत्}
{अदितिर्मङ्गलं प्रादात् तत्ते भवतु मङ्गलम्}

\twolineshloka
{त्रीन् विक्रमान् प्रक्रमतो विष्णोरमिततेजसः}
{यदासीन्मङ्गलं राम तत्ते भवतु मङ्गलम्}

\twolineshloka
{ऋषयः सागरा द्वीपा वेदा लोका दिशश्च ते}
{मङ्गलानि महाबाहो दिशन्तु तव सर्वदा}

\twolineshloka
{मङ्गलं कोसलेन्द्राय महनीयगुणाब्धये}
{चक्रवर्तितनूजाय सार्वभौमाय मङ्गलम्}

\fourlineindentedshloka*
{कायेन वाचा मनसेन्द्रियैर्वा}
{बुद्‌ध्याऽऽत्मना वा प्रकृतेः स्वभावात्}
{करोमि यद्यत् सकलं परस्मै}
{नारायणायेति समर्पयामि}
