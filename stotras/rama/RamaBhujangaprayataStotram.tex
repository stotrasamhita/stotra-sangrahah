% !TeX program = XeLaTeX
% !TeX root = ../../shloka.tex

\sect{रामभुजङ्गप्रयातस्तोत्रम्}

\fourlineindentedshloka
{विशुद्धं परं सच्चिदानन्दरूपम्}
{गुणाधारमाधारहीनं वरेण्यम्}
{महान्तं विभान्तं गुहान्तं गुणान्तम्}
{सुखान्तं स्वयं धाम रामं प्रपद्ये}%१

\fourlineindentedshloka
{शिवं नित्यमेकं विभुं तारकाख्यम्}
{सुखाकारमाकारशून्यं सुमान्यम्}
{महेशं कलेशं सुरेशं परेशम्}
{नरेशं निरीशं महीशं प्रपद्ये}%२

\fourlineindentedshloka
{यदावर्णयत् कर्णमूलेऽन्तकाले}
{शिवो राम रामेति रामेति काश्याम्}
{तदेकं परं तारकब्रह्मरूपम्}
{भजेऽहं भजेऽहं भजेऽहं भजेऽहम्}%३

\fourlineindentedshloka
{महारत्नपीठे शुभे कल्पमूले}
{सुखासीनमादित्यकोटिप्रकाशम्}
{सदा जानकीलक्ष्मणोपेतमेकम्}
{सदा रामचन्द्रं भजेऽहं भजेऽहम्}%४

\fourlineindentedshloka
{क्वणद्रत्नमञ्जीरपादारविन्दम्}
{लसन्मेखलाचारुपीताम्बराढ्यम्}
{महारत्नहारोल्लसत्‌ कौस्तुभाङ्गम्}
{नदच्चञ्चरीमञ्जरीलोलमालम्}%५

\fourlineindentedshloka
{लसच्चन्द्रिकास्मेरशोणाधराभम्}
{समुद्यत् पतङ्गेन्दुकोटिप्रकाशम्}
{नमद्‌ब्रह्मरुद्रादिकोटीररत्न-}
{स्फुरत् कान्तिनीराजनाराधिताङ्घ्रिम्}%६

\fourlineindentedshloka
{पुरः प्राञ्जलीनाञ्जनेयादिभक्तान्‌}
{स्वचिन्मुद्रया भद्रया बोधयन्तम्‌}
{भजेऽहं भजेऽहं सदा रामचन्द्रम्}
{त्वदन्यं न मन्ये न मन्ये न मन्ये}%७

\fourlineindentedshloka
{यदा मत्समीपं कृतान्तः समेत्य}
{प्रचण्डप्रतापैर्भटैर्भीषयेन्माम्}
{तदाऽऽविष्करोषि त्वदीयं स्वरूपम्}
{तदापत् प्रणाशं सकोदण्डबाणम्}%८

\fourlineindentedshloka
{निजे मानसे मन्दिरे सन्निधेहि}
{प्रसीद प्रसीद प्रभो रामचन्द्र}
{ससौमित्रिणा कैकयीनन्दनेन}
{स्वशक्त्याऽऽनुभक्त्या च संसेव्यमान}%९

\fourlineindentedshloka
{स्वभक्ताग्रगण्यैः कपीशैर्महीशैः}
{अनीकैरनेकैश्च राम प्रसीद}
{नमस्ते नमोऽस्त्वीश राम प्रसीद}
{प्रशाधि प्रशाधि प्रकाशं प्रभो माम्}%१०

\fourlineindentedshloka
{त्वमेवासि दैवं परं मे यदेकम्}
{सुचैतन्यमेतत् त्वदन्यं न मन्ये}
{यतोऽभूदमेयं वियद्वायुतेजो-}
{जलोर्व्यादिकार्यं चरं चाचरं च}%११

\fourlineindentedshloka
{नमः सच्चिदानन्दरूपाय तस्मै}
{नमो देवदेवाय रामाय तुभ्यम्}
{नमो जानकीजीवितेशाय तुभ्यम्}
{नमः पुण्डरीकायताक्षाय तुभ्यम्}%१२

\fourlineindentedshloka
{नमो भक्तियुक्तानुरक्ताय तुभ्यम्}
{नमः पुण्यपुञ्जैकलभ्याय तुभ्यम्}
{नमो वेदवेद्याय चाद्याय पुंसे}
{नमः सुन्दरायेन्दिरावल्लभाय}%१३

\fourlineindentedshloka
{नमो विश्वकर्त्रे नमो विश्वहर्त्रे}
{नमो विश्वभोक्त्रे नमो विश्वमात्रे}
{नमो विश्वनेत्रे नमो विश्वजेत्रे}
{नमो विश्वपित्रे नमो विश्वमात्रे}%१४

\fourlineindentedshloka
{शिलाऽपि त्वदङ्घ्रिक्षमासङ्गिरेणु-}
{प्रसादाद्धि चैतन्यमाधत्त राम}
{नरस्त्वत् पदद्वन्द्वसेवाविधानात्}
{सुचैतन्यमेतेति किं चित्रमद्य}%१५

\fourlineindentedshloka
{पवित्रं चरित्रं विचित्रं त्वदीयम्}
{नरा ये स्मरन्त्यन्वहं रामचन्द्र}
{भवन्तं भवान्तं भरन्तं भजन्तो}
{लभन्ते कृतान्तं न पश्यन्त्यतोऽन्ते}%१६

\fourlineindentedshloka
{स पुण्यः स गण्यः शरण्यो ममायम्}
{नरो वेद यो देवचूडामणिं त्वाम्}
{सदाकारमेकं चिदानन्दरूपम्}
{मनोवागगम्यं परन्धाम राम}%१७

\fourlineindentedshloka
{प्रचण्डप्रतापप्रभावाभिभूत-}
{प्रभूतारिवीर प्रभो रामचन्द्र}
{बलं ते कथं वर्ण्यतेऽतीव बाल्ये}
{यतोऽखण्डि चण्डीशकोदण्डदण्डः}%१८

\fourlineindentedshloka
{दशग्रीवमुग्रं सपुत्रं समित्रम्}
{सरिद्दुर्गमध्यस्थरक्षोगणेशम्}
{भवन्तं विना राम वीरो नरो वा-}
{ऽसुरो वाऽमरो वा जयेत् कस्त्रिलोक्याम्}%१९

\fourlineindentedshloka
{सदा राम रामेति रामामृतं ते}
{सदाराममानन्दनिष्यन्दकन्दम्}
{पिबन्तं नमन्तं सुदन्तं हसन्तम्}
{हनूमन्तमन्तर्भजे तं नितान्तम्}%२०

\fourlineindentedshloka
{सदा राम रामेति रामामृतं ते}
{सदाराममानन्दनिष्यन्दकन्दम्}
{पिबन्नन्वहं नन्वहं नैव मृत्योः}
{बिभेमि प्रसादादसादात् तवैव}%२१

\fourlineindentedshloka
{असीतासमेतैरकोदण्डभूषैः}
{असौमित्रिवन्द्यैरचण्डप्रतापैः}
{अलङ्केशकालैरसुग्रीवमित्रैः}
{अरामाभिधेयैरलं देवतैर्नः}%२२

\fourlineindentedshloka
{अवीरासनस्थैरचिन्मुद्रिकाढ्यैः}
{अभक्ताञ्जनेयादितत्त्वप्रकाशैः}
{अमन्दारमूलैरमन्दारमालैः}
{अरामाभिधेयैरलं देवतैर्नः}%२३

\fourlineindentedshloka
{असिन्धुप्रकोपैरवन्द्यप्रतापैः}
{अबन्धुप्रयाणैरमन्दस्मिताढ्यैः}
{अदण्डप्रवासैरखण्डप्रबोधैः}
{अरामभिदेयैरलं देवतैर्नः}%२४

\fourlineindentedshloka
{हरे राम सीतापते रावणारे}
{खरारे मुरारेऽसुरारे परेति}
{लपन्तं नयन्तं सदा कालमेव}
{समालोकयालोकयाशेषबन्धो}%२५

\fourlineindentedshloka
{नमस्ते सुमित्रासुपुत्राभिवन्द्य}
{नमस्ते सदा कैकयीनन्दनेड्य}
{नमस्ते सदा वानराधीशवन्द्य}
{नमस्ते नमस्ते सदा रामचन्द्र}%२६

\fourlineindentedshloka
{प्रसीद प्रसीद प्रचण्डप्रताप}
{प्रसीद प्रसीद प्रचण्डारिकाल}
{प्रसीद प्रसीद प्रपन्नानुकम्पिन्‌}
{प्रसीद प्रसीद प्रभो रामचन्द्र}%२७

\fourlineindentedshloka
{भुजङ्गप्रयातं परं वेदसारम्}
{मुदा रामचन्द्रस्य भक्त्या च नित्यम्}
{पठन्‌ सन्ततं चिन्तयन्‌ स्वान्तरङ्गे}
{स एव स्वयं रामचन्द्रः स धन्यः}%२८

{॥इति~श्रीमच्छङ्कराचार्यविरचितं श्री~रामभुजङ्गप्रयातस्तोत्रं  सम्पूर्णम्॥}
\closesection
