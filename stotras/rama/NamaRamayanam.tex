% !TeX program = XeLaTeX
% !TeX root = ../../shloka.tex

\sect{नामरामायणम्}
\newcommand{\jaya}{
\smallskip
\twolineshloka*
{राम राम जय राजा राम}
{राम राम जय सीता राम}
\vspace{0.5cm}}
\begin{large}
\begin{multicols}{2}
\dnsub{बालकाण्डः}
शुद्धब्रह्मपरात्पर\hfill राम\\
कालात्मकपरमेश्वर\hfill राम\\
शेषतल्पसुखनिद्रित\hfill राम\\
ब्रह्माद्यमरप्रार्थित\hfill राम\\
चण्डकिरणकुलमण्डन\hfill राम\\
श्रीमद्दशरथनन्दन\hfill राम\\
कौसल्यासुखवर्धन\hfill राम\\
विश्वामित्रप्रियधन\hfill राम\\
घोरताटकाघातक\hfill राम\\
मारीचादिनिपातक\hfill राम\\
कौशिकमखसंरक्षक\hfill राम\\
श्रीमदहल्योद्धारक\hfill राम\\
गौतममुनिसम्पूजित\hfill राम\\
सुरमुनिवरगणसंस्तुत\hfill राम\\
नाविकधावितमृदुपद\hfill राम\\
मिथिलापुरजनमोहक\hfill राम\\
विदेहमानसरञ्जक\hfill राम\\
त्र्यम्बककार्मुखभञ्जक\hfill राम\\
सीतार्पितवरमालिक\hfill राम\\
कृतवैवाहिककौतुक\hfill राम\\
भार्गवदर्पविनाशक\hfill राम\\
श्रीमदयोध्यापालक\hfill राम\\
\jaya
\dnsub{अयोध्याकाण्डः}
अगणितगुणगणभूषित\hfill राम\\
अवनीतनयाकामित\hfill राम\\
राकाचन्द्रसमानन\hfill राम\\
पितृवाक्याश्रितकानन\hfill राम\\
प्रियगुहविनिवेदितपद\hfill राम\\
तत्क्षालितनिजमृदुपद\hfill राम\\
भरद्वाजमुखानन्दक\hfill राम\\
चित्रकूटाद्रिनिकेतन\hfill राम\\
दशरथसन्ततचिन्तित\hfill राम\\
कैकेयीतनयार्थित\hfill राम\\
विरचितनिजपितृकर्मक\hfill राम\\
भरतार्पितनिजपादुक\hfill राम\\
\jaya
\dnsub{अरण्यकाण्डः}
दण्डकावनजनपावन\hfill राम\\
दुष्टविराधविनाशन\hfill राम\\
शरभङ्गसुतीक्ष्णार्चित\hfill राम\\
अगस्त्यानुग्रहवर्धित\hfill राम\\
गृध्राधिपसंसेवित\hfill राम\\
पञ्चवटीतटसुस्थित\hfill राम\\
शूर्पणखार्त्तिविधायक\hfill राम\\
खरदूषणमुखसूदक\hfill राम\\
सीताप्रियहरिणानुग\hfill राम\\
मारीचार्तिकृताशुग\hfill राम\\
विनष्टसीतान्वेषक\hfill राम\\
गृध्राधिपगतिदायक\hfill राम\\
कबन्धबाहुच्छेदन\hfill राम\\
शबरीदत्तफलाशन\hfill राम\\
\jaya

\dnsub{किष्किन्धाकाण्डः}
हनुमत्सेवितनिजपद\hfill राम\\
नतसुग्रीवाभीष्टद\hfill राम\\
गर्वितवालिसंहारक\hfill राम\\
वानरदूतप्रेषक\hfill राम\\
हितकरलक्ष्मणसंयुत\hfill राम\\
\jaya

\dnsub{सुन्दरकाण्डः}
कपिवरसन्ततसंस्मृत\hfill राम\\
तद्गतिविघ्नध्वंसक\hfill राम\\
सीताप्राणाधारक\hfill राम\\
दुष्टदशाननदूषित\hfill राम\\
शिष्टहनूमद्भूषित\hfill राम\\
सीतावेदितकाकावन\hfill राम\\
कृतचूडामणिदर्शन\hfill राम\\
कपिवरवचनाश्वासित\hfill राम\\
\jaya

\dnsub{युद्धकाण्डः}
रावणनिधनप्रस्थित\hfill राम\\
वानरसैन्यसमावृत\hfill राम\\
शोषितसरिदीशार्थित\hfill राम\\
विभीषणाभयदायक\hfill राम\\
पर्वतसेतुनिबन्धक\hfill राम\\
कुम्भकर्णशिरश्छेदक\hfill राम\\
राक्षससङ्घविमर्दक\hfill राम\\
अहिमहिरावणचारण\hfill राम\\
संहृतदशमुखरावण\hfill राम\\
विधिभवमुखसुरसंस्तुत\hfill राम\\
खःस्थितदशरथवीक्षित\hfill राम\\
सीतादर्शनमोदित\hfill राम\\
अभिषिक्तविभीषणनत\hfill राम\\
पुष्पकयानारोहण\hfill राम\\
भरद्वाजाभिनिषेवण\hfill राम\\
भरतप्राणप्रियकर\hfill राम\\
साकेतपुरीभूषण\hfill राम\\
सकलस्वीयसमानत\hfill राम\\
रत्नलसत्पीठास्थित\hfill राम\\
पट्टाभिषेकालङ्कृत\hfill राम\\
पार्थिवकुलसम्मानित\hfill राम\\
विभीषणार्पितरङ्गक\hfill राम\\
कीशकुलानुग्रहकर\hfill राम\\
सकलजीवसंरक्षक\hfill राम\\
समस्तलोकाधारक\hfill राम\\
\jaya

\dnsub{उत्तरकाण्डः}
आगतमुनिगणसंस्तुत\hfill राम\\
विश्रुतदशकण्ठोद्भव\hfill राम\\
सितालिङ्गननिर्वृत\hfill राम\\
नीतिसुरक्षितजनपद\hfill राम\\
विपिनत्याजितजनकज\hfill राम\\
कारितलवणासुरवध\hfill राम\\
स्वर्गतशम्बुकसंस्तुत\hfill राम\\
स्वतनयकुशलवनन्दित\hfill राम\\
अश्वमेधक्रतुदीक्षित\hfill राम\\
कालावेदितसुरपद\hfill राम\\
आयोध्यकजनमुक्तिद\hfill राम\\
विधिमुखविबुधानन्दक\hfill राम\\
तेजोमयनिजरूपक\hfill राम\\
संसृतिबन्धविमोचक\hfill राम\\
धर्मस्थापनतत्पर\hfill राम\\
भक्तिपरायणमुक्तिद\hfill राम\\
सर्वचराचरपालक\hfill राम\\
सर्वभवामयवारक\hfill राम\\
वैकुण्ठालयसंस्थित\hfill राम\\
नित्यानन्दपदस्थित\hfill राम\\
\jaya
\end{multicols}
\vspace{-0.5cm}
॥इति श्रीमन्नारदविरचितं नामरामायणं सम्पूर्णम्॥
\end{large}

\closesection

\fourlineindentedshloka*
{वैदेहीसहितं सुरद्रुमतले हैमे महामण्डपे}
{मध्ये पुष्पकमासने मणिमये वीरासने सुस्थितम्}
{अग्रे वाचयति प्रभञ्जनसुते तत्त्वं मुनिभ्यः परम्}
{व्याख्यान्तं भरतादिभिः परिवृतं रामं भजे श्यामलम्}

\fourlineindentedshloka*
{वामे भूमिसुता पुरश्च हनुमान् पश्चात् सुमित्रासुतः}
{शत्रुघ्नो भरतश्च पार्श्वदलयोर्वाय्वादिकोणेषु च}
{सुग्रीवश्च विभीषणश्च युवराट् तारासुतो जाम्बवान्}
{मध्ये नीलसरोजकोमलरुचिं रामं भजे श्यामलम्}
