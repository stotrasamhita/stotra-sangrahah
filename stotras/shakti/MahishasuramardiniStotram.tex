% !TeX program = XeLaTeX
% !TeX root = ../../shloka.tex

\sect{महिषासुरमर्दिनि स्तोत्रम्}
% \setlength{\shlokaspaceskip}{10pt}
\fourlineindentedshloka
{अयि गिरिनन्दिनि नन्दितमेदिनि विश्वविनोदिनि नन्दिनुते}
{गिरिवर-विन्ध्य-शिरोधिनिवासिनि विष्णुविलासिनि जिष्णुनुते}
{भगवति हे शितिकण्ठकुटुम्बिनि भूरिकुटुम्बिनि भूरिकृते}
{जय जय हे महिषासुरमर्दिनि रम्यकपर्दिनि शैलसुते}

\fourlineindentedshloka
{सुरवरवर्षिणि दुर्धरधर्षिणि दुर्मुखमर्षिणि हर्षरते}
{त्रिभुवनपोषिणि शङ्करतोषिणि किल्बिषमोषिणि घोषरते}
{दनुज-निरोषिणि दितिसुत-रोषिणि दुर्मद-शोषिणि सिन्धुसुते}
{जय जय हे महिषासुरमर्दिनि रम्यकपर्दिनि शैलसुते}

\fourlineindentedshloka
{अयि जगदम्ब-मदम्ब-कदम्ब-वनप्रिय-वासिनि हासरते}
{शिखरि शिरोमणि तुङ्ग-हिमालय-शृङ्ग-निजालय-मध्यगते}
{मधु-मधुरे मधु-कैटभ-गञ्जिनि कैटभ-भञ्जिनि रासरते}
{जय जय हे महिषासुरमर्दिनि रम्यकपर्दिनि शैलसुते}

\fourlineindentedshloka
{अयि शतखण्ड-विखण्डित-रुण्ड-वितुण्डित-शुण्ड-गजाधिपते}
{रिपु-गज-गण्ड-विदारण-चण्ड-पराक्रम-शुण्ड-मृगाधिपते}
{निज-भुज-दण्ड-निपातित-खण्ड-विपातित-मुण्ड-भटाधिपते}
{जय जय हे महिषासुरमर्दिनि रम्यकपर्दिनि शैलसुते}

\fourlineindentedshloka
{अयि रण-दुर्मद-शत्रु-वधोदित-दुर्धर-निर्जर-शक्तिभृते}
{चतुर-विचार-धुरीण-महाशिव-दूतकृत-प्रमथाधिपते}
{दुरित-दुरीह-दुराशय-दुर्मति-दानवदूत-कृतान्तमते}
{जय जय हे महिषासुरमर्दिनि रम्यकपर्दिनि शैलसुते}

\fourlineindentedshloka
{अयि शरणागत-वैरि-वधूवर-वीर-वराभय-दायकरे}
{त्रिभुवन-मस्तक-शूल-विरोधि शिरोधि कृतामल-शूलकरे}
{दुमिदुमि-तामर-दुन्दुभिनाद-महो-मुखरीकृत-तिग्मकरे}
{जय जय हे महिषासुरमर्दिनि रम्यकपर्दिनि शैलसुते}

\fourlineindentedshloka
{अयि निज-हुङ्कृति मात्र-निराकृत-धूम्रविलोचन-धूम्रशते}
{समर-विशोषित-शोणित-बीज-समुद्भव-शोणित-बीजलते}
{शिव-शिव-शुम्भ-निशुम्भ-महाहव-तर्पित-भूत-पिशाचरते}
{जय जय हे महिषासुरमर्दिनि रम्यकपर्दिनि शैलसुते}

\fourlineindentedshloka
{धनुरनु-सङ्ग-रणक्षणसङ्ग-परिस्फुर-दङ्ग-नटत्कटके}
{कनक-पिशङ्ग-पृषत्क-निषङ्ग-रसद्भट-शृङ्ग-हतावटुके}
{कृत-चतुरङ्ग-बलक्षिति-रङ्ग-घटद्बहुरङ्ग-रटद्बटुके}
{जय जय हे महिषासुरमर्दिनि रम्यकपर्दिनि शैलसुते}

\fourlineindentedshloka
{जय जय जप्य-जयेजय-शब्द-परस्तुति-तत्पर-विश्वनुते}
{भण-भण-भिञ्जिमि-भिङ्कृत-नूपुर-सिञ्जित-मोहित-भूतपते}
{नटित-नटार्ध-नटीनट-नायक-नाटित-नाट्य-सुगानरते}
{जय जय हे महिषासुरमर्दिनि रम्यकपर्दिनि शैलसुते}

\fourlineindentedshloka
{अयि सुमनः सुमनः सुमनः सुमनः सुमनोहर-कान्तियुते}
{श्रित-रजनी-रजनी-रजनी-रजनी-रजनीकर-वक्त्रवृते}
{सुनयन-विभ्रमर-भ्रमर-भ्रमर-भ्रमर-भ्रमराधिपते}
{जय जय हे महिषासुरमर्दिनि रम्यकपर्दिनि शैलसुते}

\fourlineindentedshloka
{सहित-महाहव-मल्लम-तल्लिक-मल्लित-रल्लक-मल्लरते}
{विरचित-वल्लिक-पल्लिक-मल्लिक-भिल्लिक-भिल्लिक-वर्गवृते}
{सितकृत-फुल्लसमुल्ल-सितारुण-तल्लज-पल्लव-सल्ललिते}
{जय जय हे महिषासुरमर्दिनि रम्यकपर्दिनि शैलसुते}

\fourlineindentedshloka
{अविरल-गण्ड-गलन्मद-मेदुर-मत्त-मतङ्गज-राजपते}
{त्रिभुवन-भूषण-भूत-कलानिधि रूप-पयोनिधि राजसुते}
{अयि सुद-तीजन-लालसमानस-मोहन-मन्मथ-राजसुते}
{जय जय हे महिषासुरमर्दिनि रम्यकपर्दिनि शैलसुते}

\fourlineindentedshloka
{कमल-दलामल-कोमल-कान्ति कलाकलितामल-भाललते}
{सकल-विलास-कलानिलयक्रम-केलि-चलत्कल-हंसकुले}
{अलिकुल-सङ्कुल-कुवलय-मण्डल-मौलिमिलद्भकुलालि-कुले}
{जय जय हे महिषासुरमर्दिनि रम्यकपर्दिनि शैलसुते}

\fourlineindentedshloka
{करमुरली-रव-वीजित-कूजित-लज्जित-कोकिल-मञ्जुमते}
{मिलित-पुलिन्द-मनोहर-गुञ्जित-रञ्जितशैल-निकुञ्जगते}
{निजगुणभूत-महाशबरीगण-सद्गुण-सम्भृत-केलितले}
{जय जय हे महिषासुरमर्दिनि रम्यकपर्दिनि शैलसुते}

\fourlineindentedshloka
{कटितट-पीत-दुकूल-विचित्र-मयूख-तिरस्कृत-चन्द्ररुचे}
{प्रणत-सुरासुर-मौलिमणिस्फुर-दंशुल-सन्नख-चन्द्ररुचे}
{जित-कनकाचल-मौलिपदोर्जित-निर्भर-कुञ्जर-कुम्भकुचे}
{जय जय हे महिषासुरमर्दिनि रम्यकपर्दिनि शैलसुते}

%\begin{flushright}
\fourlineindentedshloka
{विजित-सहस्रकरैक-सहस्रकरैक-सहस्रकरैकनुते}
{कृतसुरतारक-सङ्गरतारक-सङ्गरतारक-सूनुसुते}
{सुरथ-समाधि समानसमाधि समाधिसमाधि सुजातरते}
{जय जय हे महिषासुरमर्दिनि रम्यकपर्दिनि शैलसुते}

\fourlineindentedshloka
{पदकमलं करुणानिलये वरिवस्यति योऽनुदिनं स शिवे}
{अयि कमले कमलानिलये कमलानिलयः स कथं न भवेत्}
{तव पदमेव परम्पदमित्यनुशीलयतो मम किं न शिवे}
{जय जय हे महिषासुरमर्दिनि रम्यकपर्दिनि शैलसुते}

\fourlineindentedshloka
{कनकलसत्कल-सिन्धुजलैरनुसिञ्चिनुते गुण-रङ्गभुवं}
{भजति स किं न शचीकुच-कुम्भ-तटी-परिरम्भ-सुखानुभवम्}
{तव चरणं शरणं करवाणि नतामरवाणि निवासि शिवं}
{जय जय हे महिषासुरमर्दिनि रम्यकपर्दिनि शैलसुते}

\fourlineindentedshloka
{तव विमलेन्दुकुलं वदनेन्दुमलं सकलं ननु कूलयते}
{किमु पुरुहूत-पुरीन्दुमुखी-सुमुखीभिरसौ विमुखी क्रियते}
{मम तु मतं शिवनामधने भवती कृपया किमुत क्रियते}
{जय जय हे महिषासुरमर्दिनि रम्यकपर्दिनि शैलसुते}

\fourlineindentedshloka
{अयि मयि दीनदयालुतया कृपयैव त्वया भवितव्यमुमे}
{अयि जगतो जननी कृपयाऽसि यथाऽसि तथाऽनुमितासिरते}
{यदुचितमत्र भवत्युररि कुरुतादुरुतापमपाकुरुते}
{जय जय हे महिषासुरमर्दिनि रम्यकपर्दिनि शैलसुते}

॥इति~श्रीमत्परमहंसपरिव्राजकाचार्यस्य श्री-गोविन्द-भगवत्पूज्य-पाद-शिष्यस्य
श्रीमच्छङ्करभगवतः कृतौ  श्री-महिषासुरमर्दिनि-स्तोत्रं~सम्पूर्णम्॥
\setlength{\shlokaspaceskip}{24pt}