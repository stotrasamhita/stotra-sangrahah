% !TeX program = XeLaTeX
% !TeX root = ../../shloka.tex

\sect{कामाक्षी-स्तोत्रम्}
\setlength{\shlokaspaceskip}{16pt}
\fourlineindentedshloka
{कल्पानोकह-पुष्प-जाल-विलसन्-नीलालकां मातृकाम्}
{कान्तां कञ्ज-दलेक्षणां कलि-मल-प्रध्वंसिनीं कालिकाम्}
{काञ्ची-नूपुर-हार-दाम-सुभगां काञ्ची-पुरी-नायिकाम्}
{कामाक्षीं करि-कुम्भ-सन्निभ-कुचां वन्दे महेशप्रियाम्}%॥१॥

\fourlineindentedshloka
{काशाभांशुक-भासुरां प्रविलसत्-कोशातकी-सन्निभाम्}
{चन्द्रार्कानल-लोचनां सुरुचिरालङ्कार-भूषोज्ज्वलाम्}
{ब्रह्म-श्रीपति-वासवादि-मुनिभिः संसेविताङ्घ्रि-द्वयाम्}
{कामाक्षीं गज-राज-मन्द-गमनां वन्दे महेशप्रियाम्}%॥२॥

\fourlineindentedshloka
{ऐं क्लीं सौर्-इति यां वदन्ति मुनयस्तत्त्वार्थ-रूपां पराम्}
{वाचाम् आदिम-कारणं हृदि सदा ध्यायन्ति यां योगिनः}
{बालां फाल-विलोचनां नव-जपा-वर्णां सुषुम्नाश्रिताम्}
{कामाक्षीं कलितावतंस-सुभगां वन्दे महेशप्रियाम्}%॥३॥

\fourlineindentedshloka
{यत्पादाम्बुज-रेणु-लेशम् अनिशं लब्ध्वा विधत्ते विधिर्-}
{विश्वं तत् परिपाति विष्णुरखिलं यस्याः प्रसादाच्चिरम्}
{रुद्रः संहरति क्षणात् तदखिलं यन्मायया मोहितः}
{कामाक्षीम् अति-चित्र-चारु-चरितां वन्दे महेशप्रियाम्}%॥४॥

\fourlineindentedshloka
{सूक्ष्मात् सूक्ष्म-तरां सुलक्षित-तनुं क्षान्ताक्षरैर्लक्षिताम्}
{वीक्षा-शिक्षित-राक्षसां त्रिभुवन-क्षेमङ्करीम् अक्षयाम्}
{साक्षाल्लक्षण-लक्षिताक्षर-मयीं दाक्षायणीं साक्षिणीम्}
{कामाक्षीं शुभलक्षणैः सुललितां वन्दे महेशप्रियाम्}%॥५॥

\fourlineindentedshloka
{ओङ्काराङ्गण-दीपिकाम् उपनिषत्-प्रासाद-पारावतीम्}
{आम्नायाम्बुधि-चन्द्रिकाम् अघ-तमः-प्रध्वंस-हंस-प्रभाम्}
{काञ्ची-पट्टण-पञ्जरान्तर-शुकीं कारुण्य-कल्लोलिनीम्}
{कामाक्षीं शिव-कामराज-महिषीं वन्दे महेशप्रियाम्}%॥६॥

\fourlineindentedshloka
{ह्रीङ्कारात्मक-वर्ण-मात्र-पठनादैन्द्रीं श्रियं तन्वतीम्}
{चिन्मात्रां भुवनेश्वरीम् अनुदिनं भिक्षा-प्रदान-क्षमाम्}
{विश्वाघौघ-निवारिणीं विमलिनीं विश्वम्भरां मातृकाम्}
{कामाक्षीं परिपूर्ण-चन्द्र-वदनां वन्दे महेशप्रियाम्}%॥७॥

\fourlineindentedshloka
{वाग्-देवीति च यां वदन्ति मुनयः क्षीराब्धि-कन्येति च}
{क्षोणी-भृत्-तनयेति च श्रुति-गिरो याम् आमनन्ति स्फुटम्}
{एकानेक-फल-प्रदां बहु-विधाऽऽकारास्तनूस्तन्वतीम्}
{कामाक्षीं सकलार्ति-भञ्जन-परां वन्दे महेशप्रियाम्}%॥८॥

\fourlineindentedshloka
{मायाम् आदिम-कारणं त्रिजगताम् आराधिताङ्घ्रि-द्वयाम्}
{आनन्दामृत-वारि-राशि-निलयां विद्यां विपश्चिद्धियाम्}
{माया-मानुष-रूपिणीं मणि-लसन्मध्यां महामातृकाम्}
{कामाक्षीं करिराज-मन्दगमनां वन्दे महेशप्रियाम्}%॥९॥

\fourlineindentedshloka
{कान्ता काम-दुघा करीन्द्र-गमना कामारि-वामाङ्क-गा}
{कल्याणी कलितावतार-सुभगा कस्तूरिका-चर्चिता}
{कम्पा-तीर-रसाल-मूल-निलया कारुण्य-कल्लोलिनी}
{कल्याणानि करोतु मे भगवती काञ्ची-पुरी-देवता}%॥१०॥

{॥इति श्रीमत्परमहंसपरिव्राजकाचार्यस्य श्री-गोविन्द-भगवत्पूज्य-पाद-शिष्यस्य
श्रीमच्छङ्करभगवतः कृतौ श्री-कामाक्षीस्तोत्रं सम्पूर्णम्॥}
\setlength{\shlokaspaceskip}{24pt}
