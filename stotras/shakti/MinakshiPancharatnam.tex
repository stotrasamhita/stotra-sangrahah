% !TeX program = XeLaTeX
% !TeX root = ../../shloka.tex

\sect{मीनाक्षीपञ्चरत्नम्}

\fourlineindentedshloka
{उद्यद्भानु-सहस्रकोटिसदृशां केयूरहारोज्ज्वलां}
{बिम्बोष्ठीं स्मितदन्तपङ्क्तिरुचिरां पीताम्बरालङ्कृताम्}
{विष्णुब्रह्मसुरेन्द्रसेवितपदां तत्त्वस्वरूपां शिवां}
{मीनाक्षीं प्रणतोऽस्मि सन्ततमहं कारुण्यवारान्निधिम्}%॥ १॥

\fourlineindentedshloka
{मुक्ताहारलसत्किरीटरुचिरां पूर्णेन्दुवक्त्रप्रभां}
{शिञ्जन्नूपुरकिङ्किणीमणिधरां पद्मप्रभाभासुराम्}
{सर्वाभीष्टफलप्रदां गिरिसुतां वाणीरमासेवितां}
{मीनाक्षीं प्रणतोऽस्मि सन्ततमहं कारुण्यवारान्निधिम्}%॥ २॥

\fourlineindentedshloka
{श्रीविद्यां शिववामभागनिलयां ह्रीङ्कारमन्त्रोज्ज्वलां}
{श्रीचक्राङ्कित-बिन्दुमध्यवसतीं श्रीमत्सभानायकीम्}
{श्रीमत्षण्मुखविघ्नराजजननीं श्रीमज्जगन्मोहिनीं}
{मीनाक्षीं प्रणतोऽस्मि सन्ततमहं कारुण्यवारान्निधिम्}%॥ ३॥

\fourlineindentedshloka
{श्रीमत्सुन्दरनायकीं भयहरां ज्ञानप्रदां निर्मलां}
{श्यामाभां कमलासनार्चितपदां नारायणस्यानुजाम्}
{वीणावेणुमृदङ्गवाद्यरसिकां नानाविधाडाम्बिकां}
{मीनाक्षीं प्रणतोऽस्मि सन्ततमहं कारुण्यवारान्निधिम्}%॥ ४॥

\fourlineindentedshloka
{नानायोगिमुनीन्द्रहृन्निवसतीं नानार्थसिद्धिप्रदां}
{नानापुष्पविराजिताङ्घ्रियुगलां नारायणेनार्चिताम्}
{नादब्रह्ममयीं परात्परतरां नानार्थतत्त्वात्मिकां}
{मीनाक्षीं प्रणतोऽस्मि सन्ततमहं कारुण्यवारान्निधिम्}%॥ ५॥

॥इति श्रीमत्परमहंसपरिव्राजकाचार्यस्य श्री-गोविन्द-भगवत्पूज्य-पाद-शिष्यस्य
श्रीमच्छङ्करभगवतः कृतौ श्री-मीनाक्षीपञ्चरत्नं सम्पूर्णम्॥