% !TeX program = XeLaTeX
% !TeX root = ../../shloka.tex

\sect{दुर्गास्तोत्रम्}

\uvacha{श्री-नारायण उवाच}
\twolineshloka*
{स्तोत्रं च श्रूयतां ब्रह्मन् सर्वविघ्नविनाशकम्}
{सुखदं मोक्षदं सारं भवसन्तारकारणम्}

\uvacha{श्रीकृष्ण उवाच}
\twolineshloka
{त्वमेव सर्वजननी मूलप्रकृतिरीश्वरी}
{त्वमेवाऽऽद्या सृष्टिविधौ स्वेच्छया त्रिगुणात्मिका}

\twolineshloka
{कार्यार्थे सगुणा त्वं च वस्तुतो निर्गुणा स्वयम्}
{परब्रह्मस्वरूपा त्वं सत्या नित्या सनातनी}

\twolineshloka
{तेजःस्वरूपा परमा भक्तानुग्रहविग्रहा}
{सर्वस्वरूपा सर्वेशा सर्वाधारा परात्परा}

\twolineshloka
{सर्वबीजस्वरूपा च सर्वपूज्या निराश्रया}
{सर्वज्ञा सर्वतोभद्रा सर्वमङ्गलमङ्गला}

\twolineshloka
{सर्वबुद्धिस्वरूपा च सर्वशक्तिस्वरूपिणी}
{सर्वज्ञानप्रदा देवी सर्वज्ञा सर्वभाविनी}

\twolineshloka
{त्वं स्वाहा देवदाने च पितृदाने स्वधा स्वयम्}
{दक्षिणा सर्वदाने च सर्वशक्तिस्वरूपिणी}

\twolineshloka
{निद्रा त्वं च दया त्वं च तृष्णा त्वं चाऽऽत्मनः प्रिया}
{क्षुत्क्षान्तिः शान्तिरीशा च कान्तिस्तुष्टिश्च शाश्वती}

\twolineshloka
{श्रद्धा पुष्टिश्च तन्द्रा च लज्जा शोभा प्रभा तथा}
{सतां सम्पत्स्वरूपा श्रीर्विपत्तिरसतामिह}

\twolineshloka
{प्रीतिरूपा पुण्यवतां पापिनां कलहाङ्कुरा}
{शश्वत्कर्ममयी शक्तिः सर्वदा सर्वजीविनाम्}

\twolineshloka
{देवेभ्यः स्वपदो दात्री धातुर्धात्री कृपामयी}
{हिताय सर्वदेवानां सर्वासुरविनाशिनी}

\twolineshloka
{योगनिद्रा योगरूपा योगदात्री च योगिनाम्}
{सिद्धिस्वरूपा सिद्धानां सिद्धिदा सिद्धियोगिनी}

\twolineshloka
{माहेश्वरी च ब्रह्माणी विष्णुमाया च वैष्णवी}
{भद्रदा भद्रकाली च सर्वलोकभयङ्करी}

\twolineshloka
{ग्रामे ग्रामे ग्रामदेवी गृहदेवी गृहे गृहे}
{सतां कीर्तिः प्रतिष्ठा च निन्दा त्वमसतां सदा}

\twolineshloka
{महायुद्धे महामारी दुष्टसंहाररूपिणी}
{रक्षास्वरूपा शिष्टानां मातेव हितकारिणी}

\twolineshloka
{वन्द्या पूज्या स्तुता त्वं च ब्रह्मादीनां च सर्वदा}
{ब्राह्मण्यरूपा विप्राणां तपस्या च तपस्विनाम्}

\twolineshloka
{विद्या विद्यावतां त्वं च बुद्धिर्बुद्धिमतां सताम्}
{मेधा स्मृतिस्वरूपा च प्रतिभा प्रतिभावताम्}

\twolineshloka
{राज्ञां प्रतापरूपा च विशां वाणिज्यरूपिणी}
{सृष्टौ सृष्टिस्वरूपा त्वं रक्षारूपा च पालने}

\twolineshloka
{तथाऽन्ते त्वं महामारी विश्वे विश्वैश्च पूजिते}
{कालरात्रिर्महारात्रिर्मोहरात्रिश्च मोहिनी}

\twolineshloka
{दुरत्यया मे माया त्वं यया सम्मोहितं जगत्}
{यया मुग्धो हि विद्वांश्च मोक्षमार्गं न पश्यति}

\dnsub{फलश्रुतिः}

\twolineshloka
{इत्यात्मना कृतं स्तोत्रं दुर्गाया दुर्गनाशनम्}
{पूजाकाले पठेद्यो हि सिद्धिर्भवति वाञ्छिता}

\twolineshloka
{वन्ध्या च काकवन्ध्या च मृतवत्सा च दुर्भगा}
{श्रुत्वा स्तोत्रं वर्षमेकं सुपुत्रं लभते ध्रुवम्}

\twolineshloka
{कारागारे महाघोरे यो बद्धो दृढबन्धने}
{श्रुत्वा स्तोत्रं मासमेकं बन्धनान्मुच्यते ध्रुवम्}

\twolineshloka
{यक्ष्मग्रस्तो गलत्कुष्ठी महाशूली महाज्वरी}
{श्रुत्वा स्तोत्रं वर्षमेकं सद्यो रोगात् प्रमुच्यते}

\twolineshloka
{पुत्रभेदे प्रजाभेदे पत्‍‌नीभेदे च दुर्गतः}
{श्रुत्वा स्तोत्रं मासमेकं लभते नात्र संशयः}

\twolineshloka
{राजद्वारे श्मशाने च महारण्ये रणस्थले}
{हिंस्रजन्तुसमीपे च श्रुत्वा स्तोत्रं प्रमुच्यते}

\twolineshloka
{गृहदाहे च दावाग्नौ दस्युसैन्यसमन्विते}
{स्तोत्रश्रवणमात्रेण लभते नात्र संशयः}

\twolineshloka
{महादरिद्रो मूर्खश्च वर्षं स्तोत्रं पठेत्तु यः}
{विद्यावान् धनवांश्चैव स भवेन्नात्र संशयः}

{॥इति~श्रीब्रह्मवैवर्तमहापुराणे~प्रकृतिखण्डे षट्षष्टितमेऽध्याये श्री-नारद-नारायण-संवादे दुर्गोपाख्याने श्री-कृष्णविरचितं श्री-दुर्गास्तोत्रं सम्पूर्णम्॥}