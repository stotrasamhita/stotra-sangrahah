% !TeX program = XeLaTeX
% !TeX root = ../../shloka.tex

\begin{center}
\sect{श्यामलादण्डकम्}
\begin{Large}
\dnsub{ध्यानम्}
\fourlineindentedshloka
{माणिक्यवीणामुपलालयन्तीं}
{मदालसां मञ्जुलवाग्विलासाम्}
{माहेन्द्रनीलद्युतिकोमलाङ्गीं}
{मातङ्गकन्यां मनसा स्मरामि}

\twolineshloka
{चतुर्भुजे चन्द्रकलावतंसे कुचोन्नते कुङ्कुमरागशोणे}
{पुण्ड्रेक्षुपाशाङ्कुशपुष्पबाणहस्ते नमस्ते जगदेकमातः}

\dnsub{विनियोगः}
\twolineshloka*
{माता मरकतश्यामा मातङ्गी मदशालिनी}
{कुर्यात् कटाक्षं कल्याणी कदम्बवनवासिनी}

\dnsub{स्तुतिः}
\twolineshloka*
{जय मातङ्गतनये जय नीलोत्पलद्युते}
{जय सङ्गीतरसिके जय लीलाशुकप्रिये}
\end{Large}
\end{center}
\end{center}
\setlength{\emergencystretch}{3em}
\dnsub{दण्डकम्}
% \raggedright
जय जननि सुधासमुद्रान्-तरुद्यन्-मणिद्वीप-संरूढ-बिल्वाटवी-मध्य-कल्प-द्रुमाकल्प-कादम्ब-कान्तार-वासप्रिये कृत्तिवासप्रिये सर्वलोकप्रिये'
सादरारब्ध-सङ्गीत-सम्भावना-सम्भ्रमालोल-नीपस्रगाबद्ध-चूलीसनाथत्रिके सानुमत्पुत्रिके'
शेखरीभूत-शीतांशुरेखा-मयूखावली-बद्ध-सुस्निग्ध-नीलालकश्रेणि-शृङ्गारिते लोकसम्भाविते'
कामलीला-धनुः सन्निभ-भ्रूलता-पुष्प-सन्दोह-सन्देह-कृल्लोचने वाक्सुधासेचने'
चारुगोरोचनापङ्क-केलीलला-माभिरामे सुरामे रमे'
प्रोल्लसद्-वालिका-मौक्तिकश्रेणिका-चन्द्रिका-मण्डलोद्भासि-गण्डस्थलन्यस्त-कस्तूरिका-पत्ररेखा-समुद्भूत-सौरभ्य-सम्भ्रान्त-भृङ्गाङ्गनागीत-सान्द्रीभवन्-मन्द्रतन्त्रीस्वरे सुस्वरे भास्वरे'
वल्लकी-वादन-प्रक्रिया-लोल-तालीदलाबद्ध-ताटङ्क-भूषाविशेषान्विते सिद्ध-सम्मानिते'
दिव्यहालाम-दोद्वेलहेलाल-सच्चक्षुरान्दोलन-श्रीसमाक्षिप्त-कर्णैक-नीलोत्पले श्यामले पूरिताशेष-लोकाभि-वाञ्छाफले श्रीफले'
स्वेद-बिन्दूल्लसद्-भाल-लावण्य-निष्यन्द-सन्दोह-सन्देह-कृन्नासिका-मौक्तिके सर्वमन्त्रात्मिके कालिके'
मुग्ध-मन्दस्मितो-दारवक्त्रस्फुरत्-पूग-कर्पूर-ताम्बूल-खण्डोत्करे
ज्ञानमुद्राकरे सर्वसम्पत्करे
पद्मभास्वत्करे श्रीकरे'
कुन्द-पुष्पद्युतिस्निग्ध-दन्तावली-निर्मलालोल-कल्लोल-सम्मेल-नस्मेरशोणाधरे चारुवीणाधरे पक्वबिम्बाधरे'
सुललित-नवयौवनारम्भ-चन्द्रोदयोद्वेल-लावण्य-दुग्धार्णवाविर्भवत्कम्बु-बिम्बोक-भृत्कन्थरे सत्कला-मन्दिरे मन्थरे'
दिव्य-रत्नप्रभा-बन्धुरच्छन्न-हारादि-भूषा-समुद्योतमाना-नवद्याङ्गशोभे शुभे'
रत्न-केयूर-रश्मिच्छटा-पल्लव-प्रोल्लसद्-दोल्लता-राजिते योगिभिः पूजिते'
विश्व-दिङ्मण्डलव्याप्त-माणिक्य-तेजः स्फुरत्-कङ्कणालङ्कृते
विभ्रमालङ्कृते साधुभिः सत्कृते'
वासरारम्भ-वेला-समुज्जृम्भ-माणारविन्द-प्रतिद्वन्द्वि-पाणिद्वये
सन्ततोद्यद्वये अद्वये'
दिव्य-रत्नोर्मिका-दीधिति-स्तोम-सन्ध्यायमा-नाङ्गुली-पल्लवोद्यन्न-खेन्दु-प्रभा-मण्डले सन्नुताखण्डले
चित्प्रभामण्डले प्रोल्लसत्कुण्डले'
तारकाराजि-नीकाश-हारावलिस्मेर-चारुस्तना-भोगभारानमन्मध्य-वल्लीवलिच्छेद-वीची-समुद्यत्-समुल्लास-सन्दर्शिताकार-सौन्दर्य-रत्नाकरे वल्लकी-भृत्करे किङ्कर-श्रीकरे'
हेम-कुम्भोप-मोत्तुङ्ग-वक्षोजभारावनम्रे त्रिलोकावनम्रे'
लसद्वृत्त-गम्भीर-नाभी-सरस्तीर-शैवाल-शङ्काकर-श्यामरोमावली-भूषणे मञ्जुसम्भाषणे'
चारुशिञ्चत्कटीसूत्र-निर्भत्सितानङ्ग-लीला-धनुश्शिञ्चिनी-डम्बरे दिव्यरत्नाम्बरे'
पद्मरागोल्लसन्-मेखला-भास्वर-श्रोणि-शोभाजित-स्वर्ण-भूभृत्तले चन्द्रिका-शीतले'
विकसित-नवकिंशुकाताम्र-दिव्यांशु-कच्छन्न-चारूरु-शोभा-पराभूत-सिन्दूर-शोणाय-मानेन्द्र-मातङ्ग-हस्तार्गले वैभवानर्गले श्यामले'
कोमलस्निग्ध-नीलोत्पलोत्-पादितानङ्ग-तूणीर-शङ्काकरोदाम-जङ्घालते चारुलीलागते'
नम्र-दिक्पाल-सीमन्तिनि
कुन्तलस्निग्ध-नीलप्रभा-पुञ्चसञ्जात-दूर्वाङ्कु-राशङ्क-सारङ्ग-संयोग-रिङ्खन्न-खेन्दूज्ज्वले प्रोज्ज्वले निर्मले'
प्रह्वदेवेश-लक्ष्मीश-भूतेश-तोयेश-वागीश-कीनाश-दैत्येश-यक्षेश-वाय्वग्नि-माणिक्य-संहृष्ट-कोटीर-बाला-तपोद्दामलाक्षा-रसारुण्य-तारुण्य-लक्ष्मी-गृहीताङ्घ्रि-पद्मे
सुपद्मे उमे'
सुरुचिर-नवरत्न-पीठस्थिते सुस्थिते
रत्नपद्मासने रत्नसिंहासने
शङ्खपद्मद्वयोपाश्रिते विश्रिते
तत्र विघ्नेश-दुर्गावटु-क्षेत्रपालैर्युते
मत्तमातङ्ग-कन्या-समूहान्विते'
मञ्जुलामेनकाद्यङ्गनामानिते
भैरवैरष्टभिर्वेष्टिते देवि
वामादिभिः शक्तिभिः सेविते
धात्रि-लक्ष्म्यादि-शक्त्यष्टकैः संयुते
मातृकामण्डलैर्मण्डिते'
यक्ष-गन्धर्व-सिद्धाङ्गना-मण्डलैरर्चिते
पञ्चबाणात्मिके पञ्चबाणेन रत्या च सम्भाविते
प्रीतिभाजा वसन्तेन चानन्दिते'
भक्तिभाजां परं श्रेयसे कल्पसे'
योगिनां मानसे द्योतसे'
छन्दसामोजसा भ्राजसे'
गीत-विद्या-विनोदादि तृष्णेन कृष्णेन सम्पूज्यसे
भक्तिमच्चेतसा वेधसा स्तूयसे
विश्वहृद्येन वाद्येन विद्याधरैर्गीयसे'
श्रवणहरदक्षिणक्वाणया वीणया
किन्नरैर्गीयसे यक्षगन्धर्व-सिद्धाङ्गना-मण्डलैरर्च्यसे'
सर्वसौभाग्य-वाञ्छावतीभिर्वधूभिः
सुराणां समाराध्यसे सर्वविद्याविशेषात्मकं
चाटुगाथा-समुच्चारणं कण्ठ-मूलोल्ल-सद्वर्णराजित्रयं कोमलश्यामलो-दारपक्षद्वयं तुण्डशोभाति-धूरीभवत्
किंशुकाभं तं शुकं लालयन्ती परिक्रीडसे'
पाणिपद्मद्वयेना-क्षमालामपि स्फाटिकीं
ज्ञानसारात्मकं पुस्तकं चापरेणाङ्कुशं
पाशमाबिभ्रति येन सञ्चिन्त्यसे चेतसा तस्य वक्त्रान्तरात् गद्यपद्यात्मिका भारती निःसरेत्' येन वा यावका भाकृतिर्भाव्यसे
तस्य  वश्या भवन्ति स्त्रियः पूरुषाः'
येन वा शातकुम्भद्युतिर्भाव्यसे'
सोऽपि लक्ष्मीसहस्रैः परिक्रीडते'
किं न सिद्‌ध्येद्वपुः श्यामलं कोमलं
चन्द्र-चूडान्वितं तावकं ध्यायतः'
तस्य लीला सरोवारिधिः'
तस्य केलीवनं नन्दनं'
तस्य भद्रासनं भूतलं'
तस्य गीर्देवता किङ्करी'
तस्य चाऽऽज्ञाकरी श्री-स्वयम्'
सर्वतीर्थात्मिके सर्वमन्त्रात्मिके सर्वतन्त्रात्मिके सर्वयन्त्रात्मिके'
सर्वपीठात्मिके सर्वसत्त्वात्मिके सर्वशक्त्यात्मिके'
सर्वविद्यात्मिके सर्वयोगात्मिके सर्वरागात्मिके'
सर्वशब्दात्मिके सर्ववर्णात्मिके सर्वविश्वात्मिके सर्वगे'
हे जगन्मातृके' पाहि मां पाहि मां पाहि मां'
देवि तुभ्यं नमो देवि तुभ्यं नमो देवि तुभ्यं नमः॥

\begin{center}
॥इति महाकवि-कालिदासविरचितं श्री-श्यामलादण्डकं सम्पूर्णम्॥