% !TeX program = XeLaTeX
% !TeX root = ../../shloka.tex

\sect{गौरीदशकम्}


\fourlineindentedshloka
{लीलारब्धस्थापितलुप्ताखिललोकां}
{लोकातीतैर्योगिभिरन्तश्चिरमृग्याम्}
{बालादित्यश्रेणिसमानद्युतिपुञ्जां}
{गौरीमम्बामम्बुरुहाक्षीमहमीडे}

\fourlineindentedshloka
{प्रत्याहारध्यानसमाधिस्थितिभाजां}
{नित्यं चित्ते निर्वृतिकाष्ठां कलयन्तीम्}
{सत्यज्ञानानन्दमयीं तां तनुरूपां}
{गौरीमम्बामम्बुरुहाक्षीमहमीडे}

\fourlineindentedshloka
{चन्द्रापीडानन्दितमन्दस्मितवक्त्रां}
{चन्द्रापीडालङ्कृतनीलालकभाराम्}
{इन्द्रोपेन्द्राद्यर्चितपादाम्बुजयुग्मां}
{गौरीमम्बामम्बुरुहाक्षीमहमीडे}

\fourlineindentedshloka
{आदिक्षान्तामक्षरमूर्त्या विलसन्तीं}
{भूते भूते भूतकदम्बप्रसवित्रीम्}
{शब्दब्रह्मानन्दमयीं तां तडिदाभां}
{गौरीमम्बामम्बुरुहाक्षीमहमीडे}

\fourlineindentedshloka
{मूलाधारादुत्थितवीथ्या विधिरन्ध्रं}
{सौरं चान्द्रं व्याप्य विहारज्वलिताङ्गीम्}
{येयं सूक्ष्मात् सूक्ष्मतनुस्तां सुखरूपां}
{गौरीमम्बामम्बुरुहाक्षीमहमीडे}

\fourlineindentedshloka
{नित्यः शुद्धो निष्कल एको जगदीशः}
{साक्षी यस्याः सर्गविधौ संहरणे च}
{विश्वत्राणक्रीडनलोलां शिवपत्नीं}
{गौरीमम्बामम्बुरुहाक्षीमहमीडे}

\fourlineindentedshloka
{यस्याः कुक्षौ लीनमखण्डं जगदण्डं}
{भूयो भूयः प्रादुरभूदुत्थितमेव}
{पत्या सार्धं तां रजताद्रौ विहरन्तीं}
{गौरीमम्बामम्बुरुहाक्षीमहमीडे}

\fourlineindentedshloka
{यस्यामोतं प्रोतमशेषं मणिमाला}
{सूत्रे यद्वत् क्वापि चरं चाप्यचरं च}
{तामध्यात्मज्ञानपदव्या गमनीयां}
{गौरीमम्बामम्बुरुहाक्षीमहमीडे}

\fourlineindentedshloka
{नानाकारैः शक्तिकदम्बैर्भुवनानि}
{व्याप्य स्वैरं क्रीडति येयं स्वयमेका}
{कल्याणीं तां कल्पलतामानतिभाजां}
{गौरीमम्बामम्बुरुहाक्षीमहमीडे}

\fourlineindentedshloka
{आशापाशक्लेशविनाशं विदधानां}
{पादाम्भोजध्यानपराणां पुरुषाणाम्}
{ईशामीशार्धाङ्गहरां तामभिरामां}
{गौरीमम्बामम्बुरुहाक्षीमहमीडे}

\fourlineindentedshloka*
{प्रातःकाले भावविशुद्धः प्रणिधानाद्}
{भक्त्या नित्यं जल्पति गौरीदशकं यः}
{वाचां सिद्धिं सम्पदमग्र्यां शिवभक्तिं}
{तस्यावश्यं पर्वतपुत्री विदधाति}

॥इति श्रीमत्परमहंसपरिव्राजकाचार्यस्य श्री-गोविन्द-भगवत्पूज्य-पाद-शिष्यस्य
श्रीमच्छङ्करभगवतः कृतौ श्री-गौरीदशकं सम्पूर्णम्॥