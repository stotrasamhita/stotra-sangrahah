% !TeX program = XeLaTeX
% !TeX root = ../../shloka.tex

\sect{श्री-परमशिवेन्द्र-कृतं सुब्रह्मण्य-भुजङ्गम्}

\fourlineindentedshloka
{गणेशं नमस्कृत्य गौरी-कुमारं}
{गजास्यं गुहस्याग्र-जातं गभीरम्}
{प्रलम्बोदरं शूर्प-कर्णं त्रि-णेत्रं}
{प्रवक्ष्ये भुजङ्ग-प्रयातं गुहस्य} %॥ १ ॥

\fourlineindentedshloka
{पृथक्-षट्-किरीट-स्फुरद्-दिव्य-रत्न-}
{प्रभाक्षिप्त-मार्तण्ड-कोटि-प्रकाशम्}
{चलत्-कुण्डलोद्यत्-सु-गण्ड-स्थलान्तं}
{महानर्घ-हारोज्ज्वलत्-कम्बु-कण्ठम्} %॥ २ ॥

\fourlineindentedshloka
{शरत्-पूर्ण-चन्द्र-प्रभा-चारु-वक्त्रं}
{विराजल्ललाटं कृपा-पूर्ण-नेत्रम्}
{लसद्-भ्रू-सु-नासा-पुटं विद्रुमोष्ठं}
{सु-दन्तावलिं सुस्मितं प्रेम-पूर्णम्} %॥ ३ ॥

\fourlineindentedshloka
{द्वि-षड्-बाहु-दण्डाग्र-देदीप्यमानं}
{क्वणत्-कङ्कणालङ्कृतोदार-हस्तम्}
{लसन्मुद्रिका-रत्न-राजत्-कराग्रं}
{क्वणत्-किङ्किणी-रम्य-काञ्ची-कलापम्} %॥ ४ ॥

\fourlineindentedshloka
{विशालोदरं विस्फुरत्-पूर्ण-कुक्षिं}
{कटौ-स्वर्ण-सूत्रं तटिद्-वर्ण-गात्रम्}
{सु-लावण्य-नाभी-सरस्तीर-राजत्-}
{सु-शैवाल-रोमावली-रोचमानम्} %॥ ५ ॥

\fourlineindentedshloka
{सु-कल्लोल-वीची-वली-रोचमानं}
{लसन्मध्य-सुस्निग्ध-वासो वसानम्}
{स्फुरच्चारु-दिव्योरु-जङ्घा-सु-गुल्फं}
{विकस्वत्-पदाब्जं नखेन्दु-प्रभाढ्यम्} %॥ ६ ॥

\fourlineindentedshloka
{द्वि-षट्-पङ्कजाक्षं महा-शक्ति-युक्तं}
{त्रि-लोक-प्रशस्तं सु-कुक्के-पुर-स्थम्}
{प्रपन्नार्ति-नाशं प्रसन्नं फणीशं}
{पर-ब्रह्म-रूपं प्रकाशं परेशम्} %॥ ७ ॥

\fourlineindentedshloka
{कुमारं वरेण्यं शरण्यं सु-पुण्यं}
{सु-लावण्य-पण्यं सुरेशानुवर्ण्यम्}
{लसत्-पूर्ण-कारुण्य-लक्ष्मीश-गण्यं}
{सु-कारुण्यमार्याग्र-गण्यं नमामि} %॥ ८ ॥

\fourlineindentedshloka
{स्फुरद्-रत्न-पीठोपरि भ्राजमानं}
{हृदम्भोज-मध्ये महा-सन्निधानम्}
{समावृत्त-जानु-प्रभा-शोभमानं}
{सुरैः सेव्यमानं भजे बर्हि-यानम्} %॥ ९ ॥

\fourlineindentedshloka
{ज्वल-च्चारु-चामीकरादर्श-पूर्णं}
{चलच्चामर-च्छत्र-चित्र-ध्वजाढ्यम्}
{सुवर्णामलान्दोलिका-मध्य-संस्थं}
{महाहीन्द्र-रूपं भजे सु-प्रतापम्} %॥ १० ॥

\fourlineindentedshloka
{धनुर्बाण-चक्राभयं वज्र-खेटं}
{त्रि-शूलासि-पाशाङ्कुशाभीति-शङ्खम्}
{ज्वलत्-कुक्कुटं प्रोल्लसद्-द्वादशाक्षं}
{प्रशस्तायुधं षण्मुखं तं भजेऽहम्} %॥ ११ ॥

\fourlineindentedshloka
{स्फुरच्चारु-गण्डं द्वि-षड्-बाहु-दण्डं}
{श्रितामर्त्य-षण्डं सुसम्पत्-करण्डम्}
{द्विषद्-वंश-खण्डं सदा दान-शौण्डं}
{भव-प्रेम-पिण्डं भजे सु-प्रचण्डम्} %॥ १२ ॥

\fourlineindentedshloka
{सदा दीन-पक्षं सुर-द्विड्-विपक्षं}
{सुमृष्टान्न-भक्ष्य-प्रदानैक-दक्षम्}
{श्रितामर्त्य-वृक्षं महा-दैत्य-शिक्षं}
{बहु-क्षीण-पक्षं भजे द्वादशाक्षम्} %॥ १३ ॥

\fourlineindentedshloka
{त्रि-मूर्ति-स्वरूपं त्रयी-सत्-कलापं}
{त्रि-लोकाधिनाथं त्रिणेत्रात्म-जातम्}
{त्रि-शक्त्या प्रयुक्तं सु-पुण्य-प्रशस्तं}
{त्रि-काल-ज्ञमिष्टार्थ-दं तं भजेऽहम्} %॥ १४ ॥

\fourlineindentedshloka
{विराजद्-भुजङ्गं विशालोत्तमाङ्गं}
{विशुद्धात्म-सङ्गं विवृद्ध-प्रसङ्गम्}
{विचिन्त्यं शुभाङ्गं विकृत्तासुराङ्गं}
{भव-व्याधि-भङ्गं भजे कुक्क-लिङ्गम्} %॥ १५ ॥

\fourlineindentedshloka
{गुह स्कन्द गाङ्गेय गौरी-सुतेश-}
{प्रिय क्रौञ्च-भित् तारकारे सुरेश}
{मयूरासनाशेष-दोष-प्रणाश}
{प्रसीद प्रसीद प्रभो चित्-प्रकाश} %॥ १६ ॥

\fourlineindentedshloka
{लपन् देव-सेनेश भूतेश शेष-}
{स्वरूपाग्नि-भूः कार्त्तिकेयान्न-दातः}
{यदेत्थं स्मरिष्यामि भक्त्या भवन्तं}
{तदा मे षडास्य प्रसीद प्रसीद} %॥ १७ ॥

\fourlineindentedshloka
{भुजे शौर्य-धैर्यं करे दान-धर्मः}
{कटाक्षेऽतिशान्तिः षडास्येषु हास्यम्}
{हृदब्जे दया यस्य तं देवमन्यं}
{कुमारान्न जाने न जाने न जाने} %॥ १८ ॥

\fourlineindentedshloka
{मही-निर्जरेशान्महा-नृत्य-तोषात्}
{विहङ्गाधिरूढाद् बिलान्तर्विगूढात्}
{महेशात्म-जातान्महा-भोगि-नाथाद्}
{गुहाद् दैवमन्यन्न मन्ये न मन्ये} %॥ १९ ॥

\fourlineindentedshloka
{सुरोत्तुङ्ग-शृङ्गार-सङ्गीत-पूर्ण-}
{प्रसङ्ग-प्रियासङ्ग-सम्मोहनाङ्ग}
{भुजङ्गेश भूतेश भृङ्गेश तुभ्यं}
{नमः कुक्क-लिङ्गाय तस्मै नमस्ते} %॥२० ॥

\fourlineindentedshloka
{नमः काल-कण्ठ-प्ररूढाय तस्मै}
{नमो नीलकण्ठाधिरूढाय तस्मै}
{नमः प्रोल्लसच्चारु-चूडाय तस्मै}
{नमो दिव्य-रूपाय शान्ताय तस्मै} %॥२१॥

\fourlineindentedshloka
{नमस्ते नमः पार्वती-नन्दनाय}
{स्फुरच्चित्र-बर्ही-कृत-स्यन्दनाय}
{नमश्चर्चिताङ्गोज्ज्वलच्चन्दनाय}
{प्रविच्छेदित-प्राण-भृद्-बन्धनाय} %॥२२॥

\fourlineindentedshloka
{नमस्ते नमस्ते जगत्-पावनात्त-}
{स्वरूपाय तस्मै जगज्जीवनाय}
{नमस्ते नमस्ते जगद्-वन्दिताय}
{ह्यरूपाय तस्मै जगन्मोहनाय} %॥२३॥

\fourlineindentedshloka
{नमस्ते नमस्ते नमः क्रौञ्च-भेत्त्रे}
{नमस्ते नमस्ते नमो विश्व-कर्त्रे}
{नमस्ते नमस्ते नमो विश्व-गोप्त्रे}
{नमस्ते नमस्ते नमो विश्व-हन्त्रे} %॥२४॥

\fourlineindentedshloka
{नमस्ते नमस्ते नमो विश्व-भर्त्रे}
{नमस्ते नमस्ते नमो विश्व-धात्रे}
{नमस्ते नमस्ते नमो विश्व-नेत्रे}
{नमस्ते नमस्ते नमो विश्व-शास्त्रे} %॥२५॥

\fourlineindentedshloka
{नमस्ते नमश्शेष-रूपाय तुभ्यं}
{नमस्ते नमो दिव्य-चापाय तुभ्यम्}
{नमस्ते नमः सत्-प्रतापाय तुभ्यं}
{नमस्ते नमः सत्-कलापाय तुभ्यम्} %॥२६॥

\fourlineindentedshloka
{नमस्ते नमः सत्-किरीटाय तुभ्यं}
{नमस्ते नमः स्वर्ण-पीठाय तुभ्यम्}
{नमस्ते नमः सल्ललाटाय तुभ्यं}
{नमस्ते नमो दिव्य-रूपाय तुभ्यम्} %॥२७॥

\fourlineindentedshloka
{नमस्ते नमो लोक-रक्षाय तुभ्यं}
{नमस्ते नमो दीन-रक्षाय तुभ्यम्}
{नमस्ते नमो दैत्य-शिक्षाय तुभ्यं}
{नमस्ते नमो द्वादशाक्षाय तुभ्यम्} %॥२८॥

\fourlineindentedshloka
{भुजङ्गाकृते त्वत्-प्रियार्थं मयेदं}
{भुजङ्ग-प्रयातेन वृत्तेन कॢप्तम्}
{तव स्तोत्रमेतत् पवित्रं सु-पुण्यं}
{परानन्द-सन्दोह-संवर्धनाय} %॥२९॥

\fourlineindentedshloka
{त्वदन्यत् परं दैवतं नाभिजाने}
{प्रभो पाहि सम्पूर्ण-दृष्ट्यानुगृह्य}
{यथा-शक्ति भक्त्या कृतं स्तोत्रमेकं}
{विभो मेऽपराधं क्षमस्वाखिलेश} %॥३०॥

\fourlineindentedshloka
{इदं तारकारेर्गुण-स्तोत्र-राजं}
{पठन्तस्त्रि-कालं प्रपन्ना जना ये}
{सुपुत्राष्ट-भोगानिह त्वेव भुक्त्वा}
{लभन्ते तदन्ते परं स्वर्ग-भोगम्} %॥३१॥

॥ इति श्रीमत्-काञ्ची-कामकोटि-मूलाम्नाय-सर्वज्ञ-पीठाधिपति-जगद्गुरु-शङ्कराचार्य-श्रीमत्-सदाशिव-बोधेन्द्र-सरस्वती-श्रीचरण-अन्तेवासिवर्य-श्रीमत्-परमशिवेन्द्र-सरस्वती-श्रीचरण-विरचितं सुब्रह्मण्य-भुजङ्गं सम्पूर्णम् ॥