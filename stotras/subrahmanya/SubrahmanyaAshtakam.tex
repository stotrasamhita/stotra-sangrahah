% !TeX program = XeLaTeX
% !TeX root = ../../shloka.tex

\sect{श्रीसुब्रह्मण्याष्टकम्}

\fourlineindentedshloka
{हे स्वामिनाथ करुणाकर दीनबन्धो}
{श्रीपार्वतीशमुखपङ्कजपद्मबन्धो}
{श्रीशादिदेवगणपूजितपादपद्म}
{वल्लीशनाथ मम देहि करावलम्बम्} %॥१॥

\fourlineindentedshloka
{देवाधिदेवसुत देवगणाधिनाथ}
{देवेन्द्रवन्द्यमृदुपङ्कजमञ्जुपाद}
{देवर्षिनारदमुनीन्द्रसुगीतकीर्ते}
{वल्लीशनाथ मम देहि करावलम्बम्} %॥२॥

\fourlineindentedshloka
{नित्यान्नदाननिरताखिलरोगहारिन्}
{भाग्यप्रदानपरिपूरितभक्तकाम}
{श‍ृत्यागमप्रणववाच्यनिजस्वरूप}
{वल्लीशनाथ मम देहि करावलम्बम्} %॥३॥

\fourlineindentedshloka
{क्रौञ्चासुरेन्द्रपरिखण्डन शक्तिशूल-}
{चापादिशस्त्रपरिमण्डितदिव्यपाणे}
{श्रीकुण्डलीशधरतुण्डशिखीन्द्रवाह }
{वल्लीशनाथ मम देहि करावलम्बम्} %॥४॥

\fourlineindentedshloka
{देवाधिदेवरथमण्डलमध्यवेऽद्य}
{देवेन्द्रपीठनकरं दृढचापहस्तम्}
{शूरं निहत्य सुरकोटिभिरीड्यमान}
{वल्लीशनाथ मम देहि करावलम्बम्} %॥५॥

\fourlineindentedshloka
{हारादिरत्नमणियुक्तकिरीटहार}
{केयूरकुण्डललसत्कवचाभिरामम्}
{हे वीर तारकजयामरवृन्दवन्द्य}
{वल्लीशनाथ मम देहि करावलम्बम्} %॥६॥

\fourlineindentedshloka
{पञ्चाक्षरादिमनुमन्त्रितगाङ्गतोयैः}
{पञ्चामृतैः प्रमुदितेन्द्रमुखैर्मुनीन्द्रैः}
{पट्टाभिषिक्तहरियुक्त परासनाथ}
{वल्लीशनाथ मम देहि करावलम्बम्} %॥७॥

\fourlineindentedshloka
{श्रीकार्तिकेय करुणामृतपूर्णदृष्ट्या}
{कामादिरोगकलुषीकृतदुष्टचित्तम्}
{सिक्त्वा तु मामव कलाधरकान्तिकान्त्या}
{वल्लीशनाथ मम देहि करावलम्बम्} %॥८॥

\twolineshloka*
{सुब्रह्मण्याष्टकं पुण्यं ये पठन्ति द्विजोत्तमाः}
{ते सर्वे मुक्तिमायान्ति सुब्रह्मण्यप्रसादतः}

\twolineshloka*
{सुब्रह्मण्याष्टकमिदं प्रातरुत्थाय यः पठेत्}
{कोटिजन्मकृतं पापं तत्क्षणादेव नश्यति}

॥इति श्रीसुब्रह्मण्याष्टकं सम्पूर्णम्॥
