% !TeX program = XeLaTeX
% !TeX root = ../../shloka.tex
\sect{स्वस्ति-वचनम्/गुरुवन्दनम्}
\begin{center}
{॥ॐ श्री-गुरुभ्यो नमः॥}\\
{॥श्री-महात्रिपुरसुन्दरी-समेत-श्री-चन्द्रमौलीश्वराय नमः॥}\\
{॥श्री-काञ्ची-कामकोटि-पीठाधिपति-जगद्गुरु-\\ श्री-शङ्कराचार्य श्री-चरणयोः प्रणामाः॥}

\catcode`==\active
\def={\discretionary{-}{}{-}}

\noindent स्वस्ति श्रीमद्=अखिल=भूमण्डलालङ्कार=त्रयस्त्रिंशत्=कोटि=देवता=सेवित=श्री=कामाक्षी=देवी=सनाथ=श्रीमद्=एकाम्रनाथ=श्री=महादेवी=सनाथ=श्री=हस्तिगिरिनाथ=साक्षात्कार=परमाधिष्ठान=सत्यव्रत=नामाङ्कित=काञ्ची=दिव्य=क्षेत्रे, शारदामठ=सुस्थितानाम्,
अतुलित=सुधारस=माधुर्य=कमलासन=कामिनी=धम्मिल्ल=सम्फुल्ल=मल्लिका=मालिका=निःष्यन्द=मकरन्द=झरी=सौवस्तिक=वाङ्निगुम्फ=विजृम्भणानन्द=तुन्दिलित=मनीषि=मण्डलानाम्,
अनवरताद्वैत=विद्या=विनोद=रसिकानां निरन्तरालङ्कृतीकृत=शान्ति=दान्ति=भूम्नाम्,
सकल=भुवन=चक्र=प्रतिष्ठापक=श्रीचक्र=प्रतिष्ठा=विख्यात=यशोऽलङ्कृतानाम्,
निखिल=पाषण्ड=षण्ड=कण्टकोत्पाटनेन विशदीकृत=वेद=वेदान्त=मार्ग=षण्मत=प्रतिष्ठापकाचार्याणाम्,
परमहंस=परिव्राजकाचार्यवर्य=जगद्गुरु=श्रीमत्=शङ्कर=भगवत्पादा=चार्याणाम्,
अधिष्ठाने सिंहासनाभिषिक्त=श्रीमद्=जयेन्द्र=सरस्वती=श्रीपादानाम्, 
अन्ते=वासि=वर्य=श्रीमत्=शङ्कर=विजयेन्द्र=सरस्वती=श्रीपादानां, 
तदन्तेवासि=वर्य=श्रीमत्=सत्य=चन्द्रशेखरेन्द्र=सरस्वती=श्रीपादानां च 
चरण=नलिनयोः स=प्रश्रयं साञ्जलि=बन्धं च नमस्कुर्मः॥

\end{center}