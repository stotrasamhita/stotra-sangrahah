% !TeX program = XeLaTeX
% !TeX root = ../../shloka.tex

\sect{गुरुपरम्परा-स्तुतिः}

\centerline{॥ॐ श्री-गणेशाय नमः॥}
\centerline{॥ॐ श्री-गुरुभ्यो नमः॥}
\centerline{॥हरिः ॐ॥}


\setlength{\shlokaspaceskip}{0pt}
\fourlineindentedshloka*
{नारायणं पद्मभुवं वसिष्ठं शक्तिं च तत्पुत्रपराशरं च}%
{व्यासं शुकं गौडपदं महान्तं गोविन्दयोगीन्द्रमथास्य शिष्यम्}%
{श्री-शङ्कराचार्यमथास्य पद्मपादं च हस्तामलकं च शिष्यं}%
{तं तोटकं वार्तिककारमन्यानस्मद्गुरून् सन्ततमानतोऽस्मि}%
\setlength{\shlokaspaceskip}{24pt}
\twolineshloka*
{श्रुतिस्मृतिपुराणानाम् आलयं करुणालयम्}
{नमामि भगवत्पादशङ्करं लोकशङ्करम्}

\twolineshloka*
{लक्ष्मीनारायण इति पूर्वाश्रमनामभूषितं शान्तम्}
{ऋग्वेदे सम्यगधीतिनं महादेवमाश्रयामि गुरुम्}

\twolineshloka*
{अपारकरुणासिन्धुं ज्ञानदं शान्तरूपिणम्}
{श्रीचन्द्रशेखरगुरुं प्रणमामि मुदाऽन्वहम्}

\twolineshloka*
{देवे देहे च देशे च भक्त्यारोग्य-सुख-प्रदम्}
{बुध-पामर-सेव्यं तं श्री-जयेन्द्रं नमाम्यहम्}

\twolineshloka*
{नमामः शङ्करान्वाख्य-विजयेन्द्रसरस्वतीम्}
{श्रीगुरुं शिष्टमार्गानुनेतारं सन्मतिप्रदम्}

\twolineshloka*
{सदाशिवसमारम्भां शङ्कराचार्यमध्यमाम्}
{अस्मदाचार्यपर्यन्तां वन्दे गुरुपरम्पराम्}

\fourlineindentedshloka*
{ऐङ्कार-ह्रीङ्कार-रहस्ययुक्त-}
{श्रीङ्कार-गूढार्थ-महाविभूत्या}
{ओङ्कार-मर्म-प्रतिपादिनीभ्यां}
{नमो नमः श्रीगुरुपादुकाभ्याम्}

\twolineshloka*
{अचिन्त्याव्यक्तरूपाय निर्गुणाय गुणात्मने}
{समस्तजगदाधारमूर्तये ब्रह्मणे नमः}
