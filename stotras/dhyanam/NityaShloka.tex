% !TeX program = XeLaTeX
% !TeX root = ../../shloka.tex

\sect{चिरञ्जीविस्तोत्रम्}
\fourlineindentedshloka*
{अश्वत्थामा बलिर्व्यासो हनुमांश्च विभीषणः}
{कृपः परशुरामश्च सप्तैते चिरञ्जीविनः}
{सप्तैतान् संस्मरेन्नित्यं मार्कण्डेयमथाष्टमम्}
{जीवेद्वर्षशतं प्राज्ञ अपमृत्युविवर्जितः}

\sect{पञ्चकन्यास्मरणम्}
\twolineshloka*
{अहल्या द्रौपदी सीता तारा मन्दोदरी तथा}
{पञ्चकन्याः स्मरेन्नित्यं महापातकनाशनम्}

\sect{साम्बशिव-स्मरणम्}
\fourlineindentedshloka*
{शिवनामनि भावितेऽन्तरङ्गे}
{महति ज्योतिषि मानिनीमयार्धे}
{दुरितान्यपयान्ति दूरदूरे}
{मुहुरायान्ति महान्ति मङ्गलानि}

\sect{हरि-स्मरणम्}

\twolineshloka*
{स्मृते सकलकल्याणभाजनं यत्र जायते}
{पुरुषस्तमजं नित्यं व्रजामि शरणं हरिम्}

\sect{चक्षुरारोग्य-श्लोकः}

\twolineshloka*
{शर्यातिं च सुकन्यां च च्यवनं शक्रमश्विनौ}
{भुक्तमात्रे स्मरेद् यस्तु चक्षुस्तस्य न हीयते}


\sect{भागवतस्मरणम्}

\uvacha{पाण्डवा ऊचुः}
\fourlineindentedshloka*
{प्रह्लाद-नारद-पराशर-पुण्डरीक}
{व्यासाम्बरीष-शुक-शौनक-भीष्म-दाल्भ्यान्}
{रुक्माङ्गदार्जुन-वसिष्ठ-विभीषणादीन्}
{पुण्यानिमान् परमभागवतान् समरामि}


\sect{गो-वन्दनम्}

\twolineshloka*
{गां च दृष्ट्वा नमस्कृत्य कृत्वा चैव प्रदक्षिणम्}
{प्रदक्षिणी कृता तेन सप्तद्वीपा वसुन्धरा}

\twolineshloka*
{सर्वकामदुघे देवि सर्वतीर्थाभिषेचिनी}
{पावने सुरभिश्रेष्ठे देवि तुभ्यं नमोऽस्तु ते}

\closesection

\sect{तुलसी-वन्दनम्}
\fourlineindentedshloka
{पापानि यानि रविसूनुपटस्थितानि}
{गो-ब्रह्म-बाल-पितृ-मातृ-वधादिकानि}
{नश्यन्ति तानि तुलसीवनदर्शनेन}
{गोकोटिदानसदृशे फलमाशु च स्यात्}
{(तुलसीवनगमने प्रोक्तव्यम्)}

\fourlineindentedshloka
{या दृष्टा निखिलाघसङ्घशमनी स्पृष्टा वपुः पावनी}
{रोगाणामभिवन्दिता निरसनी सिक्तान्तकत्रासिनी}
{प्रत्यासत्ति विधायिनी भगवतः कृष्णस्य संरोपिता}
{न्यस्ता तच्चरणे विमुक्तिफलदा तस्यै तुलस्यै नमः}
(तुलसी-जल-प्रोक्षणम्)

\twolineshloka
{ललाटे यस्य दृश्येत तुलसीमूलमृत्तिका}
{यमस्तं नेक्षितुं शक्तः किमु दूता भयङ्कराः}
(मृत्तिका-धारणम्)

\twolineshloka
{तुलसीकाननं यत्र यत्र पद्मवनानि च}
{वसन्ति वैष्णवा यत्र तत्र सन्निहितो हरिः}

\twolineshloka
{पुष्कराद्यानि तीर्थानि गङ्गाद्याः सरितस्तथा}
{वासुदेवादयो देवा वसन्ति तुलसीवने}
(प्रदक्षिणम्)
\twolineshloka
{तुलसी श्रीसखि शुभे पापहारिणि पुण्यदे}
{नमस्ते नारदनुते नारायणमनःप्रिये}
 (नमस्कारः)

\closesection

\sect{अर्जुन-नामानि   (विराटपर्वान्तर्गतम्)}

\twolineshloka*
{अर्जुनः फाल्गुनो जिष्णुः किरीटी श्वेतवाहनः}
{बीभत्सुर्विजयः कृष्णः सव्यसाची धनञ्जयः}

\closesection

\sect{यमधर्मराजस्य १४ नामानि}

\twolineshloka*
{यमाय   धर्मराजाय   मृत्यवे   चान्तकाय   च}
{वैवस्वताय   कालाय   सर्वभूतक्षयाय   च}

\twolineshloka*
{औदुम्बराय   दध्नाय   नीलाय   परमेष्ठिने}
{वृकोदराय   चित्राय   चित्रगुप्ताय   वै  नमः}
\closesection
