% !TeX program = XeLaTeX
% !TeX root = ../../shloka.tex

\sect{वेङ्कटेश सुप्रभातम्}
\twolineshloka
{कौसल्या सुप्रजा राम पूर्वा सन्ध्या प्रवर्तते}
{उत्तिष्ठ नरशार्दूल कर्तव्यं दैवमाह्निकम्}

\twolineshloka
{उत्तिष्ठोत्तिष्ठ गोविन्द उत्तिष्ठ गरुडध्वज}
{उत्तिष्ठ कमलाकान्त त्रैलोक्यं मङ्गलं कुरु}

\fourlineindentedshloka
{मातः समस्तजगतां मधुकैटभारेः}
{वक्षोविहारिणि मनोहरदिव्यमूर्ते}
{श्रीस्वामिनि श्रितजनप्रियदानशीले}
{श्रीवेङ्कटेशदयिते तव सुप्रभातम्}

\fourlineindentedshloka
{तव सुप्रभातमरविन्दलोचने}
{भवतु प्रसन्नमुखचन्द्रमण्डले}
{विधिशङ्करेन्द्रवनिताभिरर्चिते}
{वृषशैलनाथदयिते दयानिधे}

\fourlineindentedshloka
{अत्र्यादिसप्तऋषयः समुपास्य सन्ध्याम्}
{आकाशसिन्धुकमलानि मनोहराणि}
{आदाय पादयुगमर्चयितुं प्रपन्नाः}
{शेषाद्रिशेखरविभो तव सुप्रभातम्}

\fourlineindentedshloka
{पञ्चाननाब्जभवषण्मुखवासवाद्याः}
{त्रैविक्रमादिचरितं विबुधाः स्तुवन्ति}
{भाषापतिः पठति वासरशुद्धिमारात्}
{शेषाद्रिशेखरविभो तव सुप्रभातम्}

\fourlineindentedshloka
{ईषत्प्रफुल्ल-सरसीरुह-नारिकेल-}
{पूगद्रुमादि-सुमनोहरपालिकानाम्}
{आवाति मन्दमनिलः सह दिव्यगन्धैः}
{शेषाद्रिशेखरविभो तव सुप्रभातम्}

\fourlineindentedshloka
{उन्मील्य नेत्रयुगमुत्तमपञ्जरस्थाः}
{पात्रावशिष्टकदलीफलपायसानि}
{भुक्त्वा सलीलमथ केलिशुकाः पठन्ति}
{शेषाद्रिशेखरविभो तव सुप्रभातम्}

\fourlineindentedshloka
{तन्त्रीप्रकर्षमधुरस्वनया विपञ्च्या}
{गायत्यनन्तचरितं तव नारदोऽपि}
{भाषासमग्रमसकृत्करचाररम्यम्}
{शेषाद्रिशेखरविभो तव सुप्रभातम्}

\fourlineindentedshloka
{भृङ्गावली च मकरन्दरसानुविद्ध-}
{झङ्कारगीत निनदैः सह सेवनाय}
{निर्यात्युपान्तसरसीकमलोदरेभ्यः}
{शेषाद्रिशेखरविभो तव सुप्रभातम्}

\fourlineindentedshloka
{योषागणेन वरदध्निविमथ्यमाने}
{घोषालयेषु दधिमन्थनतीव्रघोषाः}
{रोषात्कलिं विदधते ककुभश्च कुम्भाः}
{शेषाद्रिशेखरविभो तव सुप्रभातम्}

\fourlineindentedshloka
{पद्मेशमित्रशतपत्रगतालिवर्गाः}
{हर्तुं श्रियं कुवलयस्य निजाङ्गलक्ष्म्या}
{भेरीनिनादमिव बिभ्रति तीव्रनादम्}
{शेषाद्रिशेखरविभो तव सुप्रभातम्}

\fourlineindentedshloka
{श्रीमन्नभीष्टवरदाखिललोकबन्धो}
{श्रीश्रीनिवास जगदेकदयैकसिन्धो}
{श्रीदेवतागृहभुजान्तरदिव्यमूर्ते}
{श्रीवेङ्कटाचलपते तव सुप्रभातम्}

\fourlineindentedshloka
{श्रीस्वामिपुष्करिणिकाऽऽप्लवनिर्मलाङ्गाः}
{श्रेयोऽर्थिनो हरविरिञ्चसनन्दनाद्याः}
{द्वारे वसन्ति वरवेत्रहतोत्तमाङ्गाः}
{श्रीवेङ्कटाचलपते तव सुप्रभातम्}

\fourlineindentedshloka
{श्रीशेषशैल-गरुडाचल-वेङ्कटाद्रि-}
{नारायणाद्रि-वृषभाद्रि-वृषाद्रि-मुख्याम्}
{आख्यां त्वदीय वसतेरनिशं वदन्ति}
{श्रीवेङ्कटाचलपते तव सुप्रभातम्}

\fourlineindentedshloka
{सेवापराः शिव-सुरेश-कृशानु-धर्म-}
{रक्षोऽम्बुनाथ-पवमान-धनाधिनाथाः}
{बद्धाञ्जलि-प्रविलसन्निजशीर्ष-देशाः}
{श्रीवेङ्कटाचलपते तव सुप्रभातम्}

\fourlineindentedshloka
{धाटीषु ते विहगराज-मृगाधिराज-}
{नागाधिराज-गजराज-हयाधिराजाः}
{स्वस्वाधिकार-महिमाऽधिकमर्थयन्ते}
{श्रीवेङ्कटाचलपते तव सुप्रभातम्}

\fourlineindentedshloka
{सूर्येन्दु-भौम-बुध-वाक्पति-काव्य-सौरि-}
{स्वर्भानु-केतु-दिविषत्परिषत्प्रधानाः}
{त्वद्दास-दास-चरमावधि-दासदासाः}
{श्रीवेङ्कटाचलपते तव सुप्रभातम्}

\fourlineindentedshloka
{त्वत् पादधूलिभरितस्फुरितोत्तमाङ्गाः}
{स्वर्गापवर्गनिरपेक्ष-निजान्तरङ्गाः}
{कल्पागमाऽऽकलनयाऽऽकुलतां लभन्ते}
{श्रीवेङ्कटाचलपते तव सुप्रभातम्}

\fourlineindentedshloka
{त्वद्गोपुराग्रशिखराणि निरीक्षमाणाः}
{स्वर्गापवर्गपदवीं परमां श्रयन्तः}
{मर्त्या मनुष्यभुवने मतिमाश्रयन्ते}
{श्रीवेङ्कटाचलपते तव सुप्रभातम्}

\fourlineindentedshloka
{श्रीभूमिनायक दयादिगुणामृताब्धे}
{देवाधिदेव जगदेकशरण्यमूर्ते}
{श्रीमन्ननन्त-गरुडादिभिरर्चिताङ्घ्रे}
{श्रीवेङ्कटाचलपते तव सुप्रभातम्}

\fourlineindentedshloka
{श्रीपद्मनाभ पुरुषोत्तम वासुदेव}
{वैकुण्ठ माधव जनार्दन चक्रपाणे}
{श्रीवत्सचिह्न शरणागत-पारिजात}
{श्रीवेङ्कटाचलपते तव सुप्रभातम्}

\fourlineindentedshloka
{कन्दर्पदर्पहरसुन्दरदिव्यमूर्ते}
{कान्ताकुचाम्बुरुह-कुङ्मल-लोलदृष्टे}
{कल्याणनिर्मलगुणाकर दिव्यकीर्ते}
{श्रीवेङ्कटाचलपते तव सुप्रभातम्}

\fourlineindentedshloka
{मीनाकृते कमठ कोल नृसिंह वर्णिन्}
{स्वामिन् परश्वथ तपोधन रामचन्द्र}
{शेषांशराम यदुनन्दन कल्किरूप}
{श्रीवेङ्कटाचलपते तव सुप्रभातम्}

\fourlineindentedshloka
{एला-लवङ्ग-घनसार-सुगन्धि-तीर्थम्}
{दिव्यं वियत्सरिति हेमघटेषु पूर्णम्}
{धृत्वाऽद्य वैदिकशिखामणयः प्रहृष्टाः}
{तिष्ठन्ति वेङ्कटपते तव सुप्रभातम्}

\fourlineindentedshloka
{भास्वानुदेति विकचानि सरोरुहाणि}
{सम्पूरयन्ति निनदैः ककुभो विहङ्गाः}
{श्रीवैष्णवाः सततमर्थित-मङ्गलास्ते}
{धामाऽऽश्रयन्ति तव वेङ्कट सुप्रभातम्}

\fourlineindentedshloka
{ब्रह्मादयः सुरवराः समहर्षयस्ते}
{सन्तः सनन्दन मुखास्तव योगिवर्याः}
{धामान्तिके तव हि मङ्गलवस्तुहस्ताः}
{श्रीवेङ्कटाचलपते तव सुप्रभातम्}

\fourlineindentedshloka
{लक्ष्मीनिवास निरवद्यगुणैकसिन्धो}
{संसार-सागर-समुत्तरणैकसेतो}
{वेदान्तवेद्य निजवैभव भक्तभोग्य}
{श्रीवेङ्कटाचलपते तव सुप्रभातम्}

\fourlineindentedshloka
{इत्थं वृषाचलपतेरिह सुप्रभातम्}
{ये मानवाः प्रतिदिनं पठितुं प्रवृत्ताः}
{तेषां प्रभातसमये स्मृतिरङ्गभाजाम्}
{प्रज्ञां परार्थसुलभां परमां प्रसूते}
॥इति श्री वेङ्कटेश सुप्रभातम् सम्पूर्णम्॥