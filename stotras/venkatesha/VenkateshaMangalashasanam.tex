% !TeX program = XeLaTeX
% !TeX root = ../../shloka.tex

\sect{वेङ्कटेश मङ्गलाशासनम्}
\twolineshloka
{श्रियः कान्ताय कल्याणनिधये निधयेऽर्थिनाम्}
{श्रीवेङ्कटनिवासाय श्रीनिवासाय मङ्गलम्}

\twolineshloka
{लक्ष्मी-सविभ्रमालोक-सुभ्रू-विभ्रमचक्षुषे}
{चक्षुषे सर्वलोकानां वेङ्कटेशाय मङ्गलम्}

\twolineshloka
{श्रीवेङ्कटाद्रि-शृङ्गाग्र-मङ्गलाभरणाङ्घ्रये}
{मङ्गलानां निवासाय श्रीनिवासाय मङ्गलम्}

\twolineshloka
{सर्वावयवसौन्दर्य-सम्पदा सर्वचेतसाम्}
{सदा सम्मोहनायास्तु वेङ्कटेशाय मङ्गलम्}

\twolineshloka
{नित्याय निरवद्याय सत्यानन्दचिदात्मने}
{सर्वान्तरात्मने श्रीमद्-वेङ्कटेशाय मङ्गलम्}

\twolineshloka
{स्वतस्सर्वविदे सर्वशक्तये सर्वशेषिणे}
{सुलभाय सुशीलाय वेङ्कटेशाय मङ्गलम्}

\twolineshloka
{परस्मै ब्रह्मणे पूर्णकामाय परमात्मने}
{प्रयुञ्जे परतत्त्वाय वेङ्कटेशाय मङ्गलम्}

\twolineshloka
{आकालतत्त्वमश्रान्तमात्मनामनुपश्यताम्}
{अतृप्त्यमृतरूपाय वेङ्कटेशाय मङ्गलम्}

\twolineshloka
{प्रायः स्वचरणौ पुंसां शरण्यत्वेन पाणिना}
{कृपयाऽऽदिशते श्रीमद्-वेङ्कटेशाय मङ्गलम्}

\twolineshloka
{दयामृत-तरङ्गिण्यास्तरङ्गैरिव शीतलैः}
{अपाङ्गैः सिञ्चते विश्वं वेङ्कटेशाय मङ्गलम्}

\twolineshloka
{स्रग्भूषाम्बरहेतीनां सुषमावहमूर्तये}
{सर्वार्तिशमनायास्तु वेङ्कटेशाय मङ्गलम्}

\twolineshloka
{श्रीवैकुण्ठविरक्ताय स्वामिपुष्करिणीतटे}
{रमया रममाणाय वेङ्कटेशाय मङ्गलम्}

\twolineshloka
{श्रीमत् सुन्दरजामातृमुनिमानसवासिने}
{सर्वलोकनिवासाय श्रीनिवासाय मङ्गलम्}

\twolineshloka
{मङ्गलाशासनपरैर्मदाचार्य-पुरोगमैः}
{सर्वैश्च पूर्वैराचार्यैः सत्कृतायास्तु मङ्गलम्}

॥इति श्री~वेङ्कटेश मङ्गलाशासनं सम्पूर्णम्॥