% !TeX program = XeLaTeX
% !TeX root = ../../shloka.tex

\sect{वसिष्ठोपदेशः}

\uvacha{वसिष्ठ  उवाच}

{आत्मा नित्योऽव्ययः शुद्धो जन्मनाशादिवर्जितः॥९५॥} %7-95
\addtocounter{shlokacount}{95}

\twolineshloka
{शरीरं जडमत्यर्थमपवित्रं विनश्वरम्}
{विचार्यमाणे शोकस्य नावकाशः कथञ्चन} %7-96

\twolineshloka
{पिता वा तनयो वाऽपि यदि मृत्युवशं गतः}
{मूढास्तमनुशोचन्ति स्वात्मताडनपूर्वकम्} %7-97

\twolineshloka
{निःसारे खलु संसारे वियोगो ज्ञानिनां यदा}
{भवेद्वैराग्यहेतुः स शान्तिसौख्यं तनोति च} %7-98

\twolineshloka
{जन्मवान् यदि लोकेऽस्मिन्स्तर्हि तं मृत्युरन्वगात्}
{तस्मादपरिहार्योऽयं मृत्युर्जन्मवतां सदा} %7-99

\twolineshloka
{स्वकर्मवशतः सर्वजन्तूनां प्रभवाप्ययौ}
{विजानन्नप्यविद्वान् यः कथं शोचति बान्धवान्} %7-100

\twolineshloka
{ब्रह्माण्डकोटयो नष्टाः सृष्टयो बहुशो गताः}
{शुष्यन्ति सागराः सर्वे कैवास्था क्षणजीविते} %7-101

\twolineshloka
{चलपत्रान्तलग्नाम्बुबिन्दुवत्क्षणभङ्गुरम्}
{आयुस्त्यजत्यवेलायां कस्तत्र प्रत्ययस्तव} %7-102

\twolineshloka
{देही प्राक्तनदेहोत्थकर्मणा देहवान् पुनः}
{तद्देहोत्थेन च पुनरेवं देहः सदात्मनः} %7-103

\twolineshloka
{यथा त्यजति वै जीर्णं वासो गृह्णाति नूतनम्}
{तथा जीर्णं परित्यज्य देही देहं पुनर्नवम्} %7-104

\twolineshloka
{भजत्येव सदा तत्र शोकस्यावसरः कुतः}
{आत्मा न म्रियते जातु जायते न च वर्धते} %7-105

\twolineshloka
{षड्भावरहितोऽनन्तः सत्यप्रज्ञानविग्रहः}
{आनन्दरूपो बुद्ध्यादिसाक्षी लयविवर्जितः} %7-106

\twolineshloka
{एक एव परो ह्यात्मा ह्यद्वितीयः समः स्थितः}
{इत्यात्मानं दृढं ज्ञात्वा त्यक्त्वा शोकं कुरु क्रियाम्} %7-107

\twolineshloka
{तैलद्रोण्याः पितुर्देहमुद्धृत्य सचिवैः सह}
{कृत्यं कुरु यथान्यायमस्माभिः कुलनन्दन} %7-108

{॥इति श्रीमदध्यात्मरामायणे उमामहेश्वरसंवादे
अयोध्याकाण्डे सप्तमे सर्गे  वसिष्ठोपदेशः  सम्पूर्णः॥}
