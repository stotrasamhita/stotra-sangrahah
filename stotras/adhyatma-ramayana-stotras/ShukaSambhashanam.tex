% !TeX program = XeLaTeX
% !TeX root = ../../shloka.tex

\sect{शुकसम्भाषणम्}


\addtocounter{shlokacount}{39}

\uvacha{शुक उवाच}

\twolineshloka
{रामो न मानुषः साक्षादादिनारायणः परः}
{सीता साक्षाज्जगद्धेतुश्चिच्छक्तिर्जगदात्मिका} %4-40

\twolineshloka
{ताभ्यामेव समुत्पन्नं जगत्स्थावरजङ्गमम्}
{तस्माद्रामश्च सीता च जगतस्तस्थुषश्च तौ} %4-41

\twolineshloka
{पितरौ पृथिवीपाल तयोर्वैरी कथं भवेत्}
{अजानता त्वयाऽऽनीता जगन्मातैव जानकी} %4-42

\twolineshloka
{क्षणनाशिनि संसारे शरीरे क्षणभङ्गुरे}
{पञ्चभूतात्मके राजंश्चतुर्विंशतितत्त्वके} %4-43

\twolineshloka
{मलमांसास्थिदुर्गन्धभूयिष्ठेऽहङ्कृतालये}
{कैवास्था व्यतिरिक्तस्य काये तव जडात्मके} %4-44

\twolineshloka
{यत्कृते ब्रह्महत्यादिपातकानि कृतानि च}
{भोगभोक्ता तु यो देहः स देहोऽत्र पतिष्यति} %4-45

\twolineshloka
{पुण्यपापे समायातो जीवेन सुखदुःखयोः}
{कारणे देहयोगादिनाऽऽत्मनः कुरुतोऽनिशम्} %4-46

\twolineshloka
{यावद्देहोऽस्मि कर्ताऽस्मीत्यात्माऽहं कुरुतेऽवशः}
{अध्यासात्तावदेव स्याज्जन्मनाशादिसम्भवः} %4-47

\twolineshloka
{तस्मात्त्वं त्यज देहादावभिमानं महामते}
{आत्मातिऽनिर्मलः शुद्धो विज्ञानात्माऽचलोऽव्ययः} %4-48

\twolineshloka
{स्वाज्ञानवशतो बन्धं प्रतिपद्य विमुह्यति}
{तस्मात्त्वं शुद्धभावेन ज्ञात्वाऽऽत्मानं सदा स्मर} %4-49

\twolineshloka
{विरतिं भज सर्वत्र पुत्रदारगृहादिषु}
{निरयेष्वपि भोगः स्याच्छ्वशूकरतनावपि} %4-50

\twolineshloka
{देहं लब्ध्वा विवेकाढ्यं द्विजत्वं च विशेषतः}
{तत्रापि भारते वर्षे कर्मभूमौ सुदुर्लभम्} %4-51

\twolineshloka
{को विद्वानात्मसात्कृत्वा देहं भोगानुगो भवेत्}
{अतस्त्वं ब्राह्मणो भूत्वा पौलस्त्यतनयश्च सन्} %4-52

\twolineshloka
{अज्ञानीव सदा भोगाननुधावसि किं मुधा}
{इतः परं वा त्यक्त्वा त्वं सर्वसङ्गं समाश्रय} %4-53

\twolineshloka
{राममेव परात्मनं भक्तिभावेन सर्वदा}
{सीतां समर्प्य रामाय तत्पादानुचरो भव} %4-54

\threelineshloka
{विमुक्तः सर्वपापेभ्यो विष्णुलोकं प्रयास्यसि}
{नो चेद्गमिष्यसेऽधोऽधः पुनरावृत्तिवर्जितः}
{अङ्गीकुरुष्व मद्वाक्यं हितमेव वदामि ते} %4-55

\fourlineindentedshloka
{सत्सङ्गतिं कुरु भजस्व हरिं शरण्यम्}
{श्रीराघवं मरकतोपलकान्तिकान्तम्}
{सीतासमेतमनिशं धृतचापबाणम्}
{सुग्रीवलक्ष्मणविभीषणसेविताङ्घ्रिम्} %4-56

{॥इति श्रीमदध्यात्मरामायणे उमामहेश्वरसंवादे युद्धकाण्डे
चतुर्थे सर्गे शुकसम्भाषणं  सम्पूर्णम्॥}
