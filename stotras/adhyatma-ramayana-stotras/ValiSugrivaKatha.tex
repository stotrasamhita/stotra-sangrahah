% !TeX program = XeLaTeX
% !TeX root = ../../shloka.tex

\sect{वाली-सुग्रीव-कथा}

% \addtocounter{shlokacount}{28}

\uvacha{श्री-राम उवाच}

\twolineshloka
{वालिसुग्रीवयोर्जन्म श्रोतुमिच्छामि तत्त्वतः}
{रवीन्द्रौ वानराकारौ जज्ञाताविति नः श्रुतम्} %3-1

\uvacha{अगस्त्य उवाच}

\twolineshloka
{मेरोः स्वर्णमयस्याद्रेर्मध्यशृङ्गे मणिप्रभे}
{तस्मिन् सभाऽऽस्ते विस्तीर्णा ब्रह्मणः शतयोजना} %3-2

\twolineshloka
{तस्यां चतुर्मुखः साक्षात्कदाचिद्योगमास्थितः}
{नेत्राभ्यां पतितं दिव्यमानन्दसलिलं बहु} %3-3

\twolineshloka
{तद्गृहीत्वा करे ब्रह्मा ध्यात्वा किञ्चित्तदत्यजत्}
{भूमौ पतितमात्रेण तस्माज्जातो महाकपिः} %3-4

\twolineshloka
{तमाह द्रुहिणो वत्स किञ्चित्कालं वसात्र मे}
{समीपे सर्वशोभाढ्ये ततः श्रेयो भविष्यति} %3-5

\twolineshloka
{इत्युक्तो न्यवसत्तत्र ब्रह्मणा वानरोत्तमः}
{एवं बहुतिथे काले गते ऋक्षाधिपः सुधीः} %3-6

\twolineshloka
{कदाचित्पर्यटन्नद्रौ फलमूलार्थमुद्यतः}
{अपश्यद्दिव्यसलिलां वापीं मणिशिलान्विताम्} %3-7

\twolineshloka
{पानीयं पातुमागच्छत्तत्र छायामयं कपिम्}
{दृष्ट्वा प्रतिकपिं मत्वा निपपात जलान्तरे} %3-8

\twolineshloka
{तत्रादृष्ट्वा हरिं शीघ्रं पुनरुत्प्लुत्य वानरः}
{अपश्यत्सुन्दरीं रामामात्मानं विस्मयं गतः} %3-9

\twolineshloka
{ततः सुरेशो देवेशं पूजयित्वा चतुर्मुखम्}
{गच्छन् मध्याह्नसमये दृष्ट्वा नारीं मनोरमाम्} %3-10

\twolineshloka
{कन्दर्पशरविद्धाङ्गस्त्यक्तवान् वीर्यमुत्तमम्}
{तामप्राप्यैव तद्बीजं वालदेशेऽपतद्भुवि} %3-11

\twolineshloka
{वाली समभवत्तत्र शक्रतुल्यपराक्रमः}
{तस्य दत्त्वा सुरेशानः स्वर्णमालां दिवं गतः} %3-12

\twolineshloka
{भानुरप्यागतस्तत्र तदानीमेव भामिनीम्}
{दृष्ट्वा कामवशो भूत्वा ग्रीवादेशेऽसृजन्महत्} %3-13

\twolineshloka
{बीजं तस्यास्ततः सद्यो महाकायोऽभवद्धरिः}
{तस्य दत्त्वा हनूमन्तं सहायार्थं गतो रविः} %3-14

\twolineshloka
{पुत्रद्वयं समादाय गत्वा सा निद्रिता क्वचित्}
{प्रभातेऽपश्यदात्मानं पूर्ववद्वानराकृतिम्} %3-15

\twolineshloka
{फलमूलादिभिः सार्धं पुत्राभ्यां सहितः कपिः}
{नत्वा चतुर्मुखस्याग्रे ऋक्षराजः स्थितः सुधीः} %3-16

\twolineshloka
{ततोऽब्रवीत्समाश्वास्य बहुशः कपिकुञ्जरम्}
{तत्रैकं देवतादूतमाहूयामरसन्निभम्} %3-17

\twolineshloka
{गच्छ दूत मयाऽऽदिष्टो गृहीत्वा वानरोत्तमम्}
{किष्किन्धां दिव्यनगरीं निर्मितां विश्वकर्मणा} %3-18

\twolineshloka
{सर्वसौभाग्यवलितां देवैरपि दुरासदाम्}
{तस्यां सिंहासने वीरं राजानमभिषेचय} %3-19

\twolineshloka
{सप्तद्वीपगता ये ये वानराः सन्ति दुर्जयाः}
{सर्वे ते ऋक्षराजस्य भविष्यन्ति वशेऽनुगाः} %3-20

\twolineshloka
{यदा नारायणः साक्षाद्रामो भूत्वा सनातनः}
{भूभारासुरनाशाय सम्भविष्यति भूतले} %3-21

\twolineshloka
{तदा सर्वे सहायार्थे तस्य गच्छन्तु वानराः}
{इत्युक्तो ब्रह्मणा दूतो देवानां स महामतिः} %3-22

\twolineshloka
{यथाऽऽज्ञप्तस्तथा चक्रे ब्रह्मणा तं हरीश्वरम्}
{देवदूतस्ततो गत्वा ब्रह्मणे तन्न्यवेदयत्} %3-23

\onelineshloka
{तदादि वानराणां सा किष्किन्धाऽभून्नृपाश्रयः} %3-24

\threelineshloka
{सर्वेश्वरस्त्वमेवासीरिदानीं ब्रह्मणार्थितः}
{भूमेर्भारो हृतः कृत्स्नस्त्वया लीलानृदेहिना}
{सर्वभूतान्तरस्थस्य नित्यमुक्तचिदात्मनः} %3-25

\twolineshloka
{अखण्डानन्तरूपस्य कियानेष पराक्रमः}
{तथाऽपि वर्ण्यते सद्भिर्लीलामानुषरूपिणः} %3-26

\twolineshloka
{यशस्ते सर्वलोकानां पापहत्यै सुखाय च}
{य इदं कीर्तयेन्मर्त्यो वालिसुग्रीवयोर्महत्} %3-27

\onelineshloka
{जन्म त्वदाश्रयत्वात्स मुच्यते सर्वपातकैः} %3-28

{॥इति श्रीमदध्यात्मरामायणे उमामहेश्वरसंवादे उत्तरकाण्डे तृतीये  सर्गे
वाली-सुग्रीव-कथा सम्पूर्णा॥}
