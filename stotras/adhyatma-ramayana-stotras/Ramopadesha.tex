% !TeX program = XeLaTeX
% !TeX root = ../../shloka.tex

\sect{ज्ञान-विज्ञानोपदेशः}


\addtocounter{shlokacount}{18}
\uvacha{श्रीराम उवाच}

\twolineshloka
{शृणु वक्ष्यामि ते वत्स गुह्याद्गुह्यतरं परम्}
{यद्विज्ञाय नरो जह्यात्सद्यो वैकल्पकं भ्रमम्} %4-19

\twolineshloka
{आदौ मायास्वरूपं ते वक्ष्यामि तदनन्तरम्}
{ज्ञानस्य साधनं पश्चाज्ज्ञानविज्ञानसंयुतम्} %4-20

\twolineshloka
{ज्ञेयं च परमात्मानं यज्ज्ञात्वा मुच्यते भयात्}
{अनात्मनि शरीरादावात्मबुद्धिस्तु या भवेत्} %4-21

\twolineshloka
{सैव माया तयैवासौ संसारः परिकल्प्यते}
{रूपे द्वे निश्चिते पूर्वे मायायाः कुलनन्दन} %4-22

\twolineshloka
{विक्षेपावरणे तत्र प्रथमं कल्पयेज्जगत्}
{लिङ्गाद्यब्रह्मपर्यन्तं स्थूलसूक्ष्मविभेदतः} %4-23

\twolineshloka
{अपरं त्वखिलं ज्ञानरूपमावृत्य तिष्ठति}
{मायया कल्पितं विश्वं परमात्मनि केवले} %4-24

\twolineshloka
{रज्जौ भुजङ्गवद्भ्रान्त्या विचारे नास्ति किञ्चन}
{श्रूयते दृश्यते यद्यत्स्मर्यते वा नरैः सदा} %4-25

\twolineshloka
{असदेव हि तत्सर्वं यथा स्वप्नमनोरथौ}
{देह एव हि संसारवृक्षमूलं दृढं स्मृतम्} %4-26

\onelineshloka
{तन्मूलः पुत्रदारादिबन्धः किं तेऽन्यथाऽऽत्मनः} %4-27


\twolineshloka
{देहस्तु स्थूलभूतानां पञ्च तन्मात्रपञ्चकम्}
{अहङ्कारश्च बुद्धिश्च इन्द्रियाणि तथा दश} %4-28

\twolineshloka
{चिदाभासो मनश्चैव मूलप्रकृतिरेव च}
{एतत्क्षेत्रमिति ज्ञेयं देह इत्यभिधीयते} %4-29

\twolineshloka
{एतैर्विलक्षणो जीवः परमात्मा निरामयः}
{तस्य जीवस्य विज्ञाने साधनान्यपि मे शृणु} %4-30

\twolineshloka
{जीवश्च परमात्मा च पर्यायो नात्र भेदधीः}
{मानाभावस्तथा दम्भहिंसादिपरिवर्जनम्} %4-31

\twolineshloka
{पराक्षेपादिसहनं सर्वत्रावक्रता तथा}
{मनोवाक्कायसद्भक्त्या सद्गुरोः परिसेवनम्} %4-32

\twolineshloka
{बाह्याभ्यन्तरसंशुद्धिः स्थिरता सत्क्रियादिषु}
{मनोवाक्कायदण्डश्च विषयेषु निरीहता} %4-33

\twolineshloka
{निरहङ्कारता जन्मजराद्यालोचनं तथा}
{असक्तिः स्नेहशून्यत्वं पुत्रदारधनादिषु} %4-34

\twolineshloka
{इष्टानिष्टागमे नित्यं चित्तस्य समता तथा}
{मयि सर्वात्मके रामे ह्यनन्यविषया मतिः} %4-35

\twolineshloka
{जनसम्बाधरहितशुद्धदेशनिषेवणम्}
{प्राकृतैर्जनसङ्घैश्च ह्यरतिः सर्वदा भवेत्} %4-36

\twolineshloka
{आत्मज्ञाने सदोद्योगो वेदान्तार्थावलोकनम्}
{उक्तैरेतैर्भवेज्ज्ञानं विपरीतैर्विपर्ययः} %4-37

\twolineshloka
{बुद्धिप्राणमनोदेहाहङ्कृतिभ्यो विलक्षणः}
{चिदात्माऽहं नित्यशुद्धो बुद्ध एवेति निश्चयम्} %4-38

\twolineshloka
{येन ज्ञानेन संवित्ते तज्ज्ञानं निश्चितं च मे}
{विज्ञानं च तदैवैतत्साक्षादनुभवेद्यदा} %4-39

\twolineshloka
{आत्मा सर्वत्र पूर्णः स्याच्चिदानन्दात्मकोऽव्ययः}
{बुद्ध्याद्युपाधिरहितः परिणामादिवर्जितः} %4-40

\twolineshloka
{स्वप्रकाशेन देहादीन् भासयन्ननपावृतः}
{एक एवाद्वितीयश्च सत्यज्ञानादिलक्षणः} %4-41

\twolineshloka
{असङ्गः स्वप्रभो द्रष्टा विज्ञानेनावगम्यते}
{आचार्यशास्त्रोपदेशाद्यैक्यज्ञानं यदा भवेत्} %4-42

\twolineshloka
{आत्मनोर्जीवपरयोर्मूलाविद्या तदैव हि}
{लीयते कार्यकरणैः सहैव परमात्मनि} %4-43

\twolineshloka
{सावस्था मुक्तिरित्युक्ता ह्युपचारोऽयमात्मनि}
{इदं मोक्षस्वरूपं ते कथितं रघुनन्दन} %4-44

\twolineshloka
{ज्ञानविज्ञानवैराग्यसहितं मे परात्मनः}
{किन्त्वेतद्दुर्लभं मन्ये मद्भक्तिविमुखात्मनाम्} %4-45

\twolineshloka
{चक्षुष्मतामपि तथा रात्रौ सम्यङ् न दृश्यते}
{पदं दीपसमेतानां दृश्यते सम्यगेव हि} %4-46

\twolineshloka
{एवं मद्भक्तियुक्तानामात्मा सम्यक् प्रकाशते}
{मद्भक्तेः कारणं किञ्चिद्वक्ष्यामि शृणु तत्त्वतः} %4-47

\twolineshloka
{मद्भक्तसङ्गो मत्सेवा मद्भक्तानां निरन्तरम्}
{एकादश्युपवासादि मम पर्वानुमोदनम्} %4-48

\twolineshloka
{मत्कथाश्रवणे पाठे व्याख्याने सर्वदा रतिः}
{मत्पूजापरिनिष्ठा च मम नामानुकीर्तनम्} %4-49

\twolineshloka
{एवं सततयुक्तानां भक्तिरव्यभिचारिणी}
{मयि सञ्जायते नित्यं ततः किमवशिष्यते} %4-50

\twolineshloka
{अतो मद्भक्तियुक्तस्य ज्ञानं विज्ञानमेव च}
{वैराग्यं च भवेच्छीघ्रं ततो मुक्तिमवाप्नुयात्} %4-51

\twolineshloka
{कथितं सर्वमेतत्ते तव प्रश्नानुसारतः}
{अस्मिन्मनः समाधाय यस्तिष्ठेत्स तु मुक्तिभाक्} %4-52

\twolineshloka
{न वक्तव्यमिदं यत्नान्मद्भक्तिविमुखाय हि}
{मद्भक्ताय प्रदातव्यमाहूयापि प्रयत्नतः} %4-53

\twolineshloka
{य इदं तु पठेन्नित्यं श्रद्धाभक्तिसमन्वितः}
{अज्ञानपटलध्वान्तं विधूय परिमुच्यते} %4-54

\fourlineindentedshloka
{भक्तानां मम योगिनां सुविमलस्वान्तातिशान्तात्मनां}
{मत्सेवाभिरतात्मनां च विमलज्ञानात्मनां सर्वदा}
{सङ्गं यः कुरुते सदोद्यतमतिस्तत्सेवनानन्यधीः}
{मोक्षस्तस्य करे स्थितोऽहमनिशं दृश्यो भवे नान्यथा} %4-55

{॥इति श्रीमदध्यात्मरामयणे उमामहेश्वरसंवादे
अरण्यकाण्डे चतुर्थः सर्गः॥}
