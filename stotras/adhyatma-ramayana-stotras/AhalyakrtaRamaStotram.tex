% !TeX program = XeLaTeX
% !TeX root = ../../shloka.tex
\sect{अहल्याकृत-रामस्तोत्रम्}
\addtocounter{shlokacount}{42}
\uvacha{अहल्योवाच}

\fourlineindentedshloka
{अहो कृतार्थाऽस्मि जगन्निवास ते}
{पादाब्जसंलग्नरजः कणादहम्}
{स्पृशामि यत्पद्मजशङ्करादिभिः}
{विमृग्यते रन्धितमानसैः सदा}% १

\fourlineindentedshloka
{अहो विचित्रं तव राम चेष्टितं}
{मनुष्यभावेन विमोहितं जगत्}
{चलस्यजस्रं चरणादिवर्जितः}
{सम्पूर्ण आनन्दमयोऽतिमायिकः}% २

\fourlineindentedshloka
{यत्पादपङ्कजपरागपवित्रगात्रा}
{भागीरथी भवविरिञ्चिमुखान्  पुनाति}
{साक्षात्स एव मम दृग्विषयो यदाऽऽस्ते}
{किं वर्ण्यते मम पुराकृतभागधेयम्}% ३

\fourlineindentedshloka
{मर्त्यावतारे मनुजाकृतिं हरिं}
{रामाभिधेयं रमणीयदेहिनम्}
{धनुर्धरं पद्मविशाललोचनं}
{भजामि नित्यं न परान्  भजिष्ये}% ४

\fourlineindentedshloka
{यत्पादपङ्कजरजः श्रुतिभिर्विमृग्यं}
{यन्नाभिपङ्कजभवः कमलासनश्च}
{यन्नामसाररसिको भगवान्पुरारिः}
{तं  रामचन्द्रमनिशं हृदि भावयामि}% ५

\fourlineindentedshloka
{यस्यावतारचरितानि विरिञ्चिलोके}
{गायन्ति नारदमुखा भवपद्मजाद्याः}
{आनन्दजाश्रुपरिषिक्तकुचाग्रसीमा}
{वागीश्वरी च तमहं शरणं प्रपद्ये}% ६

\fourlineindentedshloka
{सोऽयं परात्मा पुरुषः पुराणः}
{एकः स्वयं ज्योतिरनन्त आद्यः}
{मायातनुं लोकविमोहनीयां}
{धत्ते परानुग्रह एष रामः}% ७

\fourlineindentedshloka
{अयं हि विश्वोद्भवसंयमानाम्}
{एकः  स्वमायागुणबिम्बितो यः}
{विरिञ्चिविष्ण्वीश्वरनामभेदान्}
{धत्ते स्वतन्त्रः परिपूर्ण आत्मा}% ८

\fourlineindentedshloka
{नमोऽस्तु ते राम तवाङ्घ्रिपङ्कजं}
{श्रिया धृतं वक्षसि लालितं प्रियात्}
{आक्रान्तमेकेन जगत्त्रयं पुरा}
{ध्येयं मुनीन्द्रैरभिमानवर्जितैः}% ९

\twolineshloka
{जगतामादिभूतस्त्वं जगत्त्वं जगदाश्रयः}
{सर्वभूतेष्वसंयुक्त एको भाति भवान् परः}% १०

\twolineshloka
{ओङ्कारवाच्यस्त्वं राम वाचामविषयः पुमान्}
{वाच्यवाचकभेदेन भवानेव जगन्मयः}% ११

\twolineshloka
{कार्यकारणकर्तृत्वफलसाधनभेदतः}
{एको विभासि राम त्वं मायया बहुरूपया}% १२

\twolineshloka
{त्वन्मायामोहितधियस्त्वां न जानन्ति तत्त्वतः}
{मानुषं त्वाऽभिमन्यन्ते मायिनं परमेश्वरम्}% १३

\twolineshloka
{आकाशवत्त्वं सर्वत्र बहिरन्तर्गतोऽमलः}
{असङ्गो ह्यचलो नित्यः शुद्धो बुद्धः सदव्ययः}% १४

\twolineshloka
{योषिन्मूढाऽहमज्ञा ते तत्त्वं जाने कथं विभो}
{तस्मात्ते शतशो राम नमस्कुर्यामनन्यधीः}% १५

\twolineshloka
{देव मे यत्रकुत्रापि स्थिताया अपि सर्वदा}
{त्वत्पादकमले सक्ता भक्तिरेव  सदाऽस्तु मे}% १६

\twolineshloka
{नमस्ते पुरुषाध्यक्ष नमस्ते भक्तवत्सल}
{नमस्तेऽस्तु हृषीकेश नारायण नमोऽस्तु ते}% १७

\fourlineindentedshloka
{भवभयहरमेकं भानुकोटिप्रकाशं}
{करधृतशरचापं कालमेघावभासम्}
{कनकरुचिरवस्त्रं रत्नवत्कुण्डलाढ्यं}
{कमलविशदनेत्रं सानुजं राममीडे}% १८

\twolineshloka
{स्तुत्वैवं पुरुषं साक्षाद्राघवं पुरतः स्थितम्}
{परिक्रम्य प्रणम्याऽऽशु सानुज्ञाता ययौ पतिम्}% १९

\twolineshloka
{अहल्यया कृतं स्तोत्रं यः पठेद्भक्तिसंयुतः}
{स मुच्यतेऽखिलैः पापैः परं ब्रह्माधिगच्छति}% २०

\twolineshloka
{पुत्राद्यर्थे पठेद्भक्त्या रामं हृदि निधाय च}
{संवत्सरेण लभते वन्ध्या अपि सुपुत्रकम्}% २१

\onelineshloka
{सर्वान् कामानवाप्नोति रामचन्द्रप्रसादतः}% २२

\fourlineindentedshloka
{ब्रह्मघ्नो गुरुतल्पगोऽपि पुरुषः स्तेयी सुरापोऽपि वा}
{मातृभ्रातृविहिंसकोऽपि सततं भोगैकबद्धातुरः}
{नित्यं स्तोत्रमिदं जपन्  रघुपतिं भक्त्या हृदिस्थं स्मरन्}
{ध्यायन् मुक्तिमुपैति किं पुनरसौ स्वाचारयुक्तो नरः}% २३

॥इति श्रीमदध्यात्मरामायणे उमामहेश्वरसंवादे बालकाण्डे
पञ्चमे  सर्गे श्री-अहल्याविरचितं श्री-रामचन्द्रस्तोत्रं सम्पूर्णम्॥
