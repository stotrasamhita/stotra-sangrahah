% !TeX program = XeLaTeX
% !TeX root = ../../shloka.tex

\sect{रामकथाकथनम्}


\addtocounter{shlokacount}{11}

\twolineshloka
{विलोक्य हनुमान् किञ्चिद्विचार्यैतदभाषत}
{शनैः शनैः सूक्ष्मरूपो जानक्याः श्रोत्रगं वचः} %3-3

\twolineshloka
{इक्ष्वाकुवंशसम्भूतो राजा दशरथो महान्}
{अयोध्याधिपतिस्तस्य चत्वारो लोकविश्रुताः} %3-4

\twolineshloka
{पुत्रा देवसमाः सर्वे लक्षणैरुपलक्षिताः}
{रामश्च लक्ष्मणश्चैव भरतश्चैव शत्रुहा} %3-5

\twolineshloka
{ज्येष्ठो रामः पितुर्वाक्याद्दण्डकारण्यमागतः}
{लक्ष्मणेन सह भ्रात्रा सीतया भार्यया सह} %3-6

\twolineshloka
{उवास गौतमीतीरे पञ्चवट्यां महामनाः}
{तत्र नीता महाभागा सीता जनकनन्दिनी} %3-7

\twolineshloka
{रहिते रामचन्द्रेण रावणेन दुरात्मना}
{ततो रामोऽतिदुःखार्तो मार्गमाणोऽथ जानकीम्} %3-8

\twolineshloka
{जटायुषं पक्षिराजमपश्यत्पतितं भुवि}
{तस्मै दत्त्वा दिवं शीघ्रमृष्यमूकमुपागमत्} %3-9

\twolineshloka
{सुग्रीवेण कृता मैत्री रामस्य विदितात्मनः}
{तद्भार्याहारिणं हत्वा वालिनं रघुनन्दनः} %3-10

\twolineshloka
{राज्येऽभिषिच्य सुग्रीवं मित्रकार्यं चकार सः}
{सुग्रीवस्तु समानाय्य वानरान् वानरप्रभुः} %3-11

\twolineshloka
{प्रेषयामास परितो वानरान् परिमार्गणे}
{सीतायास्तत्र चैकोऽहं सुग्रीवसचिवो हरिः} %3-12

\twolineshloka
{सम्पातिवचनाच्छीघ्रमुल्लङ्घ्य शतयोजनम्}
{समुद्रं नगरीं लङ्कां विचिन्वन् जानकीं शुभाम्} %3-13

\twolineshloka
{शनैरशोकवनिकां विचिन्वन् शिंशपातरुम्}
{अद्राक्षं जानकीमत्र शोचन्तीं दुःखसम्प्लुताम्} %3-14

\twolineshloka
{रामस्य महिषीं देवीं कृतकृत्योऽहमागतः}
{इत्युक्त्वोपररामाथ मारुतिर्बुद्धिमत्तरः} %3-15

{॥इति श्रीमदध्यात्मरामायणे उमामहेश्वरसंवादे  सुन्दरकाण्डे
तृतीये सर्गे हनूमता  रामकथाकथनं सम्पूर्णम्॥}
