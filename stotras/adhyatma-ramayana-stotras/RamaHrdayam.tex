% !TeX program = XeLaTeX
% !TeX root = ../../shloka.tex

\sect{राम हृदयम्}

\fourlineindentedshloka
{यः पृथिवीभरवारणाय दिविजैः सम्प्रार्थितश्चिन्मयः}
{सञ्जातः पृथिवीतले रविकुले मायामनुष्योऽव्ययः}
{निश्चक्रं हतराक्षसः पुनरगाद् ब्रह्मत्वमाद्यं स्थिरां}
{कीर्तिं पापहरां विधाय जगतां तं जानकीशं भजे} %1-1

\fourlineindentedshloka
{विश्वोद्भवस्थितिलयादिषु हेतुमेकं}
{मायाश्रयं विगतमायमचिन्त्यमूर्तिम्}
{आनन्दसान्द्रममलं निजबोधरूपं}
{सीतापतिं विदिततत्त्वमहं नमामि} %1-2

\fourlineindentedshloka
{पठन्ति ये नित्यमनन्यचेतसः}
{शृण्वन्ति चाध्यात्मिकसंज्ञितं शुभम्}
{रामायणं सर्वपुराणसम्मतं}
{निर्धूतपापा हरिमेव यान्ति ते} %1-3

\fourlineindentedshloka
{अध्यात्मरामायणमेव नित्यं}
{पठेद्यदीच्छेद्भवबन्धमुक्तिम्}
{गवां सहस्रायुतकोटिदानात्}
{फलं लभेद्यः शृणुयात्स नित्यम्} %1-4

\twolineshloka
{पुरारिगिरिसम्भूता श्रीरामार्णवसङ्गता}
{अध्यात्मरामगङ्गेयं पुनाति भुवनत्रयम्} %1-5

\fourlineindentedshloka
{कैलासाग्रे कदाचिद्रविशतविमले मन्दिरे रत्नपीठे}
{संविष्टं ध्याननिष्ठं त्रिनयनमभयं सेवितं सिद्धसङ्घैः}
{देवी वामाङ्कसंस्था गिरिवरतनया पार्वती भक्तिनम्रा}
{प्राहेदं देवमीशं सकलमलहरं वाक्यमानन्दकन्दम्} %1-6

\uvacha{पार्वत्युवाच}

\fourlineindentedshloka
{नमोऽस्तु ते देव जगन्निवास}
{सर्वात्मदृक् त्वं परमेश्वरोऽसि}
{पृच्छामि तत्त्वं पुरुषोत्तमस्य}
{सनातनं त्वं च सनातनोऽसि} %1-7

\fourlineindentedshloka
{गोप्यं यदत्यन्तमनन्यवाच्यं}
{वदन्ति भक्तेषु महानुभावाः}
{तदप्यहोऽहं तव देव भक्ता}
{प्रियोऽसि मे त्वं वद यत्तु पृष्टम्} %1-8

\fourlineindentedshloka
{ज्ञानं सविज्ञानमथानुभक्ति-}
{वैराग्ययुक्तं च मितं विभास्वत्}
{जानाम्यहं योषिदपि त्वदुक्तं}
{यथा तथा ब्रूहि तरन्ति येन} %1-9

\fourlineindentedshloka
{पृच्छामि चान्यच्च परं रहस्यं}
{तदेव चाग्रे वद वारिजाक्ष}
{श्रीरामचन्द्रेऽखिललोकसारे}
{भक्तिर्दृढा नौर्भवति प्रसिद्धा} %1-10

\fourlineindentedshloka
{भक्तिः प्रसिद्धा भवमोक्षणाय}
{नान्यत्ततः साधनमस्ति किञ्चित्}
{तथाऽपि हृत्संशयबन्धनं मे}
{विभेत्तुमर्हस्यमलोक्तिभिस्त्वम्} %1-11

\fourlineindentedshloka
{वदन्ति रामं परमेकमाद्यं}
{निरस्तमायागुणसम्प्रवाहम्}
{भजन्ति चाहर्निशमप्रमत्ताः}
{परं पदं यान्ति तथैव सिद्धाः} %1-12

\fourlineindentedshloka
{वदन्ति केचित्परमोऽपि रामः}
{स्वाविद्यया संवृतमात्मसंज्ञम्}
{जानाति नाऽऽत्मानमतः परेण}
{सम्बोधितो वेद परात्मतत्त्वम्} %1-13

\fourlineindentedshloka
{यदि स्म जानाति कुतो विलापः}
{सीताकृतेऽनेन कृतः परेण}
{जानाति नैवं यदि केन सेव्यः}
{समो हि सर्वैरपि जीवजातैः} %1-14

\twolineshloka
{अत्रोत्तरं किं विदितं भवद्भिः\hspace{12pt}}
{\hspace{12pt}तद्ब्रूत मे संशयभेदि वाक्यम्} %1-15

\uvacha{श्री-महादेव उवाच}

\fourlineindentedshloka
{धन्याऽसि भक्ताऽसि परात्मनस्त्वं}
{यज्ज्ञातुमिच्छा तव रामतत्त्वम्}
{पुरा न केनाप्यभिचोदितोऽहं}
{वक्तुं रहस्यं परमं निगूढम्} %1-16

\fourlineindentedshloka
{त्वयाऽद्य भक्त्या परिनोदितोऽहं}
{वक्ष्ये नमस्कृत्य रघूत्तमं ते}
{रामः परात्मा प्रकृतेरनादि-}
{रानन्द एकः पुरुषोत्तमो हि} %1-17

\fourlineindentedshloka
{स्वमायया कृत्स्नमिदं हि सृष्ट्वा}
{नभोवदन्तर्बहिरास्थितो यः}
{सर्वान्तरस्थोऽपि निगूढ आत्मा}
{स्वमायया सृष्टमिदं विचष्टे} %1-18

\fourlineindentedshloka
{जगन्ति नित्यं परितो भ्रमन्ति}
{यत्सन्निधौ चुम्बकलोहवद्धि}
{एतन्न जानन्ति विमूढचित्ताः}
{स्वाविद्यया संवृतमानसा ये} %1-19

\fourlineindentedshloka
{स्वाज्ञानमप्यात्मनि शुद्धबुद्धे}
{स्वारोपयन्तीह निरस्तमाये}
{संसारमेवानुसरन्ति ते वै}
{पुत्रादिसक्ताः पुरुकर्मयुक्ताः} %1-20

\fourlineindentedshloka
{यथाऽप्रकाशो न तु विद्यते रवौ}
{ज्योतिःस्वभावे परमेश्वरे तथा}
{विशुद्धविज्ञानघने रघूत्तमे-}
{ऽविद्या कथं स्यात्परतः परात्मनि} %1-21

\fourlineindentedshloka
{यथा हि चाक्ष्णा भ्रमता गृहादिकं}
{विनष्टदृष्टेर्भ्रमतीव दृश्यते}
{तथैव देहेन्द्रियकर्तुरात्मनः}
{कृते परेऽध्यस्य जनो विमुह्यति} %1-22

\fourlineindentedshloka
{नाहो न रात्रिः सवितुर्यथा भवेत्}
{प्रकाशरूपाव्यभिचारतः क्वचित्}
{ज्ञानं तथाऽज्ञानमिदं द्वयं हरौ}
{रामे कथं स्थास्यति शुद्धचिद्घने} %1-23

\fourlineindentedshloka
{तस्मात्परानन्दमये रघूत्तमे}
{विज्ञानरूपे हि न विद्यते तमः}
{अज्ञानसाक्षिण्यरविन्दलोचने}
{मायाश्रयत्वान्न हि मोहकारणम्} %1-24

\twolineshloka
{अत्र ते कथयिष्यामि रहस्यमपि दुर्लभम्}
{सीताराममरुत्सूनुसंवादं मोक्षसाधनम्} %1-25

\twolineshloka
{पुरा रामायणे रामे रावणं देवकण्टकम्}
{हत्वा रणे रणश्लाघी सपुत्रबलवाहनम्} %1-26

\twolineshloka
{सीतया सह सुग्रीवलक्ष्मणाभ्यां समन्वितः}
{अयोध्यामगमद्रामो हनूमत्प्रमुखैर्वृतः} %1-27

\twolineshloka
{अभिषिक्तः परिवृतो वसिष्ठाद्यैर्महात्मभिः}
{सिंहासने समासीनः कोटिसूर्यसमप्रभः} %1-28

\twolineshloka
{दृष्ट्वा तदा हनूमन्तं प्राञ्जलिं पुरतः स्थितम्}
{कृतकार्यं निराकाङ्क्षं ज्ञानापेक्षं महामतिम्} %1-29

\twolineshloka
{रामः सीतामुवाचेदं ब्रूहि तत्त्वं हनूमते}
{निष्कल्मषोऽयं ज्ञानस्य पात्रं नो नित्यभक्तिमान्} %1-30

\twolineshloka
{तथेति जानकी प्राह तत्त्वं रामस्य निश्चितम्}
{हनूमते प्रपन्नाय सीता लोकविमोहिनी} %1-31

\uvacha{सीतोवाच}

\twolineshloka
{रामं विद्धि परं ब्रह्म सच्चिदानन्दमद्वयम्}
{सर्वोपाधिविनिर्मुक्तं सत्तामात्रमगोचरम्} %1-32

\twolineshloka
{आनन्दं निर्मलं शान्तं निर्विकारं निरञ्जनम्}
{सर्वव्यापिनमात्मानं स्वप्रकाशमकल्मषम्} %1-33

\twolineshloka
{मां विद्धि मूलप्रकृतिं सर्गस्थित्यन्तकारिणीम्}
{तस्य सन्निधिमात्रेण सृजामीदमतन्द्रिता} %1-34

\twolineshloka
{तत्सान्निध्यान्मया सृष्टं तस्मिन्नारोप्यतेऽबुधैः}
{अयोध्यानगरे जन्म रघुवंशेऽतिनिर्मले} %1-35

\twolineshloka
{विश्वामित्रसहायत्वं मखसंरक्षणं ततः}
{अहल्याशापशमनं चापभङ्गो महेशितुः} %1-36

\twolineshloka
{मत्पाणिग्रहणं पश्चाद्भार्गवस्य मदक्षयः}
{अयोध्यानगरे वासो मया द्वादशवार्षिकः} %1-37

\twolineshloka
{दण्डकारण्यगमनं विराधवध एव च}
{मायामारीचमरणं मायासीताहृतिस्तथा} %1-38

\twolineshloka
{जटायुषो मोक्षलाभः कबन्धस्य तथैव च}
{शबर्याः पूजनं पश्चात्सुग्रीवेण समागमः} %1-39

\twolineshloka
{वालिनश्च वधः पश्चात्सीतान्वेषणमेव च}
{सेतुबन्धश्च जलधौ लङ्कायाश्च निरोधनम्} %1-40

\twolineshloka
{रावणस्य वधो युद्धे सपुत्रस्य दुरात्मनः}
{विभीषणे राज्यदानं पुष्पकेण मया सह} %1-41

\threelineshloka
{अयोध्यागमनं पश्चाद्राज्ये रामाभिषेचनम्}
{एवमादीनि कर्माणि मयैवाचरितान्यपि}
{आरोपयन्ति रामेऽस्मिन्निर्विकारेऽखिलात्मनि} %1-42

\fourlineindentedshloka
{रामो न गच्छति न तिष्ठति नानुशोच-}
{त्याकाङ्क्षते त्यजति नो न करोति किञ्चित्}
{आनन्दमूर्तिरचलः परिणामहीनो}
{मायागुणाननुगतो हि तथा विभाति} %1-43

\twolineshloka
{ततो रामः स्वयं प्राह हनूमन्तमुपस्थितम्}
{शृणु तत्त्वं प्रवक्ष्यामि ह्यात्मानात्मपरात्मनाम्} %1-44

\threelineshloka
{आकाशस्य यथा भेदस्त्रिविधो दृश्यते महान्}
{जलाशये महाकाशस्तदवच्छिन्न एव हि}
{प्रतिबिम्बाख्यमपरं दृश्यते त्रिविधं नभः} %1-45

\twolineshloka
{बुद्ध्यवच्छिन्नचैतन्यमेकं पूर्णमथापरम्}
{आभासस्त्वपरं बिम्बभूतमेवं त्रिधा चितिः} %1-46

\twolineshloka
{साभासबुद्धेः कर्तृत्वमविच्छिन्नेऽविकारिणि}
{साक्षिण्यारोप्यते भ्रान्त्या जीवत्वं च तथा बुधैः} %1-47

\twolineshloka
{आभासस्तु मृषा बुद्धिरविद्याकार्यमुच्यते}
{अविच्छिन्नं तु तद्ब्रह्म विच्छेदस्तु विकल्पतः} %1-48

\twolineshloka
{अविच्छिन्नस्य पूर्णेन एकत्वं प्रतिपाद्यते}
{तत्त्वमस्यादिवाक्यैश्च साभासस्याहमस्तथा} %1-49

\twolineshloka
{ऐक्यज्ञानं यदोत्पन्नं महावाक्येन चाऽऽत्मनोः}
{तदाऽविद्या स्वकार्यैश्च नश्यत्येव न संशयः} %1-50

\threelineshloka
{एतद्विज्ञाय मद्भक्तो मद्भावायोपपद्यते}
{मद्भक्तिविमुखानां हि शास्त्रगर्तेषु मुह्यताम्}
{न ज्ञानं न च मोक्षः स्यात्तेषां जन्मशतैरपि} %1-51

\fourlineindentedshloka
{इदं रहस्यं हृदयं ममाऽऽत्मनो}
{मयैव साक्षात्कथितं तवानघ}
{मद्भक्तिहीनाय शठाय न त्वया}
{दातव्यमैन्द्रादपि राज्यतोऽधिकम्} %1-52

\uvacha{श्री-महादेव उवाच}

\twolineshloka
{एतत्तेऽभिहितं देवि श्रीरामहृदयं मया}
{अतिगुह्यतमं हृद्यं पवित्रं पापशोधनम्} %1-53

\twolineshloka
{साक्षाद्रामेण कथितं सर्ववेदान्तसङ्ग्रहम्}
{यः पठेत्सततं भक्त्या स मुक्तो नात्र संशयः} %1-54

\twolineshloka
{ब्रह्महत्यादि पापानि बहुजन्मार्जितान्यपि}
{नश्यन्त्येव न सन्देहो रामस्य वचनं यथा} %1-55

\fourlineindentedshloka
{योऽतिभ्रष्टोऽतिपापी परधनपरदारेषु नित्योद्यतो वा}
{स्तेयी ब्रह्मघ्नमातापितृवधनिरतो योगिवृन्दापकारी}
{यः सम्पूज्याभिरामं पठति च हृदयं रामचन्द्रस्य भक्त्या}
{योगीन्द्रैरप्यलभ्यं पदमिह लभते सर्वदेवैः सुपूज्यम्} %1-56

{॥इति श्रीमदध्यात्मरामायणे उमामहेश्वरसंवादे बालकाण्डे
श्रीरामहृदयं नाम प्रथमः सर्गः॥}
