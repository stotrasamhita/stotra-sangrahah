% !TeX program = XeLaTeX
% !TeX root = ../../shloka.tex

\sect{नारद-राम-संवादः}

\uvacha{महादेव उवाच}

\twolineshloka
{एकदा सुखमासीनं रामं स्वान्तःपुराजिरे}
{सर्वाभरणसम्पन्नं रत्नसिंहासने स्थितम्} %1-1

\twolineshloka
{नीलोत्पलदलश्यामं कौस्तुभामुक्तकन्धरम्}
{सीतया रत्नदण्डेन चामरेणाथ वीजितम्} %1-2

\twolineshloka
{विनोदयन्तं ताम्बूलचर्वणादिभिरादरात्}
{नारदोऽवतरद्द्रष्टुमम्बराद्यत्र राघवः} %1-3

\twolineshloka
{शुद्धस्फटिकसङ्काशः शरच्चन्द्र इवामलः}
{अतर्कितमुपायातो नारदो दिव्यदर्शनः} %1-4

\twolineshloka
{तं दृष्ट्वा सहसोत्थाय रामः प्रीत्या कृताञ्जलिः}
{ननाम शिरसा भूमौ सीतया सह भक्तिमान्} %1-5

\threelineshloka
{उवाच नारदं रामः प्रीत्या परमया युतः}
{संसारिणां मुनिश्रेष्ठ दुर्लभं तव दर्शनम्}
{अस्माकं विषयासक्तचेतसां नितरां मुने} %1-6

\twolineshloka
{अवाप्तं मे पूर्वजन्मकृतपुण्यमहोदयैः}
{संसारिणाऽपि हि मुने लभ्यते सत्समागमः} %1-7

\twolineshloka
{अतस्त्वद्दर्शनादेव कृतार्थोऽस्मि मुनीश्वर}
{किं कार्यं ते मया कार्यं ब्रूहि तत्करवाणि भोः} %1-8

\twolineshloka
{अथ तं नारदोऽप्याह राघवं भक्तवत्सलम्}
{किं मोहयसि मां राम वाक्यैर्लोकानुसारिभिः} %1-9

\twolineshloka
{संसार्यहमिति प्रोक्तं सत्यमेतत्त्वया विभो}
{जगतामादिभूता या सा माया गृहिणी तव} %1-10

\twolineshloka
{त्वत्सन्निकर्षाज्जायन्ते तस्यां ब्रह्मादयः प्रजाः}
{त्वदाश्रया सदा भाति माया या त्रिगुणात्मिका} %1-11

\twolineshloka
{सूतेऽजस्रं शुक्लकृष्णलोहिताः सर्वदा प्रजाः}
{लोकत्रयमहागेहे गृहस्थस्त्वमुदाहृतः} %1-12

\twolineshloka
{त्वं विष्णुर्जानकी लक्ष्मीः शिवस्त्वं जानकी शिवा}
{ब्रह्मा त्वं जानकी वाणी सूर्यस्त्वं जानकी प्रभा} %1-13

\twolineshloka
{भवान् शशाङ्कः सीता तु रोहिणी शुभलक्षणा}
{शक्रस्त्वमेव पौलोमी सीता स्वाहानलो भवान्} %1-14

\twolineshloka
{यमस्त्वं कालरूपश्च सीता संयमिनी प्रभो}
{निरृतिस्त्वं जगन्नाथ तामसी जानकी शुभा} %1-15

\twolineshloka
{राम त्वमेव वरुणो भार्गवी जानकी शुभा}
{वायुस्त्वं राम सीता तु सदागतिरितीरिता} %1-16

\twolineshloka
{कुबेरस्त्वं राम सीता सर्वसम्पत्प्रकीर्तिता}
{रुद्राणी जानकी प्रोक्ता रुद्रस्त्वं लोकनाशकृत्} %1-17

\twolineshloka
{लोके स्त्रीवाचकं यावत्तत्सर्वं जानकी शुभा}
{पुन्नामवाचकं यावत्तत्सर्वं त्वं हि राघव} %1-18

\onelineshloka
{तस्माल्लोकत्रये देव युवाभ्यां नास्ति किञ्चन} %1-19

\twolineshloka
{त्वदाभासोदिताज्ञानमव्याकृतमितीर्यते}
{तस्मान्महान्स्ततः सूत्रं लिङ्गं सर्वात्मकं ततः} %1-20

\twolineshloka
{अहङ्कारश्च बुद्धिश्च पञ्चप्राणेन्द्रियाणि च}
{लिङ्गमित्युच्यते प्राज्ञैर्जन्ममृत्युसुखादिमत्} %1-21

\twolineshloka
{स एव जीवसंज्ञश्च लोके भाति जगन्मयः}
{अवाच्यानाद्यविद्यैव कारणोपाधिरुच्यते} %1-22

\twolineshloka
{स्थूलं सूक्ष्मं कारणाख्यमुपाधित्रितयं चितेः}
{एतैर्विशिष्टो जीवः स्याद्वियुक्तः परमेश्वरः} %1-23

\twolineshloka
{जाग्रत्स्वप्नसुषुप्त्याख्या संसृतिर्या प्रवर्तते}
{तस्या विलक्षणः साक्षी चिन्मात्रस्त्वं रघूत्तम} %1-24

\twolineshloka
{त्वत्त एव जगज्जातं त्वयि सर्वं प्रतिष्ठितम्}
{त्वय्येव लीयते कृत्स्नं तस्मात्त्वं सर्वकारणम्} %1-25

\twolineshloka
{रज्जावहिमिवात्मानं जीवं ज्ञात्वा भयं भवेत्}
{परात्माहमिति ज्ञात्वा भयदुःखैर्विमुच्यते} %1-26

\twolineshloka
{चिन्मात्रज्योतिषा सर्वाः सर्वदेहेषु बुद्धयः}
{त्वया यस्मात्प्रकाश्यन्ते सर्वस्यात्मा ततो भवान्} %1-27

\onelineshloka
{अज्ञानान्न्यस्यते सर्वं त्वयि रज्जौ भुजङ्गवत्} %1-28

\twolineshloka
{त्वत्पादभक्तियुक्तानां विज्ञानं भवति क्रमात्}
{तस्मात्त्वद्भक्तियुक्ता ये मुक्तिभाजस्त एव हि} %1-29

\twolineshloka
{अहं त्वद्भक्तभक्तानां तद्भक्तानां च किङ्करः}
{अतो मामनुगृह्णीष्व मोहयस्व न मां प्रभो} %1-30

\twolineshloka
{त्वन्नाभिकमलोत्पन्नो ब्रह्मा मे जनकः प्रभो}
{अतस्तवाहं पौत्रोऽस्मि भक्तं मां पाहि राघव} %1-31

\twolineshloka
{इत्युक्त्वा बहुशो नत्वा स्वानन्दाश्रुपरिप्लुतः}
{उवाच वचनं राम ब्रह्मणा नोदितोऽस्म्यहम्} %1-32

\twolineshloka
{रावणस्य वधार्थाय जातोऽसि रघुसत्तम}
{इदानीं राज्यरक्षार्थं पिता त्वामभिषेक्ष्यति} %1-33

\twolineshloka
{यदि राज्याभिसंसक्तो रावणं न हनिष्यसि}
{प्रतिज्ञा ते कृता राम भूभारहरणाय वै} %1-34

\twolineshloka
{तत्सत्यं कुरु राजेन्द्र सत्यसन्धस्त्वमेव हि}
{श्रुत्वैतद्गदितं रामो नारदं प्राह सस्मितम्} %1-35

\twolineshloka
{शृणु नारद मे किञ्चिद्विद्यतेऽविदितं क्वचित्}
{प्रतिज्ञातं च यत्पूर्वं करिष्ये तन्न संशयः} %1-36

\twolineshloka
{किन्तु कालानुरोधेन तत्तत्प्रारब्धसङ्क्षयात्}
{हरिष्ये सर्वभूभारं क्रमेणासुरमण्डलम्} %1-37

\twolineshloka
{रावणस्य विनाशार्थं श्वो गन्ता दण्डकाननम्}
{चतुर्दश समास्तत्र ह्युषित्वा मुनिवेषधृक्} %1-38

\twolineshloka
{सीतामिषेण तं दुष्टं सकुलं नाशयाम्यहम्}
{एवं रामे प्रतिज्ञाते नारदः प्रमुमोद ह} %1-39

\twolineshloka
{प्रदक्षिणत्रयं कृत्वा दण्डवत्प्रणिपत्य तम्}
{अनुज्ञातश्च रामेण ययौ देवगतिं मुनिः} %1-40

\fourlineindentedshloka
{संवादं पठति शृणोति संस्मरेद्वा}
{यो नित्यं मुनिवररामयोः सभक्त्या}
{सम्प्राप्नोत्यमरसुदुर्लभं विमोक्षम्}
{कैवल्यं विरतिपुरःसरं क्रमेण} %1-41

{॥इति श्रीमदध्यात्मरामायणे उमामहेश्वरसंवादे
अयोध्याकाण्डे प्रथमः सर्गे  नारद-राम-संवादः  सम्पूर्णः॥}
