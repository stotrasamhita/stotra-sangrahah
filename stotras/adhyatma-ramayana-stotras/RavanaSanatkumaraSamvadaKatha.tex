% !TeX program = XeLaTeX
% !TeX root = ../../shloka.tex

\sect{रावणसनत्कुमार-संवाद-कथा}

\uvacha{अगस्त्य उवाच}
\addtocounter{shlokacount}{28}

\twolineshloka
{अथान्यां सम्प्रवक्ष्यामि कथां राम त्वदाश्रयाम्}
{सीता हृता यदर्थं सा रावणेन दुरात्मना} %3-29

\threelineshloka
{पुरा कृतयुगे राम प्रजापतिसुतं विभुम्}
{सनत्कुमारमेकान्ते समासीनं दशाननः}
{विनयावनतो भूत्वा ह्यभिवाद्येदमब्रवीत्} %3-30

\twolineshloka
{को न्वस्मिन् प्रवरो लोके देवानां बलवत्तरः}
{देवाश्च यं समाश्रित्य युद्धे शत्रुं जयन्ति हि} %3-31

\twolineshloka
{कं यजन्ति द्विजा नित्यं कं ध्यायन्ति च योगिनः}
{एतन्मे शंस भगवन् प्रश्नं प्रश्नविदां वर} %3-32

\twolineshloka
{ज्ञात्वा तस्य हृदिस्थं यत्तदशेषेण योगदृक्}
{दशाननमुवाचेदं शृणु वक्ष्यामि पुत्रक} %3-33

\twolineshloka
{भर्ता यो जगतां नित्यं यस्य जन्मादिकं न हि}
{सुरासुरैर्नुतो नित्यं हरिर्नारायणोऽव्ययः} %3-34

\twolineshloka
{यन्नाभिपङ्कजाज्जातो ब्रह्मा विश्वसृजां पतिः}
{सृष्टं येनैव सकलं जगत्स्थावरजङ्गमम्} %3-35

\twolineshloka
{तं समाश्रित्य विबुधा जयन्ति समरे रिपून्}
{योगिनो ध्यानयोगेन तमेवानुजपन्ति हि} %3-36

\twolineshloka
{महर्षेर्वचनं श्रुत्वा प्रत्युवाच दशाननः}
{दैत्यदानवरक्षांसि विष्णुना निहतानि च} %3-37

\twolineshloka
{कां वा गतिं प्रपद्यन्ते प्रेत्य ते मुनिपुङ्गव}
{तमुवाच मुनिश्रेष्ठो रावणं राक्षसाधिपम्} %3-38

\twolineshloka
{दैवतैर्निहता नित्यं गत्वा स्वर्गमनुत्तमम्}
{भोगक्षये पुनस्तस्माद्भ्रष्टा भूमौ भवन्ति ते} %3-39

\twolineshloka
{पूर्वार्जितैः पुण्यपापैर्म्रियन्ते चोद्भवन्ति च}
{विष्णुना ये हतास्ते तु प्राप्नुवन्ति हरेर्गतिम्} %3-40

\twolineshloka
{श्रुत्वा मुनिमुखात्सर्वं रावणो हृष्टमानसः}
{योत्स्येऽहं हरिणा सार्धमिति चिन्तापरोऽभवत्} %3-41

\twolineshloka
{मनःस्थितं परिज्ञाय रावणस्य महामुनिः}
{उवाच वत्स तेऽभीष्टं भविष्यति न संशयः} %3-42

\twolineshloka
{कञ्चित्कालं प्रतीक्षस्व सुखी भव दशानन}
{एवमुक्त्वा महाबाहो मुनिः पुनरुवाच तम्} %3-43

\twolineshloka
{तस्य स्वरूपं वक्ष्यामि ह्यरूपस्यापि मायिनः}
{स्थावरेषु च सर्वेषु नदेषु च नदीषु च} %3-44

\twolineshloka
{ओङ्कारश्चैव सत्यं च सावित्री पृथिवी च सः}
{समस्तजगदाधारः शेषरूपधरो हि सः} %3-45

\twolineshloka
{सर्वे देवाः समुद्राश्च कालः सूर्यश्च चन्द्रमाः}
{सूर्योदयो दिवारात्री यमश्चैव तथाऽनिलः} %3-46

\twolineshloka
{अग्निरिन्द्रस्तथा मृत्युः पर्जन्यो वसवस्तथा}
{ब्रह्मा रुद्रादयश्चैव ये चान्ये देवदानवाः} %3-47

\twolineshloka
{विद्योतते ज्वलत्येष पाति चात्तीति विश्वकृत्}
{क्रीडां करोत्यव्ययात्मा सोऽयं विष्णुः सनातनः} %3-48

\twolineshloka
{तेन सर्वमिदं व्याप्तं त्रैलोक्यं सचराचरम्}
{नीलोत्पलदलश्यामो विद्युद्वर्णाम्बरावृतः} %3-49

\twolineshloka
{शुद्धजाम्बूनदप्रख्यां श्रियं वामाङ्कसंस्थिताम्}
{सदानपायिनीं देवीं पश्यन्नालिङ्ग्य तिष्ठति} %3-50

\twolineshloka
{द्रष्टुं न शक्यते कैश्चिद्देवदानवपन्नगैः}
{यस्य प्रसादं कुरुते स चैनं द्रष्टुमर्हति} %3-51

\twolineshloka
{न च यज्ञतपोभिर्वा न दानाध्ययनादिभिः}
{शक्यते भगवान् द्रष्टुमुपायैरितरैरपि} %3-52

\twolineshloka
{तद्भक्तैस्तद्गतप्राणैस्तच्चित्तैर्धूतकल्मषैः}
{शक्यते भगवान् विष्णुर्वेदान्तामलदृष्टिभिः} %3-53

\twolineshloka
{अथवा द्रष्टुमिच्छा ते शृणु त्वं परमेश्वरम्}
{त्रेतायुगे स देवेशो भविता नृपविग्रहः} %3-54

\twolineshloka
{हितार्थं देवमर्त्यानामिक्ष्वाकूणां कुले हरिः}
{रामो दाशरथिर्भूत्वा महासत्त्वपराक्रमः} %3-55

\twolineshloka
{पितुर्नियोगात्स भ्रात्रा भार्यया दण्डके वने}
{विचरिष्यति धर्मात्मा जगन्मात्रा स्वमायया} %3-56

\twolineshloka
{एवं ते सर्वमाख्यातं मया रावण विस्तरात्}
{भजस्व भक्तिभावेन सदा रामं श्रिया युतम्} %3-57

\uvacha{अगस्त्य उवाच}

\twolineshloka
{एवं श्रुत्वाऽसुराध्यक्षो ध्यात्वा किञ्चिद्विचार्य च}
{त्वया सह विरोधेप्सुर्मुमुदे रावणो महान्} %3-58

\threelineshloka
{युद्धार्थी सर्वतो लोकान् पर्यटन् समवस्थितः}
{एतदर्थं महाराज रावणोऽतीव बुद्धिमान्}
{हृतवान् जानकीं देवीं त्वयाऽऽत्मवधकाङ्क्षया} %3-59

\fourlineindentedshloka
{इमां कथां यः शृणुयात्पठेद्वा}
{संश्रावयेद्वा श्रवणार्थिनां सदा}
{आयुष्यमारोग्यमनन्तसौख्यं}
{प्राप्नोति लाभं धनमक्षयं च} %3-60


{॥इति श्रीमदध्यात्मरामायणे उमामहेश्वरसंवादे उत्तरकाण्डे तृतीये  सर्गे
रावणसनत्कुमार-संवाद-कथा सम्पूर्णा॥}
