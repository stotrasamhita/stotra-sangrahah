% !TeX program = XeLaTeX
% !TeX root = ../../shloka.tex

\sect{रामावताररहस्यम्}


\uvacha{श्री- महादेव उवाच}

{सर्वज्ञो नित्यलक्ष्मीको विज्ञानात्माऽपि राघवः॥१६॥} %5-16

\addtocounter{shlokacount}{17}

\twolineshloka
{सीतामनुशुशोचार्त्तः प्राकृतः प्राकृतामिव}
{बुद्ध्यादिसाक्षिणस्तस्य मायाकार्यातिवर्तिनः} %5-17

\twolineshloka
{रागादिरहितस्यास्य तत्कार्यं कथमुद्भवेत्}
{ब्रह्मणोक्तमृतं कर्तुं राज्ञो दशरथस्य हि} %5-18

\twolineshloka
{तपसः फलदानाय जातो मानुषवेषधृक्}
{मायया मोहिताः सर्वे जना अज्ञानसंयुताः} %5-19

\twolineshloka
{कथमेषां भवेन्मोक्ष इति विष्णुर्विचिन्तयन्}
{कथां प्रथयितुं लोके सर्वलोकमलापहाम्} %5-20

\twolineshloka
{रामायणाभिधां रामो भूत्वा मानुषचेष्टकः}
{क्रोधं मोहं च कामं च व्यवहारार्थसिद्धये} %5-21

\twolineshloka
{तत्तत्कालोचितं गृह्णन् मोहयत्यवशाः प्रजाः}
{अनुरक्त इवाशेषगुणेषु गुणवर्जितः} %5-22

\twolineshloka
{विज्ञानमूर्तिर्विज्ञानशक्तिः साक्ष्यगुणान्वितः}
{अतः कामादिभिर्नित्यमविलिप्तो यथा नभः} %5-23

\threelineshloka
{विन्दन्ति मुनयः केचिज्जानन्ति जनकादयः}
{तद्भक्ता निर्मलात्मानः सम्यग्जानन्ति नित्यदा}
{भक्तचित्तानुसारेण जायते भगवानजः} %5-24

{॥इति श्रीमदध्यात्मरामयणे उमामहेश्वरसंवादे
कीष्किन्धाकाण्डे पञ्चमे सर्गे  रामावताररहस्यं सम्पूर्णम्॥}
