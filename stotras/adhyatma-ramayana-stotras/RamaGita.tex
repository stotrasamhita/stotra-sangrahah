% !TeX program = XeLaTeX
% !TeX root = ../../shloka.tex

\sect{रामगीता}

% \addtocounter{shlokacount}{28}
\uvacha{श्री-महादेव उवाच}

\fourlineindentedshloka
{ततो जगन्मङ्गलमङ्गलात्मना}
{विधाय रामायणकीर्तिमुत्तमाम्}
{चचार पूर्वाचरितं रघूत्तमो}
{राजर्षिवर्यैरभिसेवितं यथा} %5-1

\fourlineindentedshloka
{सौमित्रिणा पृष्ट उदारबुद्धिना}
{रामः कथाः प्राह पुरातनीः शुभाः}
{राज्ञः प्रमत्तस्य नृगस्य शापतो}
{द्विजस्य तिर्यक्त्वमथाह राघवः} %5-2

\fourlineindentedshloka
{कदाचिदेकान्त उपस्थितं प्रभुम्}
{रामं रमालालितपादपङ्कजम्}
{सौमित्रिरासादितशुद्धभावनः}
{प्रणम्य भक्त्या विनयान्वितोऽब्रवीत्} %5-3

\fourlineindentedshloka
{त्वं शुद्धबोधोऽसि हि सर्वदेहिनाम्}
{आत्मास्यधीशोऽसि निराकृतिः स्वयम्}
{प्रतीयसे ज्ञानदृशां महामते}
{पादाब्जभृङ्गाहितसङ्गसङ्गिनाम्} %5-4

\fourlineindentedshloka
{अहं प्रपन्नोऽस्मि पदाम्बुजं प्रभो}
{भवापवर्गं तव योगिभावितम्}
{यथाञ्जसाऽज्ञानमपारवारिधिम्}
{सुखं तरिष्यामि तथाऽनुशाधि माम्} %5-5

\fourlineindentedshloka
{श्रुत्वाऽथ सौमित्रिवचोऽखिलं तदा}
{प्राह प्रपन्नार्तिहरः प्रसन्नधीः}
{विज्ञानमज्ञानतमःप्रशान्तये}
{श्रुतिप्रपन्नं क्षितिपालभूषणः} %5-6

\fourlineindentedshloka
{आदौ स्ववर्णाश्रमवर्णिताः क्रियाः}
{कृत्वा समासादितशुद्धमानसः}
{समाप्य तत्पूर्वमुपात्तसाधनः}
{समाश्रयेत्सद्गुरुमात्मलब्धये} %5-7

\fourlineindentedshloka
{क्रिया शरीरोद्भवहेतुरादृता}
{प्रियाप्रियौ तौ भवतः सुरागिणः}
{धर्मेतरौ तत्र पुनः शरीरकम्}
{पुनः क्रिया चक्रवदीर्यते भवः} %5-8

\fourlineindentedshloka
{अज्ञानमेवास्य हि मूलकारणम्}
{तद्ध्यानमेवात्र विधौ विधीयते}
{विद्यैव तन्नाशविधौ पटीयसी}
{न कर्म तज्जं सविरोधमीरितम्} %5-9

\fourlineindentedshloka
{नाज्ञानहानिर्न च रागसङ्क्षयो}
{भवेत्ततः कर्म सदोषमुद्भवेत्}
{ततः पुनः संसृतिरप्यवारिता}
{तस्माद्बुधो ज्ञानविचारवान् भवेत्} %5-10

\fourlineindentedshloka
{ननु क्रिया वेदमुखेन चोदिता}
{तथैव विद्या पुरुषार्थसाधनम्}
{कर्तव्यता प्राणभृतः प्रचोदिता}
{विद्यासहायत्वमुपैति सा पुनः} %5-11

\fourlineindentedshloka
{कर्माकृतौ दोषमपि श्रुतिर्जगौ}
{तस्मात्सदा कार्यमिदं मुमुक्षुणा}
{ननु स्वतन्त्रा ध्रुवकार्यकारिणी}
{विद्या न किञ्चिन्मनसाऽप्यपेक्षते} %5-12

\fourlineindentedshloka
{न सत्यकार्योऽपि हि यद्वदध्वरः}
{प्रकाङ्क्षतेऽन्यानपि कारकादिकान्}
{तथैव विद्या विधितः प्रकाशितैः}
{विशिष्यते कर्मभिरेव मुक्तये} %5-13

\fourlineindentedshloka
{केचिद्वदन्तीति वितर्कवादिन-}
{स्तदप्यसद्दृष्टविरोधकारणात्}
{देहाभिमानादभिवर्धते क्रिया}
{विद्या गताहङ्कृतितः प्रसिद्ध्यति} %5-14

\fourlineindentedshloka
{विशुद्धविज्ञानविरोचनाञ्चिता}
{विद्यात्मवृत्तिश्चरमेति भण्यते}
{उदेति कर्माखिलकारकादिभिः}
{निहन्ति विद्याखिलकारकादिकम्} %5-15

\fourlineindentedshloka
{तस्मात्त्यजेत्कार्यमशेषतः सुधीः}
{विद्याविरोधान्न समुच्चयो भवेत्}
{आत्मानुसन्धानपरायणः सदा}
{निवृत्तसर्वेन्द्रियवृत्तिगोचरः} %5-16

\fourlineindentedshloka
{यावच्छरीरादिषु माययाऽऽत्मधी-}
{स्तावद्विधेयो विधिवादकर्मणाम्}
{नेतीति वाक्यैरखिलं निषिध्य तत्}
{ज्ञात्वा परात्मानमथ त्यजेत्क्रियाः} %5-17

\fourlineindentedshloka
{यदा परात्मात्मविभेदभेदकम्}
{विज्ञानमात्मन्यवभाति भास्वरम्}
{तदैव माया प्रविलीयतेऽञ्जसा}
{सकारका कारणमात्मसंसृतेः} %5-18

\fourlineindentedshloka
{श्रुतिप्रमाणाभिविनाशिता च सा}
{कथं भविष्यत्यपि कार्यकारिणी}
{विज्ञानमात्रादमलाद्वितीयत-}
{स्तस्मादविद्या न पुनर्भविष्यति} %5-19

\fourlineindentedshloka
{यदि स्म नष्टा न पुनः प्रसूयते}
{कर्ताहमस्येति मतिः कथं भवेत्}
{तस्मात्स्वतन्त्रा न किमप्यपेक्षते}
{विद्या विमोक्षाय विभाति केवला} %5-20

\fourlineindentedshloka
{सा तैत्तिरीयश्रुतिराह सादरम्}
{न्यासं प्रशस्ताखिलकर्मणां स्फुटम्}
{एतावदित्याह च वाजिनां श्रुतिः}
{ज्ञानं विमोक्षाय न कर्म साधनम्} %5-21

\fourlineindentedshloka
{विद्यासमत्वेन तु दर्शितस्त्वया}
{क्रतुर्न दृष्टान्त उदाहृतः समः}
{फलैः पृथक्त्वाद्बहुकारकैः क्रतुः}
{संसाध्यते ज्ञानमतो विपर्ययम्} %5-22

\fourlineindentedshloka
{सप्रत्यवायो ह्यहमित्यनात्मधी-}
{रज्ञप्रसिद्धा न तु तत्त्वदर्शिनः}
{तस्माद्बुधैस्त्याज्यमविक्रियात्मभिः}
{विधानतः कर्म विधिप्रकाशितम्} %5-23

\fourlineindentedshloka
{श्रद्धान्वितस्तत्त्वमसीति वाक्यतो}
{गुरोः प्रसादादपि शुद्धमानसः}
{विज्ञाय चैकात्म्यमथाऽऽत्मजीवयोः}
{सुखी भवेन्मेरुरिवाप्रकम्पनः} %5-24

\fourlineindentedshloka
{आदौ पदार्थावगतिर्हि कारणम्}
{वाक्यार्थविज्ञानविधौ विधानतः}
{तत्त्वम्पदार्थौ परमात्मजीवका-}
{वसीति चैकात्म्यमथानयोर्भवेत्} %5-25

\fourlineindentedshloka
{प्रत्यक्परोक्षादिविरोधमात्मनोः}
{विहाय सङ्गृह्य तयोश्चिदात्मताम्}
{संशोधितां लक्षणया च लक्षिताम्}
{ज्ञात्वा स्वमात्मानमथाद्वयो भवेत्} %5-26

\fourlineindentedshloka
{एकात्मकत्वाज्जहती न सम्भवेत्}
{तथाऽजहल्लक्षणता विरोधतः}
{सोऽयम्पदार्थाविव भागलक्षणा}
{युज्येत तत्त्वम्पदयोरदोषतः} %5-27

\fourlineindentedshloka
{रसादिपञ्चीकृतभूतसम्भवम्}
{भोगालयं दुःखसुखादिकर्मणाम्}
{शरीरमाद्यन्तवदादिकर्मजम्}
{मायामयं स्थूलमुपाधिमात्मनः} %5-28

\fourlineindentedshloka
{सूक्ष्मं मनोबुद्धिदशेन्द्रियैर्युतम्}
{प्राणैरपञ्चीकृतभूतसम्भवम्}
{भोक्तुः सुखादेरनुसाधनं भवेत्}
{शरीरमन्यद्विदुरात्मनो बुधाः} %5-29

\fourlineindentedshloka
{अनाद्यनिर्वाच्यमपीह कारणम्}
{मायाप्रधानं तु परं शरीरकम्}
{उपाधिभेदात्तु यतः पृथक् स्थितम्}
{स्वात्मानमात्मन्यवधारयेत्क्रमात्} %5-30

\fourlineindentedshloka
{कोशेष्वयं तेषु तु तत्तदाकृतिः}
{विभाति सङ्गात् स्फटिकोपलो यथा}
{असङ्गरूपोऽयमजो यतोऽद्वयो}
{विज्ञायतेऽस्मिन् परितो विचारिते} %5-31

\fourlineindentedshloka
{बुद्धेस्त्रिधा वृत्तिरपीह दृश्यते}
{स्वप्नादिभेदेन गुणत्रयात्मनः}
{अन्योन्यतोऽस्मिन् व्यभिचारतो मृषा}
{नित्ये परे ब्रह्मणि केवले शिवे} %5-32

\fourlineindentedshloka
{देहेन्द्रियप्राणमनश्चिदात्मनाम्}
{सङ्घादजस्रं परिवर्तते धियः}
{वृत्तिस्तमोमूलतयाज्ञलक्षणा}
{यावद्भवेत्तावदसौ भवोद्भवः} %5-33

\fourlineindentedshloka
{नेतिप्रमाणेन निराकृताखिलो}
{हृदा समास्वादितचिद्घनामृतः}
{त्यजेदशेषं जगदात्तसद्रसम्}
{पीत्वा यथाम्भः प्रजहाति तत्फलम्} %5-34

\fourlineindentedshloka
{कदाचिदात्मा न मृतो न जायते}
{न क्षीयते नापि विवर्धतेऽनवः}
{निरस्तसर्वातिशयः सुखात्मकः}
{स्वयम्प्रभः सर्वगतोऽयमद्वयः} %5-35

\fourlineindentedshloka
{एवंविधे ज्ञानमये सुखात्मके}
{कथं भवो दुःखमयः प्रतीयते}
{अज्ञानतोऽध्यासवशात्प्रकाशते}
{ज्ञाने विलीयेत विरोधतः क्षणात्} %5-36

\fourlineindentedshloka
{यदन्यदन्यत्र विभाव्यते भ्रमा-}
{दध्यासमित्याहुरमुं विपश्चितः}
{असर्पभूतेऽहिविभावनं यथा}
{रज्ज्वादिके तद्वदपीश्वरे जगत्} %5-37

\fourlineindentedshloka
{विकल्पमायारहिते चिदात्मके-}
{ऽहङ्कार एष प्रथमः प्रकल्पितः}
{अध्यास एवात्मनि सर्वकारणे}
{निरामये ब्रह्मणि केवले परे} %5-38

\fourlineindentedshloka
{इच्छादिरागादिसुखादिधर्मिकाः}
{सदा धियः संसृतिहेतवः परे}
{यस्मात्प्रसुप्तौ तदभावतः परः}
{सुखस्वरूपेण विभाव्यते हि नः} %5-39

\fourlineindentedshloka
{अनाद्यविद्योद्भवबुद्धिबिम्बितो}
{जीवः प्रकाशोऽयमितीर्यते चितः}
{आत्मा धियः साक्षितया पृथक् स्थितो}
{बुद्ध्यापरिच्छिन्नपरः स एव हि} %5-40

\fourlineindentedshloka
{चिद्बिम्बसाक्ष्यात्मधियां प्रसङ्गत-}
{स्त्वेकत्र वासादनलाक्तलोहवत्}
{अन्योन्यमध्यासवशात्प्रतीयते}
{जडाजडत्वं च चिदात्मचेतसोः} %5-41

\fourlineindentedshloka
{गुरोः सकाशादपि वेदवाक्यतः}
{सञ्जातविद्यानुभवो निरीक्ष्य तम्}
{स्वात्मानमात्मस्थमुपाधिवर्जितम्}
{त्यजेदशेषं जडमात्मगोचरम्} %5-42

\fourlineindentedshloka
{प्रकाशरूपोऽहमजोऽहमद्वयो-}
{ऽसकृद्विभातोऽहमतीव निर्मलः}
{विशुद्ध विज्ञानघनो निरामयः}
{सम्पूर्ण आनन्दमयोऽहमक्रियः} %5-43

\fourlineindentedshloka
{सदैव मुक्तोऽहमचिन्त्यशक्तिमान्}
{अतीन्द्रियज्ञानमविक्रियात्मकः}
{अनन्तपारोऽहमहर्निशं बुधैः}
{विभावितोऽहं हृदि वेदवादिभिः} %5-44

\fourlineindentedshloka
{एवं सदात्मानमखण्डितात्मना}
{विचारमाणस्य विशुद्धभावना}
{हन्यादविद्यामचिरेण कारकै}
{रसायनं यद्वदुपासितं रुजः} %5-45

\fourlineindentedshloka
{विविक्त आसीन उपारतेन्द्रियो}
{विनिर्जितात्मा विमलान्तराशयः}
{विभावयेदेकमनन्यसाधनो}
{विज्ञानदृक्केवल आत्मसंस्थितः} %5-46

\fourlineindentedshloka
{विश्वं यदेतत्परमात्मदर्शनम्}
{विलापयेदात्मनि सर्वकारणे}
{पूर्णश्चिदानन्दमयोऽवतिष्ठते}
{न वेद बाह्यं न च किञ्चिदान्तरम्} %5-47

\fourlineindentedshloka
{पूर्वं समाधेरखिलं विचिन्तये-}
{दोङ्कारमात्रं सचराचरं जगत्}
{तदेव वाच्यं प्रणवो हि वाचको}
{विभाव्यतेऽज्ञानवशान्न बोधतः} %5-48

\fourlineindentedshloka
{अकारसंज्ञः पुरुषो हि विश्वको}
{ह्युकारकस्तैजस ईर्यते क्रमात्}
{प्राज्ञो मकारः परिपठ्यतेऽखिलैः}
{समाधिपूर्वं न तु तत्त्वतो भवेत्} %5-49

\fourlineindentedshloka
{विश्वं त्वकारं पुरुषं विलापये-}
{दुकारमध्ये बहुधा व्यवस्थितम्}
{ततो मकारे प्रविलाप्य तैजसम्}
{द्वितीयवर्णं प्रणवस्य चान्तिमे} %5-50

\fourlineindentedshloka
{मकारमप्यात्मनि चिद्घने परे}
{विलापयेद्प्राज्ञमपीह कारणम्}
{सोऽहं परं ब्रह्म सदा विमुक्तिमद्-}
{विज्ञानदृङ्मुक्त उपाधितोऽमलः} %5-51

\fourlineindentedshloka
{एवं सदा जातपरात्मभावनः}
{स्वानन्दतुष्टः परिविस्मृताखिलः}
{आस्ते स नित्यात्मसुखप्रकाशकः}
{साक्षाद्विमुक्तोऽचलवारिसिन्धुवत्} %5-52

\fourlineindentedshloka
{एवं सदाभ्यस्तसमाधियोगिनो}
{निवृत्तसर्वेन्द्रियगोचरस्य हि}
{विनिर्जिताशेषरिपोरहं सदा}
{दृश्यो भवेयं जितषड्गुणात्मनः} %5-53

\fourlineindentedshloka
{ध्यात्वैवमात्मानमहर्निशं मुनि-}
{स्तिष्ठेत्सदा मुक्तसमस्तबन्धनः}
{प्रारब्धमश्नन्नभिमानवर्जितो}
{मय्येव साक्षात्प्रविलीयते ततः} %5-54

\fourlineindentedshloka
{आदौ च मध्ये च तथैव चान्ततो}
{भवं विदित्वा भयशोककारणम्}
{हित्वा समस्तं विधिवादचोदितम्}
{भजेत्स्वमात्मानमथाखिलात्मनाम्} %5-55

\fourlineindentedshloka
{आत्मन्यभेदेन विभावयन्निदम्}
{भवत्यभेदेन मयाऽऽत्मना तदा}
{यथा जलं वारिनिधौ यथा पयः}
{क्षीरे वियद्व्योम्न्यनिले यथाऽनिलः} %5-56

\fourlineindentedshloka
{इत्थं यदीक्षेत हि लोकसंस्थितो}
{जगन्मृषैवेति विभावयन्मुनिः}
{निराकृतत्वाच्छ्रुतियुक्तिमानतो}
{यथेन्दुभेदो दिशि दिग्भ्रमादयः} %5-57

\fourlineindentedshloka
{यावन्न पश्येदखिलं मदात्मकम्}
{तावन्मदाराधनतत्परो भवेत्}
{श्रद्धालुरत्यूर्जितभक्तिलक्षणो}
{यस्तस्य दृश्योऽहमहर्निशं हृदि} %5-58

\fourlineindentedshloka
{रहस्यमेतच्छ्रुतिसारसङ्ग्रहम्}
{मया विनिश्चित्य तवोदितं प्रिय}
{यस्त्वेतदालोचयतीह बुद्धिमान्}
{स मुच्यते पातकराशिभिः क्षणात्} %5-59

\fourlineindentedshloka
{भ्रातर्यदीदं परिदृश्यते जगन्-}
{मायैव सर्वं परिहृत्य चेतसा}
{मद्भावनाभावितशुद्धमानसः}
{सुखी भवानन्दमयो निरामयः} %5-60

\fourlineindentedshloka
{यः सेवते मामगुणं गुणात्परम्}
{हृदा कदा वा यदि वा गुणात्मकम्}
{सोऽहं स्वपादाञ्चितरेणुभिः स्पृशन्}
{पुनाति लोकत्रितयं यथा रविः} %5-61

\fourlineindentedshloka
{विज्ञानमेतदखिलं श्रुतिसारमेकम्}
{वेदान्तवेद्यचरणेन मयैव गीतम्}
{यः श्रद्धया परिपठेद्गुरुभक्तियुक्तो}
{मद्रूपमेति यदि मद्वचनेषु भक्तिः} %5-62

{॥इति श्रीमदध्यात्मरामायणे उमामहेश्वरसंवादे उत्तरकाण्डे पञ्चमे  सर्गे
रामगीता सम्पूर्णा॥}
