% !TeX program = XeLaTeX
% !TeX root = ../../shloka.tex

\sect{वाणीस्तवनम्}
श्री नारायण उवाच 

\twolineshloka
{वाग्देवतायाः स्तवनं श्रूयतां सर्वकामदम्}
{महामुनिर्याज्ञवल्क्यो येन तुष्टाव तां पुरा}% ॥ १॥

\twolineshloka
{गुरुशापाच्च स मुनिर्हतविद्यो बभूव ह}
{तदा जगाम दुःखार्तो रविस्थानं च पुण्यदम्}% ॥ २॥

\twolineshloka
{सम्प्राप्य तपसा सूर्यं कोणार्के दृष्टिगोचरे}
{तुष्टाव सूर्यं शोकेन रुरोद स पुनः पुनः}% ॥ ३॥

\twolineshloka
{सूर्यस्तं पाठयामास वेदवेदाङ्गमीश्वरः}
{उवाच स्तुहि वाग्देवीं भक्त्या च स्मृतिहेतवे}% ॥ ४॥

\twolineshloka
{तमित्युक्त्वा दीननाथो ह्यन्तर्धानं जगाम सः}
{मुनिः स्नात्वा च तुष्टाव भक्तिनम्रात्मकन्धरः}% ॥ ५॥

याज्ञवल्क्य उवाच

\twolineshloka
{कृपां कुरु जगन्मातर्मामेवं हततेजसम्}
{गुरुशापात्स्मृतिभ्रष्टं विद्याहीनं च दुःखितम्}% ॥ ६॥

\twolineshloka
{ज्ञानं देहि स्मृतिं देहि विद्यां विद्याधिदेवते}
{प्रतिष्ठां कवितां देहि शाक्तं शिष्यप्रबोधिकाम्}% ॥ ७॥

\twolineshloka
{ग्रन्थनिर्मितिशक्तिं च सच्छिष्यं सुप्रतिष्ठितम्}
{प्रतिभां सत्सभायां च विचारक्षमतां शुभाम्}% ॥ ८॥

\twolineshloka
{लुप्तां सर्वां दैववशान्नवं कुरु पुनः पुनः}
{यथाऽङ्कुरं जनयति भगवान्योगमायया}% ॥ ९॥

\twolineshloka
{ब्रह्मस्वरूपा परमा ज्योतिरूपा सनातनी}
{सर्वविद्याधिदेवी या तस्यै वाण्यै नमो नमः}% ॥ १०॥

\twolineshloka
{यया विना जगत्सर्वं शश्वज्जीवन्मृतं सदा}
{ज्ञानाधिदेवी या तस्यै सरस्वत्यै नमो नमः}% ॥ ११॥

\twolineshloka
{यया विना जगत्सर्वं मूकमुन्मत्तवत्सदा}
{वागधिष्ठातृदेवी या तस्यै वाण्यै नमो नमः}% ॥ १२॥

\twolineshloka
{हिमचन्दनकुन्देन्दुकुमुदाम्भोजसन्निभा}
{वर्णाधिदेवी या तस्यै चाक्षरायै नमो नमः}% ॥ १३॥

\twolineshloka
{विसर्ग बिन्दुमात्राणां यदधिष्ठानमेव च}
{इत्थं त्वं गीयसे सद्भिर्भारत्यै ते नमो नमः}% ॥ १४॥

\twolineshloka
{यया विनाऽत्र सङ्ख्याकृत्सङ्ख्यां कर्तुं न शक्नुते}
{काल सङ्ख्यास्वरूपा या तस्यै देव्यै नमो नमः}% ॥ १५॥

\twolineshloka
{व्याख्यास्वरूपा या देवी व्याख्याधिष्ठातृदेवता}
{भ्रमसिद्धान्तरूपा या तस्यै देव्यै नमो नमः}% ॥ १६॥

\twolineshloka
{स्मृतिशक्तिर्ज्ञानशक्तिर्बुद्धिशक्तिस्वरूपिणी}
{प्रतिभा कल्पनाशक्तिर्या च तस्यै नमो नमः}% ॥ १७॥

\twolineshloka
{सनत्कुमारो ब्रह्माणं ज्ञानं पप्रच्छ यत्र वै}
{बभूव जडवत्सोऽपि सिद्धान्तं कर्तुमक्षमः}% ॥ १८॥

\twolineshloka
{तदाऽऽजगाम भगवानात्मा श्रीकृष्ण ईश्वरः}
{उवाच स च तं स्तौहि वाणीमिति प्रजापते}% ॥ १९॥

\twolineshloka
{स च तुष्टाव तां ब्रह्मा चाऽऽज्ञया परमात्मनः}
{चकार तत्प्रसादेन तदा सिद्धान्तमुत्तमम्}% ॥ २०॥

\twolineshloka
{यदाप्यनन्तं पप्रच्छ ज्ञानमेकं वसुन्धरा}
{बभूव मूकवत्सोऽपि सिद्धान्तं कर्तुमक्षमः}% ॥ २१॥

\twolineshloka
{तदा त्वां च स तुष्टाव सन्त्रस्तः कश्यपाज्ञया}
{ततश्चकार सिद्धान्तं निर्मलं भ्रमभञ्जनम्}% ॥ २२॥

\twolineshloka
{व्यासः पुराणसूत्रं समपृच्छद्वाल्मिकिं यदा}
{मौनीभूतः स सस्मार त्वामेव जगदम्बिकाम्}% ॥ २३॥

\twolineshloka
{तदा चकार सिद्धान्तं त्वद्वरेण मुनीश्वरः}
{स प्राप निर्मलं ज्ञानं प्रमादध्वंसकारणम्}% ॥ २४॥

\twolineshloka
{पुराणसूत्रं श्रुत्वा स व्यासः कृष्णकलोद्भवः}
{त्वां सिषेवे च दध्यौ तं शतवर्षं च पुष्करे}% ॥ २५॥

\twolineshloka
{तदा त्वत्तो वरं प्राप्य स कवीन्द्रो बभूव ह}
{तदा वेदविभागं च पुराणानि चकार ह}% ॥ २६॥

\twolineshloka
{यदा महेन्द्रे पप्रच्छ तत्त्वज्ञानं शिवा शिवम्}
{क्षणं त्वामेव सञ्चिन्त्य तस्यै ज्ञानं ददौ विभुः}% ॥ २७॥

\twolineshloka
{पप्रच्छ शब्दशास्त्रं च महेन्द्रश्च बृहस्पतिम्}
{दिव्यं वर्षसहस्रं च स त्वां दध्यौ च पुष्करे}% ॥ २८॥

\twolineshloka
{तदा त्वत्तो वरं प्राप्य दिव्यं वर्षसहस्रकम्}
{उवाच शब्दशास्त्रं च तदर्थं च सुरेश्वरम्}% ॥ २९॥

\twolineshloka
{अध्यापिताश्च यैः शिष्या यैरधीतं मुनीश्वरैः}
{ते च त्वां परिसञ्चिन्त्य प्रवर्तन्ते सुरेश्वरि}% ॥ ३०॥

\twolineshloka
{त्वं संस्तुता पूजिता च मुनीन्द्रमनुमानवैः}
{दैत्येन्द्रैश्च सुरैश्चापि ब्रह्मविष्णुशिवादिभिः}% ॥ ३१॥

\twolineshloka
{जडीभूतः सहस्रास्यः पञ्चवक्त्रश्चतुर्मुखः}
{यां स्तोतुं किमहं स्तौमि तामेकास्येन मानवः}% ॥ ३२॥

\twolineshloka
{इत्युक्त्वा याज्ञवल्क्यश्च भक्तिनम्रात्मकन्धरः}
{प्रणनाम निराहारो रुरोद च मुहुर्मुहुः}% ॥ ३३॥

\twolineshloka
{तदा ज्योतिः स्वरूपा सा तेनाऽदृष्टाऽप्युवाच तम्}
{सुकवीन्द्रो भवेत्युक्त्वा वैकुण्ठं च जगाम ह}% ॥ ३४॥

\twolineshloka
{महामूर्खश्च दुर्मेधा वर्षमेकं च यः पठेत्}
{स पण्डितश्च मेधावी सुकविश्च भवेद्ध्रुवम्}% ॥ ३५॥

{॥इति~श्रीब्रह्मवैवर्तमहापुराणे~प्रकृतिखण्डे श्री~नारद-नारायण-संवादे श्री~याज्ञवल्क्योक्तं श्री~वाणीस्तवनं सम्पूर्णम्॥}