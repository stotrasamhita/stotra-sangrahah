% !TeX program = XeLaTeX
% !TeX root = ../../shloka.tex

\sect{द्वादशार्यासूर्यस्तुतिः}

\twolineshloka
{उद्यन्नद्य विवस्वानारोहन्नुत्तरां दिवं देवः}
{हृद्रोगं मम सूर्यो हरिमाणं चाऽऽशु नाशयतु}% ॥ १ ॥

\twolineshloka
{निमिषार्धेनैकेन द्वे च शते द्वे सहस्रे द्वे}
{क्रममाण योजनानां नमोऽस्तु ते नलिननाथाय}% ॥ २ ॥

\twolineshloka
{कर्म-ज्ञान-ख-दशकं मनश्च जीव इति विश्व-सर्गाय}
{द्वादशधा यो विचरति स द्वादश-मूर्तिरस्तु मोदाय}% ॥ ३ ॥

\twolineshloka
{त्वं हि यजुर्ऋक् साम त्वमागमस्त्वं वषट्कारः}
{त्वं विश्वं त्वं हंसस्त्वं भानो परमहंसश्च}% ॥ ४ ॥

\twolineshloka
{शिवरूपाज्ज्ञानमहं त्वत्तो मुक्तिं जनार्दनाकारात्}
{शिखिरूपादैश्वर्यं त्वत्तश्चारोग्यमिच्छामि}% ॥ ५ ॥

\twolineshloka
{त्वचि दोषा दृशि दोषा हृदि दोषा येऽखिलेन्द्रियज-दोषाः}
{तान् पूषा हतदोषः किञ्चिद्रोषाग्निना दहतु}% ॥ ६ ॥

\twolineshloka
{धर्मार्थ-काम-मोक्ष-प्रतिरोधानुग्र-ताप-वेग-करान्}
{बन्दी-कृतेन्द्रिय-गणान् गदान् विखण्डयतु चण्डांशुः}% ॥ ७ ॥

\twolineshloka
{येन विनेदं तिमिरं जगदेत्य ग्रसति चरमचरमखिलम्}
{धृतबोधं तं नलिनीभर्तारं हर्तारमापदामीडे}% ॥ ८ ॥

\twolineshloka
{यस्य सहस्राभीशोरभीशु-लेशो हिमांशु-बिम्बगतः}
{भासयति नक्तमखिलं भेदयतु विपद्-गणानरुणः}% ॥ ९ ॥

\twolineshloka
{तिमिरमिव नेत्र-तिमिरं पटलमिवाशेष-रोग-पटलं नः}
{काशमिवाधि-निकायं कालपिता रोगयुक्ततां हरतात्}% ॥ १० ॥

\twolineshloka
{वाताश्मरी-गदार्शस्त्वग्-दोष-महोदर-प्रमेहांश्च}
{ग्रहणी-भगन्दराख्या महतीस्त्वं मे रुजो हंसि}% ॥ ११ ॥

\twolineshloka
{त्वं माता त्वं शरणं त्वं धाता त्वं धनं त्वमाचार्यः}
{त्वं त्राता त्वं हर्ता विपदामर्क प्रसीद मम भानो}% ॥ १२ ॥

\twolineshloka
{इत्यार्या-द्वादशकं साम्बस्य पुरो नभःस्थलात् पतितम्}
{पठतां भाग्य-समृद्धिः समस्त-रोग-क्षयश्च स्यात्}% ॥ १३ ॥


॥इति श्री-द्वादशार्यासूर्यस्तुतिः सम्पूर्णः॥