% !TeX program = XeLaTeX
% !TeX root = ../../shloka.tex

\sect{सूर्यार्यास्तोत्रम्}

\twolineshloka
{शुकतुण्डच्छवि-सवितुश्चण्डरुचेः पुण्डरीकवनबन्धोः}
{मण्डलमुदितं वन्दे कुण्डलमाखण्डलाशायाः}% ॥१॥

\twolineshloka
{यस्योदयास्तसमये सुरमुकुट-निघृष्ट-चरणकमलोऽपि}
{कुरुतेञ्जलिं त्रिनेत्रः स जयति धाम्नां निधिः सूर्यः}% ॥२॥

\twolineshloka
{उदयाचल-तिलकाय प्रणतोऽस्मि विवस्वते ग्रहेशाय}
{अम्बर-चूडामणये दिग्वनिताकर्णपूराय}% ॥३॥

\twolineshloka
{जयति जनानन्दकरः करनिकर-निरस्त-तिमिर-सङ्घातः}
{लोकालोकालोकः कमलारुणमण्डलः सूर्यः}% ॥४॥

\twolineshloka
{प्रतिबोधित-कमलवनः कृतघटनश्चक्रवाकमिथुनानाम्}
{दर्शित-समस्तभुवनः परहितनिरतो रविः सदा जयति}% ॥५॥

\twolineshloka
{अपनयतु सकलकलिकृतमलपटलं सप्रतप्तकनकाभः}
{अरविन्द-वृन्द-विघटन-पटुतरकिरणोत्करः सविता}% ॥६॥

\twolineshloka
{उदयाद्रि-चारुचामरहरित हयखुरपरिहितरेणुराग}
{हरितहय हरितपरिकर गगनाङ्गनदीपक नमस्तेऽस्तु}% ॥७॥

\twolineshloka
{उदितवति त्वयि विकसति मुकुलीयति समस्त-मस्तमितबिम्बे}
{न ह्यन्यस्मिन्दिनकर सकलं कमलायते भुवनम्}% ॥८॥

\twolineshloka
{जयति रविरुदयसमये बालातपः कनकसन्निभो यस्य}
{कुसुमाञ्जलिरिव जलधौ तरन्ति रथसप्तयः सप्त}% ॥९॥

\twolineshloka
{आर्याः साम्बपुरे सप्त आकाशात्पतिता भुवि}
{यस्य कण्ठे गृहे वाऽपि न स लक्ष्म्या वियुज्यते}% ॥१०॥

\twolineshloka
{आर्याः सप्त सदा यस्तु सप्तम्यां सप्तधा जपेत्}
{तस्य गेहं च देहं च पद्मा सत्यं न मुञ्चति}% ॥११॥

\twolineshloka
{निधिरेश दरिद्राणां रोगिणां परमौषधम्}
{सिद्धिः सकलकार्याणां गाथेयं संसृता रवेः}% ॥१२॥

॥इति श्रीयाज्ञवल्क्यविरचितं सूर्यार्यास्तोत्रं सम्पूर्णम्॥

