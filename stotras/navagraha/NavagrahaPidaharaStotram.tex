% !TeX program = XeLaTeX
% !TeX root = ../../shloka.tex

\sect{नवग्रहपीडाहरस्तोत्रम्}

\twolineshloka
{ग्रहाणामादिरादित्यो लोकरक्षणकारकः}%
{विषमस्थानसम्भूतां पीडां हरतु मे रविः}%। १॥

\twolineshloka
{रोहिणीशः सुधामूर्तिः सुधागात्रः सुधाशनः}%
{विषमस्थानसम्भूतां पीडां हरतु मे विधुः}%। २॥

\twolineshloka
{भूमिपुत्रो महातेजा जगतां भयकृत् सदा}%
{वृष्टिकृद्वृष्टिहर्ता च पीडां हरतु मे कुजः}%। ३॥

\twolineshloka
{उत्पातरूपो जगतां चन्द्रपुत्रो महाद्युतिः}%
{सूर्यप्रियकरो विद्वान् पीडां हरतु मे बुधः}%। ४॥

\twolineshloka
{देवमन्त्री विशालाक्षः सदा लोकहिते रतः}%
{अनेकशिष्यसम्पूर्णः पीडां हरतु मे गुरुः}%। ५॥

\twolineshloka
{दैत्यमन्त्री गुरुस्तेषां प्राणदश्च महामतिः}%
{प्रभुस्ताराग्रहाणां च पीडां हरतु मे भृगुः}%। ६॥

\twolineshloka
{सूर्यपुत्रो दीर्घदेहो विशालाक्षः शिवप्रियः}%
{मन्दचारः प्रसन्नात्मा पीडां हरतु मे शनिः}%। ७॥

\twolineshloka
{महाशिरा महावक्त्रो दीर्घदंष्ट्रो महाबलः}%
{अतनुश्चोर्ध्वकेशश्च पीडां हरतु मे शिखी}%। ८॥

\twolineshloka
{अनेकरूपवर्णैश्च शतशोऽथ सहस्रशः}%
{उत्पातरूपो जगतां पीडां हरतु मे तमः}%। ९॥
॥इति ब्रह्माण्डपुराणोक्तं नवग्रहपीडाहरस्तोत्रं सम्पूर्णम्॥

\closesection

\fourlineindentedshloka*
{आरोग्यं प्रददातु नो दिनकरश्चन्द्रो यशो निर्मलं}
{भूतिं भूमिसुतः सुधांशुतनयः प्रज्ञां गुरुर्गौरवम्‌}
{काव्यः कोमलवाग्विलासमतुलं मन्दो मुदं सर्वदा}
{राहुर्बाहुबलं विरोधशमनं केतुः कुलस्योन्नतिम्}
