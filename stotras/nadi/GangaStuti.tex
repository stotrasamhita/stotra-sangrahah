% !TeX program = XeLaTeX
% !TeX root = ../../shloka.tex

\sect{गङ्गादशहरास्तोत्रम्}


\uvacha{ब्रह्मोवाच}
\twolineshloka
{नमः शिवायै गङ्गायै शिवदायै नमो नमः}
{नमस्ते रुद्ररूपिण्यै शाङ्कर्यै ते नमो नमः}%  ॥ १॥

\twolineshloka
{नमस्ते विश्वरूपिण्यै ब्रह्ममूर्त्यै नमो नमः}
{सर्वदेवस्वरूपिण्यै नमो भेषजमूर्तये}%  ॥ २॥

\twolineshloka
{सर्वस्य सर्वव्याधीनां भिषक्-श्रेष्ठे नमोऽस्तु ते}
{स्थाणुजङ्गमसम्भूतविषहन्त्र्यै नमो नमः}%  ॥ ३॥

\twolineshloka
{भोगोपभोगदायिन्यै भोगवत्यै नमो नमः}
{मन्दाकिन्यै नमस्तेऽस्तु स्वर्गदायै नमो नमः}%  ॥ ४॥

\twolineshloka
{नमस्त्रैलोक्यभूषायै जगद्धात्र्यै नमो नमः}
{नमः त्रिशुक्लसंस्थायै तेजोवत्यै नमो नमः}%  ॥ ५॥

\twolineshloka
{नन्दायै लिङ्गधारिण्यै नारायण्यै नमो नमः}
{नमस्ते विश्वमुख्यायै रेवत्यै ते नमो नमः}%  ॥ ६॥

\twolineshloka
{बृहत्यै ते नमस्तेऽस्तु लोकधात्र्यै नमो नमः}
{नमस्ते विश्वमित्रायै नन्दिन्यै ते नमो नमः}%  ॥ ७॥

\twolineshloka
{पृथ्व्यै शिवामृतायै च सुवृषायै नमो नमः}
{शान्तायै च वरिष्ठायै वरदायै नमो नमः}%  ॥ ८॥

\twolineshloka
{उस्रायै सुखदोग्ध्र्यै च सञ्जीविन्यै नमो नमः}
{ब्रह्मिष्ठायै ब्रह्मदायै दुरितघ्न्यै नमो नमः}

\twolineshloka
{प्रणतार्तिप्रभञ्जिन्यै जगन्मात्रे नमोऽस्तु ते}%  ॥ ९॥
{सर्वापत्प्रतिपक्षायै मङ्गलायै नमो नमः}%  ॥ १०॥

\twolineshloka
{शरणागतदीनार्तपरित्राणपरायणे}
{सर्वस्यार्तिहरे देवि नारायणि नमोऽस्तुते}%  ॥ ११॥

\twolineshloka
{निर्लेपायै दुर्गहन्त्र्यै दक्षायै ते नमो नमः}
{परात्परतरे तुभ्यं नमस्ते मोक्षदे सदा}%  ॥ १२॥

\twolineshloka
{गङ्गे ममाग्रतो भूया गङ्गे मे देवि पृष्ठतः}
{गङ्गे मे पार्श्वयोरेहि त्वयि गङ्गेऽस्तु मे स्थितिः}%  ॥ १३॥

\threelineshloka
{आदौ त्वमन्ते मध्ये च सर्वं त्वं गां गते शुभे}
{त्वमेव मूलप्रकृतिस्त्वं हि नारायणः परः}%  ॥ १४॥
{गङ्गे त्वं परमात्मा च शिवस्तुभ्यं नमः शिवे}

\twolineshloka
{य इदं पठति स्तोत्रं भक्त्या नित्यं नरोऽपि यः}%  ॥ १५॥
{शृणुयाच्छ्रद्धया युक्तः कायवाक्चित्तसम्भवैः}

\twolineshloka
{दशधा  संस्थितैर्दोषैः सर्वैरेव प्रमुच्यते}%  ॥ १६॥
{सर्वान्कामानवाप्नोति प्रेत्य ब्रह्मणि लीयते}

\twolineshloka
{ज्येष्ठे मासि सिते पक्षे दशमी हस्तसंयुता}%  ॥ १७॥
{तस्यां दशम्यामेतच्च स्तोत्रं गङ्गाजले स्थितः}

\twolineshloka
{यः पठेद्दशकृत्वस्तु दरिद्रो वाऽपि चाक्षमः}%  ॥ १८॥
{सोऽपि तत्फलमाप्नोति गङ्गां सम्पूज्य यत्नतः}

\twolineshloka
{अदत्तानामुपादानं हिंसा चैवाविधानतः}%  ॥ १९॥
{परदारोपसेवा च कायिकं त्रिविधं स्मृतम्}

\twolineshloka
{पारुष्यमनृतं चैव पैशुन्यं चापि सर्वशः}%  ॥ २०॥
{असम्बद्धप्रलापश्च वाङ्मयं स्याच्चतुर्विधम्}

\twolineshloka
{परद्रव्येष्वभिध्यानं मनसाऽनिष्टचिन्तनम्}%  ॥ २१॥
{वितथाभिनिवेशश्च मानसं त्रिविधं स्मृतम्}

\twolineshloka
{एतानि दश पापानि हर त्वं मम जाह्नवि}%  ॥ २२॥
{दशपापहरा यस्मात्तस्माद्दशहरा स्मृता}

\twolineshloka
{त्रयस्त्रिंशच्छतं पूर्वान् पितॄनथ पितामहान्}%  ॥ २३॥
{उद्धरत्येव संसारान्मन्त्रेणानेन पूजिता}%॥ २४॥

% \pagebreak

{नमो भगवत्यै दशपापहरायै गङ्गायै नारायण्यै रेवत्यै}\\
{शिवायै दक्षायै अमृतायै विश्वरूपिण्यै नन्दिन्यै ते नमो नमः॥}% २५॥

\fourlineindentedshloka
{सितमकरनिषण्णां शुभ्रवर्णां त्रिनेत्रां}
{करधृतकलशोद्यत्सोत्पलामत्यभीष्टाम्}%
{विधिहरिहररूपां सेन्दुकोटीरजुष्टां}
{कलितसितदुकूलां जाह्नवीं तां नमामि}%  ॥ २६॥

\fourlineindentedshloka
{आदावादिपितामहस्य निगमव्यापारपात्रे जलं}
        {पश्चात्पन्नगशायिनो भगवतः पादोदकं पावनम्}
{भूयः शम्भुजटाविभूषणमणिर्जह्नोर्महर्षेरियं}
        {देवी कल्मषनाशिनी भगवती भागीरथी दृश्यते}  %॥ २७॥

\twolineshloka
{गङ्गा गङ्गेति यो ब्रूयाद्योजनानां शतैरपि}
{मुच्यते सर्वपापेभ्यो विष्णुलोकं स गच्छति}%  ॥ २८॥

॥इति श्री-स्कान्दे महापुराणे एकाशीति साहस्र्यां संहितायां तृतीये काशीखण्डे धर्माब्धिस्था श्रीगङ्गादशहरास्तुतिः सम्पूर्णा॥