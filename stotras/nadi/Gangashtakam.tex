% !TeX program = XeLaTeX
% !TeX root = ../../shloka.tex

\sect{गङ्गाष्टकम्}

\fourlineindentedshloka
{भगवति भवलीलामौलिमाले तवाम्भः}
{कणमणुपरिमाणं प्राणिनो ये स्पृशन्ति}
{अमरनगरनारीचामरमरग्राहिणीनां}
{विगतकलिकलङ्कातङ्कमङ्के लुठन्ति}% ॥ २॥

\fourlineindentedshloka
{ब्रह्माण्डं खण्डयन्ती हरशिरसि जटावल्लिमुल्लासयन्ती}
{स्वर्लोकादापतन्ती कनकगिरिगुहागण्डशैलात् स्खलन्ती}
{क्षोणी पृष्ठे लुठन्ती दुरितचयचमूं निर्भरं भर्त्सयन्ती}
{पाथोधिं पूरयन्ती सुरनगरसरित् पावनी नः पुनातु}% ॥ ३॥

\fourlineindentedshloka
{मज्जनमातङ्गकुम्भच्युतमदमदिरामोदमत्तालिजालम्}
{स्नानैः सिद्धाङ्गनानां कुचयुगविगलत् कुङ्कुमासङ्गपिङ्गम्}
{सायं प्रातर्मुनीनां कुशकुसुमचयैश्छिन्नतीरस्थनीरम्}
{पायान्नो गाङ्गमम्भः करिकलभकराक्रान्तरं हस्तरङ्गम्}% ॥ ४॥

\fourlineindentedshloka
{आदावादिपितामहस्य नियमव्यापारपात्रे जलं}
{पश्चात् पन्नगशायिनो भगवतः पादोदकं पावनम्}
{भूयः शम्भुजटाविभूषणमणिर्जह्नोर्महर्षेरियं}
{कन्या कल्मषनाशिनी भगवती भागीरथी पातु माम्}% ॥ ५॥

\fourlineindentedshloka
{शैलेन्द्रादवतारिणी निजजल-मज्जज्जनोत्तारिणी}
{पारावारविहारिणी भवभयश्रेणी-समुत्सारिणी}
{शेषाहेरनुकारिणी हरशिरोवल्लीदलाकारिणी}
{काशीप्रान्तविहारिणी विजयते गङ्गा मनोहारिणी}% ॥ ६॥

\fourlineindentedshloka
{कुतो वीची वीचिस्तव यदि गता लोचनपथम्}
{त्वमापीता पीताम्बरपुग्निवासं वितरसि}
{त्वदुत्सङ्गे गङ्गे पतति यदि कायस्तनुभृताम्}
{तदा मातः शान्तक्रतवपदलाभोऽप्यतिलघुः}% ॥ ७॥

\fourlineindentedshloka
{भगवति तव तीरे नीरमात्राशनोऽहम्}
{विगतविषयतृष्णः कृष्णमाराधयामि}
{सकलकलुषभङ्गे स्वर्गसोपानगङ्गे}
{तरलतरतरङ्गे देवि गङ्गे प्रसीद}% ॥ १॥

\fourlineindentedshloka
{मातर्जाह्नवी शम्भुसङ्गवलिते मौलौ निधायाञ्जलिं}
{त्वत्तीरे वपुषोऽवसानसमये नारायणाङ्घ्रिद्वयम्}
{सानन्दं स्मरतो भविष्यति मम प्राणप्रयाणोत्सवे}
{भूयाद् भक्तिरविच्युता हरिहराद्वैतात्मिका शाश्वती}% ॥ ९॥

\twolineshloka
{गङ्गाष्टकमिदं पुण्यं यः पठेत् प्रयतो नरः}
{सर्वपापविनिर्भुक्तो विष्णुलोकं स गच्छति}% ॥ १०॥


॥इति श्रीमत्परमहंसपरिव्राजकाचार्यस्य श्री-गोविन्द-भगवत्पूज्य-पाद-शिष्यस्य
श्रीमच्छङ्करभगवतः कृतौ श्री-गङ्गाष्टकं सम्पूर्णम्॥

\fourlineindentedshloka*
{गङ्गे त्रैलोक्यसारे सकलसुरवधूधौतविस्तीर्णतोये}
{पूर्णब्रह्मस्वरूपे हरिचरणरजोहारिणि स्वर्गमार्गे}
{प्रायश्चितं यदि स्यात् तव जलकाणिक्रा ब्रह्महत्यादिपापे}
{कस्त्वां स्तोतुं समर्थः त्रिजगदघहरे देवि गङ्गे प्रसीद}% ॥ ८॥