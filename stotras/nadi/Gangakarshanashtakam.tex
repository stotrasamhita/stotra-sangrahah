% !TeX program = XeLaTeX
% !TeX root = ../../shloka.tex

\sect{गङ्गाकर्षणाष्टकम्}

\twolineshloka
{शम्भो भवन्नामनिरन्तरानुसन्धानभाग्येन भवन्तमेव}
{यद्येष सर्वत्र तथाऽन्त्यजेऽद्य पश्यत्यहो कोऽत्र कृतोऽपराधः}% ॥१॥

\twolineshloka
{अस्त्वेष मन्तुः पितृयज्ञनिष्ठे गङ्गाप्लवो यो विहितोऽपचित्यै}
{दूरात्तु तन्नामजपेन शुद्धिर्न स्यात् कथं मे स्मृतिरर्थवादः}% ॥२॥

\twolineshloka
{त्वन्नामनिष्ठा न हि तावती मे श्रद्धा यतः कर्मसु न प्रदग्धा}
{त्रैशङ्कवं मे पशुपान्तरायो मुच्येय तस्मात्कथमार्तबन्धो}% ॥३॥

\twolineshloka
{यद्यद्य ते श्राद्धविनष्टिरिष्टा कोऽहं ततोऽन्यच्चरितुं समर्थः}
{श्राद्धे वृताः पूर्वदिनोपवासा नान्यत्र भुञ्जीयुरिदं तु खिद्ये}% ॥४॥

\twolineshloka
{श्रद्धालवः श्राद्धविघातभीत्या स्वात्मोपरोधं विगणय्य धीराः}
{यत्प्रोचुरत्रापचितिं महान्तः तत्रोचितं यद्दयया विधेहि}% ॥५॥

\twolineshloka
{गङ्गाधर त्वद्भजनान्तरायभीत्या गृहे कूपकृतावगाहः}
{जाने न तीर्थान्तरमद्य गङ्गामासादयेयं कथमार्तबन्धो}% ॥६॥

\twolineshloka
{नाहं तपस्वी सगरान्ववायो जाने न जह्वश्चरति क्व वेति}
{शम्भो जटाजूटमपावृणुष्वेत्यभ्यर्थने नालमयं वराकः}% ॥७॥

\twolineshloka
{गङ्गाधराख्या गतिरत्र नान्या तामाश्रये सङ्कटमोचनाय}
{हन्त प्रवाहः कथमत्र कूपे विस्फूर्जतीशः खलु मे प्रसन्नः}% ॥८॥

\twolineshloka
{गङ्गेति गङ्गेति हरेति गृह्णन् आप्लावितोऽहं दयया पुरारेः}
{कूपोत्थितोऽयं करुणाप्रवाहः गाङ्गश्चिरायात्र जनान् पुनातु}% ॥९॥


॥इति श्रीमच्छ्रीधरार्यकृतं गङ्गाकर्षणाष्टकं सम्पूर्णम्॥