% !TeX program = XeLaTeX
% !TeX root = ../../shloka.tex

\sect{महालक्ष्मीस्तुतिः (अगस्त्यकृता)}

\uvacha{अगस्तिरुवाच}

\fourlineindentedshloka
{मातर्नमामि कमले कमलायताक्षि}
{श्रीविष्णुहृत्कमलवासिनि विश्वमातः}
{क्षीरोदजे कमलकोमलगर्भगौरि}
{लक्ष्मि प्रसीद सततं नमतां शरण्ये}% ॥ १ ॥

\fourlineindentedshloka
{त्वं श्रीरुपेन्द्रसदने मदनैकमातर्-}
{ज्योत्स्नाऽसि चन्द्रमसि चन्द्रमनोहरास्ये}
{सूर्ये प्रभाऽसि च जगत्त्रितये प्रभाऽसि}
{लक्ष्मि प्रसीद सततं नमतां शरण्ये}% ॥ २ ॥


\fourlineindentedshloka
{त्वं जातवेदसि सदा दहनात्मशक्तिर्-}
{वेधास्त्वया जगदिदं विविधं विदध्यात्}
{विश्वम्भरोऽपि बिभृयादखिलं भवत्या}
{लक्ष्मि प्रसीद सततं नमतां शरण्ये}% ॥ ३ ॥

\fourlineindentedshloka
{त्वत्त्यक्तमेतदमले हरते हरोऽपि}
{त्वं पासि हंसि विदधासि परावरासि}
{ईड्यो बभूव हरिरप्यमले त्वदाप्त्या}
{लक्ष्मि प्रसीद सततं नमतां शरण्ये}% ॥ ४ ॥

\fourlineindentedshloka
{शूरः स एव स गुणी स बुधः स धन्यो}
{मान्यः स एव कुलशीलकलाकलापैः}
{एकः शुचिः स हि पुमान् सकलेऽपि लोके}
{यत्रापतेत्तव शुभे करुणाकटाक्षः}% ॥ ५ ॥

\fourlineindentedshloka
{यस्मिन्वसेः क्षणमहो पुरुषे गजेऽश्वे}
{स्त्रैणे तृणे सरसि देवकुले गृहेऽन्ने}
{रत्ने पतत्रिणि पशौ शयने धरायां}
{सश्रीकमेव सकले तदिहास्ति नान्यत्}% ॥ ६ ॥

\fourlineindentedshloka
{त्वत्स्पृष्टमेव सकलं शुचितां लभेत}
{त्वत्त्यक्तमेव सकलं त्वशुचीह लक्ष्मि}
{त्वन्नाम यत्र च सुमङ्गलमेव तत्र}
{श्रीविष्णुपत्नि कमले कमलालयेऽपि}% ॥ ७ ॥

\fourlineindentedshloka
{लक्ष्मीं श्रियं च कमलां कमलालयां च}
{पद्मां रमां नलिनयुग्मकरां च मां च}
{क्षीरोदजाममृतकुम्भकरामिरां च}
{विष्णुप्रियामिति सदा जपतां क्व दुःखम्}% ॥ ८ ॥

\twolineshloka
{ये पठिष्यन्ति च स्तोत्रं त्वद्भक्त्या मत्कृतं सदा}
{तेषां कदाचित् सन्तापो माऽस्तु माऽस्तु दरिद्रता}% ॥ ९ ॥

\twolineshloka
{माऽस्तु चेष्टवियोगश्च माऽस्तु सम्पत्तिसङ्क्षयः}
{सर्वत्र विजयश्चाऽस्तु विच्छदो माऽस्तु सन्ततेः}% ॥ १० ॥

\uvacha{श्रीरुवाच}

\twolineshloka
{एवमस्तु मुने सर्वं यत्त्वया परिभाषितम्}
{एतत् स्तोत्रस्य पठनं मम सान्निध्यकारणम्}% ॥ ११ ॥

\twolineshloka
{अलक्ष्मीः कालकर्णी च तद्गेहे न विशेत् क्वचित्}
{गजाश्वपशुशान्त्यर्थमेतत्स्तोत्रं सदा जपेत्}% ॥ १२ ॥

\twolineshloka
{इदं बीजरहस्यं मे रक्षणीयं प्रयत्नतः}
{श्रद्धाहीने न दातव्यं न देयश्चाशुचौ क्वचित्}% ॥ १३ ॥

॥इति श्रीस्कन्दमहापुराणे एकाशीतिसाहस्र्यां संहितायां चतुर्थे काशीखण्डेऽगस्त्यप्रस्थानं नाम पञ्चमेऽध्याये श्री-अगस्तिकृता महालक्ष्मीस्तुतिः सम्पूर्णा॥

