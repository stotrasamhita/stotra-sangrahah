% !TeX program = XeLaTeX
% !TeX root = ../../shloka.tex

\sect{गणेशभुजङ्गम्}

\fourlineindentedshloka
{रणत् क्षुद्रघण्टानिनादाभिरामं}
{चलत् ताण्डवोद्दण्डवत्पद्मतालम्}
{लसत् तुन्दिलाङ्गोपरिव्यालहारं}
{गणाधीशमीशानसूनुं तमीडे}% १}%}

\fourlineindentedshloka
{ध्वनिध्वंसवीणालयोल्लासिवक्त्रं}
{स्फुरच्छुण्डदण्डोल्लसद्‌बीजपूरम्}
{गलद्दर्पसौगन्ध्यलोलालिमालं}
{गणाधीशमीशानसूनुं तमीडे}% २}%}

\fourlineindentedshloka
{प्रकाशज्जपारक्तरन्तप्रसून-}
{प्रवालप्रभातारुणज्योतिरेकम्}
{प्रलम्बोदरं वक्रतुण्डैकदन्तं}
{गणाधीशमीशानसूनुं तमीडे}% ३}%}

\fourlineindentedshloka
{विचित्रस्फुरद्रत्नमालाकिरीटं}
{किरीटोल्लसच्चन्द्ररेखाविभूषम्}
{विभूषैकभूषं भवध्वंसहेतुं}
{गणाधीशमीशानसूनुं तमीडे}% ४}%}

\fourlineindentedshloka
{उदञ्चद्भुजावल्लरीदृश्यमूलोच्-}
{चलद्‌-भ्रूलता-विभ्रमभ्राजदक्षम्}
{मरुत् सुन्दरीचामरैः सेव्यमानं}
{गणाधीशमीशानसूनुं तमीडे}% ५}%}

\fourlineindentedshloka
{स्फुरन्निष्ठुरालोलपिङ्गाक्षितारं}
{कृपाकोमलोदारलीलावतारम्}
{कलाबिन्दुगं गीयते योगिवर्यैर्-}
{गणाधीशमीशानसूनुं तमीडे}% ६}%}

\fourlineindentedshloka
{यमेकाक्षरं निर्मलं निर्विकल्पं}
{गुणातीतमानन्दमाकारशून्यम्}
{परं पारमोङ्कारमाम्नायगर्भं}
{वदन्ति प्रगल्भं पुराणं तमीडे}% ७}%}

\fourlineindentedshloka
{चिदानन्दसान्द्राय शान्ताय तुभ्यं}
{नमो विश्वकर्त्रे च हर्त्रे च तुभ्यम्}
{नमोऽनन्तलीलाय कैवल्यभासे}
{नमो विश्वबीज प्रसीदेशसूनो}% ८}%}

\fourlineindentedshloka
{इमं सुस्तवं प्रातरुत्थाय भक्त्या}
{पठेद्यस्तु मर्त्यो लभेत्सर्वकामान्}
{गणेशप्रसादेन सिद्ध्यन्ति वाचो}
{गणेशे विभौ दुर्लभं किं प्रसन्ने}% ९}%}

॥इति श्रीमत्परमहंसपरिव्राजकाचार्यस्य श्री-गोविन्द-भगवत्पूज्य-पाद-शिष्यस्य
श्रीमच्छङ्करभगवतः कृतौ श्री-गणेशभुजङ्गं सम्पूर्णम्॥
