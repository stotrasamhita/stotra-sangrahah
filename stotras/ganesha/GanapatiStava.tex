% !TeX program = XeLaTeX
% !TeX root = ../../shloka.tex

\sect{गणपतिस्तवः}

\uvacha{गर्ग ऋषिरुवाच}

\fourlineindentedshloka
{अजं निर्विकल्पं निराकारमेकं}
{निरानन्दमानन्दमद्वैतपूर्णम्}
{परं निर्गुणं निर्विशेषं निरीहं}
{परब्रह्मरूपं गणेशं भजेम}%॥ १॥

\fourlineindentedshloka
{गुणातीतमानं चिदानन्दरूपं}
{चिदाभासकं सर्वगं ज्ञानगम्यम्}
{मुनिध्येयमाकाशरूपं परेशं}
{परब्रह्मरूपं गणेशं भजेम}%॥ २॥

\fourlineindentedshloka
{जगत्कारणं कारणज्ञानरूपं}
{सुरादिं सुखादिं गुणेशं गणेशम्}
{जगद्व्यापिनं विश्ववन्द्यं सुरेशं}
{परब्रह्मरूपं गणेशं भजेम}%॥ ३॥

\fourlineindentedshloka
{रजोयोगतो ब्रह्मरूपं श्रुतिज्ञं}
{सदा कार्यसक्तं हृदाऽचिन्त्यरूपम्}
{जगत्कारणं सर्वविद्यानिदानं}
{परब्रह्मरूपं गणेशं नताः स्मः}%॥ ४॥

\fourlineindentedshloka
{सदा सत्ययोग्यं मुदा क्रीडमानः}
{सुरारीन् हरन्तं जगत्पालयन्तम्}
{अनेकावतारं निजज्ञानहारं}
{सदा विश्वरूपं गणेशं नमामः}%॥ ५॥


\fourlineindentedshloka
{तमोयोगिनं रुद्ररूपं त्रिनेत्रं}
{जगद्धारकं तारकं ज्ञानहेतुम्}
{अनेकागमैः स्वं जनं बोधयन्तं}
{सदा सर्वरूपं गणेशं नमामः}%॥ ६॥

\fourlineindentedshloka
{तमः स्तोमहारं जनाज्ञानहारं}
{त्रयीवेदसारं परब्रह्मसारम्}
{मुनिज्ञानकारं विदूरे विकारं}
{सदा ब्रह्मरूपं गणेशं नमामः}%॥ ७॥

\fourlineindentedshloka
{निजैरोषधीस्तर्पयन्तं कराद्यैः}
{सुरौघान्कलाभिः सुधास्राविणीभिः}
{दिनेशांशुसन्तापहारं द्विजेशं}
{शशाङ्कस्वरूपं गणेशं नमामः}%॥ ८॥

\fourlineindentedshloka
{प्रकाशस्वरूपं नभो वायुरूपं}
{विकारादिहेतुं कलाधाररूपम्}
{अनेकक्रियानेकशक्तिस्वरूपं}
{सदा शक्तिरूपं गणेशं नमामः}%॥ ९॥

\fourlineindentedshloka
{प्रधानस्वरूपं महत्तत्त्वरूपं}
{धराचारिरूपं दिगीशादिरूपम्}
{असत्सत्स्वरूपं जगद्धेतुरूपं}
{सदा विश्वरूपं गणेशं नताः स्मः}%॥ १०॥

\fourlineindentedshloka
{त्वदीये मनः स्थापयेदङ्घ्रियुग्मे}
{जनो विघ्नसङ्घातपीडां लभेत}
{लसत्सूर्यबिम्बे विशाले स्थितोऽयं}
{जनो ध्वान्तपीडां कथं वा लभेत}%॥ ११॥

\fourlineindentedshloka
{वयं भ्रामिताः सर्वथाऽज्ञानयोगा-}
{दलब्धास्तवाङ्घ्रिं बहून् वर्षपूगान्}
{इदानीमवाप्तास्तवैव प्रसादात्}
{प्रपन्नान् सदा पाहि विश्वम्भराद्य}%॥ १२॥

\twolineshloka*
{एवं स्तुतो गणेशस्तु सन्तुष्टोऽभून्महामुने}
{कृपया परयोपेतोऽभिधातुमुपचक्रमे}%॥ १३॥

॥इति श्रीमद्-गर्गऋषिकृतो श्री-गणपतिस्तवः सम्पूर्णः॥
