% !TeX program = XeLaTeX
% !TeX root = ../../shloka.tex

\chapt{कार्तवीर्यार्जुन-द्वादशनामस्तोत्रम्}

ॐ श्रीं क्रों कार्तवीर्यार्जुनाय नमः।
\twolineshloka
{कार्तवीर्यार्जुनो नाम राजा बाहुसहस्रवान्}
{तस्य स्मरणमात्रेण गतं नष्टं च लभ्यते}% ॥१॥

\twolineshloka
{कार्तवीर्यः खलद्वेषी कृतवीर्यसुतो बली}
{सहस्रबाहुः शत्रुघ्नो रक्तवासा धनुर्धरः}% ॥२॥

\twolineshloka
{रक्तगन्धो रक्तमाल्यो राजा स्मर्तुरभीष्टदः}
{द्वादशैतानि नामानि कार्तवीर्यस्य यः पठेत्}% ॥३॥

\twolineshloka
{सम्पदस्तत्र जायन्ते जनस्तत्र वशं गतः}
{आनयत्याशु दूरस्थं क्षेमलाभयुतं प्रियम्}% ॥४॥

\fourlineindentedshloka
{सहस्रबाहुसशरं महितं सचापं}
{रक्ताम्बरं रक्तकिरीटकुण्डलम्}
{चोरादि-दुष्टभय-नाशम् इष्टदं तं}
{ध्यायेन्महाबल-विजृम्भित-कार्तवीर्यम्}% ॥५॥

\twolineshloka
{यस्य स्मरणमात्रेण सर्वदुःखक्षयो भवेत्}
{यन्नामानि महावीर्यश्चार्जुनः कृतवीर्यवान्}% ॥६॥

\twolineshloka
{हैहयाधिपतेः स्तोत्रं सहस्रावृत्तिकारितम्}
{वाञ्चितार्थप्रदं नृणां स्वराज्यं सुकृतं यदि}% ॥७॥

॥इति कार्तवीर्य द्वादशनाम स्तोत्रम्॥