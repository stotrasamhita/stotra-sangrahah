% !TeX program = XeLaTeX
% !TeX root = shloka.tex
{\font \x="Sanskrit 2003:script=deva" at 12pt\x
\centerline{॥ॐ श्री-गणेशाय नमः॥}
\centerline{॥ॐ श्री-गुरुभ्यो नमः॥}
\centerline{॥हरिः ॐ॥}
}

\setmainfont{Candara}
\begin{center}
\chapter[\texorpdfstring{\large{\fontspec{Candara}{Preface}}}{Preface}]{\texorpdfstring{\scshape{Preface}}{Preface}}
\end{center}

{\font \x="Sanskrit 2003:script=deva" at 12pt\x
\twolineshloka*{सदाशिवसमारम्भां शङ्कराचार्यमध्यमाम्}
{अस्मदाचार्यपर्यन्तां वन्दे गुरुपरम्पराम्}
}

This book has been primarily inspired by two pieces of work --- one, my \textit{thāthā's} beautiful hand-written composition of \textit{ślokās}, for his grandchildren, and \textit{mantrapushpam}, the excellent compilation of
\textit{veda\-mantrās} and \textit{stotrās}, from Ramakrishna Mutt.

In this book, several wonderful stotras have been compiled. One of the aims of this book is to provide ready access to a number of \textit{stotrās} in a compact form. I've often had to refer to a bundle of books for each \textit{stotram}. This book, I hope, would prove to be really useful for people who would like to carry around these \textit{stotrās} when they travel around, or would like a handy small book containing all these \textit{stotrās}. The other important feature of this book is that all the \textit{stotrās} are in \textit{devanāgarī lipi}. I've often had access to a large number of \textit{stotrās}, but in Tamil script. I find the \textit{devanāgarī lipi} more conducive to correct pronunciation. There are several simple \textit{shlokās} in this book, which I am sure children would be able to pick up easily. \textit{Stotrās} such as \textit{Nāma Rāmāyanam} should certainly be taught to children. Many of these stotras have been rendered wonderfully by \textit{Śrīmatī} \textit{M.~S.~Subbulakshmi}; one just needs to listen to her for both \textit{bhakti} and inspiration. While the foremost importance is to be given to \textit{bhakti}, one must certainly give importance to accurate pronunciation as well, and \textit{MSS} is exemplary in that regard.

One should make it a point to chant at least few of these every day, and most of these in a month. One should certainly recite the \textit{Vishnu Sahasranāmam} everyday. Of course, it must be emphasised that one's \textit{nityakarmā} takes precedence over all these ({\font \x="Sanskrit 2003:script=deva" \x सन्ध्याहीनः अशुचिः नित्यमनर्हहः सर्व\-कर्मसु}) and one must make time for \textit{sandhyāvandanādi} \textit{nityakarmās} and such prayers everyday:
{\font \x="Sanskrit 2003:script=deva" at 12pt\x
\twolineshloka*{विप्रो वृक्षस्तस्य मूलं च सन्ध्या वेदाः शाखा धर्मकर्माणि पत्रम्}
{तस्मान्मूलं यत्नतो रक्षणीयं छिन्ने मूले नैव शाखा न पत्रम्}
}

\noindent In \textit{Kaliyuga}, foremost importance is given to \textit{nāmasankīrtanam}, and hence, stotras such as these should be recited with \textit{bhakti}, regularly:
{\font \x="Sanskrit 2003:script=deva" at 12pt\x
\twolineshloka*{ध्यायन् कृते यजन् यज्ञैः त्रेतायां द्वापरेऽर्चयन्}
{यदाप्नोति तदाप्नोति कलौ सङ्कीर्त्य केशवम्}
}

There are several people whom I must thank for their contributions to this book. I cannot undermine the importance of the Sanskrit Documents Website\footnote{\scriptsize \url{http://sanskritdocuments.org/}}, which happens to be the source for almost all of the texts contained in here. Many thanks to volunteers to build and present such a wonderful collection online.
\thispagestyle{fancy}
I must acknowledge the efforts of my friend \textit{Prasād}, who has been instrumental (and almost wholly responsible) for the improved formatting in this book. I consulted him several times for help with \XeLaTeX. But for his \TeX\ macros, some of the alignments would have never happened! I must also thank the writers of the software ITranslator\footnote{\scriptsize \url{http://www.omkarananda-ashram.org/Sanskrit/Itranslt.html}}, which has been the hammer-and-nail for compiling this book. The other tool critical for this book was \XeLaTeX, and it was indeed the release of MiK\TeX\ 2.7 that led me to experiment with \XeLaTeX, which I think has been a success.

I take this opportunity to seek the blessings of my \textit{Appā}, \textit{Ammā}, my \textit{Guru} \textit{Shri S.~Ananthakrishnan}, and my \textit{Māmā}, who have inspired me and taught me all that I know. I must definitely thank \textit{Sāketh} too, who has been inspirational in several ways.

I must specially thank my \textit{ammā}, who has encouraged and inspired me a lot through the course of compiling this book. I also must thank her for proof-reading the text, and particularly helping with \textit{Śyāmalā dandakam}. Thanks are also due to my wife, for her support and encouragement throughout.

Although we have put in efforts to remove any typographical errors in this book, I must emphasise that the errors in this book are solely due to my ignorance and I would be glad to rectify them. Please drop me a \textit{gmail} at \textit{karthik.raman} to notify me of even the smallest of errors.
\thispagestyle{fancy}
{\font \x="Sanskrit 2003:script=deva" at 12pt\x
\twolineshloka*{यदक्षरपदभ्रष्टं मात्राहीनं तु यद्भवेत्}
{तत्सर्वं क्षम्यतां देव नारायण नमोऽस्तु ते}
}

This book is dedicated to Śrī Krishna.
{\font \x="Sanskrit 2003:script=deva" at 12pt\x
\twolineshloka*{यत्करोषि यदश्नासि यज्जुहोषि ददासि यत्}
{यत्तपस्यसि कौन्तेय तत्कुरुष्व मदर्पणम्}
\hfill — श्रीमद्भगवद्गीता ९-२७
\centerline{सर्वम् श्री-कृष्णार्पणमस्तु॥}
}

\medskip
\noindent{May 16, 2008} \hfill \textsc{Karthik Raman}
