% !TeX program = XeLaTeX
% !TeX root = stotramanjari-nv.tex

\end{center}
\begingroup
\setlength{\emergencystretch}{3em}
\fontspec[Script=Devanagari]{Siddhanta}
\sect{देव्यथर्वशीर्षम्}
ॐ सर्वे वै देवा देवीमुपतस्थुः कासि त्वं महादेवीति साऽब्रवीदहं ब्रह्मस्वरूपिणी।
मत्तः प्रकृतिपुरुषात्मकं जगत्।
शून्यं चाशून्यं च।
अहमानन्दानानन्दौ।
अहं विज्ञानाविज्ञाने।
अहं ब्रह्माब्रह्मणी।
द्वे ब्रह्मणी वेदितव्ये।
इति चाथर्वणी श्रुतिः।

अहं पञ्चभूतानि।
अहं पञ्चतन्मात्राणि।
अहमखिलं जगत्।
वेदोऽहमवेदोऽहम्।
विद्याहमविद्याहम्।
अजाहमनजाहम्।
अधश्चोर्ध्वं च तिर्यक्चाहम्।
अहꣳ रुद्रेभिर्वसुभिश्चरामि।
अहमादित्यैरुत विश्वदेवैः।
अहं मित्रावरुणावुभौ बिभर्मि।
अहमिन्द्राग्नी अहमश्विना उभौ।
अहꣳ सोमं त्वष्टारं भगं दधामि।
अहं विष्णुमुरुक्रमम्।
ब्रह्माणमुत प्रजापतिं दधामि।
अहं दधामि द्रविणꣳ हविष्मते सुप्राव्ये यजमानाय सुव्रते।

अहꣳ राज्ञी सङ्गमनी वसूनां चिकितुषी प्रथमा यज्ञियानाम्।
अहꣳ सुवे पितरमस्य मूर्धन्मम योनिरप्स्वन्तः समुद्रे।
य एवं वेद स दैवी सम्पदमाप्नोति।
ते देवा अब्रुवन् नमो देव्यै महादेव्यै शिवायै सततं नमः।
नमः प्रकृत्यै भद्रायै नियताः प्रणताः स्म ताम्।

तामग्निवर्णां तपसा ज्वलन्तीं वैरोचनीं कर्मफलेषु जुष्टाम्।
दुर्गां देवीं शरणं प्रपद्यामहेऽसुरान्नाशयित्र्यै ते नमः।

देवीं वाचमजनयन्त देवास्तां विश्वरूपाः पशवो वदन्ति।
सा नो मन्द्रेषमूर्जं दुहाना धेनुर्वागस्मानुपसुष्टुतैतु॥
कालरात्री ब्रह्मस्तुतां वैष्णवीं स्कन्दमातरम् ।
सरस्वतीमदितिं दक्षदुहितरं नमामः पावनां शिवाम्।

महालक्ष्म्यै च विद्महे सर्वशक्त्यै च धीमहि।
तन्नो देवी प्रचोदयात्।

अदितिर्ह्यजनिष्ट दक्ष या दुहिता तव।
तां देवा अन्वजायन्त भद्रा अमृतबन्धवः॥
कामे योनिः कमला वज्रपाणिर्गुहा हंसा मातलिश्चाभ्रमिन्द्रः।
पुनर्गुहा सकला मायया चापृथक् क्लेशा विश्वमातादिविद्याः॥

एषात्मशक्तिः।
एषा विश्वमोहिनी पाशाङ्कुशधनुर्बाणधरा।
एषा श्रीमहाविद्या।
य एवं वेद स शोकं तरति।
नमस्ते भगवति मातरस्मान्पाहि सर्वतः।
सैषा वैष्णव्यष्टौ वसवः सैवैकादश रुद्राः सैषा द्वादशादित्याः सैषा विश्वेदेवाः सोमपा असोमपाश्च सैषा यातुधाना असुरा रक्षांसि पिशाचयक्षसिद्धाः।
सैषा सत्त्वरजस्तमांसि सैषा ब्रह्मविष्णुरुद्ररूपिणी सैषा प्रजापतीन्द्र सैषा ग्रहनक्षत्रज्योतिः कलाकाष्टादिविश्वरूपिणी तामहं प्रणौमि नित्यम्।
पापापहारिणीं देवीं भुक्तिमुक्तिप्रदायिनीम्।
अनन्तां विजयां शुद्धां शरण्यां सर्वदां शिवाम्।

वियदाकारसंयुक्तं वीतिहोत्रसमन्वितम्।
अर्धेदुलसितं देव्या बीजं सर्वार्थसाधकम्।

एवमेकाक्षरं मन्त्रं यतयः शुद्धचेतसः।
ध्यायन्ति परमानन्दमया ज्ञानाम्बुराशयः।

वाङ्मया ब्रह्मभूस्तस्मात्षष्ठवत्रसमन्वितम्।
सूर्यो वाम\-श्रोत्र\-बिन्दु\-संयुक्ताष्ट\-तृतीयकम्।

नारायणेन सम्मिश्रो वायुश्वाधारयुक्ततः।
विच्चेनवार्णकोणस्य महानानन्ददायकः।

हृत्पुण्डरीकमध्यस्थां प्रातः सूर्यसमप्रभाम्।
पाशाङ्कुशधरां सौम्या वरदाभय\-हस्तकाम्।
त्रिनेत्रां रक्तवसनां भक्तकामदुहं भजे।

भजामि त्वां महादेवि महाभयविनाशिनि।
महादारिद्र्यशमनि महाकारुण्यरूपिणि।

यस्याः स्वरूपं ब्रह्मादयो न जानन्ति तस्मादुच्यते अज्ञेया।
यस्या अन्तो न लभ्यते तस्मादुच्यते अनन्ता।
यस्या लक्षं नोपलक्ष्यते तस्मादुच्यते यस्या जननं नोपलक्ष्यते तस्मादुच्यते अजा।
एकैव सर्वत्र वर्तते तस्मादुच्यत एका।
एकैव विश्वरूपिणी तस्मादुच्यतेऽनेका।
अत एवोच्यतेऽज्ञेयाऽनन्तालक्ष्याऽजैकानेका।
मन्त्राणां मातृका देवी शब्दानां ज्ञानरूपिणी।
ज्ञानानां चिन्मयातीता शून्यानां शून्यसाक्षिणी॥

यस्याः परतरं नास्ति सैषा दुर्गा प्रकीर्तिता।
तां दुर्गा दुर्गमां देवीं दुराचारविघातिनीम्।
नमामि भवभीतोऽहं संसारार्णवतारिणीम्।

इदमथर्वशीर्षं योऽधीते।
स पञ्चाथर्वशीर्षफलमाप्नोति।
इदमथर्वशीर्षं ज्ञात्वा योऽर्चाꣳ स्थापयति।
शतलक्षं प्रजप्तापि नार्चाशुद्धिं च विन्दति।
शतमष्टोत्तरं चास्य पुरश्चर्याविधिः स्मृतः।

दशवारं पठेद्यस्तु सद्यः पापैः प्रमुच्यते।
महादुर्गाणि तरति महादेव्याः प्रसादतः।

सायमधीयानो दिवसकृतं पापं नाशयति।
प्रातरधीयानो रात्रिकृतं पापं नाशयति।
सायं प्रातः प्रयुञ्जानोऽपापो भवति।
निशीथे तुरीय\-सन्ध्यायां जप्त्वा वाक्सिद्धिर्भवति।
नूतनायां प्रतिमायां जप्त्वा देवता\-सान्निध्यं भवति।
भौमाश्विन्यां महादेवीसन्निधौ जप्त्वा महामृत्युं तरति स महामृत्युं तरति।
य एवं वेद।
इत्युपनिषत्॥

॥इति देव्यथर्वशीर्ष सम्पूर्णम्॥ \endgroup
\begin{center}