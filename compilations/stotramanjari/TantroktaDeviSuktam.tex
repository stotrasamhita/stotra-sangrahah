\sect{तन्त्रोक्तं देवीसूक्तम्}


\twolineshloka
{नमो देव्यै महादेव्यै शिवायै सततं नमः}
{नमः प्रकृत्यै भद्रायै नियताः प्रणताः स्म ताम्}% ॥ १॥

\twolineshloka
{रौद्रायै नमो नित्यायै गौर्यै धात्र्यै नमो नमः}
{ज्योत्स्नायै चेन्दुरूपिण्यै सुखायै सततं नमः}% ॥ २॥

\twolineshloka
{कल्याण्यै प्रणतां वृद्ध्यै सिद्ध्यै कुर्मो नमो नमः}
{नैरृत्यै भूभृतां लक्ष्म्यै शर्वाण्यै ते नमो नमः}% ॥ ३॥

\twolineshloka
{दुर्गायै दुर्गपारायै सारायै सर्वकारिण्यै}
{ख्यात्यै तथैव कृष्णायै धूम्रायै सततं नमः}% ॥ ४॥

\twolineshloka
{अतिसौम्यातिरौद्रायै नतास्तस्यै नमो नमः}
{नमो जगत्प्रतिष्ठायै देव्यै कृत्यै नमो नमः}% ॥ ५॥

\twolineshloka
{या देवी सर्वभूतेषु विष्णुमायेति शब्दिता}
{नमस्तस्यै नमस्तस्यै नमस्तस्यै नमो नमः}% ॥ ६॥

\twolineshloka
{या देवी सर्वभूतेषु चेतनेत्यभिधीयते}
{नमस्तस्यै नमस्तस्यै नमस्तस्यै नमो नमः}% ॥ ७॥

\twolineshloka
{या देवी सर्वभूतेषु बुद्धिरूपेण संस्थिता}
{नमस्तस्यै नमस्तस्यै नमस्तस्यै नमो नमः}% ॥ ८॥

\twolineshloka
{या देवी सर्वभूतेषु निद्रारूपेण संस्थिता}
{नमस्तस्यै नमस्तस्यै नमस्तस्यै नमो नमः}% ॥ ९॥

\twolineshloka
{या देवी सर्वभूतेषु क्षुधारूपेण संस्थिता}
{नमस्तस्यै नमस्तस्यै नमस्तस्यै नमो नमः}% ॥ १०॥

\twolineshloka
{या देवी सर्वभूतेषु छायारूपेण संस्थिता}
{नमस्तस्यै नमस्तस्यै नमस्तस्यै नमो नमः}% ॥ ११॥

\twolineshloka
{या देवी सर्वभूतेषु शक्तिरूपेण संस्थिता}
{नमस्तस्यै नमस्तस्यै नमस्तस्यै नमो नमः}% ॥ १२॥

\twolineshloka
{या देवी सर्वभूतेषु तृष्णारूपेण संस्थिता}
{नमस्तस्यै नमस्तस्यै नमस्तस्यै नमो नमः}% ॥ १३॥

\twolineshloka
{या देवी सर्वभूतेषु क्षान्तिरूपेण संस्थिता}
{नमस्तस्यै नमस्तस्यै नमस्तस्यै नमो नमः}% ॥ १४॥

\twolineshloka
{या देवी सर्वभूतेषु जातिरूपेण संस्थिता}
{नमस्तस्यै नमस्तस्यै नमस्तस्यै नमो नमः}% ॥ १५॥

\twolineshloka
{या देवी सर्वभूतेषु लज्जारूपेण संस्थिता}
{नमस्तस्यै नमस्तस्यै नमस्तस्यै नमो नमः}% ॥ १६॥

\twolineshloka
{या देवी सर्वभूतेषु शान्तिरूपेण संस्थिता}
{नमस्तस्यै नमस्तस्यै नमस्तस्यै नमो नमः}% ॥ १७॥

\twolineshloka
{या देवी सर्वभूतेषु श्रद्धारूपेण संस्थिता}
{नमस्तस्यै नमस्तस्यै नमस्तस्यै नमो नमः}% ॥ १८॥

\twolineshloka
{या देवी सर्वभूतेषु कान्तिरूपेण संस्थिता}
{नमस्तस्यै नमस्तस्यै नमस्तस्यै नमो नमः}% ॥ १९॥

\twolineshloka
{या देवी सर्वभूतेषु लक्ष्मीरूपेण संस्थिता}
{नमस्तस्यै नमस्तस्यै नमस्तस्यै नमो नमः}% ॥ २०॥

\twolineshloka
{या देवी सर्वभूतेषु वृत्तिरूपेण संस्थिता}
{नमस्तस्यै नमस्तस्यै नमस्तस्यै नमो नमः}% ॥ २१॥

\twolineshloka
{या देवी सर्वभूतेषु स्मृतिरूपेण संस्थिता}
{नमस्तस्यै नमस्तस्यै नमस्तस्यै नमो नमः}% ॥ २२॥

\twolineshloka
{या देवी सर्वभूतेषु दयारूपेण संस्थिता}
{नमस्तस्यै नमस्तस्यै नमस्तस्यै नमो नमः}% ॥ २३॥

\twolineshloka
{या देवी सर्वभूतेषु तुष्टिरूपेण संस्थिता}
{नमस्तस्यै नमस्तस्यै नमस्तस्यै नमो नमः}% ॥ २४॥

\twolineshloka
{या देवी सर्वभूतेषु मातृरूपेण संस्थिता}
{नमस्तस्यै नमस्तस्यै नमस्तस्यै नमो नमः}% ॥ २५॥

\twolineshloka
{या देवी सर्वभूतेषु भ्रान्तिरूपेण संस्थिता}
{नमस्तस्यै नमस्तस्यै नमस्तस्यै नमो नमः}% ॥ २६॥

\twolineshloka
{इन्द्रियाणामधिष्ठात्री भूतानां चाखिलेषु या}
{भूतेषु सततं तस्यै व्याप्त्यै दैव्यै नमो नमः}% ॥ २७॥

\twolineshloka
{चित्तिरूपेण या कृत्स्नमेतद्व्याप्य स्थितां जगत्}
{नमस्तस्यै नमस्तस्यै नमस्तस्यै नमो नमः}% ॥ २८॥

\fourlineindentedshloka
{स्तुता सुरैः पूर्वमभीष्टसंश्रयात्}
{तथा सुरेन्द्रेण दिनेषु सेविता}
{करोतु सा नः शुभहेतुरीश्वरी}
{शुभानि भद्राण्यभिहन्तु चापदः}% ॥ २९॥

\fourlineindentedshloka
{या साम्प्रतं चोद्धतदैत्यतापितै-}
{रस्माभिरीशा च सुरैर्नमस्यते}
{या च स्मृता तत्क्षणमेव हन्ति नः}
{सर्वापदो भक्तिविनम्रमूर्तिभिः}% ॥ ३०॥