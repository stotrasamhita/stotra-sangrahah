%! TeX program = XeLaTeX
%!TeX root = nityaparayanam-vedam.tex
%\part{वेदमन्त्राः}
% \fontsize{19pt}{23pt}\selectfont
\newcounter{anuvakam}
\newcommand{\anuvakamend}{\refstepcounter{anuvakam}\\[-0.8ex]
\rule[0.5ex]{0.93\textwidth}{1.5pt}\bfseries{[\devanumber{\arabic{anuvakam}}]}%\hrulefill
}
\setlength{\emergencystretch}{3em}
\input{"../../../vedamantra-book/mantras/MrtyunjayaHomaMantrah.tex"}
\clearpage
\input{"../../../vedamantra-book/mantras/Ganapatyatharvashirsham.tex"}
\clearpage
\input{"../../../vedamantra-book/mantras/DurgaSuktam.tex"} 
\clearpage
\input{"../../../vedamantra-book/mantras/MedhaSuktam.tex"} 
\closesection

\clearpage
\input{"../../../vedamantra-book/mantras/NavagrahaSuktam.tex"} 
\clearpage
\input{"../../../vedamantra-book/mantras/AyushyaSuktam.tex"} 
\clearpage
\chapt{वेदविस्मरणाय जपमन्त्राः}
\centerline{\scriptsize (तैत्तिरीयारण्यकम्/प्रपाठकः – १०/अनुवाकः – ८–९)}
यश्छन्द॑सामृष॒भो वि॒श्वरू॑प॒श्छन्दोभ्य॒श्छन्दास्यावि॒वेश॑। सता शिक्यः पुरोवाचो॑पनि॒षदिन्द्रो ज्ये॒ष्ठ इ॑न्द्रि॒याय॒ ऋषि॑भ्यो॒ नमो॑ दे॒वेभ्य॑ स्व॒धा पि॒तृभ्यो॒ भूर्भुव॒ सुव॒श्छन्द॒ ओम्॥२२॥ नमो॒ ब्रह्म॑णे धा॒रणं॑ मे अ॒स्त्वनि॑राकरणं धा॒रयि॑ता भूयासं॒ कर्ण॑योः श्रु॒तं मा च्योढ्वं॒ ममा॒मुष्य॒ ओम्॥२३॥

\chapt{मेधा-विलास-सिद्ध्यर्थं जपमन्त्रः}
सद॑स॒स्पति॒मद्भु॑तं प्रि॒यमिन्द्र॑स्य॒ काम्यम्। सनिं॑ मे॒धाम॑यासिषम्।
\closesection

\chapt{ब्राह्मण-वेदविद्भ्यो नमस्काराः}
ये अ॒र्वाङु॒त वा॑ पुरा॒णे वे॒दं वि॒द्वास॑म॒भितो॑ वदन्त्यादि॒त्यमे॒व ते परि॑वदन्ति॒ सर्वे॑ अ॒ग्निं द्वि॒तीयं॑ तृ॒तीयं॑ च ह॒समिति॒ याव॑ती॒र्वै दे॒वता॒स्ताः सर्वा॑ वेद॒विदि॑ ब्राह्म॒णे व॑सन्ति॒ तस्माद्ब्राह्म॒णेभ्यो॑ वेद॒विद्भ्यो॑ दि॒वे दि॑वे॒ नम॑स्कुर्या॒न्नाश्ली॒लं कीर्तयेदे॒ता ए॒व दे॒वता प्रीणाति॥

\twolineshloka*
{आब्रह्मलोकादाशेषाद् आलोकालोकपर्वतात्}
{ये वसन्ति द्विजा देवास्तेभ्यो नित्यं नमो नमः}

\chapt{समष्ट्यभिवादनम्}

नमः॒ प्राच्यै॑ दि॒शे याश्च॑ दे॒वता॑ ए॒तस्यां॒ प्रति॑वसन्त्ये॒ताभ्य॑श्च॒  नमो॒ नमो दक्षि॑णायै दि॒शे याश्च॑ दे॒वता॑ ए॒तस्यां॒ प्रति॑वसन्त्ये॒ताभ्य॑श्च॒  नमो॒ नमः॒ प्रती᳚च्यै दि॒शे याश्च॑ दे॒वता॑ ए॒तस्यां॒ प्रति॑वसन्त्ये॒ताभ्य॑श्च॒  नमो॒ नम॒ उदी᳚च्यै दि॒शे याश्च॑ दे॒वता॑ ए॒तस्यां॒ प्रति॑वसन्त्ये॒ताभ्य॑श्च॒  नमो॒ नम॑ ऊ॒र्ध्वायै॑ दि॒शे याश्च॑ दे॒वता॑ ए॒तस्यां॒ प्रति॑वसन्त्ये॒ताभ्य॑श्च॒  नमो॒ नमोऽध॑रायै दि॒शे याश्च॑ दे॒वता॑ ए॒तस्यां॒ प्रति॑वसन्त्ये॒ताभ्य॑श्च॒  नमो॒ नमो॑ऽवान्त॒रायै॑ दि॒शे याश्च॑ दे॒वता॑ ए॒तस्यां॒ प्रति॑वसन्त्ये॒ताभ्य॑श्च॒  नमो॒ नमो गङ्गायमुनयोर्मध्ये ये॑ वस॒न्ति॒ ते मे प्रसन्नात्मानश्चिरं जीवितं व॑र्धय॒न्ति॒ नमो गङ्गायमुनयोर्मुनि॑भ्यश्च॒ नमो॒ नमो गङ्गायमुनयोर्मुनि॑भ्यश्च॒ नमः॥

\twolineshloka*
{यां सदा सर्वभूतानि चराणि स्थावराणि च}
{सायं प्रातर्नमस्यन्ति सा मा सन्ध्याऽभिरक्षतु}

\twolineshloka*
{शिवाय विष्णुरूपाय शिवरूपाय विष्णवे}
{शिवस्य हृदयं विष्णुर्विष्णोश्च हृदयं शिवः}

\twolineshloka*
{यथा शिवमयो विष्णुरेवं विष्णुमयः शिवः}
{यथाऽन्तरं न पश्यामि तथा मे स्वस्तिरायुषि}

\twolineshloka*
{नमो ब्रह्मण्य-देवाय गो-ब्राह्मण-हिताय च}
{जगद्धिताय-कृष्णाय गोविन्दाय नमो नमः}

\input{"../../../vedamantra-book/mantras/AghamarshanaSuktam.tex"} 

\closesection

हिर॑ण्यवर्णा॒ शुच॑यः पाव॒का यासु॑ जा॒तः क॒श्यपो॒ यास्विन्द्र॑।
अ॒ग्निं या गर्भं॑ दधि॒रे विरू॑पा॒स्ता न॒ आप॒ श स्यो॒ना भ॑वन्तु॥ 
यासा॒ राजा॒ वरु॑णो॒ याति॒ मध्ये॑ सत्यानृ॒ते अ॑व॒पश्यं॒ जना॑नाम्।
म॒धु॒श्चुत॒ शुच॑यो॒ याः पा॑व॒कास्ता न॒ आप॒ श स्यो॒ना भ॑वन्तु॥ 
यासां दे॒वा दि॒वि कृ॒ण्वन्ति॑ भ॒क्षं या अ॒न्तरि॑क्षे बहु॒धा भव॑न्ति।
याः पृ॑थि॒वीं पय॑सो॒न्दन्ति॑ शु॒क्रास्ता न॒ आप॒ श स्यो॒ना भ॑वन्तु॥ 
शि॒वेन॑ मा॒ चक्षु॑षा पश्यताऽऽपः शि॒वया॑ त॒नुवोप॑ स्पृशत॒ त्वचं॑ मे।
सर्वा अ॒ग्नी र॑फ्सु॒षदो॑ हुवे वो॒ मयि॒ वर्चो॒ बल॒मोजो॒ नि ध॑त्त॥
\clearpage

\input{"../../../vedamantra-book/mantras/VaishvadevaMantrah.tex"} 
\clearpage

\input{"../../../vedamantra-book/mantras/TrisuparnaMantrah.tex"} 

\chapt{सूर्य-मन्त्राः}

\centerline{\scriptsize (तैत्तिरीयारण्यके प्रश्नः – १० (महानारयणोपनिषत्))}

घृणिः सूर्य॑ आदि॒त्यो न प्रभा॑ वा॒त्यक्ष॑रम्। 
मधु॑ क्षरन्ति॒ तद्र॑सम्। 
स॒त्यं वै तद्रस॒मापो॒ ज्योती॒रसो॒ऽमृतं॒ ब्रह्म॒ भूर्भुवः॒ सुव॒रोम्॥५४॥

\begin{minipage}{\linewidth}
\centerline{\scriptsize (तैत्तिरीयसंहितायां काण्डः १/प्रश्नः – ४)}
%1.4.31.1
त॒रणि॑र्वि॒श्वद॑र्\mbox{}शतो ज्योति॒ष्कृद॑सि सूर्य। विश्व॒मा भा॑सि रोच॒नम्॥ उ॒प॒या॒मगृ॑हीतो\-ऽसि॒ सूर्या॑य त्वा॒ भ्राज॑स्वत ए॒ष ते॒ योनिः॒ सूर्या॑य त्वा॒ भ्राज॑स्वते॥३२॥
\end{minipage}

\begin{minipage}{\linewidth}
\centerline{\scriptsize (तैत्तिरीयब्राह्मणे अष्टकं – ३/प्रश्नः – ७/अनुवाकः – ६/ पञ्चादयः ७६-७७)}

उ॒द्यन्न॒द्य मि॑त्रमहः। 
आ॒रोह॒न्नुत्त॑रां॒ दिवम्᳚।
हृ॒द्रो॒गं मम॑ सूर्य।
ह॒रि॒माणं॑ च नाशय।
शुके॑षु मे हरि॒माणम्᳚।
रो॒प॒णाका॑सु दध्मसि॥
अथो॑ हारिद्र॒वेषु॑ मे।
ह॒रि॒माणं॒ नि द॑ध्मसि।
उद॑गाद॒यमा॑दि॒त्यः।
विश्वे॑न॒ सह॑सा स॒ह।
द्वि॒षन्तं॒ मम॑ र॒न्धयन्।
मो अ॒हं द्वि॑ष॒तो र॑धम्।
मित्र-रवि-सूर्य-भानु-खग-पूष-हिरण्यगर्भ-मरीच्यादित्य-सवित्रर्क-भास्करेभ्यो नमः॥
\end{minipage}

\begin{minipage}{\linewidth}
\dnsub{आदित्यमण्डले परब्रह्मोपासनम्}
\centerline{\scriptsize (तैत्तिरीयारण्यके प्रश्नः – १० (महानारयणोपनिषत्))}
आ॒दि॒त्यो वा ए॒ष ए॒तन्म॒ण्डलं॒ तप॑ति॒ तत्र॒ ता ऋच॒स्तदृ॒चा म॑ण्डल॒ꣳ॒ स ऋ॒चां लो॒कोऽथ॒ य ए॒ष ए॒तस्मि॑न्म॒ण्डले॒ऽर्चिर्दी॒प्यते॒ तानि॒ सामा॑नि॒ स सा॒म्नां म॒ण्डल॒ꣳ॒ स सा॒म्नां लो॒कोऽथ॒ य ए॒ष ए॒तस्मि॑न्म॒ण्डले॒ऽर्चिषि॒ पुरु॑ष॒स्तानि॒ यजूꣳ॑षि॒ स यजु॑षा मण्डल॒ꣳ॒ स यजु॑षां लो॒कः सैषा त्र॒य्येव॑ वि॒द्या त॑पति॒ य ए॒षो᳚ऽन्तरा॑दि॒त्ये हि॑र॒ण्मयः॒ पुरु॑षः॥३१॥
%६.१४.०
\end{minipage}


\begin{minipage}{\linewidth}
\dnsub{आदित्यपुरुषस्य सर्वात्मकत्वप्रदर्शनम्}
\centerline{\scriptsize (तैत्तिरीयारण्यके प्रश्नः – १० (महानारयणोपनिषत्))}
आ॒दि॒त्यो वै तेज॒ ओजो॒ बलं॒ यश॒श्चक्षुः॒ श्रोत्र॑मा॒त्मा मनो॑ म॒न्युर्मनु॑र्मृ॒त्युः स॒त्यो मि॒त्रो वा॒युरा॑का॒शः प्रा॒णो लो॑कपा॒लः कः किं कं तथ्स॒त्यमन्न॑म॒मृतो॑ जी॒वो विश्वः॑ कत॒मः स्व॑य॒म्भु ब्रह्मै॒तदमृ॑त ए॒ष पुरु॑ष ए॒ष भू॒ताना॒मधि॑पति॒र्ब्रह्म॑णः॒ सायु॑ज्यꣳ सलो॒कता॑माप्नोत्ये॒तासा॑मे॒व दे॒वता॑ना॒ꣳ॒ सायु॑ज्यꣳ सा॒र्ष्टिताꣳ॑ समानलो॒कता॑माप्नोति॒ य ए॒वं वेदे᳚त्युप॒निषत्॥३२॥
\end{minipage}


\closesection
\clearpage
\input{"../../../vedamantra-book/mantras/BhagyaSuktam.tex"} 
\clearpage



\begin{minipage}{\linewidth}
\centerline{\scriptsize (तैत्तिरीयारण्यके प्रश्नः – १ (अरुणप्रश्नः) / अनुवाकः - १२/ पञ्चादिः ५८)}
नि॒घृष्वै॑रस॒मायु॑तैः। कालैर्‌हरित्व॑माप॒न्नैः। 
इन्द्राऽऽया॑हि स॒हस्र॑\-युक्। अ॒ग्निर्वि॒भ्राष्टि॑वसनः। 
वा॒युः श्वेत॑सिकद्रु॒कः। सं॒व॒थ्स॒रो वि॑षू॒\-वर्णैः᳚। 
नित्या॒स्तेऽ\-नु\-च॑रास्त॒व। सुब्रह्मण्योꣳ सुब्रह्मण्योꣳ सु॑ब्रह्म॒ण्योम्। 
\end{minipage}

\begin{minipage}{\linewidth}
\centerline{\scriptsize (तैत्तिरीयब्राह्मणे अष्टकं – ३/प्रश्नः – ७/अनुवाकः – ७/ पञ्चादयः ९०-९१)}
%3.7.7.12
स॒ख्यात्ते॒ मा यो॑षम्।
स॒ख्यान्मे॒ मा यो᳚ष्ठाः।
साऽसि॑ सुब्रह्मण्ये।
तस्या᳚स्ते पृथि॒वी पादः॑।
साऽसि॑ सुब्रह्मण्ये।
तस्या᳚स्ते॒\-ऽन्तरि॑क्षं॒ पादः॑।
साऽसि॑ सुब्रह्मण्ये।
तस्या᳚स्ते॒ द्यौः पादः॑।
साऽसि॑ सुब्रह्मण्ये।
तस्या᳚स्ते॒ दिशः॒ पादः॑॥९०॥

%3.7.7.13
प॒रोर॑जास्ते पञ्च॒मः पादः॑।
सा न॒ इष॒मूर्जं॑ धुक्ष्व।
तेज॑ इन्द्रि॒यम्।
ब्र॒ह्म॒व॒र्च॒सम॒न्नाद्यम्᳚।
वि मि॑मे त्वा॒ पय॑स्वतीम्।
दे॒वानां धे॒नुꣳ सु॒दुघा॒मन॑पस्फुरन्तीम्।
इन्द्रः॒ सोमं॑ पिबतु।
क्षेमो॑ अस्तु नः।
\end{minipage}

\input{"../../../vedamantra-book/mantras/PavamanaSuktam.tex"} 
\clearpage
\input{"../../../vedamantra-book/mantras/BhuSuktam.tex"} 
\clearpage
\input{"../../../vedamantra-book/mantras/ShriSuktam.tex"} 
\clearpage
\input{"../../../vedamantra-book/mantras/VishnuSuktam.tex"} 
\clearpage
\input{"../../../vedamantra-book/mantras/RudraPrashnah.tex"} 
\clearpage
\input{"../../../vedamantra-book/mantras/ChamakaPrashnah.tex"} 
\input{"../../../vedamantra-book/mantras/PurushaSuktam.tex"} 
\clearpage
\input{"../../../vedamantra-book/mantras/NarayanaSuktam.tex"} 
\clearpage
\setmainfont[Script=Devanagari,Mapping=tex-text]{Sanskrit 2003}
%\fontsize{16pt}{20pt}\selectfont